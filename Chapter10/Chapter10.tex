\section{Inductance}
\subsection{Mutual Inductance}
\begin{figure}[h]
\epsfysize=3in
\centerline{\epsffile{Chapter10/fig10.1.eps}}
\caption{\em Two inductively coupled circuits.}\label{f10.1}
\end{figure}
Consider two arbitrary conducting circuits, labelled 1 and 2. Suppose
that $I_1$ is the instantaneous current flowing around circuit 1. This current
generates a magnetic field ${\bf B}_1$ which links the second circuit, giving
rise to a magnetic flux ${\mit\Phi}_{2}$ through that circuit. 
If the current $I_1$ doubles, then the magnetic field ${\bf B}_1$ doubles
in strength at all points in space, so the magnetic flux ${\mit\Phi}_{2}$
through the second circuit also doubles. This conclusion follows from the
{\em linearity}\/ of the laws of magnetostatics, plus the definition of
magnetic flux. Furthermore, it is  obvious that the flux  through the
second circuit is zero whenever the current  flowing around the first circuit
is zero. It follows that the flux ${\mit\Phi}_{2}$ through the second
circuit is {\em directly proportional}\/ to the current $I_1$ flowing around the
first circuit. Hence, we can write
\begin{equation}\label{e10.1}
 {\mit\Phi}_{2} = M_{21}\,I_1,
\end{equation}
where the constant of proportionality $M_{21}$ is called the {\em mutual
inductance}\/ of circuit 2 with respect to circuit 1. Similarly, the flux
${\mit\Phi}_{1}$ through the first circuit due to the instantaneous
current $I_2$ flowing around the second circuit is directly proportional to
that current, so we can write
\begin{equation}\label{e10.2}
 {\mit\Phi}_{1} = M_{12}\,I_2,
\end{equation}
where $M_{12}$ is the {\em mutual
inductance} of circuit 1 with respect to circuit 2. It is possible to demonstrate
mathematically that $M_{12}= M_{21}$. In other words, the flux linking
circuit 2 when a certain current flows around circuit 1 is exactly the same
as the flux linking circuit 1 when the same current flows around circuit
2. This is true irrespective of the size, number of turns,
relative position, and relative
orientation of the two circuits. Because of this, we can write
\begin{equation}\label{e10.3}
M_{12}= M_{21} = M,
\end{equation}
where $M$ is termed the {\em mutual inductance}\/ of the two circuits. Note that
$M$ is a purely geometric quantity, depending only on the  size, number of turns,
relative position, and relative
orientation of the two circuits. The SI units of mutual inductance
are called {\em Henries}\/ (H). One henry is equivalent to a volt-second
per ampere:
\begin{equation}
1\,{\rm H} \equiv 1\,{\rm V\,s\,A}^{-1}.
\end{equation}
It turns out that a henry is a rather unwieldy unit. The mutual inductances of 
the circuits typically encountered in laboratory experiments
are measured in
 milli-henries. 
 
Suppose that the current flowing around circuit 1 changes by an amount
$d I_1$ in a time interval $d t$. It follows from
Eqs.~(\ref{e10.1}) and (\ref{e10.3}) that the flux linking circuit 2 changes by an amount
$d{\mit\Phi}_{2} = M\,d I_1$ in the same time interval.
According to Faraday's law, an emf 
\begin{equation}
{\cal E}_2 = - \frac{d{\mit\Phi}_{2}}{d t}
\end{equation}
is generated around the second circuit due to the changing magnetic flux
linking that circuit. Since, $d{\mit\Phi}_{2} = M\,d I_1$,
this emf can also be written
\begin{equation}\label{e10.6}
{\cal E}_2 = - M\, \frac{d I_1}{d  t}.
\end{equation}
Thus, the emf generated around the second circuit due to
the current flowing around  the first circuit is directly proportional
to the rate at which that  current changes.
Likewise, if the current $I_2$ flowing around the second circuit changes by an
amount $d I_2$ in a time interval $d t$ then
the emf generated around the first circuit is
\begin{equation}
{\cal E}_1 = - M\, \frac{d I_2}{d t}.
\end{equation}
Note that there is no direct physical coupling between the two circuits. The coupling
is due entirely to the magnetic field generated by the currents flowing around the
circuits. 

As a simple example, suppose that two insulated wires are wound on the
same cylindrical former, so as to form two solenoids sharing  a common 
air-filled core. Let $l$ be the length of the core, $A$ the cross-sectional
area of the core, $N_1$ the number of times the first wire is
wound around the core, and $N_2$ the number of times the second wire is
wound around the core. If a current $I_1$ flows around the first
wire then a uniform axial magnetic field of strength $B_1 = \mu_0\,N_1\,I_1/l$
is generated in the core  (see Sect.~\ref{s8.8}). The magnetic
field in  the region outside the core  is of negligible
magnitude. The flux linking a single turn of the second wire is $B_1\,A$. Thus,
the flux linking all $N_2$ turns of the second wire is ${\mit\Phi}_{2}
= N_2\,B_1\,A = \mu_0\,N_1\,N_2\,A\,I_1/l$. From Eq.~(\ref{e10.1}), the mutual
inductance of the second wire with respect to the first is
\begin{equation}
M_{21}= \frac{{\mit\Phi}_{2}}{I_1} = \frac{\mu_0\,N_1\,N_2\,A}{l}.
\end{equation}
Now, the flux linking the second wire when a current $I_2$ flows in the
first wire is ${\mit\Phi}_{1} = N_1\,B_2\,A$, where $B_2 = \mu_0\,N_2\,I_2/l$
is the associated magnetic field generated in the core.
It follows from Eq.~(\ref{e10.2}) that the mutual inductance of the first  wire with respect to the second  is 
\begin{equation}
M_{12} =  \frac{{\mit\Phi}_{1}}{I_2} = \frac{\mu_0\,N_1\,N_2\,A}{l}.
\end{equation}
Note that $M_{21} = M_{12}$, in accordance with Eq.~(\ref{e10.3}). Thus,
the mutual inductance
of the two wires is given by
\begin{equation}\label{e10.10}
M = \frac{\mu_0\,N_1\,N_2\,A}{l}.
\end{equation}
As described previously,  $M$ is a geometric quantity depending on
the dimensions of the core, and the manner in which the two wires are
wound around the core, but not on the  actual currents flowing through
the wires.

\subsection{Self Inductance}\label{s10.2}
We do not necessarily need  two circuits in order to have inductive effects. Consider
a single conducting circuit around which a current $I$ is
flowing. This current generates a magnetic field ${\bf B}$ which
gives rise to a magnetic flux ${\mit\Phi}$ linking the
circuit.  We  expect the flux ${\mit\Phi}$ to be directly proportional
to the current $I$, given the linear nature of the laws of magnetostatics,
and the definition of magnetic flux. Thus, we can write
\begin{equation}\label{e10.11}
{\mit\Phi} = L\,I,
\end{equation}
where the constant of proportionality $L$ is called the {\em self inductance}\/ of the
circuit. Like mutual inductance, the self inductance
of a circuit is measured in units of henries, and is a
purely geometric quantity,  depending only on the
shape of the circuit and number of turns in the circuit. 

If the current flowing around the circuit changes by an
amount $dI$ in a time interval $dt$ then the
magnetic flux linking the circuit changes by an amount $d{\mit\Phi}
= L\,d I$ in the same time interval. According to
Faraday's law, an emf
\begin{equation}
{\cal E} = - \frac{d{\mit\Phi}}{d t}
\end{equation}
is generated around the circuit. Since  $d{\mit\Phi}
= L\,d I$,
this emf can also be written
\begin{equation}\label{e10.13}
{\cal E} = - L\,\frac{dI}{d t}.
\end{equation}
Thus, the emf generated around the circuit due to its own current is directly
proportional to the rate at which the current changes. Lenz's law, and
common sense, demand that if the current is increasing then the emf should
always 
act to reduce the current, and {\em vice versa}. This is easily appreciated,
since if
 the emf acted to increase the
current when the current was increasing then we would clearly get an unphysical
positive feedback 
effect in which the current continued to increase without limit. It follows, from
Eq.~(\ref{e10.13}), that the
self inductance $L$ of a circuit is necessarily a {\em positive} number. This
is not the case for mutual inductances, which can be either positive or negative. 

Consider a solenoid of length $l$ and cross-sectional
area $A$. Suppose that the solenoid has $N$ turns.
When a current $I$ flows in the solenoid, a uniform axial field of magnitude
\begin{equation}\label{e10.19}
B = \frac{\mu_0\,N\,I}{l}
\end{equation}
is generated in the core of the solenoid. The field-strength outside the core
is
negligible. The magnetic flux linking a single turn of the solenoid is
${\mit\Phi} = B\,A$. Thus, the magnetic flux linking all $N$ turns of
the solenoid is
\begin{equation}
{\mit\Phi}= N\,B\,A = \frac{\mu_0\,N^2\,A\,I}{l}.
\end{equation}
According to Eq.~(\ref{e10.11}), the self inductance of the solenoid is given by
$L= {\mit\Phi}/I$, which reduces to
\begin{equation}\label{e10.21}
L = \frac{\mu_0\,N^2\,A}{l}.
\end{equation}
Note that $L$ is positive. Furthermore, $L$ is a geometric quantity depending
only on the dimensions of the solenoid, and the number of turns in the solenoid.

Engineers 
like to reduce all pieces of electrical apparatus, no matter how complicated, to
an {\em equivalent circuit}\/ consisting of a network of
just {\em four}\/ different types
of component. These four basic components are {\em emfs}, {\em resistors}, {\em capacitors},
and {\em inductors}. An inductor is simply a pure self inductance, and is usually
represented a little solenoid in circuit diagrams. In practice, inductors
generally consist of short air-cored solenoids wound from enameled copper wire. 

\subsection{Energy Stored in an Inductor}\label{s10.3}
Suppose that an inductor of inductance $L$ is connected to a
variable DC voltage supply.
The supply is adjusted so as to increase the current $i$ flowing through
the inductor from zero to some final value $I$. As the current through the
inductor is ramped up, an emf ${\cal E} = - L\,d i/dt$
is generated, which acts to oppose the increase in the current. Clearly, work must
be done against this emf by the voltage source in order to establish the
current in the inductor. The work done by the voltage source during a
time interval $d t$ is
\begin{equation}
d W = P\,dt = -{\cal E}\,i\,dt
= i\,L\,\frac{d i}{d t} \,d t = L\,i\,di.
\end{equation}
Here, $P=-{\cal E}\,i$ is the instantaneous rate at which the voltage source performs work.
To find the total work $W$ done in establishing the final current $I$ in the
inductor, we must integrate the above expression. Thus,
\begin{equation}
W = L\int_0^I i\,di,
\end{equation}
giving
\begin{equation}\label{e10.18}
W  = \frac{1}{2}\,L\,I^2.
\end{equation}
This energy is actually stored in the magnetic field generated by the current
flowing through the inductor. In a pure inductor, the energy is stored without
loss, and is returned to the rest of the circuit when the current through the
inductor is ramped down, and its associated magnetic field collapses. 

Consider a simple solenoid.
Equations~ (\ref{e10.19}), (\ref{e10.21}),  and (\ref{e10.18}) can be combined to give
\begin{equation}
W = \frac{1}{2} \,L\,I^2 =\frac{\mu_0\,N^2\,A}{2\,l} \,\left(\frac{B\,l}{\mu_0\,N}
\right)^2,
\end{equation}
which reduces to
\begin{equation}
W = \frac{B^2}{2\,\mu_0} \,\,l\,A.
\end{equation}
This represents the energy stored in the magnetic field of the solenoid. 
However, the volume of the field-filled core of the solenoid is $l\,A$, so the magnetic
energy density ({\em i.e.}, the energy per unit volume) inside the
solenoid is $w = W/(l\,A)$, or
\begin{equation}\label{e10.24}
w = \frac{B^2}{2\,\mu_0}.
\end{equation}
It turns out that this result is quite general. Thus, we can calculate the
energy content of any magnetic field by dividing space into little cubes
(in each of which the magnetic field is approximately uniform), applying the
above formula to find the energy content of each cube, and summing
the energies thus obtained to find the total energy. 

When electric and magnetic fields exist together in space, Eqs.~(\ref{e6.23}) and
(\ref{e10.24}) can be combined to give an expression for
the total energy stored in the combined
fields per unit volume:
\begin{equation}
w = \frac{\epsilon_0\,E^2}{2} + \frac{B^2}{2\,\mu_0}.
\end{equation}

\subsection{The $RL$ Circuit}
Consider a circuit in which a battery of emf $V$ is connected in series
with an inductor of inductance $L$ and a resistor of resistance $R$.
For obvious reasons, this type of circuit
 is usually called an {\em $RL$ circuit}. The resistance $R$ includes the 
resistance of the wire loops
of the inductor, in addition to any other resistances in the circuit. 

In steady-state, the current $I$ flowing around the the circuit  has the magnitude
\begin{equation}
I = \frac{V}{R}
\end{equation}
specified by Ohm's law. Note that, in a steady-state, or DC, circuit,
zero  back-emf is generated by the inductor, according to Eq.~(\ref{e10.13}),
so the inductor effectively
disappears from the circuit. In fact, inductors have no effect whatsoever
 in DC circuits.
They just act like pieces of conducting wire. 

\begin{figure}[h]
\epsfysize=2in
\centerline{\epsffile{Chapter10/fig10.2.eps}}
\caption{\em An $RL$ circuit with a switch.}\label{f10.3}
\end{figure}

Let us  now slightly modify our $RL$ circuit by introducing a switch.
The new circuit is shown in Fig.~\ref{f10.3}. Suppose that the switch is initially open, but is 
suddenly closed
at $t=0$. Obviously, we expect the instantaneous current $i$  which flows
around  the circuit, once the switch is thrown, to eventually settle
down to the steady-state 
value $I=V/R$. But, how long does this process take? Note that as the current flowing around
the circuit is building up to its final value, a non-zero back-emf is generated in the
inductor, according to Eq.~(\ref{e10.13}). Thus, although the inductor does not
affect the final steady-state value of the current flowing around the circuit,
it certainly does affect how long after the switch is closed it takes
for this final current to be established. 

If the instantaneous current $i$ flowing around the circuit changes by an
amount $di$ in a short time interval $d t$, then the
emf generated in the inductor is given by [see Eq.~(\ref{e10.13})]
\begin{equation}
{\cal E} = - L\,\frac{d i}{d t}.
\end{equation}
Applying Ohm's law around the circuit, we obtain
\begin{equation}
V + {\cal E} = i\,R,
\end{equation}
which yields
\begin{equation}\label{e10.29}
-L\,\frac{d i}{d t} = i\,R - V.
\end{equation}
Let
\begin{equation}
i' = i - I,\label{e10.30}
\end{equation}
where $I=V/R$ is the steady-state current. Equation~(\ref{e10.29}) can be rewritten
\begin{equation}\label{e10.31}
\frac{d i'}{d t} = -i'\,\frac{R}{L},
\end{equation}
since $di'=di$ (because $I$ is non-time-varying). 
At $t=0$, just after the switch is closed, we expect the current $i$ flowing
around the circuit to be zero. It follows from Eq.~(\ref{e10.30}) that
\begin{equation}\label{e10.32}
i'(t=0)=-I.
\end{equation}

Integration of Eq.~(\ref{e10.31}), subject to the initial condition
(\ref{e10.32}), yields
\begin{equation}
i'(t) = -I\,{\rm e}^{-R\,t/L}.
\end{equation}
Thus, it follows from Eq.~(\ref{e10.30}) that
\begin{equation}
i(t) = I\,(1-{\rm e}^{-R\,t/L}).
\end{equation}
The above expression specifies the current $i$ flowing around the circuit
a time interval $t$ after the switch is closed (at time $t=0$). The variation of
the current with time is sketched in Fig.~\ref{f10.4}. 
It can be seen that
when the switch is closed the current $i$ flowing in the circuit does not suddenly
jump up to its final value, $I=V/R$. Instead, the current increases smoothly
from zero, and gradually asymptotes to its final value. The current
has risen to  approximately $63\%$ of its
final value a time
\begin{equation}
\tau = \frac{L}{R}
\end{equation}
after the switch is closed (since ${\rm e}^{-1} \simeq 0.37$). By the time $t=5\,\tau$, the current has risen to
more than $99\%$ of its final value (since ${\rm e}^{-5}< 0.01$). Thus, $\tau=L/R$ is a good measure of
how long after the
switch is closed it takes for the current flowing in the circuit to attain its steady-state
value. The quantity $\tau$ is termed the {\em time-constant}, or, somewhat
unimaginatively,  the {\em L over R time}, of the circuit.

\begin{figure}
\epsfysize=2in
\centerline{\epsffile{Chapter10/fig10.3.eps}}
\caption{\em Sketch of the current rise phase in an $RL$ circuit switched
on at $t=0$.}\label{f10.4}
\end{figure}

Suppose that the current flowing in the circuit discussed above has settled
down to its steady-state value $I=V/R$. 
Consider what would happen
if we were to suddenly (at $t=0$, say) switch the battery out
of the circuit, and replace it by a conducting wire. Obviously,
we would expect the current to
eventually decay away to zero, since there is no longer a steady emf in the circuit
to maintain a steady current. But, how long does this process take?

Applying Ohm's law around the circuit, in the absence of the battery,
we obtain
\begin{equation}
{\cal E} = i\,R,
\end{equation}
where ${\cal E}= - L\,di/dt$ is the back-emf generated
by the inductor, and $i$ is the instantaneous current flowing around the circuit. 
The above equation reduces to
\begin{equation}\label{e10.42}
\frac{di}{dt} = -i\,\frac{R}{L}.
\end{equation}
At $t=0$, immediately after the battery is switched out of the circuit,
we expect the current $i$ flowing around the circuit to equal its
steady-state value $I$, so that
\begin{equation}\label{e10.43}
i(t=0)=I.
\end{equation}

Integration of Eq.~(\ref{e10.42}), subject to the boundary condition (\ref{e10.43}), yields
\begin{equation}
i(t) = I\,{\rm e}^{-R\,t/L}.
\end{equation}
According to the above formula, once the battery is switched out of
the circuit, the current decays smoothly to zero. After one $L/R$ time
({\em i.e.}, $t=L/R$), the current has decayed to $37\%$ of its initial
value. After five $L/R$ times, the current has decayed to less than
$1\%$ of its initial value. 

We can now appreciate the significance of self inductance. The back-emf
generated in an inductor, as the current flowing through it
tries to change, effectively prevents the
current from rising (or falling) much faster than the L/R time of the
circuit. This effect is
sometimes advantageous, but is often a great nuisance.
All circuits possess some self inductance, as well as some resistance, so
all have a finite $L/R$ time. This means that when we power up a DC circuit, the current
does not jump up instantaneously to its steady-state value. Instead, the
rise is spread out over the $L/R$ time of the circuit. This is a good thing.
If the current were to rise instantaneously then extremely large 
inductive electric
fields would be generated by the sudden jump in the magnetic field, leading,
inevitably, to breakdown and electric arcing. So, if there were no such thing
as self inductance then every time we switched a DC electric circuit on or off
there would be a big blue flash due to arcing between
 conductors. Self inductance
can also be a bad thing. Suppose that we possess a fancy power supply, and wish
to use it to send an electric signal down a wire.
Of course, the wire will possess both resistance and inductance,
and will, therefore, have some characteristic $L/R$ time. Suppose that we
try to send a square-wave signal down the wire. Since the current in the wire
cannot rise or fall faster than the $L/R$ time,  the leading and trailing edges of
the signal get smoothed out over an $L/R$ time. The typical difference between
the signal fed into the wire (upper trace) and that which comes out of the
other end (lower trace) is illustrated in Fig.~\ref{f10.5}. Clearly, there is little
point in us having a fancy power supply unless we also possess a low inductance
wire, so that the signal from the power supply can be
transmitted to some load device without serious distortion. 

\begin{figure}
\epsfysize=2in
\centerline{\epsffile{Chapter10/fig10.5.ps}}
\caption{\em Typical difference between the input wave-form (top)
and the output wave-form (bottom) when a square-wave is sent down a
line with finite $L/R$ time, $\tau$.}\label{f10.5}
\end{figure}

\subsection{The $RC$ Circuit}\label{s10.6}
Let us now discuss a topic which, admittedly, has nothing whatsoever to
do with inductors, but is mathematically so similar to the topic just discussed
that it seems sensible to consider it at
this point. 

Consider a circuit in which a battery of emf $V$ is connected in series with
a capacitor of capacitance $C$, and a resistor of resistance $R$.
For fairly obvious reasons, such a circuit is generally referred to as
an {\em $RC$ circuit}. In steady-state, the charge on the positive plate
of the capacitor is given by $Q=C\,V$, and zero current flows around the circuit
(since current cannot flow across the insulating gap between the capacitor
plates). 

\begin{figure}
\epsfysize=2in
\centerline{\epsffile{Chapter10/fig10.4.eps}}
\caption{\em An $RC$ circuit with a switch.}\label{f10.6}
\end{figure}

Let us now introduce a switch into the circuit, as shown in Fig.~\ref{f10.6}. Suppose
that the switch is initially open, but is suddenly closed at $t=0$. It is
assumed that the capacitor plates are uncharged when the switch is thrown. 
We expect a transient current $i$ to flow around the circuit until
the charge $q$ on the positive plate of the capacitor attains its
final steady-state value $Q=C\,V$. But, how long does this process take?

The potential difference $v$ between the positive and negative plates
of the capacitor is given by
\begin{equation}
v = V - i\,R.
\end{equation}
In other words, the potential difference between the plates is the emf
of the battery minus the potential drop across the resistor. The charge
$q$ on the positive plate of the capacitor is written
\begin{equation}\label{e10.46}
q = C\, v = Q -i\,R\,C,
\end{equation}
where $Q=C\,V$ is the final charge. Now, if $i$ is the instantaneous current
flowing around the circuit, then in a short time interval $dt$ the
charge on the positive plate of the capacitor increases by a
small amount $d q = i\,dt$ (since all of the charge 
which flows around the circuit must accumulate on the plates of the
capacitor). It follows that
\begin{equation}\label{e10.47}
i = \frac{d q}{d t}.
\end{equation}
Thus, the instantaneous current flowing around the circuit is
numerically equal to the rate at which the charge accumulated
on the positive plate of
the capacitor increases with time. Equations~(\ref{e10.46}) and
(\ref{e10.47}) can be combined together to give
\begin{equation}\label{e10.48}
\frac{d q'}{dt} = - \frac{q'}{R\,C},
\end{equation}
where 
\begin{equation}\label{e10.49}
q' = q -Q.
\end{equation}
At $t=0$, just after the switch is closed, the charge on the positive
plate of the capacitor is zero, so
\begin{equation}\label{e10.50}
q'(t=0) = -Q.
\end{equation}

Integration of Eq.~(\ref{e10.48}), subject to the boundary condition
(\ref{e10.50}), yields
\begin{equation}\label{e10.51}
q'(t) = -Q \,{\rm e}^{-t/R\,C}.
\end{equation}
It follows from Eq.~(\ref{e10.49}) that
\begin{equation}
q(t) = Q\,(1- {\rm e}^{-t/R\,C}).
\end{equation}
The above expression specifies the charge $q$ on the positive plate of
the capacitor a time interval $t$ after the switch is closed
(at time $t=0$). The variation of the charge with time is
sketched in Fig.~\ref{f10.7}. 
It can be seen that when the switch is closed the charge $q$ on the
positive plate of the capacitor
does not suddenly jump up to its final value, $Q=C\,V$. Instead, the charge
increases smoothly from zero, and gradually asymptotes to its final value.
The charge has risen to approximately $63\%$ of its final value a 
time 
\begin{equation}
\tau = R\,C
\end{equation}
after the switch is closed. By the time $t=5\,\tau$, the charge has risen to
more than $99\%$ of its final value. Thus, $\tau=R\,C$ is a good measure of
how long after the
switch is closed it takes for the capacitor to fully charge up.
 The quantity $\tau$ is termed the {\em time-constant},  or the {\em  $RC$ time}, of the circuit.

\begin{figure}
\epsfysize=2in
\centerline{\epsffile{Chapter10/fig10.6.eps}}
\caption{\em Sketch of the charging phase in an $RC$ circuit switched on at $t=0$.}\label{f10.7}
\end{figure}

According to Eqs.~(\ref{e10.47}) and (\ref{e10.48}),
\begin{equation}
i = \frac{dq}{d t} = \frac{dq'}{dt}
= - \frac{q'}{R\,C}.
\end{equation}
It follows from Eq.~(\ref{e10.51}) that
\begin{equation}
i(t) = I\,{\rm e}^{-t/R\,C},
\end{equation}
where $I=V/R$. The above expression specifies the current $i$ flowing
around the circuit a time interval $t$ after the switch is closed
(at time $t=0$). It can be seen that, immediately after the switch is
thrown, the current $I=V/R$ which flows in the circuit is that which would
flow if the capacitor were replaced by a conducting wire. However, this
current is only transient, and rapidly decays away to a negligible value.
After one $RC$ time, the current has decayed to 37\% of its initial value.
After five $RC$ times, the current has decayed to less than 1\% of its initial
value. It is interesting to note that for a short instant of time,
just  after the
switch is closed, the current in the circuit acts as if there is no insulating
gap between the capacitor plates. It essentially takes an $RC$ time for the
information about the break in the
circuit to propagate around the circuit, and cause the current to stop
flowing. 

Suppose that we take a capacitor of capacitance $C$, which is charged to a voltage
$V$, and discharge it by connecting a resistor of resistance $R$ across
its terminals at $t=0$. How long does it take for the capacitor to
discharge? By analogy with the previous analysis, the charge $q$ on the
positive plate of the capacitor at time $t$ is given by
\begin{equation}
q(t) = Q\,{\rm e}^{-t/R\,C},
\end{equation}
where $Q=C\,V$ is the initial charge on the positive plate. It can be seen that
it takes a few $RC$ times for the capacitor to fully discharge. 
The current $i$ which flows through the resistor is
\begin{equation}
i(t) = I\,{\rm e}^{-t/R\,C},
\end{equation}
where $I = V/R$ is the initial current. It can be seen that the capacitor
initially acts like a battery of emf $V$ (since it drives the current
$I=V/R$ across the resistor), but that, as it discharges, its effective 
emf decays to a negligible value on a few $RC$ times. 

\subsection{Transformers}
A transformer is a device for stepping-up, or stepping-down, the voltage  of
an alternating electric 
signal. Without efficient transformers, the  transmission and
distribution of AC
electric power over long distances would be impossible. Figure~\ref{f10.8}
shows the circuit diagram of a typical transformer.
There are two  circuits. Namely, the {\em primary circuit}, and the {\em secondary circuit}.
There is no direct electrical connection between the two circuits, but 
each circuit contains a coil which links it {\em inductively}\/ to the other circuit.
In real transformers, the two coils are wound onto the same iron core.
The purpose of the iron core is to channel the magnetic flux generated by
the current flowing around the primary coil, so that 
as much of it as possible also links the
secondary coil. The common  magnetic flux linking the two coils is conventionally
denoted in circuit diagrams by a number of parallel straight lines drawn between the coils.

\begin{figure}[h]
\epsfysize=2.5in
\centerline{\epsffile{Chapter10/fig10.7.eps}}
\caption{\em Circuit diagram of a transformer.}\label{f10.8}
\end{figure}

Let us consider a particularly simple transformer in which the primary and secondary
coils are {\em solenoids}\/ sharing the same air-filled core. Suppose that
$l$ is the length of the core, and $A$ is its cross-sectional area. Let $N_1$ be
the total  number of turns in the primary coil, and let $N_2$ be the 
total number of turns
in the secondary coil. Suppose that an alternating voltage
\begin{equation}\label{e10.58}
v_1 = V_1\,\cos (\omega\, t)
\end{equation}
is fed into the primary circuit from some external AC power source. Here,
$V_1$ is the peak voltage in the primary circuit, and $\omega$ is the
alternation frequency (in radians per second). The current driven around the
primary circuit is written
\begin{equation}\label{e10.59}
i_1 = I_1\,\sin (\omega\, t),
\end{equation}
where $I_1$ is the peak current. This current generates a
changing  magnetic flux, 
in the core of the solenoid, which links the secondary coil, and, thereby, 
inductively generates the alternating emf
\begin{equation}\label{e10.60}
v_2= V_2\,\cos (\omega\, t)
\end{equation}
in the secondary circuit, where $V_2$ is the peak voltage. Suppose that this
emf drives an alternating current
\begin{equation}\label{e10.61}
i_2 = I_2\,\sin (\omega\, t)
\end{equation}
around the secondary circuit, where $I_2$ is the peak current. 

The circuit equation for the primary circuit is written
\begin{equation}\label{e10.62}
v_1 - L_1\,\frac{di_1}{d t} - M\,\frac{di_2}{d t}=0,
\end{equation}
assuming that there is negligible resistance in this circuit. The first term
in the above equation is the externally generated emf. The second term is
the back-emf due to the self inductance $L_1$ of the primary coil. The
final term is the emf due to the mutual inductance $M$ of the primary
and secondary coils. In the absence of any significant resistance in the primary
circuit, these three emfs must add up to zero. Equations (\ref{e10.58}), (\ref{e10.59}),
(\ref{e10.61}), and (\ref{e10.62}) can be combined to give
\begin{equation}\label{e10.63}
V_1 = \omega\,(L_1\,I_1 + M\,I_2),
\end{equation}
since
\begin{equation}\label{e10.64}
\frac{d\sin(\omega\,t)}{dt} = \omega\,\cos(\omega\,t).
\end{equation}

The alternating emf generated in the secondary circuit consists of the
emf generated by the self inductance $L_2$ of the secondary coil, plus the
emf generated by the mutual inductance of the primary and secondary coils.
Thus,
\begin{equation}\label{e10.65}
v_2 = L_2\,\frac{di_2}{d t} + M\,\frac{di_1}{d t}.
\end{equation}
Equations  (\ref{e10.59}), (\ref{e10.60}), 
(\ref{e10.61}), (\ref{e10.64}), and (\ref{e10.65}) yield
\begin{equation}\label{e10.66}
V_2 = \omega \,(L_2\,I_2 + M\,I_1).
\end{equation}

Now, the instantaneous power output of the external AC power source which drives the
primary circuit is
\begin{equation}
P_1 = i_1\,v_1.
\end{equation}
Likewise, the instantaneous electrical energy per unit time transfered inductively from the
primary to the secondary circuit is
\begin{equation}
P_2 = i_2\,v_2.
\end{equation}
If resistive losses in the primary
and secondary circuits are negligible, as is assumed to be the case, then,
by energy conservation, these
two powers must  equal one another at all times. Thus,
\begin{equation}
i_1\,v_1= i_2\,v_2,
\end{equation}
which easily reduces to
\begin{equation}\label{e10.70}
I_1\,V_1 = I_2\, V_2.
\end{equation}
Equations (\ref{e10.63}), (\ref{e10.66}), and (\ref{e10.70}) yield
\begin{equation}
I_1\,V_1 = \omega\,(L_1\,I_1^{~2} + M\,I_1\,I_2)
= \omega \,(L_2\,I_2^{~2} + M\,I_1\,I_2) = I_2\,V_2,
\end{equation}
which gives 
\begin{equation}
\omega\,L_1\,I_1^{~2} = \omega \,L_2\,I_2^{~2},
\end{equation}
and, hence, 
\begin{equation}\label{e10.73}
\frac{I_1}{I_2} = \sqrt{\frac{L_2}{L_1}}.
\end{equation}
Equations~(\ref{e10.70}) and (\ref{e10.73}) can be combined to give
\begin{equation}
\frac{V_1}{V_2} =  \sqrt{\frac{L_1}{L_2}}.
\end{equation}
Note that, although the mutual inductance of the two coils is 
entirely responsible for the transfer of
energy  between the primary and secondary circuits, it is the self inductances
of the two coils which determine the ratio of the peak voltages and
peak currents in these circuits. 

Now, from Sect.~\ref{s10.2}, the self inductances of the primary and
secondary  coils  are given by $L_1=\mu_0\,N_1^{~2} \,A/l$
and $L_2= \mu_0\,N_2^{~2} \,A/l$, respectively. It follows
that
\begin{equation}
\frac{L_1}{L_2} = \left(\frac{N_1}{N_2}\right)^2,
\end{equation}
and, hence, that 
\begin{equation}\label{e10.76}
\frac{V_1}{V_2} = \frac{I_2}{I_1} = \frac{N_1}{N_2}.
\end{equation}
In other words, the ratio of the peak voltages and peak currents
in the primary and secondary circuits is determined by the ratio of
the number of turns in the primary and secondary coils. This latter ratio
is usually called the {\em turns-ratio} of the transformer. If
the secondary coil contains {\em more}\/ turns than the primary coil then the
peak voltage in the secondary circuit {\em  exceeds}\/ that in the primary circuit.
This type of transformer is called a {\em step-up transformer,} because
it steps up the voltage of an AC signal. Note that in a step-up
transformer the peak current in the secondary
circuit is {\em less}\/ than the peak current in the primary circuit  (as must be the case if energy is to be conserved). Thus,
a step-up transformer actually steps down the current. Likewise,
if the secondary coil contains {\em less}\/ turns than the primary coil
then the peak voltage in the secondary circuit is {\em  less}\/ than that
in the primary circuit. This type of transformer is called a {\em step-down
transformer}. Note that a step-down transformer actually steps up the
current ({\em i.e.}, the peak current in the secondary circuit
exceeds that in the primary circuit).

AC electricity is generated in power stations at a fairly low peak voltage
({\em i.e.}, something like 440\,V), and is consumed by the domestic
user at a peak voltage of 110\,V (in the U.S.). However, AC electricity
is transmitted from the power station to the location where it is consumed
at a very high peak voltage (typically 50\,kV). In fact, as soon as an AC signal
comes out of a generator in a power station it is fed into a step-up
transformer which boosts its peak voltage from a few hundred volts to many tens
of kilovolts. The output from the step-up transformer is fed into a
high tension transmission line, which typically transports the electricity over
many tens of kilometers, and, once the electricity has reached its
point of consumption, it is fed through a series of step-down transformers
until, by the time it emerges from a domestic power socket, its peak voltage is
only 110\,V. But, if AC electricity is both generated and consumed at
comparatively low peak voltages, why go to the trouble of
stepping up the peak voltage to a very high value at the
power station, and then stepping down the voltage again once the electricity
has reached its point of consumption? Why not generate, transmit, and
distribute the electricity at a peak voltage of 110\,V?
Well, consider an electric
power line which transmits a peak electric power $P$ between a power station
and  a city. We can think of $P$, which
depends on the number of consumers in the city, and the nature of the
electrical devices which they operate, as essentially a fixed number.
Suppose that $V$ and $I$ are the peak voltage and peak current 
of the AC signal transmitted along the line,
respectively. We can think of these numbers as being variable, since we can change
them using a transformer. However, since $P=I \,V$, the product of the peak 
voltage and the peak current must remain constant. Suppose that the resistance
of the line is $R$. The peak rate at which electrical energy is lost due
to ohmic heating in the line is $P_R= I^2 \,R$, which can be written
\begin{equation}
P_R = \frac{P^2\,R}{V^2}.
\end{equation}
Thus, if the power $P$ transmitted down the line is a fixed quantity, 
as is the resistance $R$ of the line, then the
power lost in the line due to ohmic heating varies like the {\em inverse square}\/
 of
the peak voltage in the line. It turns out that even at very high voltages,
such as 50\,kV, the ohmic power losses in 
transmission lines which run over tens of kilometers
can amount to up to 20\% of the transmitted power. It can readily be
appreciated that if an attempt were made to transmit AC electric power
at a peak voltage of 110\,V then the ohmic losses would be so severe that virtually none of
the power would reach its destination. Thus, it is only possible to generate
electric power at a central location, transmit it over large distances, 
and then distribute it at its point of consumption, if the transmission
is performed at a very high peak voltages (the higher, the better). Transformers
play a vital role in this process because they allow us to step-up
and step-down the voltage of an AC electric signal
 very efficiently (a well-designed
transformer typically has a  power loss which is only a few percent of the
total power  flowing through it). 

Of course, transformers do not work for DC electricity, because the
magnetic flux generated by the primary coil does not vary in time,
and, therefore, does not induce an emf in the secondary coil. 
In fact, there is no efficient method of stepping-up or
stepping-down the voltage of a DC electric signal. Thus, it is
impossible to efficiently transmit DC electric power over larger distances.
 This is the main reason why
commercially generated electricity is AC, rather than DC. 

\subsection{Impedance Matching}
The principle use of transformers is in
 the transmission and distribution of commercially generated electricity. 
However, a second, very important use of transformers is as {\em impedance matching}\/
devices. Recall, from Sect.~\ref{s7.9}, that for maximum
 power delivery from a source
to a load, the load must have the same resistance as the internal resistance
of the source. This can be accomplished by using a transformer to match the two
resistances. Suppose that the power source is connected to the primary
circuit, and the load to the secondary.
If the resistance of the load is $R$, then
$R= V_2/I_2$. However,  from the {\em transformer equation}, (\ref{e10.76}), we have
\begin{equation}
V_1 = \frac{N_1}{N_2}\,V_2,
\end{equation}
and
\begin{equation}
I_1 =\frac{N_2}{N_1}\,I_2.
\end{equation}
Now the effective resistance $R'$ of the load in the primary circuit
is given by
\begin{equation}
R' = \frac{V_1}{I_1} = \left(\frac{N_1}{N_2}\right)^2\,\frac{V_2}{I_2},
\end{equation}
which easily reduces to
\begin{equation}\label{e10.81}
R' = \left(\frac{N_1}{N_2}\right)^2\,R.
\end{equation}
Thus, by choosing the appropriate turns ratio, the effective load resistance 
$R'$ can be made equal to the internal resistance of the source, no matter what 
value the
actual load resistance $R$ takes. This process is called {\em impedance matching}.

\subsection{Worked Examples}
\subsection*{\em Example 10.1: Mutual induction}
{\em Question:} Suppose that two insulated wires are wound onto a common
cylindrical former of length $l=0.1$\,m and cross-sectional area
$A=0.05\,{\rm m}^2$. The number of turns in the first wire is $N_1=100$, and
the number of turns in the second wire is $N_2=300$. What is the mutual
inductance of the two wires? If the current $I_1$ flowing in the
first wire increases uniformly from 0 to $10$\,A in $0.1$\,s, what emf is
generated in the second wire? Does this emf act to drive a current in the
second wire which circulates in the same sense as the current in the first wire,
or the opposite sense? \\
~\\
{\em Answer:} From Eq.~(\ref{e10.10}), the mutual inductance of the two wires is
$$
M = \frac{\mu_0\,N_1\,N_2\,A}{l} = \frac{(1.26\times 10^{-6})\,(100)\,(300)\,(0.05)}{0.1}
= 0.0188\,{\rm H}.
$$
From Eq.~(\ref{e10.6}), the emf generated around the second loop by the changing current
in the first loop is
$$
{\cal E}_2 = - M\,\frac{dI_1}{d t}=
-(0.0188)\,\frac{(10-0)}{(0.1)} = -1.88 \,{\rm V}.
$$
The minus sign indicates that this emf acts so as to drive a current in the
second wire which circulates in the {\em opposite}\/ sense to the current flowing
in the first wire, in accordance with Lenz's law. If the current in the
first wire were decreased, instead of increased, then the emf in the second
wire would act to drive a current which circulates in the same sense as
the former current.


\subsection*{\em Example 10.2: Energy density of electric and magnetic fields}
{\em Question:} In a certain region of space, the magnetic field has a value
of $1.0\times 10^{-2}$\,T, and the electric field has a value
of $2.0\times 10^6\,{\rm V\,m}^{-1}$. What is the combined energy density of
the electric and magnetic fields?\\
~\\
{\em Answer:} For the electric field, the energy density is
$$
w_E = \frac{1}{2}\,\epsilon_0\,E^2 =
(0.5)\,(8.85\times 10^{-12}) \,(2.0\times 10^6)^2 = 18\,\,{\rm J\,m}^{-3}.
$$
For the magnetic field, the energy density is
$$
w_B = \frac{1}{2} \,\frac{B^2}{\mu_0} = \frac{(0.5)\,(1.0\times 10^{-2})^2}
{(4\pi\times 10^{-7})} = 40\,\,{\rm J\,m}^{-3}.
$$
The net energy density is the sum of the energy density due to the electric
field and the energy density due to the magnetic field:
$$
w = w_E + w_B = (18 + 40) = 58\,\,{\rm J\,m}^{-3}.
$$


\subsection*{\em Example 10.3: The RL circuit}
{\em Question:} A coil has a resistance of $R=5.0\,\Omega$ and
an inductance of $L= 100\,{\rm mH}$. At a particular instant in time
after a battery is connected across the coil, the current is $i=2.0\,{\rm A}$,
and is increasing at a rate of $d i/dt=20\,{\rm A\,s}^{-1}$. What is the voltage
$V$ of the battery? What is the time-constant of the circuit? What is
the final value of the current?\\
~\\
{\em Answer:} Application of Ohm's law around the circuit gives [see Eq.~(\ref{e10.29})]
$$
V = i\,R + L\,\frac{di}{d t}
= (2.0)\,(5.0) + (0.1)\,(20) = 12\,{\rm V}.
$$
The time-constant of the circuit is simply
$$
\tau = \frac{L}{R} = \frac{(0.1)}{(5.0)} = 0.020\,{\rm s}.
$$
The final steady-state current $I$ is given by Ohm's law, with the inductor acting
like a conducting wire, so
$$
I= \frac{V}{R} = \frac{(12)}{(5)} = 2.4\, {\rm A}.
$$


\subsection*{\em Example 10.4: The RC circuit}
{\em Question:} A capacitor of capacitance $C= 15\,\mu{\rm F}$ is charged up to a
voltage of $V = 800\,{\rm V}$, and then discharged by connecting a
resistor of resistance $R=8\,{\rm M}\Omega$ across its terminals. How
long does it take for the charge on the positive plate of the capacitor to
be reduced to $10\%$ of its original value?\\
~\\
{\em Answer:} Suppose that the resistor is first connected across the 
capacitor at $t=0$. The charge $q$ on the positive plate of the capacitor
is given by
$$
q(t) = Q\,{\rm e}^{-t/R\,C},
$$
which can be rearranged to give
$$
\frac{Q}{q} = {\rm e}^{\,t/R\,C}.
$$
Taking the natural logarithm of both sides, we obtain
$$
\ln\left(\frac{Q}{q}\right) = \frac{t}{R\,C}.
$$
Hence,
$$
t = \tau\,\ln\left(\frac{Q}{q}\right),
$$
where
$$
\tau = R\,C = (8)\,(15) = 120\,\,{\rm s}
$$
is the $RC$ time. Since $q/Q = 0.1$, in this case, it follows that
$$
t = (120) \,(\ln 10)= 276.3\,\,{\rm s}.
$$


\subsection*{\em Example 10.5: The step-up transformer}
{\em Question:} An electric power plant produces $P=1$\,GW of AC
electric power at a peak voltage of $V_1=500$\,V. If it is desired to
transmit this power at a peak voltage of $V_2= 50$\,kV,
what is the appropriate turns-ratio of the step-up transformer? 
What peak current $I_1$ would be sent over the transmission line if the peak
voltage were $V_1=500$\,V? What peak current $I_2$ would be sent over the transmission
 line if the peak
voltage were $V_2=50$\,kV? What is the ratio of the ohmic powers losses in
the line in the two cases?\\
~\\
{\em Answer:} The appropriate turns-ratio
is
$$
\frac{N_2}{N_1} = \frac{(5\times 10^4)}{(500)} = 100.
$$
Since the peak power is given by $P=I_1\,V_1$, it follows that
$$
I_1 = \frac{P}{V_1} = \frac{(1\times 10^9)}{(500)} = 2\,{\rm MA}.
$$
Since the peak power remains unchanged after the signal
passes through the transformer (assuming that there are no
power losses in the transformer), we have
$$
I_2 = \frac{P}{V_2} = \frac{(1\times 10^9)}{(5\times 10^4)} = 20\,{\rm kA}.
$$
The ratio of the power lost to ohmic heating in the two
cases is
$$
\frac{P_1}{P_2} = \frac{I_1^{~2}\,R}{I_2^{~2}\,R} = \left(\frac{2\times 10^6}
{2\times 10^4}\right)^2 = 10000,
$$
where $R$ is the resistance of the transmission line. Note that the ohmic
power loss is much greater at low peak voltage than at high peak voltage. 


\subsection*{\em Example 10.6: Impedance matching}
{\em Question:} An audio amplifier with an internal
resistance of $2.0\,{\rm k}\Omega$ is used to drive
a loudspeaker with a resistance of $R=5.0\,\Omega$. A transformer is
used to connect the amplifier to the loudspeaker. What is the
appropriate turns-ratio of the transformer for optimal power transfer
between the amplifier and the loudspeaker?\\
~\\
{\em Answer:} We require the transformer to convert the actual resistance $R$ of
the loudspeaker into an effective resistance $R'$ which matches 
the internal resistance $2.0\,{\rm k}\Omega$ of the amplifier. Thus,
from  Eq.~(\ref{e10.81}),
$$
\frac{N_1}{N_2} = \sqrt{\frac{R'}{R} } = \sqrt{\frac{2\times 10^3}{5}}
= 20.
$$

