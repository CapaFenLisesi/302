\section{Paraxial Optics}
 
\subsection{Spherical Mirrors}
A spherical mirror is a mirror which has the shape of a piece
cut out of a spherical surface. There are two types of spherical
 mirrors: {\em concave}, and {\em convex}. These are illustrated
 in Fig.~\ref{f13.1}. The most commonly occurring examples of concave mirrors
are  shaving mirrors and makeup mirrors.
As is well-known,  these types of mirrors
magnify objects placed close to them. The most commonly
occurring examples of convex mirrors are the passenger-side wing
mirrors of  cars. These type of mirrors have wider fields of view than
equivalent flat mirrors, but objects which appear in them
generally  look
smaller (and, therefore, farther away) than they actually are. 

\begin{figure}[h]
\epsfysize=2.5in
\centerline{\epsffile{Chapter13/fig13.01.eps}}
\caption{\em A concave (left) and a convex (right) mirror}.\label{f13.1}
\end{figure}

Let us now introduce  a few key concepts which are needed to
study  image formation by a concave spherical mirror. 
As illustrated in Fig.~\ref{f13.2}, the normal to the centre of
the mirror is called the {\em principal axis}.
The mirror is assumed to be {\em rotationally symmetric}\/ about
this axis. Hence, we can  represent a three-dimensional
mirror in a two-dimensional diagram, without loss of generality. 
The point $V$ at which the principal axis touches the surface of the
mirror is called the {\em vertex}. The point $C$, on the principal
axis, which is equidistant from all points on the reflecting
surface of the mirror is called the {\em centre of curvature}.
 The distance along the principal axis from point $C$
to point $V$ is called the {\em radius of curvature}\/ of the mirror,
and is denoted $R$. It is found experimentally that rays striking a
concave mirror parallel to its principal axis, and not too far away
from this axis, are reflected by the mirror such that they all pass
through the same point $F$ on the principal axis. This
point, which is lies between the centre of curvature and the vertex, is
called the {\em focal point}, or {\em focus}, of the mirror.
The distance along the principal axis from the focus to the
vertex is called the {\em focal length}\/ of the mirror, and is
denoted $f$. 

\begin{figure}[h]
\epsfysize=3in
\centerline{\epsffile{Chapter13/fig13.02.eps}}
\caption{\em Image formation by a concave mirror.}\label{f13.2}
\end{figure}

In our study of concave mirrors, we are going to
assume  that all light-rays which strike
a mirror parallel to its principal axis ({\em e.g.}, all rays
emanating from a distant object) are brought to a focus at the same
point $F$.
Of course, as mentioned above, this is only an approximation.
It turns out that as rays from a distant object depart further
from the principal axis of a concave mirror they are brought
to a focus ever closer to the mirror, as shown in Fig.~\ref{f13.3}. This
lack of perfect focusing of a spherical mirror is called
{\em spherical aberration}. The approximation in which
we neglect spherical aberration is called the {\em paraxial
approximation}.\footnote{``Paraxial'' is derived from
ancient Greek roots, and means ``close to the axis''.}
 Likewise, the  study of image formation under this approximation
is known as {\em paraxial optics}. This
field of optics was first investigated systematically by
the famous  German mathematician
Karl Friedrich Gauss in 1841.

\begin{figure}
\epsfysize=3in
\centerline{\epsffile{Chapter13/fig13.03.eps}}
\caption{\em Spherical aberration in a concave mirror.}\label{f13.3}
\end{figure}

It can be demonstrated, by geometry, that the
only type of mirror which does not suffer from
spherical aberration is a {\em parabolic} mirror ({\em i.e.}, a mirror
whose reflecting surface is the surface of revolution of a
parabola). Thus, a ray traveling parallel to the principal
axis of a parabolic mirror is brought to a focus at the same point $F$,
no matter how far the ray is from the axis. Since the path
of a light-ray is completely {\em reversible}, it follows
that a light source placed at the focus $F$ of a parabolic
mirror yields a perfectly parallel beam of light, after the light has reflected
off the surface of the mirror. Parabolic mirrors are more
difficult, and, therefore, more expensive, to make than
spherical mirrors. Thus, parabolic mirrors are only
used in situations where the spherical aberration of
a conventional spherical mirror would be a serious problem. 
The receiving dishes of  radio telescopes are generally
parabolic. They reflect
the incoming radio waves from (very) distant astronomical
sources,  and bring them
to a focus at a single point, where a detector is placed. In this
case, since the sources are extremely faint, it is imperative to
avoid the signal losses which would be associated with spherical
aberration. A car headlight consists of a light-bulb placed at the
focus of a parabolic reflector. The use of a parabolic reflector
enables the headlight to cast a very straight beam of light ahead of
the car. The beam would be nowhere near as well-focused were a
spherical reflector used instead. 

\subsection{Image Formation by Concave Mirrors}
There are two alternative methods of locating the image
formed by a concave mirror. The first is purely graphical, and the
second uses simple algebraic analysis. 

The graphical method of locating the image produced by a
concave mirror consists of drawing light-rays emanating from
key points on the object, and finding where these rays are brought
to a focus by the mirror. This task can be accomplished
using just {\em four}\/ simple rules:
\begin{enumerate}
\item An incident ray which is parallel to the principal axis is
reflected through the focus $F$ of the mirror.
\item An incident ray which passes through the focus $F$
of the mirror is reflected parallel to the principal axis.
\item An incident ray which passes through the centre of
curvature $C$ of the mirror is reflected back along its own
path (since it is normally incident on the mirror). 
\item An incident ray which strikes the mirror at its
vertex  $V$ is reflected such that its angle of incidence with respect to
the principal axis is equal to its angle of reflection. 
\end{enumerate}
The validity of these rules in the
paraxial approximation is fairly self-evident. 

Consider an object $ST$ which is placed a distance $p$
from a concave spherical mirror, as shown in Fig.~\ref{f13.4}. For the sake of
definiteness, let us suppose that the object distance $p$ is
greater than the focal length $f$ of the mirror. Each point
on the object is assumed to radiate light-rays in all directions.
Consider four light-rays emanating from the tip $T$ of the
object which strike the mirror, as shown in the figure. 
The reflected rays are constructed using rules 1--4 above, and the
rays are labelled accordingly. It can be seen
that the reflected rays all come together at some point $T'$. Thus,
$T'$ is the image of $T$ ({\em i.e.}, if we were to place a small
projection screen at $T'$ then we would see an image of the tip on the
screen). As is easily demonstrated, rays emanating from other parts
of the object are brought into focus in the vicinity of $T'$ such
that a complete image of the object is produced between 
$S'$ and $T'$ (obviously, point $S'$ is the image of point $S$). 
This image could be  viewed by
projecting it onto a screen placed between points
$S'$ and $T'$. Such an image is termed a {\em real image}. 
Note that the image $S'T'$ would also be directly
visible to an observer looking
straight at the mirror from a distance greater than the image
distance $q$ (since the observer's eyes could not
 tell that the light-rays
diverging from the image  were in anyway different from those
which would emanate from a real object). According to the figure, the image is {\em inverted}\/ with respect to the object, and is
also {\em magnified}. 

\begin{figure}
\epsfysize=3.5in
\centerline{\epsffile{Chapter13/fig13.04.eps}}
\caption{\em Formation of a real image by a concave mirror.}\label{f13.4}
\end{figure}

Figure~\ref{f13.5} shows what happens when the object distance $p$
is less than the focal length $f$. In this case, the image 
 appears to an observer looking straight
at the mirror to be located {\em behind}\/ the mirror.
For instance, rays emanating from the tip $T$ of the object
appear, after reflection from the
mirror, to come from a point $T'$  which is behind the
mirror.  Note that only two rays are used to locate $T'$, for
the sake of clarity. In fact, {\em two}\/ is the minimum number of rays
needed to locate a point image. 
Of course,
the image behind the mirror
cannot be viewed by projecting it onto a screen, because
there are no real light-rays behind the mirror. This
type of image is termed a {\em virtual image}. The characteristic
difference
between a real image and a virtual image is that, immediately after 
reflection from the mirror, light-rays emitted by the object {\em converge}\/
on a real image, but {\em diverge}\/ from a virtual image. 
According to Fig.~\ref{f13.5}, the image is {\em upright}\/ with
respect to the object, and is also {\em magnified}.

\begin{figure}
\epsfysize=3in
\centerline{\epsffile{Chapter13/fig13.05.eps}}
\caption{\em Formation of a virtual image by a concave mirror.}\label{f13.5}
\end{figure}

The graphical method described above is fine for developing an
intuitive understanding of image formation by concave mirrors,
 or for 
checking a calculation, but is a bit  too cumbersome for
everyday use. The analytic method described below is far more
flexible.

Consider an object $ST$ placed a distance $p$ in front of
a concave mirror of radius of curvature $R$. In order to find
the image $S'T'$ produced by the mirror, we draw two rays from
$T$ to the mirror---see Fig.~\ref{f13.6}. The first, labelled 1, travels from $T$ to the
vertex $V$ and is reflected such that its angle of
incidence $\theta$ equals its angle of reflection. The second
ray, labelled 2, passes through the centre of curvature $C$ of
the mirror, strikes the mirror at point $B$, and is reflected
back along its own path. The two rays meet at point $T'$.
Thus, $S'T'$ is the image of $ST$, since point $S'$ must lie on the
principal axis.

\begin{figure}
\epsfysize=3in
\centerline{\epsffile{ Chapter13/fig13.06.eps}}
\caption{\em Image formation by a concave mirror.}\label{f13.6}
\end{figure}

In the triangle $STV$, we have $\tan\theta=h/p$, and in the
triangle $S'T'V$ we have $\tan\theta=-h'/q$, where $p$ is
the object distance, and $q$ is the image distance. Here,
$h$ is the height of the object, and $h'$ is the height of
the image. By convention, $h'$ is a negative number, since
the image is inverted (if the image were upright then $h'$
would be a positive number). It follows that
\begin{equation}
\tan\theta = \frac{h}{p}= \frac{-h'}{q}.
\end{equation}
Thus, the {\em magnification} $M$ of the image with respect
to the object is given by
\begin{equation}\label{e13.12}
M = \frac{h'}{h} = -\frac{q}{p}.
\end{equation}
By convention, $M$ is negative if the image is inverted with
respect to the object,  and
positive if the image is upright. It is clear that the
magnification of the image is just determined by the
ratio of the image and object distances from the vertex. 

From triangles $STC$ and $S'T'C$, we have $\tan\alpha
=h/(p-R)$ and $\tan \alpha = -h'/(R-q)$, respectively.
These expressions yield
\begin{equation}\label{e13.13}
\tan\alpha = \frac{h}{p-R} = - \frac{h'}{R-q}.
\end{equation}
Equations~(\ref{e13.12}) and (\ref{e13.13}) can be combined to give
\begin{equation}
\frac{-h'}{h} = \frac{R-q}{p-R} = \frac{q}{p},
\end{equation}
which easily reduces to
\begin{equation}\label{e13.15}
\frac{1}{p}+\frac{1}{q} = \frac{2}{R}.
\end{equation}
This expression relates the object distance, the image distance,
and the radius of curvature of the mirror. 

For an object which is very far away from
the mirror ({\em i.e.}, $p\rightarrow\infty$),
so that light-rays from the object are parallel to the principal
axis, we expect the image to form at the focal point
$F$ of the mirror. Thus, in this case, $q=f$, where $f$ is
the focal length of the mirror, and Eq.~(\ref{e13.15}) reduces to
\begin{equation}
0 + \frac{1}{f} = \frac{2}{R}.
\end{equation}
The above expression yields
\begin{equation}\label{e13.17}
f = \frac{R}{2}.
\end{equation}
In other words,  in the paraxial approximation, the focal length
of a concave spherical mirror is {\em half}\/ of its radius of
curvature.
 Equations~(\ref{e13.15}) and (\ref{e13.17}) can be combined to give
\begin{equation}\label{e13.18}
\frac{1}{p} + \frac{1}{q} = \frac{1}{f}.
\end{equation}

The above expression was derived for the case of a real
image. However, as is easily demonstrated, it also applies
to virtual images provided that the following sign convention
is adopted. For real images, which always form {\em in front}\/
of the mirror, the image distance $q$ is {\em positive}. For 
virtual images, which always form {\em behind}\/ the mirror,
the image distance $q$ is {\em negative}. It immediately follows,
from Eq.~(\ref{e13.12}), that real images are always inverted, and 
virtual images are always upright. Table~\ref{t13.1}
shows how the location  and character of the image formed
in a concave spherical mirror depend on the location of
the object, according to Eqs.~(\ref{e13.12}) and (\ref{e13.18}). It is
clear that the {\em modus operandi} of a shaving mirror,
or a makeup mirror, is to place the object ({\em i.e.}, a
face) between the mirror and the focus of the mirror. The image
is upright, (apparently) located behind the mirror, and magnified.
\begin{table}\centering
\begin{tabular}{lll}\hline
{\em Position of object} & {\em Position of image} &
{\em Character of image}\\ \hline
At $\infty$ & At $F$ & Real, zero size\\
Between $\infty$ and $C$ & Between $F$ and $C$ &
Real, inverted, diminished\\
At $C$ & At $C$ & Real, inverted, same size \\
Between $C$ and $F$ & Between $C$ and $\infty$ &
Real, inverted, magnified\\
At $F$ & At $\infty$ & \\
Between $F$ and $V$ & From $-\infty$ to $V$ & Virtual,
upright, magnified\\
At $V$ & At $V$ & Virtual, upright, same size\\
\end{tabular}
\caption{\em Rules for image formation by concave mirrors.}\label{t13.1}
\end{table}

\subsection{Image Formation by Convex Mirrors}
The definitions of the principal axis, centre of curvature $C$,
radius of curvature $R$, and the vertex $V$, of a convex mirror
are analogous  to the corresponding definitions
 for a concave mirror.
When parallel light-rays strike a convex mirror they are reflected
such that they appear to emanate from a single point $F$
located behind the mirror, as shown in Fig.~\ref{f13.7}. This point is called the {\em virtual focus}
of the mirror. The focal length
$f$ of the mirror is simply the distance between $V$ and $F$. 
As is easily demonstrated, in the paraxial approximation,  the focal length of a convex mirror
is half of its radius of curvature.

\begin{figure}
\epsfysize=3.5in
\centerline{\epsffile{Chapter13/fig13.07.eps}}
\caption{\em  The virtual focus of a convex mirror.}\label{f13.7}
\end{figure}

There are, again, two alternative methods of locating the image
formed by a convex mirror. The first is graphical, and the second
analytical.

According to the graphical method, 
the image produced by a convex mirror can always be located 
by drawing a ray diagram according to {\em four}\/ simple rules:
\begin{enumerate}
\item An incident ray which is parallel to the principal axis
is reflected as if it came from the virtual focus $F$ of the
mirror.
\item An incident ray which is directed towards the virtual
focus $F$ of the mirror is reflected parallel to the
principal axis.
\item An incident ray which is directed towards the centre of
curvature $C$ of the mirror is reflected back along its own
path (since it is normally incident on the mirror).
\item An incident ray which strikes the mirror at its vertex $V$
is reflected such that its angle of incidence with respect to the
principal axis is equal to its angle of reflection.
\end{enumerate}
The validity of these rules in the paraxial approximation is,
again, fairly self-evident.

In the example shown in Fig.~\ref{f13.8}, two rays are used to locate the image
$S'T'$ of an object $ST$ placed in front of the mirror. 
It can be seen that the image is virtual, upright, and diminished.

\begin{figure}
\epsfysize=3in
\centerline{\epsffile{Chapter13/fig13.08.eps}}
\caption{\em Image formation by a convex mirror.}\label{f13.8}
\end{figure}

As is easily demonstrated, 
application of the analytical method to image formation by convex mirrors again
yields Eq.~(\ref{e13.12}) for
the magnification of the image, and Eq.~(\ref{e13.18}) for the location of the
image, provided that we adopt the following sign convention.
The focal length $f$ of a convex mirror is redefined to be {\em
minus} the distance between points $V$ and $F$. In other words,
the focal length of a concave mirror, with a real focus, is always positive,
and the focal length of a convex mirror, with a virtual focus,
is always negative. Table~\ref{t13.2}
shows how the location  and character of the image formed
in a convex spherical mirror depend on the location of
the object, according to Eqs.~(\ref{e13.12}) and (\ref{e13.18}) (with $f<0$).
\begin{table}\centering
\begin{tabular}{lll}\hline
{\em Position of object} & {\em Position of image} &
{\em Character of image}\\ \hline
At $\infty$  & At $F$& Virtual, zero size\\
Between $\infty$ and $V$ & Between $F$ and $V$ &
Virtual, upright, diminished\\
At $V$ & At $V$ & Virtual, upright, same size\\
\end{tabular}
\caption{\em Rules for image formation by convex mirrors.}\label{t13.2}
\end{table}

In summary, the formation of an image by a spherical mirror involves the
{\em crossing} of light-rays emitted by the object and reflected off
the mirror. If the light-rays actually cross in front of the
mirror then the image is real. If the light-rays do not
actually cross, but appear to cross when projected backwards
behind the mirror, then the image is virtual. A real image
can be projected onto a screen, a virtual image cannot. 
However, both
types of images can be seen by an observer, and photographed by a
camera. The magnification of the image
is specified by Eq.~(\ref{e13.12}), and the location of the image is determined
 by
Eq.~(\ref{e13.18}). These two formulae can be used to characterize both real
and virtual images formed by either concave or convex mirrors, provided
that the following sign conventions are observed:
\begin{enumerate}
\item The height $h'$ of the image is positive if the image is
upright, with respect to the object, and negative if the image
is inverted. 
\item The magnification $M$ of the image is positive if the image is
upright, with respect to the object, and negative if the image
is inverted. 
\item The image distance $q$ is positive if the image is real, and,
therefore, located in front of the mirror, and  negative if the
image is virtual, and, therefore, located behind the mirror. 
\item The focal length $f$  of
the mirror is positive if the mirror is concave, so that the focal
point $F$  is located in front of
the mirror, and negative if the mirror is convex, so that the focal
point $F$  is located behind
the mirror.
\end{enumerate}
Note that  the front side of the mirror is defined
to be  the side from which the light is incident. 

\subsection{Image Formation by Plane Mirrors}
Both concave and convex spherical mirrors asymptote to plane mirrors
in the limit in which their radii of curvature $R$ tend to infinity.
In other words, a plane mirror can be treated as either 
 a concave or a convex mirror
for which $R\rightarrow \infty$. Now, if $R\rightarrow\infty$, then
$f=\pm R/2\rightarrow\infty$, so $1/f\rightarrow 0$, and Eq.~(\ref{e13.18})
yields
\begin{equation}
\frac{1}{p} + \frac{1}{q} = \frac{1}{f} = 0,
\end{equation}
or 
\begin{equation}
q = -p.
\end{equation}
Thus, for a plane mirror the image is {\em virtual}, and is
located as far behind the mirror as the object is in front of
the mirror. According to Eq.~(\ref{e13.12}), the magnification of the
image is given by
\begin{equation}
M = -\frac{q}{p} = 1.
\end{equation}
Clearly, the image is {\em upright}, since $M>0$, and is
the {\em same size}
as the object, since $|M|=1$. However, an  image seen
in a plane mirror does differ from the original object in one 
important respect: {\em i.e.}, 
left and right are {\em swapped over}. In other words, 
a right-hand looks like a left-hand in a plane mirror, and {\em vice
versa}. 

\subsection{Thin Lenses}
A lens is a transparent medium (usually glass) bounded by two
curved surfaces (generally either spherical, cylindrical, or
plane surfaces). As illustrated in Fig.~\ref{f13.9}, the
line which passes normally  through
both bounding surfaces of a lens   is called the {\em optic axis}.
The point $O$ on the optic axis which lies midway between the two bounding
surfaces  is called the {\em optic centre}.

\begin{figure}
\epsfysize=1.5in
\centerline{\epsffile{Chapter13/fig13.09.eps}}
\caption{\em The optic axis of a lens.}\label{f13.9}
\end{figure}

 There are two basic
kinds of lenses: {\em converging}, and {\em diverging}. A converging lens 
brings all incident light-rays parallel to its optic axis together at
a point $F$, behind the lens,
called the {\em focal point}, or {\em focus}, of the lens.
A diverging lens spreads out all incident light-rays parallel
 to its optic axis so that they appear to diverge from a {\em  virtual
focal point} $F$ in front of the lens. Here, the front side of the lens
is conventionally defined to be the side from which the light is
incident. The differing effects of a converging and a diverging lens
on incident light-rays parallel to the optic axis ({\em i.e.}, 
emanating from a
distant object) are illustrated in Fig.~\ref{f13.10}. 

\begin{figure}
\epsfysize=4in
\centerline{\epsffile{Chapter13/fig13.10.eps}}
\caption{\em The focii of converging (top) and diverging (bottom) lens.}\label{f13.10}
\end{figure}

Lenses, like mirrors,
 suffer from {\em spherical aberration}, which causes
light-rays parallel to the optic axis, but a relatively long way from
the axis, to be brought to a focus, or a virtual focus, {\em closer}\/ to the
lens than light-rays which are relatively close to the axis. It turns out that
spherical aberration in lenses can be completely cured by using lenses
whose bounding surfaces are {\em non-spherical}. However, such lenses
are more difficult, and, therefore,  more expensive, to manufacture than
conventional  lenses
whose bounding surfaces are spherical. Thus, the former sort of lens is only
employed in situations where the spherical aberration of a  conventional
lens would be a serious problem. The usual method of curing
spherical aberration is to use {\em combinations}\/ of conventional
lenses ({\em i.e.}, compound lenses). 
In the
following, we shall make use  of the {\em  paraxial approximation}, in which
spherical aberration is completely ignored, and all light-rays parallel
to the optic axis are assumed to be brought to a focus, or a virtual
focus, at the same point $F$. This approximation is valid as long
as the radius of the lens is small compared to the object distance
and the image distance.

The {\em focal length}\/ of a lens, which is usually denoted $f$, is
defined as the distance between the optic centre $O$ and the focal
point $F$, as shown in Fig.~\ref{f13.10}. However, by convention, {\em converging}\/ lenses have 
{\em positive}\/
focal lengths, and {\em diverging}\/ lenses have {\em negative}\/ focal lengths. In other
words, if the focal point lies behind the lens then the focal length is
positive, and if the focal point lies in front of the lens then the focal
length is negative. 

Consider a conventional lens whose bounding surfaces are {\em spherical}.
Let $C_f$ be the centre of curvature of the front
surface, and $C_b$ the centre of curvature of the back surface.
The radius of curvature  $R_f$ of the front surface is the
distance between the optic centre $O$ and the point $C_f$. Likewise,
the radius of curvature $R_b$ of the back surface is the distance
between points $O$ and $C_b$. However, by convention, the
radius of curvature of a bounding surface is {\em positive}\/ if its centre of
curvature lies {\em  behind} the lens, and {\em negative}\/ if its centre of
curvature lies {\em in front}
 of the lens. Thus, in Fig.~\ref{f13.11},
$R_f$ is positive and $R_b$ is negative.

\begin{figure}
\epsfysize=2.5in
\centerline{\epsffile{Chapter13/fig13.11.eps}}
\caption{\em A thin lens.}\label{f13.11}
\end{figure}

In the paraxial approximation, it is possible to find a simple
formula relating 
 the focal length $f$  of a lens to the
radii of curvature, $R_f$ and $R_b$, of its front and back bounding surfaces.
This formula is written
\begin{equation}\label{e13.22}
\frac{1}{f} = (n-1)\left(\frac{1}{R_f} - \frac{1}{R_b}\right),
\end{equation}
where $n$ is the refractive index of the lens. The above formula
is usually  called the {\em lens-maker's formula}, and was discovered by
Descartes. Note that the lens-maker's formula is only valid for a
{\em thin lens}\/ whose thickness is small compared to its focal length.
What Eq.~(\ref{e13.22}) is basically
telling us is that light-rays
which pass from air to glass through a {\em convex}\/ surface are {\em focused}, whereas light-rays which pass from air to glass through a {\em concave}
surface are {\em defocused}. Furthermore, since light-rays are
{\em reversible}, it follows that rays which pass from  glass to air
through a {\em convex} surface are {\em defocused}, whereas rays
 which pass from air to glass through a {\em concave}\/
surface are {\em focused}. Note that the net focusing or defocusing action
of a lens is due to the {\em difference}\/ in the radii of curvature of
its two bounding surfaces. 

Suppose that a certain lens has a focal length $f$. What happens
to the focal length if we turn the lens around, so that its front
bounding surface becomes its back bounding surface, and {\em vice
versa}? It is easily seen that when the lens is turned around
$R_f\rightarrow -R_b$ and $R_b \rightarrow -R_f$. However, the focal
length $f$ of the lens is invariant under this transformation, according
to Eq.~(\ref{e13.22}). Thus, the focal length of a lens is the same for
light incident from either side. In particular, a converging
lens remains a converging lens when it is turned around, and likewise
for a diverging lens. 

The most commonly occurring type of converging lens is a {\em bi-convex},
or {\em double-convex}, lens, for which $R_f>0$ and $R_b<0$. In this
type of lens, both bounding surfaces have a focusing effect on light-rays
passing through the lens. Another fairly common type of
converging lens is a {\em plano-convex}\/ lens, for which
$R_f>0$ and $R_b=\infty$. In this type of
lens, only the curved bounding surface has a focusing effect on light-rays. The plane surface has no focusing or defocusing effect. 
A less common type of converging lens is a {\em convex-meniscus}\/
lens, for which $R_f>0$ and $R_b>0$, with $R_f<R_b$. In this type
of lens, the front bounding surface has a focusing effect on light-rays,
whereas the back bounding surface has a defocusing effect, but the
focusing effect of the front surface wins out. 

The most commonly occurring type of diverging lens is a {\em bi-concave},
or {\em double-concave}, lens, for which $R_f<0$ and $R_b>0$. In this
type of lens, both bounding surfaces have a defocusing effect on light-rays
passing through the lens. Another fairly common type of
converging lens is a {\em plano-concave}\/ lens, for which
$R_f<0$ and $R_b=\infty$. In this type of
lens, only the curved bounding surface has a defocusing effect on light-rays. The plane surface has no focusing or defocusing effect. 
A less common type of converging lens is a {\em concave-meniscus}\/
lens, for which $R_f<0$ and $R_b<0$, with $R_f<|R_b|$. In this type
of lens, the front bounding surface has a defocusing effect on light-rays,
whereas the back bounding surface has a focusing effect, but the
defocusing effect of the front surface wins out. 

Figure~\ref{f13.12} shows the various types of lenses mentioned above. Note
that, as a general rule, converging lenses are thicker at the centre
than at the edges, whereas diverging lenses are thicker at the
edges than at the centre.

\begin{figure}[h]
\epsfysize=3.5in
\centerline{\epsffile{Chapter13/fig13.12.eps}}
\caption{\em Various different types of thin lens.}\label{f13.12}
\end{figure}

\subsection{Image Formation by Thin Lenses}
There are two alternative methods of locating the image formed by
a thin lens. 
Just as for
spherical mirrors, the first method is {\em graphical}, and the second  {\em analytical}. 

The graphical method of locating the image formed by a
thin lens involves drawing light-rays emanating from key points
on the object, and finding where these rays are brought to
a focus by the lens. This task can be accomplished using a
small number of simple rules. 

Consider a converging lens. It is helpful to define {\em two}\/
focal points for such a lens. The first, the so-called {\em image
focus}, denoted $F_i$, is defined as the point behind the
lens to which all incident light-rays parallel to the optic
axis converge after passing through the lens.
 This is the same as the focal point $F$ defined
previously. The second, the so-called {\em object
focus}, denoted $F_o$,  is defined as the position in front of
the lens for  which rays emitted from a point source of light placed at
that position 
would be refracted parallel to the optic axis after passing through
the lens. It is easily demonstrated that the object focus $F_o$
is as far in front of the optic centre $O$ of the lens as the image focus
$F_i$ is behind $O$. The distance from the optic centre to either
focus is, of course, equal to the focal length
$f$ of the lens. The image produced by a converging lens can be
located using just {\em three}\/ simple rules:
\begin{enumerate}
\item An incident ray which is parallel to the optic axis is
refracted through the image focus $F_i$ of the lens.
\item An incident ray which passes through the object focus
$F_o$ of the lens is refracted parallel to the optic axis.
\item An incident ray which passes through the optic
centre $O$ of the lens is not refracted at all.
\end{enumerate}
The last rule is only an approximation. It turns out that although
a light-ray which passes through the optic centre of the
lens does not change direction, it is displaced slightly to one
side. However, this displacement is negligible for a thin lens. 

Figure~\ref{f13.13} illustrates how the image $S'T'$ of an object $ST$
placed in front of a converging lens
is located using the above rules. 
In fact, the three rays, 1--3, emanating from
the tip $T$ of the object, are constructed using rules 1--3, respectively. 
Note that the image is real (since light-rays actually cross), inverted, and
diminished. 

\begin{figure}[h]
\epsfysize=2.5in
\centerline{\epsffile{Chapter13/fig13.13.eps}}
\caption{\em Image formation by a converging lens.}\label{f13.13}
\end{figure}

Consider a diverging lens. It is again helpful to
define two focal points for such a lens. The image focus $F_i$ is
defined as the point in front of the lens from which all
incident light-rays parallel to the optic axis appear to diverge after 
passing through the lens. This is the same as the focal point $F$
defined earlier. 
The object focus $F_o$ is
defined as the point behind the lens to which all
incident light-rays which are refracted parallel to the optic
axis after passing through the lens appear to converge. Both foci
are located a distance $f$ from the optic centre, where $f$ is the
focal length of the lens. The image produced by a diverging lens
can be located using the following three rules:
\begin{enumerate}
\item An incident ray which is parallel to the optic
axis is refracted as if it came from the image focus $F_i$ of the lens.
\item An incident ray which is directed towards the object focus 
$F_o$ of the lens is refracted parallel to the optic axis.
\item An incident ray which passes through the optic centre $O$ of
the lens is not refracted at all. 
\end{enumerate}

Figure~\ref{f13.14} illustrates how the image  $S'T'$ of an object $ST$
placed in front of a diverging  lens
is located using the above rules. 
In fact, the three rays, 1--3, emanating from
the tip $T$ of the object, are constructed using rules 1--3, respectively. 
Note that the image is virtual (since light-rays do not
actually cross), upright, and
diminished. 

\begin{figure}
\epsfysize=3in
\centerline{\epsffile{Chapter13/fig13.14.eps}}
\caption{\em Image formation by a diverging lens.}\label{f13.14}
\end{figure}

Let us now investigate the analytical method.
Consider an object of height $h$ placed a distance $p$ in
front of a converging lens.
Suppose that  a real image of height $h'$ is formed a distance
$q$ behind the lens. As is illustrated in Fig.~\ref{f13.15},
the image can be located using rules 1 and 3,
 discussed
above. 

\begin{figure}
\epsfysize=3in
\centerline{\epsffile{Chapter13/fig13.15.eps}}
\caption{\em Image formation by a converging lens.}\label{f13.15}
\end{figure}

Now, the right-angled triangles $SOT$ and $S'OT'$ are similar, so
\begin{equation}
\frac{-h'}{h} = \frac{OS'}{OS} = \frac{q}{p}.
\end{equation}
Here, we have adopted the convention that the image height $h'$ is
{\em negative}\/ if the image is {\em inverted}. The magnification of a thin 
converging lens is given by
\begin{equation}\label{e13.24}
M = \frac{h'}{h} = -\frac{q}{p}.
\end{equation}
This is the same as the expression (\ref{e13.12}) for the magnification 
of a spherical mirror. Note that we are again adopting the convention
that the magnification is {\em negative} if the image is {\em inverted}. 

The right-angled triangles $OPF$ and $S'T'F$ are also similar, and
so
\begin{equation}
\frac{S'T'}{OP} = \frac{FS'}{OF},
\end{equation}
or
\begin{equation}
\frac{-h'}{h} = \frac{q}{p} = \frac{q-f}{f}.
\end{equation}
The above expression can be rearranged to give
\begin{equation}\label{e13.27}
\frac{1}{p} + \frac{1}{q} = \frac{1}{f}.
\end{equation}
Note that this is exactly the same as the formula (\ref{e13.18}) relating  the
image and object distances in a spherical mirror. 

Although formulae (\ref{e13.24}) and (\ref{e13.27}) were derived for the case of a real
image formed by a converging lens, they also apply to virtual images, and
to images formed by diverging lenses, provided that the following sign conventions are adopted. First of all, as we have already mentioned, the focal
length $f$ of a {\em converging}\/ lens is {\em positive}, and the focal length of
a {\em diverging}\/ lens is {\em negative}. Secondly, the image distance $q$ is
{\em positive}\/ if the image is {\em real}, and, therefore, located
{\em behind} the lens, and {\em negative}\/ if the image is {\em virtual},
and, therefore, located {\em in front}\/ of the lens. It immediately follows,
from Eq.~(\ref{e13.24}), that {\em real}\/ images are always {\em inverted}, and
{\em virtual}\/ images are always {\em upright}. 

Table~\ref{t13.3} shows how the location and character of
the image formed by a converging lens depend on the location of the
object. Here, the point $V_o$ is located on the optic axis two focal lengths
in front of the optic centre, and the point $V_i$ is located on the optic
axis two focal lengths behind the optic centre. Note the almost exact analogy between the image forming properties of
a converging lens and those of a concave spherical mirror.
\begin{table}\centering
\begin{tabular}{lll}\hline
{\em Position of object} & {\em Position of image} &
{\em Character of image}\\ \hline
At $+\infty$ & At $F$ & Real, zero size\\
Between $+\infty$ and $V_o$ & Between $F$ and $V_i$ &
Real, inverted, diminished\\
At $V_o$ & At $V_i$ & Real, inverted, same size \\
Between $V_o$ and $F$ & Between $V_i$ and $-\infty$ &
Real, inverted, magnified\\
At $F$ & At $-\infty$ & \\
Between $F$ and $O$ & From $+\infty$ to $O$ & Virtual,
upright, magnified\\
At $O$ & At $O$ & Virtual, upright, same size\\
\end{tabular}
\caption{\em Rules for image formation by converging lenses.}\label{t13.3}
\end{table}

Table~\ref{t13.4} shows how the location and character of
the image formed by a diverging  lens depend on the location of the
object. Note the almost exact analogy between the image forming properties of
a diverging lens and those of a convex spherical mirror.
\begin{table}\centering
\begin{tabular}{lll}\hline
{\em Position of object} & {\em Position of image} &
{\em Character of image}\\ \hline
At $\infty$  & At $F_i$& Virtual, zero size\\
Between $\infty$ and $O$ & Between $F_i$ and $O$ &
Virtual, upright, diminished\\
At $O$ & At $O$ & Virtual, upright, same size\\
\end{tabular}
\caption{\em Rules for image formation by diverging lenses.}\label{t13.4}
\end{table}

Finally, let us reiterate the sign conventions used to determine
 the positions and characters of the images formed by thin lenses:
\begin{enumerate}
\item The height $h'$ of the image is positive if the image is
upright, with respect to the object, and negative if the image
is inverted. 
\item The magnification $M$ of the image is positive if the image is
upright, with respect to the object, and negative if the image
is inverted. 
\item The image distance $q$ is positive if the image is real, and,
therefore, located behind the lens, and  negative if the
image is virtual, and, therefore, located in front of the lens. 
\item The focal length $f$  of
the lens  is positive if the lens is converging, so that the image focus
 $F_i$ is located behind the
lens, and negative if the lens is diverging, so that the
image focus
 $F_i$  is located in front of the lens.
\end{enumerate}
Note that  the front side of the lens is defined
to be  the side from which the light is incident. 

\subsection{Chromatic aberration}
We have seen that both mirrors and lenses suffer from spherical
aberration, an effect which limits the clarity and sharpness of the
images formed by such devices. However, lenses also suffer from another
type of abberation called {\em chromatic abberation}. This occurs
because the index of refraction of the glass in a lens is different for
different wavelengths. We have seen that a prism refracts violet light
more than red light. The same is true of lenses. As a result, a simple
lens focuses violet light closer to the lens than it focuses
red light. Hence, white light  produces a slightly blurred image of
an object, with coloured edges. 

For many years, chromatic abberation was a sufficiently serious 
problem for lenses that scientists tried to find ways of reducing
the number of lenses in scientific instruments,
or even eliminating them all together. For instance, Isaac Newton
developed a type of telescope, now called the
Newtonian telescope,  which uses a mirror instead of a lens to collect
light. However, in 1758, John Dollond, an English optician, discovered
a way to eliminate chromatic abberation. He combined two lenses,
one converging, the other diverging, to make an {\em achromatic
doublet}. The two lenses in an achromatic
doublet are made of different type of glass with indices of refraction
chosen such that the combination brings any two chosen colours to the
same sharp focus. 

Modern scientific instruments use {\em compound lenses} ({\em i.e.},
combinations of simple lenses) to simultaneously
 eliminate both chromatic and
spherical aberration. 

\subsection{Worked Examples}
\subsection*{\em Example 13.1: Concave mirrors}
{\em Question:} An object of height $h=4$\,cm is placed
a distance $p=15$\,cm in front of a concave mirror of
focal length $f=20$\,cm. What is the height, location, and
nature of the image? Suppose that the object is
moved to a new position a distance $p=25$\,{\rm cm} in front
of the mirror. What now is the height, location, and
nature of the image? \\
~\\
{\em Answer:} According to Eq.~(\ref{e13.18}), the image distance $q$ is
given by
$$
q = \frac{1}{1/f - 1/p} = \frac{1}{(1/20-1/15)} = - 60\,{\rm cm}.
$$
Thus, the image is {\em virtual} (since $q<0$), and is located $60$\,cm {\em behind}
the mirror. According to Eq.~(\ref{e13.12}), the magnification $M$ of
the image is given by
$$
M = -\frac{q}{p} = -\frac{(-60)}{(15)} = 4.
$$
Thus, the image is {\em upright}\/ (since $M>0$), and
{\em magnified}\/ by a factor of $4$. It follows that the height 
$h'$ of
the image is given by 
$$ 
h' = M\,h = (4)\,(4) = 16\,{\rm cm}.
$$

If the object is moved such that $p=\,25$\,{\rm cm} then the
new image distance is given by
$$
q = \frac{1}{1/f - 1/p} = \frac{1}{(1/20-1/25)} = 100\,{\rm cm}.
$$
Thus, the new image is {\em real}\/ (since $q>0$), and is
located 100\,{\rm cm} {\em in front of} the mirror. The new magnification is
given by
$$
M = -\frac{q}{p} = -\frac{(100)}{(15)} = -6.67.
$$
Thus, the image is {\em inverted}\/ (since $M<0$), and
{\em magnified}\/ by a factor of $6.67$. It follows that the
new height of the image
is
$$ 
h' = M\,h = -(6.67)\,(4) = -26.67\,{\rm cm}.
$$
Note that the height is negative because the image is inverted.

\subsection*{\em Example 13.2: Convex mirrors}
{\em Question:} How far must an object be
placed in front of a convex mirror
of radius of curvature $R=50$\,cm in order to ensure that the
size of the image is ten times less than the
size of the object? How far behind the
mirror is the image located?\\
~\\
{\em Answer:} The focal length $f$ of a convex mirror is {\em minus}
half of its radius of curvature (taking the sign convention for
the focal lengths of convex mirrors into account). Thus, $f=-25$\,cm.
If the image is ten times smaller than the object then the magnification
is 
$M=0.1$. We can be sure that $M=+0.1$, as opposed to $-0.1$, because
we know that images formed in convex mirrors are always virtual and
upright. 
According to Eq.~(\ref{e13.12}), the image  distance $q$
is given by
$$
q = - M\,p,
$$
where $p$ is the object distance. This can be combined with Eq.~(\ref{e13.18})
to give
$$
p = f\left(1-\frac{1}{M}\right) = -(25)\,(1-10) = 225\,{\rm cm}.
$$
Thus, the object must be placed $225$\,cm in front of the mirror. 
The image distance is given by
$$
q = -M\,p = -(0.1)\,(225) = -22.5\,{\rm cm}.
$$
Thus, the image is located $22.5$\,cm behind the mirror.


\subsection*{\em Example 13.3: Converging lenses}
{\em Question:} An object of height $h=7$\,cm is placed
a distance $p=25$\,cm in front of a thin converging lens of focal length
$f=35$\,cm. What is the height, location, and nature
of the image? Suppose that the object is moved to a
new location a distance $p=90$\,cm in front of the
lens. What now is the height, location, and nature of the image?\\
~\\
{\em Answer:} According to Eq.~(\ref{e13.27}), the image distance $q$ 
is given by 
$$
q = \frac{1}{1/f - 1/p} = \frac{1}{(1/35-1/25)} = - 87.5\,{\rm cm}.
$$
Thus, the image is {\em virtual}\/ (since $q<0$), and is located
$87.5$\,cm {\em in front}\/ of the lens. According to Eq.~(10.24),
the magnification $M$ of the image is given by
$$
M = -\frac{q}{p} = -\frac{(-87.5)}{(25)} = 3.5.
$$
Thus, the image is {\em upright}\/ (since $M>0$), and {\em magnified}\/
by a factor of $3.5$. It follows that the height $h'$ of the image
is given by
$$
h' = M\,h =(3.5)\,(7) =24.5\,{\rm cm}.
$$

If the object is moved such that $p=\,90$\,{\rm cm} then the
new image distance is given by
$$
q = \frac{1}{1/f - 1/p} = \frac{1}{(1/35-1/90)} = 57.27\,{\rm cm}.
$$
Thus, the new image is {\em real}\/ (since $q>0$), and is
located 57.27\,{\rm cm} {\em behind}\/ the lens. The new magnification is
given by
$$
M = -\frac{q}{p} = -\frac{(57.27)}{(90)} = -0.636.
$$
Thus, the image is {\em inverted}\/ (since $M<0$), and
{\em diminished}\/ by a factor of $0.636$. It follows that the
new height of the image
is
$$ 
h' = M\,h = -(9.636)\,(7) = -4.45\,{\rm cm}.
$$
Note that the height is negative because the image is inverted.

\subsection*{\em Example 13.4: Diverging lenses}
{\em Question:} How far must an object be
placed in front of a diverging lens
of focal length $45$\,cm in order to ensure that the
size of the image is fifteen times less than the
size of the object? How far in front of the lens is
 the image located?\\
~\\
{\em Answer:} The focal length $f$ of a diverging
lens  is {\em negative} by convention,
so  $f=-45$\,cm, in this case. 
If the image is fifteen times smaller than the object then the magnification
is 
$M=0.0667$. We can be sure that $M=+0.0667$, as opposed to $-0.0667$, because
we know that images formed in diverging lenses are always virtual and
upright. 
According to Eq.~(\ref{e13.24}), the image  distance $q$
is given by
$$
q = - M\,p,
$$
where $p$ is the object distance. This can be combined with Eq.~(\ref{e13.27})
to give
$$
p = f\left(1-\frac{1}{M}\right) = -(45)\,(1-15) = 630\,{\rm cm}.
$$
Thus, the object must be placed $630$\,cm in front of the lens. 
The image distance is given by
$$
q = -M\,p = -(0.0667)\,(630) = -42\,{\rm cm}.
$$
Thus, the image is located $42$\,cm {\em in front}\/ of the lens.
