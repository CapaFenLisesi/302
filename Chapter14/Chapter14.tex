\section{Wave Optics}\label{s13}
\subsection{Introduction}\label{s14.1}
Geometric optics is an incredibly successful theory. Probably its
most important application  is in describing and explaining
 the operation of 
 commonly occurring  optical instruments: {\em 
e.g.}, the camera, the telescope, and the microscope. 
Although geometric optics does not make any explicit  assumption about
the nature of light, it tends to suggest that light consists of a
stream of massless particles. This is certainly what scientists, including,
most notably, Isaac Newton,  
generally assumed up until about the year 1800.

Let us examine how the particle theory of light accounts for the three basic
laws of geometric optics:
\begin{enumerate}
\item{\sf The law of geometric propagation:} This is easy. Massless
particles obviously move in straight-lines in free space.
\item{\sf The law of reflection:} This is also fairly easy. We merely
have to assume that light particles bounce {\em elastically}\/
({\em i.e.}, without energy loss) off reflecting surfaces. 
\item {\sf The law of refraction:} This is the tricky one. Let us
assume that the speed of light particles propagating through a transparent
dielectric medium is proportional to the 
index of refraction, $n\equiv \sqrt{K}$. 
Let us further assume that at a general 
 interface between two different dielectric
media,  light particles crossing the interface conserve
momentum in the plane parallel to the interface. In general, this implies
that the particle momenta normal to the interface are not conserved:
{\em i.e.}, the interface exerts a normal reaction force on crossing
particles, but no parallel force. From Fig.~\ref{f14.1}, 
parallel momentum conservation for  light particles crossing the interface
yields
\begin{equation}
v_1\,\sin\theta_1 = v_2\,\sin\theta_2.
\end{equation}
However, by assumption, $v_1=n_1\,c$ and $v_2=n_2\,c$, so
\begin{equation}
n_1\,\sin\theta_1 = n_2\,\sin\theta_2.
\end{equation}
This highly contrived (and incorrect) derivation of the law of refraction
was first proposed by Descartes in 1637. Note that it depends
crucially on the (incorrect) assumption that light travels
{\em faster}\/ in dense media ({\em e.g.}, glass)
than in rarefied media ({\em e.g.}, water). This assumption
 appears very strange to us nowadays, but it seemed eminently reasonable
to scientists in the 17th and 18th centuries. After all, they knew that
sound travels faster in dense media ({\em e.g.}, water) than in rarefied
media ({\em e.g.}, air). 
\end{enumerate}

\begin{figure}
\epsfysize=2.5in
\centerline{\epsffile{Chapter14/fig14.1.eps}}
\caption{\em Descartes' model of refraction}\label{f14.1}
\end{figure}

The wave theory of light, which became established in the first
half of the 19th century, initially encountered tremendous resistance.
Let us briefly
examine the reasons why scientists in the early 1800s refused 
 to think of light as a wave phenomenon? Firstly, the particle theory of
light was intimately associated with Isaac Newton, so any attack on this
theory was considered to be a slight to his memory. Secondly, all of the
waves that scientists were familiar with at that time manifestly
did not travel in straight-lines. For instance, water waves are 
diffracted as they pass through the narrow mouth of a harbour, as shown
in Fig.~\ref{f14.2}. In other words, the ``rays'' associated with such waves
are bent as they traverse the harbour mouth.
Scientists thought that if light were a wave phenomenon then it would also not
travel in straight-lines: {\em i.e.}, it would not cast straight, sharp
shadows, any more than water waves cast straight, sharp ``shadows.''
Unfortunately, they did not appreciate that if the wavelength of light
is much shorter than that of water waves then light can  be a wave
phenomenon and still propagate in a largely geometric manner.

\begin{figure}
\epsfysize=3in
\centerline{\epsffile{Chapter14/fig14.2.eps}}
\caption{\em Refraction of water waves through the entrance of a harbour.}\label{f14.2}
\end{figure}

\subsection{Huygens' principle}
The first person to explain how wave theory can also account for the
laws of geometric optics was Christiaan Huygens in 1670. At the time,
of course, nobody took the slightest notice of him. His work was
later rediscovered after the eventual triumph of wave theory. 

Huygens had a very important insight into the nature of wave propagation
which is nowadays called {\em Huygens' principle}. When
applied to the propagation of light waves, this principle states
that:
\begin{quote}{\sf Every point on a wave-front may be considered a source
of secondary spherical wavelets which spread out in the forward direction
at the speed of light. The new wave-front is the tangential surface to
all of these secondary wavelets.
}\end{quote}

According to Huygens' principle, a plane light wave propagates though
free space at the speed of light, $c$. The light rays associated with
this wave-front propagate in straight-lines, as shown in Fig.~\ref{f14.3}. It is also fairly straightforward to account for the
laws of reflection and refraction using Huygens' principle.

\begin{figure}
\epsfysize=3in
\centerline{\epsffile{Chapter14/fig14.3.eps}}
\caption{\em Huygen's principle.}\label{f14.3}
\end{figure}

\subsection{Young's Double-Slit Experiment}
The first serious challenge to the particle theory of light was made 
by the English scientist Thomas Young in 1803. Young possessed one of
the most brilliant minds in the history of science. A physician
by training, he was the first to describe how the lens of the human
eye changes shape in order to focus on objects at differing distances.
He also studied Physics, and, amongst other things,
definitely established the wave theory of light, as described
below. Finally, he also studied Egyptology, and helped
decipher the Rosetta Stone.

Young knew that sound was a wave phenomenon, and, hence,  that if two sound
waves of equal intensity, but $180^\circ$ out of phase, reach the ear
then they cancel one another out, and no sound is heard. This phenomenon
is called {\em interference}. Young reasoned that if light were actually
a wave phenomenon, as he suspected, then a similar interference effect
should occur for light. This line of reasoning lead Young to perform
an experiment which is nowadays referred to as {\em Young's double-slit
experiment}. 

In Young's experiment, two very narrow parallel slits, separated by
a distance $d$,  are cut into a
thin sheet  of metal. Monochromatic light, from  a distant light-source, passes
through the slits and eventually hits a screen a comparatively large
distance $L$ from the slits. The experimental setup is sketched
in Fig.~\ref{f14.4}.

\begin{figure}
\epsfysize=3in
\centerline{\epsffile{Chapter14/fig14.4.eps}}
\caption{\em Young's double-slit experiment.}\label{f14.4}
\end{figure}

According to Huygens' principle, each slit radiates spherical light waves.
The light waves emanating from each slit are superposed on the screen. If
the waves are $180^\circ$ out of phase then {\em destructive interference}\/
occurs, resulting in a dark patch on the screen. On the other hand,
if the waves are completely in phase then {\em constructive interference}\/
occurs, resulting in a light patch on the screen.

The point $P$ on the screen which lies exactly opposite to the centre point
of the
two slits, as shown in Fig.~\ref{f14.5}, is obviously associated with
a bright patch. This follows because the path-lengths from each slit to
this point are the same. The waves emanating from each slit are initially
in phase, since all points on the incident wave-front are in phase ({\em i.e.},
the wave-front is straight and parallel to the metal sheet).
The
waves are still in phase at point $P$ since they have
traveled equal distances in order to reach that point.

\begin{figure}
\epsfysize=3in
\centerline{\epsffile{Chapter14/fig14.5.eps}}
\caption{\em Interference of light in Young's double-slit experiment.}\label{f14.5}
\end{figure}

From the above discussion, the general condition for constructive interference
on the screen 
is simply that the difference in path-length ${\mit\Delta}$ between
the two waves be an {\em integer}\/ number of wavelengths. In other words,
\begin{equation}
{\mit\Delta} = m\,\lambda,
\end{equation}
where $m=0,1,2,\cdots$. Of course, the point $P$ corresponds to the special
case where $m=0$. It follows, from Fig.~\ref{f14.5}, that the angular
location  of the $m$th bright patch on the screen is given by
\begin{equation}
\sin\theta_m = \frac{\mit\Delta}{d} = \frac{m\,\lambda}{d}.
\end{equation}

Likewise, the general condition for destructive interference 
on the screen is that
the difference in path-length between the two waves be a {\em half-integer}\/ 
number of wavelengths. In other words,
\begin{equation}
{\mit\Delta} = (m+1/2)\,\lambda,
\end{equation}
where $m=1,2,3,\cdots$. It follows that the angular coordinate of the
$m$th dark patch on the screen is given by
\begin{equation}
\sin\theta_m' = \frac{\mit\Delta}{d} = \frac{(m+1/2)\,\lambda}{d}.
\end{equation}

Usually, we expect the wavelength $\lambda$ of the incident light
to be much less than the
perpendicular distance $L$ to the screen. Thus,
\begin{equation}
\sin\theta_m \simeq \frac{y_m}{L},
\end{equation}
where $y_m$ measures position on the screen relative to the point $P$. 

It is clear that the interference pattern on the screen consists of
{\em alternating light and dark bands}, running parallel to the slits. The
distances of the centers of the various light bands from the point $P$
are given by 
\begin{equation}\label{e14.8}
y_m = \frac{m\,\lambda\,L}{d},
\end{equation}
where $m=0,1,2,\cdots$. Likewise, the distances of the centres of the
various dark bands from the point $P$ are given by 
\begin{equation}
y_m' = \frac{(m+1/2)\,\lambda\,L}{d},
\end{equation}
where $m=1,2,3,\cdots$. 
The bands are {\em equally spaced}, and of thickness $\lambda\,L/d$. 
Note that if the distance from the screen $L$ is much larger than the
spacing $d$ between the two slits then the thickness of the bands
on the screen greatly exceeds the wavelength $\lambda$ of the light. Thus,
given a sufficiently large ratio $L/d$, it should be possible to observe
a banded
  interference pattern on the screen, despite the fact that the wavelength
of visible light is only of order 1 micron. Indeed, when Young performed this
experiment in 1803  he observed an interference pattern of the type
described above. Of course, this pattern is a {\em direct consequence}\/
 of the wave nature of light, and
  is completely inexplicable on the basis
of geometric optics.

It is interesting to note that when Young first presented his findings to
the Royal Society of London he was ridiculed. His work only achieved
widespread
acceptance when it was confirmed, and greatly extended, by the French
physicists Augustin Fresnel and Francois Argo in the 1820s. 
The particle theory of light was dealt its final death-blow
 in 1849 when the French physicists Fizeau and Foucault independently
demonstrated that light propagates  {\em more slowly}\/ though water than
though air. Recall (from Sect.~\ref{s14.1}),
 that the particle theory of light can only
account for the law of refraction on the assumption that light
propagates {\em faster}\/ through dense media, such as water, than through
rarefied media, such as air. 

\subsection{Interference in Thin Films}
In everyday life, the interference of light  most commonly gives rise to
easily observable effects when
light impinges on a thin film of some transparent material. For instance,
the brilliant colours seen in soap bubbles,  in oil
films floating on puddles of water,  and in the feathers of a peacock's
tail, are due to interference of this type. 

Suppose that a very thin film of air is trapped between two
pieces of glass, as shown in Fig.~\ref{f14.6}. If monochromatic light
({\em e.g.}, the yellow light from a sodium lamp) is incident {\em almost
normally}\/ to the film then some of the light is reflected from the
interface between the bottom of the upper plate and the air, and
some is reflected from the interface between the air and the top of the
lower plate. The eye focuses these two parallel light beams at one
spot on the retina. The two beams produce either destructive or constructive
interference, depending on whether their path difference is equal
to an odd or an even number of half-wavelengths, respectively.

\begin{figure}
\epsfysize=3in
\centerline{\epsffile{Chapter14/fig14.6.eps}}
\caption{\em Interference of light due to a thin film of air trapped between
two pieces of glass.}\label{f14.6}
\end{figure}

Let $t$ be the thickness of the air film. The difference in path-lengths
between the two light rays shown in the figure is clearly ${\mit\Delta}=2\,t$.
Naively, we might expect that constructive interference, and, hence,
 {\em brightness},
would  occur if ${\mit \Delta}=m\,\lambda$, where $m$ is an integer,
and destructive interference, and, hence, {\em  darkness}, would occur if
${\mit\Delta}=(m+1/2)\,\lambda$. However, this is not the
entire picture, since an additional phase difference is introduced between
the two rays on reflection. The first ray is reflected at an interface
between an optically dense medium (glass), through which the
ray travels,  and a less dense medium (air).
There is no phase change on reflection from such an interface, just
as there is no phase change when a wave on a string is reflected from a
free end of the string. (Both waves on strings and electromagnetic
waves are {\em transverse waves}, and, therefore, have analogous
properties.)
The second ray is reflected at an interface
between an optically less dense medium (air),  through which the
ray travels,  and a  dense medium (glass). There is a
$180^\circ$ phase change on reflection from such an interface, just as
there is a $180^\circ$ phase change when a wave on a string
is reflected from a fixed end. Thus, an additional $180^\circ$ phase change
is introduced between the two rays, which is equivalent to an
additional path difference of $\lambda/2$. When this additional
phase change is taken into account, the condition for constructive
interference becomes 
\begin{equation}
2\,t=(m+1/2)\,\lambda,
\end{equation}
where $m$ is an integer. Similarly, the condition for
destructive interference becomes
\begin{equation}
2\,t = m\,\lambda.
\end{equation}

For white light, the above criteria yield constructive interference for
some wavelengths, and destructive interference for others. Thus,
the light reflected back from the film exhibits those colours
for which the constructive interference occurs.

If the thin film consists of water, oil, or some other transparent material
of refractive index $n$ then the results are basically the same as those for
an air film, except that the wavelength of the light in the
film is reduced from $\lambda$ (the vacuum wavelength) to $\lambda/n$.
It follows that the modified  criteria for constructive and destructive
interference are
\begin{equation}
2\,n\,t=(m+1/2)\,\lambda,
\end{equation}
and
\begin{equation}
2\,n\,t = m\,\lambda,
\end{equation}
respectively.

\subsection{Worked Examples}
\subsection*{\em Example~14.1: Double slit experiment}
\noindent{\em Question:}~Coherent light of wavelength $633\,{\rm nm}$ from a
He-Ne laser falls on a double slit with a slit separation of $0.103\,{\rm mm}$. 
An interference pattern is produced on a screen $2.56\,{\rm m}$ from the slits.
Calculate the separation on the screen of the two fourth-order bright
fringes on either side of the central image.\\
~\\
\noindent{\em Solution:}~The easiest way to handle this problem is to calculate
the distance $y_4$ of the fourth-order bright fringe on one side from the
central image, and then  double this value to obtain the distance
between the two fourth-order images. From Eq.~(\ref{e14.8}),
$$
y_4 = \frac{4\,\lambda\,L}{d} =\frac{4\,(633\times 10^{-9})\,(2.65)}{(0.103\times
10^{-3})} = 6.29\,{\rm cm}.
$$
The distance between the two fourth-order fringes is therefore
$$
2\,y_4 = 12.6\,{\rm cm}.
$$


\subsection*{\em Example~14.2: Interference in thin films}
\noindent{\em Question:}~A soap bubble 250 nm thick is illuminated
by white light. The index of refraction of the soap film is $1.36$.
Which colours are {\em not}\/ seen in the reflected light? Which colours
appear strong in the reflected light? What colour does the soap film
appear at normal incidence?\\
~\\
\noindent{\em Solution:}~For destructive interference, we must have
$n\,t=m\,\lambda/2$. Thus, the wavelengths that are {\em not}\/
reflected satisfy
$$
\lambda_m = \frac{2\,n\,t}{m},
$$
where $m=1,2,3,\cdots$. It follows that
$$
\lambda_1 = \frac{(2)\,(1.36)\,(250\times 10^{-9})}{(1)} = 680\,{\rm nm},
$$
and
$$
\lambda_2 = \frac{(2)\,(1.36)\,(250\times 10^{-9})}{(2)} = 340\,{\rm nm}.
$$
These are the only wavelengths close to the visible region of the
electromagnetic spectrum for which destructive interference occurs.
In fact, 680~nm lies right in the middle of the red region of the
spectrum, whilst 340~nm lies in the ultraviolet region (and is,
therefore, invisible to the human eye). It follows that the only
non-reflected colour is {\em red}.

For constructive interference, we must have $n\,t=(m+1/2)\,\lambda/2$. Thus, the wavelengths that are {\em strongly
reflected}\/ satisfy
$$
\lambda_m' = \frac{2\,n\,t}{m+1/2},
$$
where $m=0,1,2,\cdots$. It follows that
$$
\lambda_1' = \frac{(2)\,(1.36)\,(250\times 10^{-9})}{(1/2)} = 1360\,{\rm nm},
$$
and
$$
\lambda_2' = \frac{(2)\,(1.36)\,(250\times 10^{-9})}{(3/2)} = 453\,{\rm nm},
$$
and
$$
\lambda_3' = \frac{(2)\,(1.36)\,(250\times 10^{-9})}{(5/2)} = 272\,{\rm nm}.
$$
A wavelength of 272~nm lies in the ultraviolet region whereas 
1360~nm lies in the infrared. Clearly, both wavelengths correspond
to light which is invisible to the human eye. The only strong reflection
occurs at 453~nm, which corresponds to the {\em blue-violet}\/ region
of the spectrum.

The reflected light is weak in the red region of the spectrum and
strong in the blue-violet region. The soap film will, therefore, possess
a pronounced {\em blue}\/ colour.

