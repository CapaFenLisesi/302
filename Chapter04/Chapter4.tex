\section{Gauss' Law}\label{sgauss}
\subsection{Electric Field-Lines}
An electric field can be represented diagrammatically as a set of lines with
arrows on, 
called {\em electric field-lines}, which fill space. Electric field-lines
are drawn  according to the following rules:
\begin{quote}
{\sf The direction of the electric field is everywhere tangent to the field-lines,
in the sense of the arrows on the lines. The magnitude of the field is proportional
to the number of field-lines per unit area passing through a small surface normal to the lines.}
\end{quote}
Thus, field-lines   determine the magnitude,
 as well as the direction, of the electric field.
In particular, the field
is strong at points where the field-lines are closely spaced, and weak at points
where they are far apart. 

\begin{figure}[h]
\epsfysize=2.5in
\centerline{\epsffile{Chapter04/fig4.1.eps}}
\caption{\em The electric field-lines of a positive point charge.}\label{f4.1}
\end{figure}

The electric field-lines associated with  a {\em positive}\/ point charge are a
set of unbroken, evenly spaced (in solid angle) straight-lines which radiate from the charge---see Fig.~\ref{f4.1}.
Thus, the tangent to the field-lines is always directed radially away from the
charge, giving the correct direction for the electric field. The number
of electric field-lines per unit area normal to
the lines  falls off like $1/r^2$, where $r$ is
the radial distance from the charge, since the total number of lines is fixed,
whereas the area normal to the lines increases like $r^2$. Thus, the electric
field-strength falls off like $1/r^2$,  in accordance with Coulomb's law. 

By analogy, the electric field-lines associated with a {\em negative}\/ point
charge are a set of unbroken, evenly spaced  (in solid angle) straight lines which 
converge on the charge. 

As a general rule, electric field-lines generated by fixed charges begin on positive charges, end on negative
charges, and are unbroken and never cross in the vacuum regions between charges.

\subsection{Gauss' Law}\label{s4.2}
One of the most useful results in electrostatics is named after the
celebrated German mathematician Karl Friedrich Gauss (1777--1855). 

Suppose that a positive point charge $q$ generates an electric field ${\bf E}$. Consider a spherical 
surface of radius $R$, centred on
the charge. The normal to this surface is everywhere parallel to the direction of the
electric field ${\bf E}$, since the field always points
radially away from the charge. The area of the surface is $4\pi\, R^2$. 
Finally, the strength of the electric field at radius $R$ is $E(R) = q/(4\pi\epsilon_0\,R^2)$. 
Hence, if we multiply the electric field-strength by the  area of the surface, we obtain
\begin{equation}
E(R)\,4\pi \,R^2 = \frac{q}{4\pi\epsilon_0\,R^2} \,4\pi \,R^2 = \frac{q}{\epsilon_0}.
\end{equation}
Note that the final result is independent of the radius of the sphere. Thus,
the same result would be obtained for any sphere centred on the charge. This is
the essence of {\em Gauss' law}. 

You may be wondering why it took a famous German mathematician to prove such a
trivial-seeming  law. 
Well, Gauss proved that this  law also applies to {\em any}\/ closed surface,
and {\em any}\/ distribution of electric charges. Thus, if we
multiply each outward element of a general  closed surface $S$ by the component 
of the electric field normal to that element, and then sum over the
entire surface, the result is the total charge enclosed by the surface, divided
by $\epsilon_0$. In other words,
\begin{equation}
\oint_S {\bf E}\cdot d{\bf S}=\frac{Q}{\epsilon_0},
\end{equation}
where $S$ is a closed surface, and
$Q$ is the charge enclosed by it. The integral is termed the
{\em electric flux}, ${\mit\Phi}_E$, through the surface, and
is proportional to the number of electric field-lines which pierce this surface. We adopt the convention that
the flux is positive if the electric field points outward through the surface, and
negative if the field points inward. Thus, Gauss' law can be written:
\begin{quote}
{\sf The electric flux through any closed surface is equal to the total
charge enclosed by the surface, divided by $\epsilon_0$.}
\end{quote}

Gauss' law is especially useful for  evaluating the electric fields
produced by charge distributions which possess some sort
of symmetry. Let us examine three 
examples of such distributions.

\subsection{Electric Field of a Spherical Conducting Shell}\label{snormal}
Suppose that a thin, spherical, conducting shell carries a negative charge $-Q$. We expect the
excess electrons to mutually repel one another, and, thereby, become uniformly distributed
over the surface of the shell. The electric field-lines produced outside such a
charge distribution point towards the surface of the conductor, and end on the
excess electrons. Moreover, the field-lines are {\em normal}\/ to the surface of the conductor.
This must be the case, otherwise the electric field would have a component parallel
to the conducting surface. Since the excess electrons are free to move through the conductor,
any parallel component of the field would cause a redistribution of the charges 
on the shell. This process will only cease when the parallel component 
 has been reduced to zero over the whole surface of the shell. It follows that:
\begin{quote}
{\sf The electric field immediately above the surface of a conductor is directed
normal to that surface}.
\end{quote}

\begin{figure}[h]
\epsfysize=3.5in
\centerline{\epsffile{Chapter04/fig4.2.eps}}
\caption{\em The electric field generated by a negatively charged spherical
conducting shell.}\label{f4.2}
\end{figure}

Let us consider an imaginary surface, usually referred to as a {\em gaussian surface},
which is a sphere of radius $R$ lying just above the surface of
the conductor. Since the electric field-lines are everywhere normal
to this surface, Gauss' law tells us that
\begin{equation}
{\mit\Phi}_E = E\,A = \frac{-Q}{\epsilon_0},
\end{equation}
where ${\mit\Phi}_E$ is the electric flux through the gaussian surface, 
$A=4\pi\, R^2$ the area of this surface, and $E$ the electric
field-strength just above the surface of the conductor. Note that, by symmetry, 
 $E$ is
uniform over the surface of the conductor. It follows that
\begin{equation}\label{e4.4}
E = \frac{-Q}{\epsilon_0\,A} =- \frac{Q}{4\pi\epsilon_0\,R^2}.
\end{equation}
But, this is  the same result as would be obtained from Coulomb's law for 
a point charge of magnitude $-Q$ located at the centre of the conducting
shell. Now, a simple extension of the above argument leads to the conclusion that
Eq.~(\ref{e4.4}) holds {\em everywhere}\/ outside the shell (with $R$ representing
the radial distance from the center of the shell).
Hence, we conclude the electric field outside a charged, spherical, conducting shell is the same as that generated when all the charge is concentrated at the centre of the shell.

Let us repeat the above calculation using a spherical gaussian surface which
lies just inside the conducting shell. Now, the
gaussian surface encloses no charge, since all of the charge lies on the
shell, so it follows from Gauss' law, and symmetry, that the
electric field inside the shell is zero. In fact, the electric field inside 
{\em any}\/ closed
hollow conductor is {\em zero}\/ (assuming that the region enclosed by the conductor
contains no charges). 

\subsection{Electric Field of a Uniformly Charged Wire}
Consider a long straight wire which carries the uniform
charge per unit length $\lambda$.
We expect the electric field generated by such a charge distribution
to possess cylindrical symmetry. We also expect the field to
point radially (in a cylindrical
sense) away from the wire (assuming that the wire is positively
charged).  

\begin{figure}[h]
\epsfysize=3.5in
\centerline{\epsffile{Chapter04/fig4.3.eps}}
\caption{\em The electric field generated by a uniformly charged wire.}\label{f4.2a}
\end{figure}

Let us draw a cylindrical gaussian surface, co-axial with the wire, of radius
$R$ and length $L$---see Fig.~\ref{f4.2a}. The above symmetry arguments imply that the electric field generated by the wire is everywhere perpendicular
to the curved surface of the cylinder. Thus, according to Gauss' law,
\begin{equation}
E(R)\,2\pi \,R\, L = \frac{\lambda\, L}{\epsilon_0},
\end{equation}
where $E(R)$ is the electric field-strength a perpendicular distance $R$ from the wire. Here, the
left-hand side represents the electric flux through the gaussian surface.
Note that there is no contribution from the two flat ends of the cylinder, since the
field is parallel to the surface there. The right-hand side represents the
total charge enclosed by the cylinder, divided by $\epsilon_0$. It follows
that
\begin{equation}
E(R) = \frac{\lambda}{2\pi\epsilon_0\,R}.
\end{equation}
The field points radially (in a cylindrical sense) away from the wire
if $\lambda>0$, and radially towards the wire if $\lambda<0$. 

\subsection{Electric Field of a Uniformly Charged Plane}\label{s4.5}
Consider an infinite plane which carries the  uniform charge per unit area
$\sigma$. Suppose that the plane coincides with the 
$y$--$z$ plane ({\em i.e.}, the plane which satisfies $x=0$). 
By symmetry, we expect the electric field on either side of the plane to
be a function of $x$ only, to be directed
normal to the plane, and to point away from/towards the plane
depending on whether $\sigma$ is  positive/negative. 

\begin{figure}[h]
\epsfysize=2.5in
\centerline{\epsffile{Chapter04/fig4.4.eps}}
\caption{\em The electric field generated by a uniformly charged plane.}\label{f4.2b}
\end{figure}

Let us draw a cylindrical gaussian surface, whose axis is normal
to the plane, and which is cut in half by the plane---see Fig.~\ref{f4.2b}. Let the 
cylinder run from $x=-a$ to $x=+a$, and let its cross-sectional
area be $A$. According to Gauss' law,
\begin{equation}
2 \,E(a)\, A = \frac{\sigma\, A}{\epsilon_0},
\end{equation}
where $E(a)=-E(-a)$ is the electric field strength at $x=+a$. 
Here,  the left-hand side represents the electric flux out of the surface. Note that the
only contributions to this flux come from the flat surfaces at the two ends of the cylinder.
The right-hand side represents the charge enclosed by the cylindrical
surface, divided by $\epsilon_0$. It follows that
\begin{equation}\label{e4.8}
E = \frac{\sigma}{2\,\epsilon_0}.
\end{equation}
Note that the electric field is uniform ({\em i.e.}, it does not
depend on $x$), normal to the charged plane,
and oppositely directed on either side of the plane. The electric field always
points away from a positively charged plane, and {\em vice versa}. 

\begin{figure}[h]
\epsfysize=2in
\centerline{\epsffile{Chapter04/fig4.5.eps}}
\caption{\em The electric field generated by two oppositely charged parallel
planes.}\label{f4.3}
\end{figure}

Consider the electric field produced by two parallel planes which carry equal
and opposite uniform charge densities $\pm \sigma$. We can calculate this field
by superposing the electric
fields produced by each plane taken in isolation. It is easily seen,
from the above discussion,  that in the region  between
the planes the field is uniform, normal to the planes, directed from the
positively to the negatively charged plane, and of magnitude
\begin{equation}
E= \frac{\sigma}{\epsilon_0}
\end{equation}
---see Fig.~\ref{f4.3}.
Outside this region,  the electric field cancels to zero. 
The above result is only valid for two charged planes of infinite extent. However,
the result is approximately valid for two charged planes of finite extent, provided
that the spacing between the planes is small compared to their typical dimensions. 

\subsection{Charged Conductors}\label{szero}
Suppose that we put a negative charge on an arbitrarily shaped, solid,
conducting object. Where does the excess negative charge end up after the
charges have attained their equilibrium positions?

Let us construct a gaussian surface which lies just inside the surface
of the conductor. Application of Gauss' law yields
\begin{equation}
\oint {\bf E}\cdot d{\bf S} = \frac{Q}{\epsilon_0},
\end{equation}
where $Q$ is the enclosed charge.
But, the electric field-strength inside a conductor must be {\em zero}, since the
charges are free to move through the conductor, and will, thus, continue to move until no field remains. 
Hence, the left-hand side of the above equation is zero, and, therefore,
the right-hand side must also be zero. This can only be the case if there are
no charges enclosed by the gaussian surface. In other words, there can be no excess
charge in the interior of the conductor.
Instead, all of the excess charge must be distributed
over the surface of the conductor. It follows that:
\begin{quote}
{\sf Any excess charge on a solid conductor resides entirely
on the outer surface of the conductor.}
\end{quote}

\subsection{Worked Examples}
\subsection*{\em Example 4.1: Electric field of a uniformly charged sphere}
\noindent{\em Question:} An insulating sphere of radius $a$ carries a total
charge $Q$ which is {\em uniformly} distributed over the volume of the
sphere. Use Gauss' law to find the electric field distribution both
inside and outside the sphere. \\
~\\
\noindent{\em Solution:}
By symmetry, we expect the electric field generated by a spherically symmetric
charge distribution to point radially towards, or away from, the
center of the distribution, and to depend only on
the radial distance $r$ from this point. Consider a
gaussian surface which is a sphere of radius $r$, centred on the centre of the
charge distribution. Gauss' law gives
$$
A(r)\,E_r(r) = \frac{q(r)}{\epsilon_0},
$$
where $A(r) = 4\pi\, r^2$ is the area of the surface, $E_r(r)$ the radial electric
field-strength at radius $r$, and $q(r)$ the total charge enclosed by the
surface. It is easily seen that
$$
q(r) = 	
\left\{\begin{array}{lcl}
Q&\mbox{\hspace{1cm}}&r\geq a\\[0.5ex]
Q\,(r/a)^3&&r<a
\end{array}\right..
$$
Thus,
$$
E_r(r) =
\left\{\begin{array}{lcl}
\frac{Q}{4\pi\epsilon_0\,r^2}&\mbox{\hspace{1cm}}&r\geq a\\[0.5ex]
\frac{Q\,r}{4\pi\epsilon_0\,a^3}&&r<a
\end{array}\right..
$$
Clearly, the electric field-strength is proportional to $r$ inside the
sphere, but falls off like $1/r^2$ outside the sphere.


