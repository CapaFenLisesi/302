\section{Magnetic Induction}
\subsection{Faraday's Law}
The phenomenon of magnetic induction plays a crucial role in 
three very useful electrical devices: the {\em electric generator}, the {\em electric
motor}, and the {\em transformer}. Without these devices, modern life would be
impossible in its present form. Magnetic induction was discovered in 1830 by the 
English physicist Michael Faraday.  The American physicist Joseph Henry
independently made the same discovery at about the same time. Both
physicists were intrigued by the fact that an electric current flowing around 
a circuit can generate a magnetic field. Surely, they reasoned, if an electric
current can generate  a magnetic field then a magnetic field must somehow be able to
generate an electric current.
However, it took many years of fruitless experimentation
before they were able to find the essential ingredient which allows
a magnetic field to generate an electric current. This
 ingredient is {\em time variation}.

Consider a planar loop $C$ of conducting wire of cross-sectional area $A$.
Let us place this loop in a magnetic field whose strength $B$ is approximately
uniform over the extent of the loop. Suppose that the direction of the
magnetic field subtends an angle $\theta$ with the normal direction to the
loop. The {\em magnetic flux}\/ ${\mit\Phi}_B$ through the loop is
defined as the product of the area of the loop and the component of
the magnetic field perpendicular to the loop. Thus,
\begin{equation}
{\mit\Phi}_B = A\,B_\perp= A\, B\, \cos\theta.
\end{equation}
If the loop is wrapped around itself $N$ times ({\em i.e.}, if the loop
has $N$ {\em turns}) then the magnetic flux through the loop is simply
$N$ times the magnetic flux through a single turn:
\begin{equation}
{\mit\Phi}_B = N\,A\,B_\perp.
\end{equation}
Finally, if the magnetic field is not uniform over the loop, or the loop does not
lie in one plane,
 then we must evaluate the
magnetic flux as a surface integral
\begin{equation}
{\mit\Phi}_B = \int_S {\bf B}\cdot d{\bf S}.
\end{equation}
Here, $S$ is some surface attached to $C$.
If the loop has $N$ turns then the flux is $N$ times the above value. 
The SI unit of magnetic flux is the weber (Wb). One tesla is equivalent to
one weber per meter squared:
\begin{equation}
1\,{\rm T} \equiv 1\,{\rm Wb\, m}^{-2}.
\end{equation}

Faraday discovered  that if the magnetic field through a loop of wire
{\em varies  in time} then an emf is induced around the loop. 
Faraday was able to observe this effect because the emf gives rise to
a current circulating in  the loop. Faraday found that the magnitude of
the emf is directly proportional to the time rate of change of the magnetic field.
He also discovered that an emf is generated when a loop of wire {\em moves}\/
from a region of low magnetic field-strength to one of high magnetic field-strength, and {\em vice versa}. The emf is directly proportional to
the velocity with which the loop  moves between the two regions. Finally,
Faraday discovered that an emf is generated around a loop which  {\em rotates}\/
in a uniform magnetic field of constant strength. In this case, the emf
is directly proportional to the rate at which the loop rotates.
Faraday was eventually
able to propose a single
law which could account for all of his many and varied observations. This law, which is known as
{\em Faraday's law of magnetic induction}, is as follows:
\begin{quote}
{\sf The emf induced in a circuit is proportional to the time rate of change of the
magnetic flux linking that circuit.}
\end{quote} 
SI units have been fixed so that the constant of proportionality in this
law is {\em unity}. Thus, if the magnetic flux through a circuit changes
by an amount $d{\mit\Phi}_B$ in a time interval $dt$
then the emf ${\cal E}$ generated in the circuit is
\begin{equation}
{\cal E} = \frac{d{\mit\Phi}_B}{dt}.
\end{equation}

There are many different ways in which the magnetic flux linking an
electric  circuit can
change. Either the magnetic field-strength can change, or the direction of the magnetic
field can change, or the position of the circuit can change, or the shape of the
circuit can change, or the orientation of the circuit can change. 
Faraday's law states that all of these ways are 
completely {\em equivalent}\/ as far as the generation of an emf around the
circuit is concerned. 

\subsection{Lenz's Law}
We still have not specified in which direction the emf generated 
by a time-varying magnetic flux linking an electric circuit acts. In order to
help specify this direction, we need to make use of a 
 right-hand rule. Suppose
that a current $I$  circulates around a planar loop of conducting wire, and,
thereby, generates a magnetic field ${\bf B}$. What is the direction of
this magnetic field as it passes through the middle of the loop? Well,
if the fingers of a right-hand circulate in the same direction as the current,
then the thumb indicates the direction of the
magnetic field as it passes through the centre of the loop. This is illustrated in 
Fig.~\ref{f9.1}.

\begin{figure}
\epsfysize=3in
\centerline{\epsffile{Chapter09/fig9.01.eps}}
\caption{\em Magnetic field generated by a planar current-carrying loop.}\label{f9.1}
\end{figure}

Consider a plane loop of conducting wire which is linked by magnetic
flux. By convention, the direction in which current would have to
flow around the loop in order to {\em increase}\/ the magnetic flux linking the loop
is termed the {\em positive}\/ direction. Likewise, the direction  in which current would have to
flow around the loop in order to {\em decrease}\/ the magnetic flux
linking the loop
is termed the {\em negative}\/ direction. Suppose that the magnetic flux linking the
loop is increased.  In accordance with 
Faraday's law, an emf is generated around the loop.
 Does this emf act in the positive
direction, so as to drive a current around the loop
which further increases the magnetic flux, or does it act in the
negative direction, so as to drive a current around the loop which
decreases the magnetic flux? It is easily demonstrated experimentally that the
emf acts in the negative direction. Thus:
\begin{quote}
{\sf The emf induced in an electric circuit always acts in such a direction that
the current it drives around the circuit  opposes the change 
in magnetic flux which produces the emf.}
\end{quote}
This result is known as {\em Lenz's law}, after the nineteenth century Russian
scientist Heinrich Lenz, who first formulated it. Faraday's law, combined with
Lenz's law,  is usually
written
\begin{equation}
{\cal E} = -\frac{d{\mit\Phi}_B}{dt}.
\end{equation}
The minus sign is to remind us that the emf always acts to oppose the change
in magnetic flux which generates the emf. 

\subsection{Magnetic Induction}\label{s9.3}
Consider a one-turn loop of conducting wire $C$ which is placed in a magnetic
field ${\bf B}$. The magnetic flux ${\mit\Phi}_B$ linking  loop $C$ can be written
\begin{equation}
{\mit\Phi}_B = \int_S {\bf B}\cdot d{\bf S}
\end{equation}
where $S$ is any surface attached to the loop.

Suppose that the magnetic field   changes in time,   causing the
magnetic flux ${\mit\Phi}_B$ linking circuit $C$ to vary.
Let the flux change by an amount $d{\mit\Phi}_B$ in the time interval $dt$. According to Faraday's law, the emf
${\cal E}$
induced around loop $C$ is given by
\begin{equation}\label{ex}
{\cal E} = - \frac{d{\mit\Phi}_B}{dt}.
\end{equation}
If ${\cal E}$ is
{\em positive}\/ then the emf acts around the loop in the {\em same}\/ sense as that
indicated
by the fingers of a right-hand, when the thumb points
in the  direction of the mean  magnetic field passing through the loop. Likewise,
if ${\cal E}$ is
{\em negative}\/  then the emf acts around the loop in the 
{\em opposite}\/ sense to that
indicated
by the fingers of a right-hand, when the thumb  points
in the  direction of the mean magnetic field passing through the loop. In the
former case, we say that the emf acts in the {\em positive}\/ direction, whereas
in the latter case we say it acts in the {\em negative}\/ direction.

Suppose that ${\cal E}>0$, so that the emf acts in the positive direction.
How, exactly, is this emf produced? In order to answer this question,
we need to remind ourselves what an emf actually is. When we say that
an emf ${\cal E}$ acts around the loop $C$ in the positive direction,
what we really mean is that a charge $q$ which circulates once around
the loop in the positive direction acquires the energy $q\,{\cal E}$.
How does the charge acquire this energy? Clearly, either an electric
field or a magnetic field, or some combination of the two, must perform the
work $q\,{\cal E}$ on the charge as it circulates around the loop.
However, we have already seen, from Sect.~\ref{s8.4}, that a magnetic
field is unable to do work on a charged particle. Thus, the charge must
acquire the energy $q\,{\cal E}$ from an {\em electric}\/ field as it
circulates once around the loop in the positive direction.

According to Sect.~\ref{spotn}, the work that the electric field does on the charge as it goes around
the loop is
\begin{equation}
W=q\,\oint_C {\bf E}\cdot d{\bf r},
\end{equation}
where $d{\bf r}$ is a line element of the loop. Hence, by energy conservation,
we can write $W = q\,{\cal E}$, or
\begin{equation}\label{ey}
{\cal E} = \oint_C {\bf E} \cdot d{\bf r}.
\end{equation}
 The term
on the right-hand side of the above expression can be recognized as
the {\em line integral}\/ of the electric field around loop $C$ in the
positive direction. Thus, the emf generated around the
circuit $C$ in the positive direction is equal to the line integral
of the electric field around the circuit in the same direction. 

Equations~(\ref{ex}) and (\ref{ey}) can be combined to give
\begin{equation}\label{e9.13}
 \oint_C {\bf E} \cdot d{\bf r}= -\frac{d{\mit\Phi}_B}{dt}.
\end{equation}
Thus, Faraday's law implies that the line integral of the electric field around circuit $C$
(in the positive direction) is equal to minus the time rate of change of the
magnetic flux linking this circuit. Does this law just apply to conducting
circuits, or can we apply it to an arbitrary closed loop in space? Well,
the difference between a conducting circuit and an arbitrary closed loop
is that electric current can flow around a circuit, whereas current
cannot, in general, flow around an arbitrary loop. In fact, the emf
${\cal E}$ induced around a conducting circuit drives a current $I={\cal E}/R$
around that circuit, where $R$ is the resistance of the circuit. However,
we can make this resistance arbitrarily large without invalidating Eq.~(\ref{e9.13}). 
In the limit in which $R$ tends to infinity, no current flows around the circuit,
so the circuit becomes indistinguishable from an arbitrary loop. Since we can place
such a circuit anywhere in space, and Eq.~(\ref{e9.13}) still holds, we are forced to the 
conclusion that Eq.~(\ref{e9.13}) is valid for {\em any} closed loop in space, and not just
for conducting circuits. 

Equation~(\ref{e9.13}) describes how a time-varying magnetic field {\em generates}\/
an electric field which fills space. The strength of the electric field is directly
proportional to the rate of change of the magnetic field. The 
field-lines associated with this electric
field  form loops in the plane perpendicular to the magnetic field. If the
magnetic field is increasing then the electric field-lines circulate in the
opposite sense to the fingers of a right-hand, when the thumb points 
in the direction of the field. If the
magnetic field is decreasing then the electric field-lines circulate in the
same sense as the fingers of a right-hand, when the thumb points 
in the direction of the field. This is illustrated in Fig.~\ref{f9.2}.

\begin{figure}
\epsfysize=3in
\centerline{\epsffile{Chapter09/fig9.02.eps}}
\caption{\em Inductively generated electric fields}.\label{f9.2}
\end{figure}

We can now appreciate that when a conducting circuit is placed in a
time-varying magnetic field, it is the electric field induced by the changing
magnetic field which gives rise to the emf around the circuit. If the loop has a
finite resistance then this electric field also drives a current around the circuit.
Note, however, that the electric field is generated irrespective of the
presence of a conducting circuit. The electric field generated by a time-varying
magnetic field is quite different in nature to that generated by a set 
of stationary
electric charges. In the latter case, the electric field-lines begin on
positive charges, end on negative charges, and {\em  never}\/ form closed loops
in free space. In the former case, the electric field-lines {\em never}\/ begin or
end, and {\em  always}\/ form closed loops in free space. In fact, the electric
field-lines generated by magnetic induction behave in much the same
manner as magnetic field-lines. Recall, from Sect.~\ref{s3.1}, that  an electric
field generated by fixed charges is unable to do net work on a charge
which circulates in a closed loop. On the other hand, an electric
field generated by magnetic induction certainly can do work on a charge
which circulates in a closed loop. This is basically how a current is induced in
a conducting loop placed in a time-varying magnetic field. One consequence of this
fact 
is that the work done in slowly moving a charge
between two points in an inductive electric field {\em does}\/ depend on the
path taken between the two points. It follows that we cannot
calculate  a {\em unique}\/ potential difference between two points in an inductive
electric field. In fact, the whole idea of electric potential breaks down
in a such a field (fortunately, there is a way of
salvaging the idea of electric potential in an inductive  field, but this topic
lies beyond the scope of this course). Note, however, that it is still
possible to calculate a {\em unique} value for the emf generated around a conducting
circuit by an inductive electric field, because, in this case, the path taken by
electric charges is uniquely specified: {\em i.e.}, the charges have 
to follow the 
circuit.

\subsection{Motional Emf}\label{s9.4}
We now understand how an emf is generated around a {\em fixed}\/ circuit placed in
a time-varying magnetic field. But, according to Faraday's law, an
emf is also generated around a {\em moving} circuit placed in a  magnetic
field which does not vary in time. 
According to
Eq.~(\ref{e9.13}), no space-filling inductive
electric field is generated in the latter case,
since   the magnetic field is steady. So, how do we account for the emf in the latter
 case?

In order to help answer this question, let us consider a simple
circuit in which a conducting rod of length $l$ slides along a
U-shaped conducting frame in the presence of a uniform magnetic field.
This circuit is illustrated in Fig.~\ref{f9.3}. Suppose, for the sake of simplicity, that the
magnetic field is directed perpendicular to the plane of the circuit. To be
more exact, the magnetic field is directed into the page in the figure.
Suppose, further, that we move the rod to the right with the constant velocity
$v$. 

\begin{figure}
\epsfysize=3in
\centerline{\epsffile{Chapter09/fig9.03.eps}}
\caption{\em Motional emf.}\label{f9.3}
\end{figure}

The magnetic flux linked by the circuit is simply the product of
the perpendicular magnetic field-strength, $B$, and the area of the circuit,
$l\,x$, where $x$ determines the position of the sliding rod.
Thus,
\begin{equation}
{\mit\Phi}_B = B\,l\,x.
\end{equation}
Now, the rod moves a distance $dx=v\,dt$ in a
time interval $dt$, so in the same time interval the magnetic
flux linking the circuit {\em increases}\/ by
\begin{equation}
d{\mit\Phi}_B = B\,l\,dx = B\,l\,v\,dt.
\end{equation}
It follows, from Faraday's law, that the magnitude of the emf ${\cal E}$
generated around the circuit is given by
\begin{equation}\label{e9.15}
{\cal E} = \frac{d{\mit\Phi}_B}{d t} = B\,l\,v.
\end{equation}
Thus, the emf generated in the circuit by the moving rod is simply the product of
the magnetic field-strength, the length of the rod, and the velocity of
the rod. 
If the magnetic field is not perpendicular to the circuit,
but instead subtends an angle $\theta$ with respect to the normal direction
to the plane of the circuit, then it is easily demonstrated that the 
{\em motional emf}\/ generated in the circuit by the moving rod is
\begin{equation}
{\cal E} = B_\perp\,l\,v,
\end{equation}
where $B_\perp = B\,\cos\theta$ is the component of the magnetic field 
which is perpendicular to  the plane of the circuit. 

Since the magnetic flux linking the circuit  {\em increases}\/ in time, the
emf acts in the {\em negative}\/ direction ({\em i.e.}, in the opposite sense to the
fingers of a right-hand, if the thumb  points along the
direction of the magnetic field). The emf, ${\cal E}$, therefore, acts in the
{\em anti-clockwise}\/ direction in the figure. If $R$
is the total  resistance of the circuit,
then this emf drives an anti-clockwise electric current of magnitude
$I={\cal E}/R$ around the circuit.

But, where does the emf come from? Let us again remind ourselves what an
emf is. When we say that an emf ${\cal E}$ acts around the circuit
in the anti-clockwise direction, what we really mean is that a charge $q$
which circulates once around the circuit in the anti-clockwise direction
acquires the energy $q\,{\cal E}$. The only way in which the charge
can acquire this energy is if something does {\em work}\/
 on it as it circulates.
Let us assume that the charge circulates very {\em slowly}. The magnetic
field exerts a negligibly small force on the charge when it is traversing the
non-moving part of the circuit (since the charge is moving very slowly). 
However, when the charge is traversing the moving rod
it experiences  an {\em upward}\/ (in the figure) magnetic force of magnitude $f=q\,v\,B$ 
 (assuming that
$q>0$). The net work done on the charge by this force as
it traverses the rod is 
\begin{equation}
W' = q\,v\,B\,l = q\,{\cal E},
\end{equation}
since ${\cal E} = B\,l\,v$. Thus, it would appear that the motional emf
generated around the circuit can be accounted for in terms of the
magnetic  force exerted
on charges traversing  the moving rod. 

But, if we think carefully, we can see that there is something
seriously wrong with the above explanation. 
We seem to be saying that the charge acquires the energy $q\,{\cal E}$
from the {\em magnetic field}\/ as it moves around the circuit once in the
anti-clockwise direction. But, this is impossible, because a magnetic field 
{\em cannot}\/ do  work on an electric charge. 


Let us look at the problem from the point of view of a charge
$q$ traversing the moving rod. In the frame of reference of the rod,
the charge only moves very slowly, so the magnetic force on it is negligible. In fact, only an electric field can exert a significant
force on a slowly moving charge. In order to account for the motional emf generated
around the circuit, we need the charge to experience an upward force of
magnitude $q\,v\,B$. The only way in which this is possible is if the charge
sees an upward pointing {\em electric field}\/ of magnitude
\begin{equation}
E_0 = v\,B.
\end{equation}
In other words, although there is no electric field in the laboratory frame,
there is an electric field in the frame of reference of the moving rod,
and it is this field which does the necessary  amount of work on charges
moving around the circuit to account for the existence
of the motional emf, ${\cal E}=B\,l\,v.$

More generally, if a conductor moves in the laboratory frame
with velocity ${\bf v}$ in the presence of a magnetic field ${\bf B}$ then
a charge $q$ inside  the conductor experiences a magnetic force ${\bf f} = q\,
{\bf v}\times{\bf B}$. In the frame of the conductor, in which the charge is
essentially stationary, the same force takes the form of an electric
force ${\bf f} = q\,{\bf E}_0$, where ${\bf E}_0$ is the electric field in
the frame of reference of the conductor. Thus,
if a conductor moves with velocity ${\bf v}$ through a magnetic field ${\bf B}$
then the electric field ${\bf E}_0$ which appears in the rest frame of the conductor
is given by
\begin{equation}
{\bf E}_0 = {\bf v} \times {\bf B}.
\end{equation}
This electric field is the ultimate origin of the motional emfs which are
 generated whenever
circuits move with respect to magnetic fields. 

We can now appreciate that Faraday's law is due to a combination of
two apparently distinct effects. The first is the space-filling 
electric field 
generated by a changing magnetic field. The second  is the electric
field generated inside a conductor when it moves through a magnetic field.
In reality, these effects are two aspects of the same basic phenomenon, which
explains why no real distinction is made between them in Faraday's law.

\subsection{Eddy Currents}
We have seen, in the above example, that when a conductor is moved in a magnetic
field a motional emf is generated.  Moreover, according to Worked Example 9.3, this emf drives a current 
which heats the conductor, and, when combined with the magnetic
field,  also
gives rise to a magnetic force acting on the conductor which opposes
its motion. It turns out that these results are quite general. 
Incidentally, the induced  currents which circulate inside a moving conductor in  a static magnetic
field, or a stationary conductor in a time-varying magnetic field, are usually called
{\em eddy currents}. 

Consider a  metal disk which rotates in a perpendicular magnetic field which only
extends over a small rectangular portion of the disk, as shown in Fig.~\ref{f9.4}. Such a field could be produced by the pole of a horseshoe magnetic. 
The motional emf induced in the disk, as it moves through the field-containing
region, acts in the direction ${\bf v}\times{\bf B}$, where ${\bf v}$ is
the velocity of the disk, and ${\bf B}$ the magnetic field. It follows
from Fig.~\ref{f9.4} that the emf acts downward. The emf drives  currents which are
also directed downward. However, these currents must form closed loops, and,
hence, they are directed upward in those regions of the disk immediately adjacent
to the field-containing region, as shown in the figure. It can be seen that
the induced currents flow in little eddies. Hence, the name ``eddy currents.''
According to the right-hand rule, the downward currents in the field-containing
region give rise to a magnetic force on the disk which acts to the right. 
In other words, the magnetic force acts to prevent the rotation of the disk. 
Clearly, external  work must be done on the disk in order to keep it 
rotating at a constant angular velocity. This external work is ultimately
dissipated as heat 
 by the eddy currents circulating inside  the disk. 

\begin{figure}
\epsfysize=3in
\centerline{\epsffile{Chapter09/fig9.04.eps}}
\caption{\em Eddy currents}.\label{f9.4}
\end{figure}

Eddy currents can be  very useful. For instance, some cookers work by  using
eddy currents. The cooking pots, which are usually made out of aluminium, are placed
on plates which generate oscillating magnetic fields. These fields induce
eddy currents in the pots which heat them up. The heat is then transmitted
to the food inside the pots. This type of cooker is particularly useful for
food which needs to be  cooked gradually over a long period of time: {\em i.e.},
over many hours, or even days. Eddy currents can also be used to heat small 
pieces of metal until they become  white-hot by  placing them  in a very
rapidly oscillating magnetic field. This technique is sometimes used in brazing. 
Heating conductors by means of eddy currents is called {\em inductive heating}.
Eddy currents can also be used to damp motion. This technique, which
is called {\em eddy current damping},  is often
employed in galvanometers.

\subsection{The Alternating Current Generator}\label{s9.6}
An electric generator, or dynamo,  is a device which converts mechanical energy into
electrical energy. The simplest practical generator consists of a rectangular
coil rotating in a uniform magnetic field. The magnetic field is usually supplied
by a permanent magnet. This setup is illustrated in Fig.~\ref{f9.5}.

\begin{figure}
\epsfysize=3in
\centerline{\epsffile{Chapter09/fig9.05.eps}}
\caption{\em An alternating current generator.}\label{f9.5}
\end{figure}

Let $l$ be the length of the coil along its axis of rotation, and $w$ the
width of the coil perpendicular to this axis. Suppose that the
coil rotates with  constant angular velocity $\omega$ in a uniform
magnetic field of strength $B$. The velocity $v$ with which the the two
long sides of the coil ({\em i.e.},  
sides $ab$ and $cd$) move through the magnetic field is simply the product
of the angular velocity of rotation $\omega$ and the distance $w/2$ of each
side from the axis of rotation, so $v = \omega\,w/2$. The motional emf
induced in each side is given by ${\cal E} = B_\perp\,l\,v$, where $B_\perp$ is
the component of the magnetic field perpendicular to instantaneous direction 
of motion of the side in question.
If the direction of the magnetic field subtends an
angle $\theta$ with the normal direction to
the coil, as shown in the figure, then $B_\perp = B\,\sin\theta$. 
Thus, the magnitude of the motional emf generated in sides $ab$ and $cd$
is 
\begin{equation}
{\cal E}_{ab} = \frac{B\, w\, l\, \omega\,\sin\theta}{2} = \frac{B\,A\,
\omega\,\sin\theta}{2},
\end{equation}
where $A=w\,l$ is the area of the coil. 
The emf is zero when $\theta = 0^\circ$ or $180^\circ$, since the
direction of motion of sides $ab$ and $cd$ is {\em parallel}\/ to the direction
of the magnetic field in these cases. The emf attains its maximum value when
$\theta = 90^\circ$ or $270^\circ$, since the direction of motion of
sides $ab$ and $cd$ is {\em perpendicular} to the direction of the magnetic field
in these cases. Incidentally, it is clear, from symmetry, that no net motional
emf is generated in sides $bc$ and $da$ of the coil. 

Suppose that the direction of rotation of the coil is such that side
$ab$ is moving into the page in Fig.~\ref{f9.5} (side view), whereas side
$cd$ is moving out of the page. The motional emf induced in side $ab$ acts from 
$a$ to $b$. Likewise, the motional
emf induce in side $cd$ acts from $c$ to $d$. It can be seen that both emfs
act in the clockwise direction around the coil. Thus, the net emf 
${\cal E}$ acting around the
coil is $2\,{\cal E}_{ab}$. If the coil has $N$ turns then the net emf becomes
$2\,N\,{\cal E}_{ab}$. Thus, the general expression for the emf generated around a
steadily rotating, multi-turn coil in a uniform magnetic field is
\begin{equation}\label{e9.22}
{\cal E} = N\,B\,A\,\omega\,\sin( \omega\, t),
\end{equation}
where we have written $\theta = \omega \,t$ for a steadily rotating coil (assuming that
$\theta=0$ at $t=0$). This expression can also be written
\begin{equation}\label{e9.23}
{\cal E} = {\cal E}_{\rm max}\,\sin (2\pi\, f\, t),
\end{equation}
where 
\begin{equation}\label{e9.24}
{\cal E}_{\rm max}= 2\pi\,N\,B\,A\,f
\end{equation}
 is the peak emf produced by the generator, and
$f=\omega/2\pi$ is the number of complete rotations the coils executes per second. Thus, the
peak emf is directly proportional to the area of the coil, the number of turns
in the coil, the rotation frequency of the coil,
and the magnetic field-strength.

Figure~\ref{f9.6} shows the emf specified in Eq.~(\ref{e9.23}) plotted as a function
of time. It can be seen that the variation of the emf with time is
{\em sinusoidal}\/ in nature. The emf attains its peak values when the plane of
the coil is parallel to the plane of the magnetic field,  passes through
zero when the plane of the coil is perpendicular to the magnetic field, and reverses
sign every half period of revolution of the coil. The emf is periodic
({\em i.e.}, it continually repeats the same pattern in time), with
period $T= 1/f$ (which is, of course, the rotation period of the coil). 

\begin{figure}
\epsfysize=2.5in
\centerline{\epsffile{Chapter09/fig9.06.eps}}
\caption{\em Emf generated by a steadily rotating AC generator.}\label{f9.6}
\end{figure}

Suppose that some load ({\em e.g.}, a light-bulb, or an electric heating
element) of resistance $R$ is connected across the terminals of the
generator. In practice, this is achieved by connecting the two ends of the
 coil to rotating rings which are then connected to the external circuit by means
of metal brushes. According to Ohm's law, the current $I$ which flows in the
load is given by
\begin{equation}\label{e9.25}
I = \frac{\cal E}{R} = \frac{ {\cal E}_{\rm max}}{R}\, \sin (2\pi\, f\, t).
\end{equation}
Note that this current is constantly changing direction, just like the
emf of the generator. Hence, the type of generator described above is
usually termed an {\em alternating current}, or $AC$, generator. 

The current $I$ which flows through the load must also flow around the coil.
Since the coil is situated in a magnetic field, this current gives rise to
a torque on the coil which, as is easily demonstrated, acts to slow down its
rotation. According to  Sect.~\ref{s8.11}, the braking torque $\tau$ acting
on the coil is given by
\begin{equation}
\tau = N\,I\,B_\parallel\,A,
\end{equation}
where $B_\parallel=B\,\sin\theta$ is the component of the magnetic field which
lies in the plane of the coil. It follows from Eq.~(\ref{e9.22}) that 
\begin{equation}\label{e9.27}
\tau = \frac{{\cal E}\, I}{\omega},
\end{equation}
since ${\cal E} = N\,B_\parallel\,A\,\omega$. 
An external
torque which is equal and opposite to the breaking torque must be applied to
the coil if it is to rotate {\em uniformly}, as assumed
above. The rate $P$ at which this external torque does work is equal to the
product of the torque $\tau$ and the angular velocity $\omega$ of the coil. Thus,
\begin{equation}
P= \tau\,\omega = {\cal E}\, I.
\end{equation}
Not surprisingly, the rate at which the external torque performs works exactly matches the
rate ${\cal E}\,I$ at which electrical energy is generated in the circuit comprising the rotating coil and the load.

Equations~(\ref{e9.22}), (\ref{e9.25}), and (\ref{e9.27}) yield
\begin{equation}\label{e9.29}
\tau = \tau_{\rm max}\,\sin^2(2\pi \,f\,t),
\end{equation}
where $\tau_{\rm max} = ({\cal E}_{\rm max})^2/(2\pi\,f\,R)$. Figure~\ref{f9.7} shows the breaking 
torque $\tau$ plotted as a function of time $t$, according to
Eq.~(\ref{e9.29}). It can be seen that the
torque is always of the same sign ({\em i.e.}, it always acts in the same
direction, so as to continually oppose the
rotation of the coil), but  is not constant
in time. Instead, it   {\em pulsates}\/  periodically with period $T$. The breaking
torque attains its maximum value whenever the plane of the coil is parallel to the
plane of the magnetic field, and is zero whenever the plane of the coil is perpendicular
to the magnetic field. It is clear that the external torque needed
to keep the coil rotating at a constant angular velocity must also pulsate
in time with period $T$. A constant external torque would give rise to  a non-uniformly rotating
coil, and, hence, to an alternating emf which varies with  time in a more
complicated manner than $\sin(2\pi\, f\, t)$.

\begin{figure}
\epsfysize=2.5in
\centerline{\epsffile{Chapter09/fig9.07.eps}}
\caption{\em The braking torque in a steadily rotating AC generator.}\label{f9.7}
\end{figure}

Virtually all commercial power stations generate electricity using AC generators. 
The external power needed to turn the generating coil is usually supplied by
a steam turbine (steam blasting against fan-like blades which are
forced into rotation). Water is vaporized to produce
high pressure 
steam  by burning coal, or by using the energy released  inside a nuclear 
reactor. 
Of course, in hydroelectric power stations, the power needed  
to turn the generator coil is supplied by a water turbine (which is similar
to a steam turbine, except that falling water plays the role of the steam). 
Recently, a new type of power station has been developed in which the
power needed to rotate the generating coil is supplied by a gas turbine
(basically, a large jet engine which burns natural gas). In the United States
and Canada, the alternating emf generated by power stations oscillates at
$f=60$\,Hz, which means that the
generator coils in power stations rotate exactly
sixty times a second. In Europe, and much of the rest of the world, the oscillation frequency
of commercially generated electricity is $f=50$\,Hz. 

\subsection{The Direct Current Generator}
Most common electrical appliances ({\em e.g.}, electric light-bulbs, and electric
heating elements) work fine on AC electrical power. However, there are some
situations in which DC  power is preferable. For instance, small electric
motors ({\em e.g.}, those which power food mixers and vacuum cleaners) work very well on AC
electricity, but very large electric motors ({\em e.g.}, those
which power subway trains) generally work much better on DC electricity. Let us
investigate how DC electricity can be generated. 

\begin{figure}
\epsfysize=2.5in
\centerline{\epsffile{Chapter09/fig9.08.eps}}
\caption{\em A split-ring commutator.}\label{f9.8}
\end{figure}

A simple DC generator consists of the same basic elements as a simple
AC generator: {\em i.e.}, a multi-turn coil rotating uniformly in a magnetic
field. The main difference between a DC generator and an AC generator lies
in the manner in which the rotating coil is connected to the external circuit 
containing the load. In an AC generator, both ends of the coil are connected
to separate slip-rings which co-rotate with the coil, and are connected to
the external circuit via wire brushes. In this manner, the emf ${\cal E}_{\rm ext}$
seen by the external circuit is always the same as the emf ${\cal E}$ 
 generated around the rotating
coil. In a DC generator, the two ends of the coil are attached to different halves
of a single split-ring which co-rotates with the coil. The split-ring is connected
 to
the external circuit by means of metal brushes---see Fig.~\ref{f9.8}. 
This combination of a rotating split-ring and stationary metal brushes 
is called a {\em commutator}. The purpose of the commutator is to ensure that
the emf ${\cal E}_{\rm ext}$
seen by the external circuit 
is equal to the emf  ${\cal E}$ 
 generated around the rotating
coil for {\em half}\/ the rotation period, but is equal to minus this emf for the
other half (since the connection between the external circuit and the rotating
coil is reversed by the commutator every half-period of rotation). The
positions of the metal brushes can be adjusted such that the connection between
the rotating coil and the external circuit reverses whenever the emf
${\cal E}$ generated around the coil goes through zero. In this special case,
the emf seen in the external circuit is simply
\begin{equation}
{\cal E}_{\rm ext} = |{\cal E}| = {\cal E}_{\rm max}\,|\sin (2\pi\, f\, t)|.
\end{equation}
Figure~\ref{f9.9} shows ${\cal E}_{\rm ext}$ plotted as a function of
time, according to the above formula.  The variation of the emf with time is
very similar to that of an AC generator, except that whenever the AC generator
would produce a negative emf the commutator in the DC generator reverses
the polarity of the coil with respect to the external circuit, so that the negative
half of the AC signal is reversed and made positive. The result is a bumpy
direct emf which rises and falls but never changes direction. This type of
pulsating emf can be smoothed out by using more than one coil rotating about the
same axis, or by other electrical techniques, to give a good imitation of the
direct current delivered by a battery. The {\em alternator}\/ in a car
({\em i.e.}, the DC generator which recharges the battery) is a common example
of a DC generator
of the type discussed above. Of course, in an alternator, the external torque needed to rotate
the coil is provided by the engine of the car.

\begin{figure}
\epsfysize=2.5in
\centerline{\epsffile{Chapter09/fig9.09.eps}}
\caption{\em Emf generated in a steadily rotating  DC
generator.}\label{f9.9}
\end{figure}

\subsection{The Alternating Current Motor}
The first electric dynamo was constructed in 1831 by Michael Faraday. 
An electric  dynamo is, of course, a device which transforms mechanical energy into
electrical energy. An electric motor, on the other hand, is a device which
transforms electrical energy into mechanical energy. In other words, an electric
motor is an electric dynamo run in {\em reverse}. It took a surprisingly long time for
scientists in the nineteenth century to realize this. In fact, the message only
really sank home after a fortuitous accident during  the 1873 Vienna World Exposition. 
A large hall was filled with modern gadgets. One of these gadgets,
a steam engine driven
dynamo, was producing electric power when a workman unwittingly connected the
output leads from another dynamo to the energized circuit. Almost immediately,
the latter dynamo started to whirl around at great speed. The dynamo was, 
in effect, transformed into an electric motor. 

An AC electric motor consists of the same basic elements as an
AC electric generator: {\em i.e.}, a multi-turn coil which is free
to rotate in a constant magnetic field. 
Furthermore, the rotating coil is connected to the
external circuit in just the same manner as in an AC generator: {\em i.e.}, via two
slip-rings attached to metal brushes. Suppose that an external voltage
source of emf $V$ is connected across the motor. It is assumed that $V$
is an {\em alternating}\/ emf, so that
\begin{equation}
V = V_{\rm max} \,\sin (2\pi\, f\, t),
\end{equation}
where $V_{\rm max}$ is the peak voltage, and $f$ is the alternation frequency. 
Such an emf could be obtained from an AC generator, or, more simply, from
the domestic mains supply. For the case of the mains, $V_{\rm max} = 110$\,V
and $f=60$\,Hz in the U.S.\ and Canada, whereas $V_{\rm max} = 220$\,V and
$f=50$\,Hz in Europe and Asia. The external emf drives an alternating current
\begin{equation}
I = I_{\rm max} \, \sin (2\pi\, f\, t)
\end{equation}
around the external circuit, and through the motor. As this current flows around the
coil, the magnetic field exerts a torque on the coil, which causes it
to rotate. The motor eventually  attains a steady-state in which
the rotation frequency of the coil matches the alternation frequency of the
external emf. In other words, the steady-state rotation frequency of the coil is
$f$. Now a coil rotating in a magnetic field generates an emf ${\cal E}$. 
It is easily demonstrated that this emf acts to {\em oppose}\/ the circulation of the
current around the coil: {\em i.e.}, the induced emf acts in the opposite
direction to the external emf. For an $N$-turn  coil of cross-sectional area $A,$
rotating with frequency $f$ in a magnetic field $B$, the back-emf ${\cal E}$
is given by
\begin{equation}
{\cal E} = {\cal E}_{\rm max}\,\sin (2\pi\, f\, t),
\end{equation}
where
\begin{equation}\label{e9.32}
{\cal E}_{\rm max} = 2\pi\,N\,B\,A\,f,
\end{equation}
 and use has been made of the results
of Sect.~\ref{s9.6}. 

\begin{figure}
\epsfysize=3in
\centerline{\epsffile{Chapter09/fig9.10.eps}}
\caption{\em Circuit diagram for an AC motor connected to an external
AC emf source.}\label{f9.10}
\end{figure}

Figure~\ref{f9.10} shows the circuit in question. A circle with a
wavy line inside is the conventional way of indicating an AC voltage source. 
The motor is modeled as a resistor $R$, which represents the internal
resistance of the motor, in series with the back-emf ${\cal E}$. Of course,
the back-emf acts in the opposite direction to the external emf $V$. 
Application of Ohm's law around the circuit gives
\begin{equation}
V = I\, R + {\cal E},
\end{equation}
or
\begin{equation}
V_{\rm max}\, \sin (2\pi\, f \,t) = I_{\rm max} \,R\,\sin (2\pi\, f\, t) + {\cal E}_{\rm max}\,
\sin (2\pi\, f\, t),
\end{equation}
which reduces to
\begin{equation}
V_{\rm max} = I_{\rm  max} \,R + {\cal E}_{\rm max}.
\end{equation}

The rate $P$ at which the motor {\em gains}\/ electrical energy from the external
circuit is given by
\begin{equation}\label{e9.37}
P = {\cal E}\,I = P_{\rm max}\, \sin^2(2\pi\, f\, t),
\end{equation}
where
\begin{equation}
P_{\rm max} = {\cal E}_{\rm max}\, I_{\rm max} = \frac{ {\cal E}_{\rm  max}\,
(V_{\rm max} - {\cal E}_{\rm max})}{R}.
\end{equation}
By conservation of energy, $P$ is also the rate at which the motor performs
mechanical work. Note that the rate at which the motor does mechanical
work is not constant in time, but, instead, pulsates at the rotation frequency
of the coil. It is possible to construct a motor which performs work
at a more uniform rate by employing more than one coil rotating about the
same axis. 

As long as $V_{\rm max} > {\cal E}_{\rm max}$, the rate at which the
motor performs mechanical work is positive ({\em i.e.}, the motor 
does useful work). However, if $V_{\rm max} < {\cal E}_{\rm max}$
then the rate at which the motor performs work becomes negative. This means that
we must do mechanical work on the motor in order to keep it rotating, which
is another way of saying that the motor does not do useful work.
Clearly, in order for an AC motor to do useful work, the external emf $V$ must
be able to overcome the back-emf ${\cal E}$ induced in the motor ({\em i.e.},
$V_{\rm max} > {\cal E}_{\rm max}$).


\subsection{The Direct Current Motor}
In steady-state, an AC motor always
rotates at the alternation frequency of its power supply. 
Thus, an AC motor powered by the domestic  mains supply  rotates at 60\,Hz in the
U.S.\, and Canada, and at 50\,Hz in Europe and Asia. Suppose, however, that
we require a {\em variable speed}\/ electric motor. We could always use
an AC motor driven  by a variable frequency AC power supply, but such  power
supplies are very expensive. A far cheaper
alternative is to use a DC motor driven  by a DC power supply. Let us
investigate DC motors.

A DC motor consists of the same basic elements as a DC electric generator:
{\em i.e.}, a multi-turn coil which is free to rotate in a constant magnetic
field. Furthermore, the rotating coil is connected to the external circuit
in just the same manner as in a DC generator: {\em i.e.}, via a split-ring
commutator which reverses the polarity of the coil with respect to
the external circuit whenever the coil passes through the plane perpendicular
to the direction of the  magnetic field. Suppose that an
external DC voltage source 
({\em e.g.}, a battery, or a DC generator) of emf $V$ is connected across the motor.
The voltage source drives a steady current $I$ around the external
circuit, and through the motor. As the current flows around the coil,
the magnetic field exerts a torque  on the coil, which causes it to rotate.
Let us suppose that the motor eventually attains  a steady-state rotation
frequency $f$. The rotating coil generates a back-emf ${\cal E}$ whose
magnitude is directly proportional to the frequency of rotation [see Eq.~(\ref{e9.32})]. 

\begin{figure}
\epsfysize=3in
\centerline{\epsffile{Chapter09/fig9.11.eps}}
\caption{\em Circuit diagram for an DC motor connected to an external
DC emf source.}\label{f9.11}
\end{figure}

Figure~\ref{f9.11} shows the circuit in question. The motor is modeled
as a resistor $R$, which represents the internal resistance of the
motor, in series with the back-emf ${\cal E}$. Of course, the back-emf
acts in the opposite direction to the external emf $V$. Application of Ohm's
law around the circuit gives
\begin{equation}
V = I\,R + {\cal E},
\end{equation}
which yields
\begin{equation}\label{e9.40}
I = \frac{V-{\cal E}}{R}.
\end{equation}
The rate at which the motor performs mechanical work is
\begin{equation}\label{e9.41}
P = {\cal E} \,I = \frac{{\cal E} \,(V - {\cal E})}{R}.
\end{equation}

Suppose that a DC  motor is subject to a  light external load, so that it only
has to perform mechanical work at a relatively low rate. In this case,
the motor will spin up until its back-emf ${\cal E}$ is slightly less
than the external emf $V$, so that very little current flows through the
motor [according to Eq.~(\ref{e9.40})], and, hence, the mechanical 
power output of the motor
is relatively low [according to Eq.~(\ref{e9.41})]. If the load on the motor is
increased then the motor will slow down, so that its 
back-emf is reduced, the current flowing through the motor is increased, and,
hence, the mechanical power output of the motor is raised until
it matches the new load. Note that the current flowing
through a DC motor is generally limited  by the back-emf, rather than the
internal resistance of the motor. In fact, conventional DC motors are designed
on the assumption that the back-emf will always limit  the current flowing through
the motor to a relatively small value. If the motor jams, so that the coil stops
rotating and the back-emf falls to zero, then the current $I=V/R$ which
flows through the motor is generally so large that it will burn out the motor
if allowed to flow for any appreciable length of time. For this reason, 
the  power to an electric motor
should always be shut off immediately if the motor jams. When a DC motor
is started up, the coil does not initially spin fast enough to
generate a substantial back-emf. Thus, there is a short time period,
just after the motor is switched on, in which the motor draws a relatively
large current from its power supply. This explains why the lights 
in a house sometimes
dim transiently when a large motor, such as 
an air conditioner motor, is switched on.

Suppose that a DC motor is subject to a constant, but relatively light, load. 
As mentioned above, the motor will spin up until its back emf almost
matches the external emf. If the external emf is increased then the motor
will spin up further, until its back-emf  matches the new external
emf. Likewise, if the external emf is decreased then the motor will spin down.
It can be seen that
the rotation rate of a DC motor is controlled by the
emf of the DC power supply to which the motor is attached. The higher the emf, the
higher the  rate of rotation. 
Thus, it is relatively easy to vary the speed of a DC motor, unlike an
AC motor, which is essentially a fixed speed motor. 

\subsection{Worked Examples}
\subsection*{\em Example 9.1: Faraday's law}
{\em Question:} A plane circular loop of conducting wire of radius $r=10$\,cm which
possesses $N=15$ turns is placed in a uniform magnetic field. The direction
of the magnetic field makes an angle of $30^\circ$ with respect to the normal
direction to the loop. 
The magnetic field-strength $B$ is increased at a constant
rate from $B_1=1$\,T to $B_2=5$\,T in a time interval of ${\mit\Delta}t=10$\,s. What
is the emf generated around the loop? If the electrical resistance of the
loop is $R=15\,\Omega$, what current flows around the loop as the
magnetic field is increased? \\
~\\
\noindent{\em Answer:} The area of the loop is 
$$
A = \pi\,r^2 = \pi\,(0.1)^2 = 0.0314\,{\rm m}^2.
$$
The component of the magnetic field perpendicular to the loop is
$$
B_\perp = B\,\cos\theta = B\,\cos 30^\circ = 0.8660\,B,
$$
where $B$ is the magnetic field-strength. Thus, the initial magnetic
flux linking the loop is
$$
{\mit\Phi}_{B\,1} = N\,A\,B_1\,\cos\theta = (15)\,(0.0314)\,(1)\,(0.8660) = 0.408\,{\rm Wb}.
$$
Likewise, the final flux linking the loop is
$$
{\mit\Phi}_{B\,2} = N\,A\,B_2\,\cos\theta = 
(15)\,(0.0314)\,(5)\,(0.8660) = 2.039\,{\rm Wb}.
$$
The time rate of change of the flux is
$$
\frac{d{\mit\Phi}_B}{dt} = \frac{{\mit\Phi}_{B\,2}- 
{\mit\Phi}_{B\,1}}{{\mit\Delta}t} = \frac{(2.039-0.408)}{(10)}=0.163\,{\rm Wb\,s}^{-1}.
$$
Thus, the emf generated around the loop is
$$
{\cal E} = \frac{d{\mit\Phi}_B}{dt} = 0.163\,{\rm V}.
$$
Note, incidentally, that one weber per second is equivalent to one volt. 

According to Ohm's law, the current which flows around the loop in response to the
emf is
$$
I = \frac{{\cal E}}{R} = \frac{(0.163)}{(15)} = 0.011\,{\rm A}.
$$


\subsection*{\em Worked Example 2: Lenz's law}
{\em Question:} A long solenoid with an air core has $n_1=400$ turns per meter
and a cross-sectional area of $A_1= 10\,{\rm cm}^2$. The current $I_1$ flowing 
around
the solenoid increases from 0 to $50\,{\rm A}$ in $2.0\,{\rm s}$. 
A plane loop of wire consisting of $N_2=10$ turns, which is of cross-sectional area
$A_2= 100\,{\rm cm}^2$ and resistance $R_2=0.050\,\Omega$, is placed around the
solenoid close to its centre. The loop is orientated such that
it lies in the plane perpendicular to the axis
of the solenoid. 
What is the magnitude ${\cal E}_2$ of the emf 
induced 
in the coil? What current $I_2$ does does this emf drive around the coil? 
Does this current circulate in the same direction as the current flowing in
the solenoid, or in the opposite direction?\\
~\\
\noindent{\em Answer:} We must, first of all, calculate the magnetic
flux linking the coil. The magnetic field is confined to the region
inside the solenoid (the field generated outside a {\em long}\/ solenoid
is essentially negligible). The magnetic field runs along the axis of the
solenoid, so it is directed perpendicular to the plane of the coil. Thus,
the magnetic flux linking a single turn of the coil is the product of the
area $A_1$ of the magnetic-field-containing region and the strength $B$
of the perpendicular field. Note that, in this case, the magnetic flux does
not depend on the area $A_2$ of the coil, as long as the magnetic-field-containing
region lies completely within the coil. 
 The magnetic flux ${\mit\Phi}_B$ linking the
whole coil is the flux linking a single turn times the number $N_2$ of
turns in the coil. Thus,
$$
{\mit\Phi}_B = N_2\,A_1\,B.
$$
Now, the magnitude of the magnetic field generated by the solenoid
is given by (see Sect.~\ref{s8.8}) 
$$
B = \mu_0\,n_1\,I_1,
$$
so the magnetic flux linking the coil can be written
$$
{\mit\Phi}_B = N_2\,A_1\,\mu_0\,n_1\,I_1.
$$
This magnetic flux increases because the current $I_1$ flowing in the
solenoid increases.
Thus, the time rate of change of the magnetic flux is given by
\begin{eqnarray}
\frac{d{\mit\Phi}_B}{dt} = 
 N_2\,A_1\,\mu_0\,n_1\,\frac{dI_1}{dt}
&=& (10)\,(10\times10^{-4})\,(4\pi\times 10^{-7}) \,(400)\,\frac{(50)}{(2)}
\nonumber\\[0.5ex]& =& 
1.26\times 10^{-4}\,{\rm Wb\,\,s}^{-1}.\nonumber
\end{eqnarray}
By Faraday's induction law, the emf generated around the coil is
$$
{\cal E}_2 = -\frac{ d{\mit\Phi}_B}{dt} =
 -1.26\times 10^{-4}\,{\rm V}.
$$

Ohm's law gives
$$
I_2 = \frac{{\cal E}_2}{R_2} = \frac{(-1.26\times 10^{-4})}{(0.050)} = -2.6\,{\rm mA},
$$
as the current induced in the coil. 

According to Lenz's law, the current induced  in the coil is such as to 
oppose the increase in the  magnetic flux linking  the coil. Thus, the current
in the coil must circulate in the {\em opposite} direction to the current  in
the solenoid, so that the magnetic field generated by the the former current, in the middle of
the coil, is oppositely directed to that generated by the latter current.
The fact that the current $I_2$ in the above formula is {\em negative} 
is indicative of the fact  that
this current 
runs in the opposite direction to the current flowing around the solenoid. 


\subsection*{\em Worked Example 3: Motional emf}
{\em Question:} Consider the circuit described in Sect.~\ref{s9.4}.
Suppose that the length of the moving rod is $l=0.2$\,m,  its speed
is $v=0.1\,{\rm m\,s}^{-1}$,  the magnetic field-strength is $B=1.0$\,T 
(the field is directed into the page---see Fig.~\ref{f9.3}), and the resistance of the circuit is
$R=0.020\,\Omega$. What is the emf generated around the circuit? What current
flows around the circuit? What is the magnitude and direction of the force
acting on the moving rod due to the fact that a current is flowing along it?
What is the rate at which work must be performed on the rod in order to
keep it moving at constant velocity against this force? What is the rate at
which electrical energy is generated? What is the rate at which energy is converted
into heat due to the resistivity of the circuit?\\
~\\
{\em Answer:} The emf is generated by the motion of the rod. According to
Eq.~(\ref{e9.15}), the magnitude of the motional emf is
$$
{\cal E} = B\,l\,v = (1)\,(0.2)\,(0.1) = 0.020\,{\rm V}.
$$
The emf acts in the anti-clockwise direction in Fig.~\ref{f9.3}.

The anti-clockwise current driven around the circuit by the motional
emf follows from Ohm's law:
$$
I = \frac{\cal E}{R} = \frac{(0.020)}{(0.020)} = 1.0\,{\rm A}.
$$

Since the rod carries a current $I$ which flows perpendicular to a magnetic
field $B$, the force per unit length acting on the rod is 
$F= I\,B$ (see Sect.~\ref{s8.2}). Thus, the total force acting on the rod is of magnitude
$$
f = I\,B\,l = (1)\,(1)\,(0.2) = 0.20\,{\rm N}.
$$
This force is directed parallel to the vector ${\bf I}\times {\bf B}$.
It follows that the force is to the left in Fig.~\ref{f9.3}. In other words, the force {\em opposes} the motion producing the
emf.

In order to keep the rod moving at a constant velocity, some external
agent must apply a force to the rod which is equal and opposite to
the magnetic force described above. Thus, the externally applied force acts to
the right. 
The rate $P$ at which work is done
by this force is the product of the force and the velocity of the rod in the
direction of this force. Thus,
$$
P = f\,v = (0.20)\,(0.10) = 0.020\,{\rm W}.
$$

Every charge $q$ which circulates around the circuit in the anti-clockwise
direction acquires the energy $q\,{\cal E}$. The amount of charge per unit
time which circulates around the circuit is, by definition, equal to the
current $I$ flowing around the circuit. Thus, the rate at which electric charges
acquire energy in the circuit is 
$$
P = {\cal E}\,I = (0.020)\,(1) = 0.020\,{\rm W}.
$$
Now, the rate at which electric charges acquire energy in the circuit is equal
to the rate at which mechanical work is done on the rod  by the
external force, as must be the case if energy is to
be conserved. Thus, we can think of this circuit as constituting a
primitive  generator which transforms mechanical into electrical
energy. 

The rate at which electrical energy is converted into heat energy in the
circuit is
$$
P = I^2\,R = (1)\,(1)\,(0.020) = 0.020\,{\rm W}.
$$
Thus, all of the mechanical work done on the rod eventually ends up as heat dissipated
in the circuit.


\subsection*{\em Worked Example 4: AC generators}
{\em Question:} A simple AC generator consists of an $N=10$ turn coil of
area $A= 1200\,{\rm cm}^2$ which rotates at a constant frequency of
$f=60$\,Hz in a $B=0.40$\,T magnetic field. What is the peak emf of the device?\\
~\\
\noindent{\em Answer:} The peak emf ${\cal E}_{\rm max}$ is given by [see Eq.~(\ref{e9.24})]
$$
{\cal E}_{\rm max} = 2\pi\,N\,B\,A\,f = (6.283)\,(10)\,(0.40)\,(0.12)\,(60) = 181 \,{\rm V}.
$$

\subsection*{\em Worked Example 5: AC motors}
{\em Question:} An AC motor has an internal resistance of $R=4.0\,\Omega$. When
powered by a $50\,{\rm Hz}$ AC supply of peak voltage $V=120\,{\rm V}$ it draws a peak current of
$I=5.0\,{\rm A}$. What is the peak back-emf produced by the motor? What  is the peak power
delivered to the motor by the AC supply? What is the peak rate of energy loss as heat in the 
motor? What is the peak useful power  produced by the motor? What is the efficiency
({\em i.e.}, the ratio of the peak useful power output to the peak power delivered)
of such a motor?\\
~\\
\noindent{\em Answer:} If $V$ is the peak applied voltage, and ${\cal E}$ 
the peak back-emf, then the peak applied voltage must equal the sum of the peak voltage
drops across the motor, or $V = {\cal E} + I\,R$. It follows that
$$
{\cal E} = V - I\,R = (120) - (5.0)\,(4.0) = 100\,{\rm V}.
$$
The peak power delivered by the AC supply is
$$
P_1 = V\,I = (120)\,(5.0) = 600\,{\rm W}.
$$
Energy is lost as heat in the motor at the peak rate
$$
P_2 = I^2\,R = (5.0)^2\,(4.0) = 100\,{\rm W}.
$$
The peak useful power produced by the motor is the difference between the
peak power supplied to the motor and the peak power dissipated as heat:
$$
P = P_1 - P_2 = (600) - (100) = 500\,{\rm W}.
$$
The peak useful power is also given by the product of the peak back-emf and
the peak current flowing through the motor [see Eq.~(\ref{e9.37})],
$$
P= {\cal E}\,I = (100)\,(5.0) = 500\,{\rm W}.
$$
The efficiency $\eta$ is the ratio of the peak useful power output of the motor
to the peak  power supplied, or
$$
\eta = \frac{P}{P_1} = \frac{500}{600} = 0.83 = 83\,\%.
$$
