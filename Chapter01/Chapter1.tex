\section{Introduction}
These lecture notes are designed to accompany a lower-division college survey
course, for Physics majors, covering Electricity, Magnetism, and Optics. Students are expected
to be familiar with ordinary calculus and elementary mechanics. A brief survey of
the vector calculus needed in the course is provided.

What is an electric field? What is lightning,  and what determines where it strikes? How do
simple electrical circuits work? What do we mean by the terms voltage, current, resistance, capacitance, and inductance, when talking  about such circuits? What is the difference between AC and DC electricity?
How do electric motors and generators
work? How does the national electric grid, which supplies 
electricity to homes and businesses, work? What is a magnetic field? Why do magnets point North?
What is the difference between a bar magnet and an electromagnet?
What exactly is the relationship
between electric and magnetic fields and light? How are rainbows formed?
How are images formed by mirrors and lenses? What causes the vivid
colors which sometimes appear in soap bubbles and oily puddles?

These, and other, questions regarding electricity and magnetism will be answered in this course. Starting from simple electrical, magnetic, and optical
effects which have been known for thousands of years, we shall trace
the gradual development of electromagnetic theory,  illustrating
this fascinating story with explanations of natural electromagnetic phenomena and everyday electrical
devices. By the end of the course, students should have a thorough 
understanding of the basic concepts of Electricity, Magnetism, and Optics,
and should also be able to perform a variety of simple calculations
involving these concepts.
