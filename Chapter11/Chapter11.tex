\section{Electromagnetic Waves}\label{s11}
\subsection{Maxwell's Equations}
In the latter half of the nineteenth century, the Scottish physicist
James Clerk Maxwell demonstrated that all  previously
established experimental facts regarding electric and magnetic
fields could be summed up in just {\em four}\/ equations. Nowadays,
these equations are generally known as {\em Maxwell's equations}.

The first equation is simply  Gauss'
law (see Sect.~\ref{sgauss}). This
equation describes how electric charges generate  electric fields.
Gauss' law states that:
\begin{quote}
{\sf The electric flux through any closed surface is equal to the total
charge enclosed by the surface, divided by $\epsilon_0$.}
\end{quote}
This can be written mathematically  as
\begin{equation}
\oint_S {\bf E}\cdot d{\bf S} = \frac{Q}{\epsilon_0},
\end{equation}
where $S$ is a closed surface enclosing the charge $Q$. The above
expression can also be written
\begin{equation}\label{em1}
\oint_S {\bf E}\cdot d{\bf S}= \frac{1}{\epsilon_0}\int_V \rho\,dV,
\end{equation}
where $V$ is a volume bounded by the surface $S$, and $\rho$
is the {\em charge density}: {\em i.e.}, the electric charge
per unit volume.

The second equation is the magnetic equivalent of Gauss' law
(see Sect.~\ref{s8.10}). This equation describes how the non-existence of
magnetic monopoles causes magnetic field-lines to form closed loops. Gauss' law for magnetic fields states that:
\begin{quote}
{\sf The magnetic flux through any closed surface is equal to zero.}
\end{quote}
This can be written mathematically as 
\begin{equation}\label{em2}
\int_S {\bf B} \cdot d{\bf S} = 0,
\end{equation}
where $S$ is a closed surface.

The third equation is  Faraday's law (see Sect.~\ref{s9.3}). This equation describes
how changing magnetic fields generate  electric fields. Faraday's law
states that:
\begin{quote}
{\sf The line integral of the electric field around any closed loop
is equal to minus the time rate of change of the magnetic flux
through the loop.}
\end{quote}
This can be written mathematically as
\begin{equation}\label{em3}
\oint_C {\bf E}\cdot d {\bf S} = -\frac{d}{dt}\int_{S'} {\bf B}\cdot d {\bf S}',
\end{equation}
where $S'$ is a surface attached to the loop $C$.

The fourth, and final, equation is Amp\`{e}re's circuital law (see Sect.~\ref{s8.7}). 
This equation describes how  electric currents generates  magnetic fields.
Amp\`{e}re's circuital law states that:
\begin{quote}
{\sf The line integral of the magnetic field around any closed loop is
equal to $\mu_0$ times the algebraic sum of the currents which pass
through the loop}.
\end{quote}
This can be written mathematically as
\begin{equation}
\oint_C {\bf B}\cdot d{\bf r} = \mu_0\,I,
\end{equation}
where $I$ is the net current flowing through loop $C$. This equation
can also be written 
\begin{equation}\label{em4}
\oint_C {\bf B}\cdot d{\bf r} = \mu_0\int_{S'}{\bf j}\cdot d{\bf S}',
\end{equation}
where $S'$ is a surface attached to the loop $C$, and ${\bf j}$ is the
{\em current density}: {\em i.e.}, the electrical current per unit area.

When Maxwell first wrote Eqs.~(\ref{em1}), (\ref{em2}), (\ref{em3}), 
and (\ref{em4}) he was basically 
trying to summarize everything 
which was known at the time about electric and magnetic fields
in mathematical form. However, the more Maxwell looked at his
equations, the more convinced he became that they were incomplete.
Eventually, he proposed adding a new term, called the {\em displacement
current}, to the right-hand side of his fourth equation. In fact, Maxwell was able
to show that (\ref{em1}), (\ref{em2}), (\ref{em3}), 
and (\ref{em4})  are {\em mathematically inconsistent}\/ unless
the displacement current term is added  to Eq.~(\ref{em4}). Unfortunately,
Maxwell's 
demonstration of this fact  requires some advanced mathematical
techniques  which lie well beyond the scope of this course. 
In the following,
we shall give a highly simplified version of his derivation of the
missing term. 

\begin{figure}[h]
\epsfysize=2in
\centerline{\epsffile{Chapter11/fig11.1.eps}}
\caption{\em Circuit containing a charging capacitor.}\label{f11.1}
\end{figure}

Consider a circuit
consisting of a parallel plate capacitor of capacitance $C$ in series with a
resistance  $R$ and an steady emf $V$, as shown in Fig.~\ref{f11.1}. Let $A$ be the area of
the capacitor plates, and let $d$ be their separation. Suppose that
the switch is closed at $t=0$. The current $i$ flowing around the circuit
starts from an initial value of $I=V/R$, and gradually decays to zero
on the RC time of the circuit (see Sect.~\ref{s10.6}). 
Simultaneously, the charge $q$ on the
positive plates of the capacitor starts from zero, and gradually increases
to a final value of $Q=C\,V$. As the charge $q$ varies, so does the
potential
difference $v$ between the capacitor plates, since $v=q/C$. 

The electric field in the region between the plates is approximately uniform,
directed perpendicular to the plates (running from the positively charged plate
to the negatively charged plate), and is of magnitude $E= v/d$. It follows
that
\begin{equation}
q = C\,v = C\,d\,E.
\end{equation}
In a time interval $dt$, the charge on the positive
plate of the capacitor increases by an amount
$dq = C\,d\,d E$,
 where  $dE$ is the
corresponding increase in the electric field-strength between the plates.
Note that both $C$ and $d$ are time-independent quantities. It follows
that 
\begin{equation}
\frac{dq}{d t} = C\,d\,\frac{dE}{dt}.
\end{equation}
Now, $dq/d t$ is numerically equal to the
instantaneous current $i$ flowing around the circuit (since all of the
charge which flows around the circuit must accumulate on the plates of
the capacitor). Also, $C=\epsilon_0\, A/d$ for a parallel plate capacitor.
Hence, we can write
\begin{equation}
i = \frac{dq}{d t} = C\,d\,\frac{d E}{dt} =\epsilon_0\,A\, \frac{d E}{d t}.
\end{equation}
Since the electric field $E$ is normal to the area $A$, we can also write
\begin{equation}\label{e11.8}
i =\epsilon_0\,A\, \frac{d E_\perp}{dt}.
\end{equation}

Equation~(\ref{e11.8}) relates the instantaneous current flowing around the
circuit to the time rate of change of the electric field between the
capacitor plates. According to Eq.~(\ref{em4}), the current flowing around the 
circuit generates a magnetic field.  This field circulates around the 
current carrying wires
connecting the various components of the circuit. However, since
there is no actual current flowing between the 
plates of the capacitor,  no magnetic field is generated in this region,
according to Eq.~(\ref{em4}). 
Maxwell demonstrated that for reasons of  mathematical self-consistency there
must, in fact,  be a magnetic field generated in the
region between the plates of the capacitor.
Furthermore, this magnetic field must be the same as that which
would be generated if the
current $i$ ({\em i.e.}, the same current as that which
flows around  the rest of the circuit)
flowed between the plates. Of course, there is no actual current flowing between
the plates. However, there is a changing electric field. Maxwell argued that a
changing electric field constitutes an effective current ({\em i.e.}, it
generates a magnetic field in just the same manner as an actual current). 
For historical reasons (which do not particularly interest us at the moment), Maxwell 
called this type of current a {\em displacement current}. 
Since the displacement
current $I_D$ flowing between the plates of the capacitor must equal the
current $i$ flowing around the rest of the circuit,  it follows from Eq.~(\ref{e11.8})
that
\begin{equation}\label{e11.9}
I_D = \epsilon_0\,A\, \frac{dE_\perp}{dt}.
\end{equation}

Equation~(\ref{e11.9}) was derived for the special case of the changing electric
field generated in the region
between the plates of a charging parallel plate capacitor. Nevertheless, this equation turns out to be completely general. Note that
$A\,E_\perp$ is equal to the electric flux ${\mit\Phi}_E$ between the plates of
the capacitor. Thus, the most general expression for the displacement current
passing through some closed loop is
\begin{equation}
I_D = \epsilon_0\,\frac{d{\mit\Phi}_E}{d t},
\end{equation}
where ${\mit\Phi}_E$ is the electric flux through the loop. 

According to Maxwell's argument, a displacement current is just as effective at generating a
magnetic field as a real current. Thus, we need to modify Amp\`{e}re's
circuital law to take displacement currents into account. The modified
law, which is known as the {\em Amp\`{e}re-Maxwell law}, is written
\begin{quote}
{\sf The line integral of the electric field around any closed loop is
equal to $\mu_0$ times the algebraic sum of the actual currents
and which pass
through the loop plus $\mu_0$ times the displacement current 
passing through the
loop.}
\end{quote}
This can be written mathematically as
\begin{equation}
\oint_C {\bf B}\cdot d{\bf r} = \mu_0\,(I + I_D),
\end{equation}
where $C$ is a loop through which the electric current $I$ and the 
displacement current $I_D$ pass. 
This equation can also be written
\begin{equation}\label{em5}
\oint_C {\bf B}\cdot d{\bf r} = \mu_0\int_{S'} {\bf j} \cdot d{\bf S}'
+ \mu_0\,\epsilon_0\,\frac{d}{dt}\int_{S'} {\bf E}\cdot d{\bf S}',
\end{equation}
where $S'$ is a surface attached to the loop $C$.

Equations (\ref{em1}), (\ref{em2}), (\ref{em3}), and (\ref{em5}) are known collectively as
{\em Maxwell's equations}. They constitute a complete and mathematically
self-consistent description of the behaviour of electric and magnetic
fields. 

\subsection{Electromagnetic Waves}
One of the first 
things that Maxwell did with his four equations, once he had obtained
them, was to look for wave-like solutions. 
Maxwell knew that the wave-like solutions of the equations of gas dynamics
correspond to sound waves, and the wave-like solutions of the
equations of fluid dynamics correspond to gravity waves in water, so
he reasoned that if his equations possessed wave-like solutions then  these
would correspond to a completely
new type of wave, which he  called an {\em electromagnetic wave}. 


Maxwell was primarily interested in electromagnetic
waves which can propagate through a vacuum ({\em i.e.},
a region containing no charges or currents). 
Now, in a vacuum, Maxwell's equations
 reduce to
\begin{eqnarray}
\oint_S {\bf E}\cdot d{\bf S},&= & 0,\label{emv1}\\[0.5ex]
\oint_S {\bf B}\cdot d{\bf S} &=& 0,\label{emv2}\\[0.5ex]
\oint_C {\bf E}\cdot d{\bf r}&=& - \frac{d}{dt}\int_{S'}{\bf B}\cdot d{\bf S}',\label{emv3}
\\[0.5ex]
\oint_C {\bf B}\cdot d{\bf r} &=& \mu_0\,
\epsilon_0\,\frac{d}{dt}\int_{S'} {\bf E}\cdot d {\bf S}',\label{emv4}
\end{eqnarray}
where $S$ is a closed surface, and $S'$ a surface attached to some loop $C$.
Note that, with the addition of the displacement current term
on the right-hand side of Eq.~(\ref{emv4}), these equations
exhibit a nice symmetry between electric and magnetic fields. 
Unfortunately, Maxwell's mathematical proof that the above
equations possess
wave-like solutions lies well beyond the scope of this course. We can, nevertheless,
still write down these solutions, and comment on them.

Consider a plane electromagnetic wave propagating along the
$z$-axis. According to Maxwell's calculations, the electric and
magnetic fields associated with such a wave take the form
\begin{eqnarray}\label{e11.14}
E_x &=& E_0\,\cos[2\pi\,(z/\lambda - f\,t)],\\[0.5ex]
B_y &=& B_0\,\cos[2\pi\,(z/\lambda-f\,t)].
\end{eqnarray}
Note that the fields are periodic in both time and space. 
The
oscillation frequency (in hertz) of the fields at a given point
in space is $f$.
The equation
of a wave crest is 
\begin{equation}\label{e11.15}
\frac{z}{\lambda} - f\,t = N,
\end{equation}
where $N$ is an integer. It can be seen that the distance along the
$z$-axis between successive wave crests is given by $\lambda$. This
distance is conventionally termed the {\em wavelength}.
Note that each  wave crest {\em propagates} along  
the $z$-axis. In a time interval $dt$, the $N$th wave
crest moves a distance $d z = \lambda\,f\,d t$,
according to Eq.~(\ref{e11.15}). Hence, the velocity  $c = d z/d t$
with which the wave propagates along the $z$-axis is given by
\begin{equation}
c = f\,\lambda.
\end{equation}

Maxwell was able to establish that electromagnetic waves possess the
following properties:
\begin{enumerate}
\item The magnetic field oscillates {\em in phase} with the
electric
field. In other words, a wave maximum of the magnetic field always
coincides with a wave maximum of the electric field in both time and
space.
\item The electric field is always perpendicular to the magnetic field, and
both fields are directed at right-angles to the direction of propagation of
the wave. In fact, the wave propagates in the direction ${\bf E}\times{\bf B}$. 
Electromagnetic waves are clearly a type of {\em transverse wave}. 
\item For a $z$-directed wave, the electric field is free
to oscillate in {\em any}
direction which lies in the $x$-$y$ plane. The direction in which the
electric field oscillates is conventionally
termed the direction of {\em polarization}
of the wave. Thus, Eqs.~(\ref{e11.14}) represent a plane  electromagnetic wave which
propagates along the $z$-axis, and is polarized in the $x$-direction. 
\item The maximum amplitudes of the electric and the magnetic fields
are related via
\begin{equation}
E_0= c\,B_0.
\end{equation}
\item There is no constraint on the possible frequency or wavelength of
electromagnetic waves. However, the propagation velocity of
electromagnetic waves  is {\em fixed}, and takes the value
\begin{equation}\label{e11.18}
c = \frac{1}{\sqrt{\mu_0\,\epsilon_0}}.
\end{equation} 
\end{enumerate}

According to Eqs.~(\ref{emv3}) and (\ref{emv4}), a changing magnetic field generates an electric
field, and a changing electric field generates a magnetic field. 
Thus, we can think of the propagation of an electromagnetic field through a vacuum as due
to a kind of ``leap-frog'' effect, in which a changing electric field
generates a magnetic field, which, in turn, generates an electric field,
and so on. Note that the displacement current term in Eq.~(\ref{emv4}) plays
a crucial role in the propagation of electromagnetic waves. 
Indeed, without this term,
a changing electric field is incapable of generating
 a magnetic field, and so there can be no leap-frog effect. 
Electromagnetic waves have many properties in common with other
types of wave ({\em e.g.}, sound waves). However,  they are
unique in one respect: {\em i.e.}, they are able to propagate through a vacuum. All other types of
waves require some sort of medium through which to propagate. 

Maxwell deduced that the speed of propagation of an electromagnetic
wave through a vacuum is entirely 
determined by the constants $\mu_0$ and $\epsilon_0$
[see Eq. (\ref{e11.18})]. The former constant is related to the strength of the
magnetic field generated by a steady current, whereas the latter constant
is related to the strength of the electric field generated by a stationary
charge. The values of both constants
were well known in Maxwell's day. In modern units,
$\mu_0 = 4\pi\times 10^{-7}\,{\rm N}\,{\rm s}^2 \,{\rm C}^{-2}$ and
$\epsilon_0 = 8.854\times 10^{-12}\,{\rm C}^2\,{\rm N}^{-1} \,{\rm m}^{-2}$.
Thus, when Maxwell calculated the velocity of electromagnetic waves he obtained
\begin{equation}
c = \frac{1}{\sqrt{(4\pi\times 10^{-7})\,(8.854\times 10^{-12})} }= 2.998
\times 10^8 \,{\rm m}\,{\rm s}^{-1}.
\end{equation}
Now, Maxwell knew [from the work of Fizeau (1849) and Foucault (1850)] that
the velocity of light was about $3\times 10^8\,{\rm m}\,{\rm s}^{-1}$. 
The remarkable agreement between this experimentally determined velocity  and his 
theoretical prediction for the velocity of electromagnetic waves immediately
lead Maxwell to hypothesize that {\em light is a form of electromagnetic wave}.
Of course, this hypothesis turned out to be correct. 
We can still appreciate that Maxwell's achievement in identifying light as
a form of electromagnetic wave   was
quite
remarkable. After all, his  equations were derived from 
the results of bench-top
laboratory experiments involving charges, batteries, coils, and currents,
{\em etc.},  
which apparently
had nothing
whatsoever to do with light. 

Maxwell was able to make another remarkable prediction. The wavelength of
light was well known in the late nineteenth century from studies of diffraction
through slits, {\em etc}. 
Visible light actually occupies a surprisingly
narrow range of  wavelengths. The shortest wavelength blue light which is visible
 has a wavelength of $\lambda= 0.40$ microns (one micron is $10^{-6}$ meters).
The longest wavelength red light which is visible has a wavelength of
$\lambda= 0.76$ microns. However, there is nothing in Maxwell's
 analysis which suggested that
this particular range of wavelengths is special.
In principle,  electromagnetic waves
can have any wavelength. 
Maxwell concluded that visible light forms
 a small element  of a vast spectrum of
previously undiscovered
types of electromagnetic radiation. 

Since Maxwell's time, virtually all of the
non-visible parts of the electromagnetic spectrum have been observed. 
Table~\ref{t11.1}  gives a brief guide to the electromagnetic spectrum. 
Electromagnetic waves are of particular importance because they 
are our only source of information regarding the Universe around us. 
Radio waves and microwaves (which are comparatively
hard to scatter) have provided much of
our knowledge about the centre of the Galaxy. This is completely unobservable
in visible light, which is strongly scattered by interstellar gas and dust
lying in the galactic plane. 
For the same reason, the spiral arms of the Galaxy can only be mapped out using radio waves.
Infrared radiation is useful for detecting 
proto-stars which are not yet hot enough to emit visible radiation.
Of course, visible radiation is still the mainstay of astronomy. 
Satellite based ultraviolet observations have yielded invaluable insights into
the structure and distribution of distant galaxies. Finally, X-ray and $\gamma$-ray
astronomy usually concentrates on  exotic objects in the Galaxy such as pulsars
and supernova remnants. 
\begin{table}
\centering
\begin{tabular}{|l|c|}\hline
Radiation Type& Wavelength Range ($m$)\\[0.5ex]\hline
Gamma Rays & $<10^{-11}$ \\[0.5ex]
X-Rays & $10^{-11}$--$10^{-9}$ \\[0.5ex]
Ultraviolet & $10^{-9}$--$10^{-7}$\\[0.5ex]
Visible & $10^{-7}$--$10^{-6}$\\[0.5ex]
Infrared& $10^{-6}$--$10^{-4}$\\[0.5ex]
Microwave & $10^{-4}$--$10^{-1}$\\[0.5ex]
TV-FM & $10^{-1}$--$10^1$ \\[0.5ex]
Radio& $>10^1$\\\hline
\end{tabular}
\caption{\em The electromagnetic spectrum.}\label{t11.1}
\end{table}

\subsection{Effect of Dielectric Materials}\label{s11.3}
It turns out that electromagnetic waves cannot propagate very far through
a conducting medium before they are either absorbed or reflected. However,
electromagnetic waves are able to propagate through transparent dielectric media
without difficultly. The speed of electromagnetic waves propagating through
a dielectric medium is given by
\begin{equation}
c' = \frac{c}{\sqrt{K}},
\end{equation}
where $K$ is the dielectric constant of the medium in question, and
$c$  the velocity of light in a vacuum. Since $K>1$ for dielectric
materials, we conclude that:
\begin{quote}
{\sf The  velocity with which  electromagnetic
waves propagate 
through a dielectric medium is always less than the velocity
with which they propagate through a  vacuum. }
\end{quote}

\subsection{Energy in Electromagnetic Waves}
From Sect.~\ref{e10.3}, the energy stored per unit volume in an electromagnetic
wave is given by
\begin{equation}
w = \frac{\epsilon_0\,E^2}{2} + \frac{B^2}{2\,\mu_0}.
\end{equation}
Since, $B=E/c$, for an electromagnetic wave, and $c= 1/\sqrt{\mu_0\,\epsilon_0}$,
the above expression yields
\begin{equation}
w = \frac{\epsilon_0\,E^2}{2} + \frac{E^2}{2\,\mu_0\,c^2} =
\frac{\epsilon_0\,E^2}{2} + \frac{\epsilon_0\,E^2}{2},
\end{equation}
or
\begin{equation}
w = \epsilon_0\,E^2.
\end{equation}
It is clear, from the above,  that half the energy in an
electromagnetic wave is carried by the electric field, and the other half
is carried by the magnetic field. 

As an electromagnetic field propagates it transports energy. Let $P$ be the
power per unit area carried by an electromagnetic wave:
{\em i.e.}, $P$ is the energy transported per unit time
across a unit cross-sectional area perpendicular to the
direction in which the wave is traveling.  Consider a
plane electromagnetic
wave propagating along the $z$-axis. The wave propagates
 a distance $c\,dt$
along the $z$-axis in a time interval $d t$. If we
consider a cross-sectional area $A$ at right-angles to the $z$-axis, then in
a time $d t$ the wave sweeps  through a volume $d V$ of
space, where $dV = A\,c\,d t$. The
amount of energy filling this volume is
\begin{equation}
d W = w\,d V = \epsilon_0\,E^2\, A\,c\,d t.
\end{equation}
It follows, from the definition of $P$, that the power per unit area carried by
the wave is given by
\begin{equation}
P = \frac{d W}{A\,d t} = \frac{ \epsilon_0\,E^2\, A\,c\,d t}{A\,dt},
\end{equation}
so that
\begin{equation}
P = \epsilon_0\,E^2\, c.
\end{equation}
Since half the energy in an electromagnetic wave is carried by the
electric field, and the other half is carried by the magnetic field, it
is conventional to convert the above expression into a form involving both
the electric and magnetic field strengths. Since, $E=c\,B$, we have
\begin{equation}
P = \epsilon_0\,c\,E\,(c\,B) = \epsilon_0\,c^2\,E\,B = \frac{E\,B}{\mu_0}.
\end{equation}
Thus,
\begin{equation}\label{e11.28}
P = \frac{E\,B}{\mu_0}.
\end{equation}

Equation~(\ref{e11.28}) specifies the power per unit area transported by an electromagnetic
wave at any given instant of time. The {\em peak} power  is given by
\begin{equation}
P_0 = \frac{E_0\,B_0}{\mu_0},
\end{equation}
where $E_0$ and $B_0$ are the peak amplitudes of the oscillatory
electric and
magnetic fields, respectively. It is easily demonstrated that
the {\em average} power per unit area transported by an electromagnetic wave 
is {\em half} the peak power, so that
\begin{equation}\label{e11.30}
S= \bar{P} = \frac{E_0\,B_0}{2\,\mu_0} = \frac{\epsilon_0\,c\,E_0^{~2}}{2}
= \frac{c\,B_0^{~2}}{2\,\mu_0}.
\end{equation}
The quantity $S$ is conventionally termed the {\em intensity} of the wave. 

\subsection{Worked Examples}
\subsection*{\em Example 11.1: Electromagnetic waves}
{\em Question:} Consider  electromagnetic waves of wavelength
$\lambda = 30$\,cm in air. What is the frequency of such waves? If such waves pass
from air into a block of quartz, for which $K=4.3$, what is their new
speed, frequency, and wavelength?\\
~\\
{\em Answer:} Since, $f\,\lambda=c$, assuming that the dielectric constant of air is approximately unity, it follows that
$$
f = \frac{c}{\lambda} = \frac{(3\times 10^8)}{(0.3)} = 1\times 10^9\,{\rm Hz}.
$$
The new speed of the waves as they pass propagate through the
quartz is
$$
c' = \frac{c}{\sqrt{K}} = \frac{(3\times 10^8)}{\sqrt{4.3}} = 1.4\times
 10^8\,{\rm m}\,{\rm s}^{-1}.
$$
The frequency of electromagnetic waves  does not change when the medium through
which the waves are  propagating changes. Since $c'=f\,\lambda$ for
electromagnetic waves propagating through a dielectric medium, we have
$$
\lambda_{\rm  quartz} = \frac{c'}{f} = \frac{(1.4\times 10^8)}
{(1\times 10^9)} =  14\,{\rm cm}.
$$

\subsection*{\em Example 11.2: Intensity of electromagnetic radiation}
{\em Question:} Suppose that the intensity of the sunlight falling on the
ground on a particular day is $140\,{\rm W}\,{\rm m}^{-2}$. What are the
peak values of the electric and magnetic fields associated with the
incident radiation?\\
~\\
{\em Answer:} According to Eq.~(\ref{e11.30}), the peak electric field is given by
$$
E_0 = \sqrt{\frac{2\,S}{\epsilon_0\,c}} = \sqrt{\frac{(2)\,(140)}
{(8.85\times 10^{-12})\,(3\times 10^8)}} = 324.7\,\,{\rm V}\,{\rm m}^{-1}.
$$
Likewise, the peak magnetic field is given by
$$
B_0 = \sqrt{\frac{2\,\mu_0\,S}{c}} = \sqrt{\frac{(2)\,(4\pi\times 10^{-7})\,(140)}
{(3\times 10^8)}} = 1.083\times 10^{-6}\,{\rm T}.
$$
Note, of course, that $B_0=E_0/c$. 
