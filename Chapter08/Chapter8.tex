\section{Magnetism}
\subsection{Historical Introduction}
The phenomenon of magnetism has been known to mankind for many thousands of years.
Loadstone (a magnetized form of the commonly occurring iron oxide
mineral magnetite) was the first permanent magnetic material to be identified and
studied. The ancient Greeks were aware of the ability of loadstone  to
attract small pieces of iron. The Greek word {\em magnes}\/ (\gr{m'agnhc}), which is the root of the English word
{\em magnet},  is 
derived from Magnesia, the name of an ancient city in Asia Minor, which, presumably,
was once a copious source of loadstones. 

The magnetic compass was invented some time during the first ten centuries
AD. Credit is variously given to the Chinese, the Arabs, and the Italians.
What is certain is that by the 12th century magnetic compasses were in
regular use by mariners to aid navigation at sea. In the 13th century,
Peter Perigrinus of France discovered that the magnetic effect of
a spherical loadstone is strongest at two oppositely directed
points  on the surface of the
sphere, which he termed the {\em poles}\/ of the magnet. He found that
there are two types of poles, and that like poles repel one another
whereas unlike poles attract. In 1600, the English
physician William Gilbert concluded,
quite correctly, that the reason magnets like to align themselves in
a North-South direction is that the Earth itself is a magnet.
Furthermore, the Earth's magnetic poles are aligned, more or less,
along its axis of rotation. This insight immediately
gave rise to a fairly obvious
nomenclature for the two different poles of a magnet: a magnetic
{\em north pole}\/ (N) has the same magnetic polarity as the geographic south pole of the
Earth, and a magnetic {\em south pole}\/ (S) has the same polarity as the geographic north 
 pole of the
Earth. Thus, the north pole of a magnet likes to point northwards towards
the geographic north pole of the Earth (which is its magnetic south pole). 
Another British scientist, John Michell, discovered in
1750 that the attractive and repulsive forces between the poles of
magnets vary inversely as the square of the distance of separation.
Thus, the inverse square law for forces between
 magnets was actually discovered prior to
that for forces between electric charges. 

\subsection{Amp\`{e}re's Experiments}\label{s8.2}
In 1820, the Danish physicist Hans Christian \O rsted was giving a lecture 
demonstration of various electrical and
magnetic effects. Suddenly, much to his amazement, he noticed that
the needle of a compass he was holding
was deflected  when he moved it close to  a current carrying
wire. This was a very surprising observation, since, until that moment, electricity and
magnetism had been thought of  as two quite  unrelated phenomena.
Word of this discovery spread quickly along the scientific grapevine,
and the French physicist  Andre Marie Amp\`{e}re 
immediately decided to investigate further. 
Amp\`{e}re's apparatus consisted (essentially) of a long straight wire carrying an
electric current
current $I$. Amp\`{e}re quickly discovered that the needle of a small compass maps
out a series of concentric circular loops in the plane
perpendicular to  a current carrying wire---see Fig.~\ref{f8.1}.
The direction of circulation around these magnetic loops is conventionally taken to be
the direction in which the {\em north}\/ pole of a compass needle
points.
Using  this convention, the circulation of the loops is given by a
{\em right-hand rule}. If the thumb of the right-hand points along the direction of the
current, then the fingers of the right-hand circulate in the same sense as the 
magnetic loops.

\begin{figure}[h]
\epsfysize=2.5in
\centerline{\epsffile{Chapter08/fig8.01.eps}}
\caption{\em Magnetic loops around a current carrying wire.}\label{f8.1}
\end{figure}

Amp\`{e}re's next series of experiments involved bringing a short test wire, carrying
a current $I'$,
close to the original wire, and investigating the force exerted on the test wire.
This experiment is not quite as clear cut as Coulomb's experiment because, unlike
electric charges, 
electric currents cannot exist as point entities. They
have to flow in complete circuits. We must
imagine that the circuit which connects with  the central wire is sufficiently
far away that it has no appreciable influence on the outcome of the experiment.
The circuit which connects with
 the test wire is more problematic. Fortunately, if the
feed wires are twisted around each other, as indicated in Fig.~\ref{f8.2}, then 
they effectively cancel one another out, and also do not influence the outcome of
the experiment.
\begin{figure}[h]
\epsfysize=2.5in
\centerline{\epsffile{Chapter08/fig8.02.eps}}
\caption{\em Amp\`{e}re's experiment.}\label{f8.2}
\end{figure}

Amp\`{e}re discovered that the force exerted on the test wire is directly proportional
to its length. He also made the following observations.
If the current in the test wire 
({\em i.e.}, the test current) flows parallel to the current in the central wire 
then the two wires  attract one another. If the current in the test
wire is reversed then the two wires  repel one another.
If the test current points radially towards the central wire 
(and the current in the central wire flows upward) then the test wire
is subject to a downward force. If the test current is reversed then the force is
upward. If the test current is rotated in a single plane, so that it starts
parallel to the central current and ends up pointing radially
towards it, then the force on
the test wire is of constant magnitude, and is always at right-angles to the
test current. If the test current is parallel to a magnetic loop then there is
no force exerted on the test wire. If the test current is rotated in
a single plane, so that it starts parallel to the central current, and ends up
pointing along a magnetic loop, then the magnitude of the force on the
test wire attenuates like $\cos\theta$ (where $\theta$ is the angle the current
is turned through, and $\theta=0$ corresponds to the
case where the test current is parallel to the central current),
 and its direction is again always at right-angles to
the test current. Finally, Amp\`{e}re was able to establish that the attractive
force between two parallel current carrying wires is proportional to the product of
the two currents, and
 falls off like one over the perpendicular
distance between the wires.

This rather complicated force law can be summed up succinctly in vector notation
provided that we define a vector field ${\bf B}$, called the {\em magnetic field}, 
which fills space, and
whose direction is everywhere   tangential  to the 
magnetic loops mapped out by the north
pole of a small
  compass. The dependence of the force per unit length, ${\bf F}$, acting on a
test wire with the different 
possible orientations of the test current is described  by 
\begin{equation}\label{e8.1}
{\bf F} = {\bf I}' \times{\bf B},
\end{equation}
where ${\bf I}'$ is a vector whose direction and magnitude are the same as those
of the test current. 

The variation of the force per unit length acting on
a test wire with the strength of the 
central current, and the perpendicular distance $r$ to the central wire, is 
accounted for by saying that the magnetic field-strength is proportional to $I$, and
inversely proportional to $r$. Thus, we can write 
\begin{equation}
B = \frac{\mu_0\,I}
{2 \pi\, r}.
\end{equation}
The constant of proportionality $\mu_0$ is called the
{\em permeability of free space}, and takes the value
\begin{equation}
\mu_0 = 4\pi \times 10^{-7}\,{\rm N\,A^{-2}}.
\end{equation}
Incidentally, the SI unit of magnetic field strength is the tesla (T), which is the
same as a newton per ampere per meter:
\begin{equation}
1\,{\rm T} \equiv 1\,{\rm N}\,{\rm A}^{-1}\,{\rm m}^{-1}.
\end{equation}

The concept of a magnetic field which
fills the space around a current carrying wire
allows the calculation of the force on a test
wire to be conveniently split into two parts. In the first part, we calculate the
magnetic field generated by the current flowing in the central wire. This field
circulates  in the plane normal to the wire. Its magnitude is
proportional to the central current, and inversely proportional to the  perpendicular
distance from the wire. In the second part, we use
Eq.~(\ref{e8.1}) to calculate the force per unit
length acting on a
short current carrying wire placed in the magnetic field 
generated by the central current.
This force is perpendicular to both the direction of the magnetic field and the direction of the
test current. Note that, at this stage, we have no reason to suppose that the magnetic
field has any real existence. It is introduced merely to facilitate the calculation
of the  force exerted  on  the test wire by the central wire. It turns out, however,
that the magnetic field {\em does}\/ have a real existence, since, as we shall see, there is
an energy associated with a magnetic field which fills space. 

\subsection{Amp\`{e}re's Law}
Magnetic fields, like electric fields, are completely 
{\em superposable}. So, if
a field ${\bf B}_1$ is generated by a current $I_1$ flowing through some circuit,
and a field ${\bf B}_2$ is generated by a current $I_2$ flowing through another
 circuit, then when the currents $I_1$ and $I_2$ flow through both circuits
simultaneously the generated magnetic field is ${\bf B}_1+{\bf B}_2$.
This is true at all points in space.

\begin{figure}[h]
\epsfysize=2.5in
\centerline{\epsffile{Chapter08/fig8.03.eps}}
\caption{\em Two parallel current carrying wires.}
\end{figure}

Consider two parallel wires separated by a perpendicular distance $r$,
and carrying electric currents $I_1$ and $I_2$, respectively. The magnetic field-strength at the second wire due to the current flowing in the first wire 
is $B = \mu_0 \,I_1/2\pi\, r$. This field is orientated at right-angles to the second
wire, so the force per unit length exerted on the second wire is
\begin{equation}\label{e8.6}
F= \frac{\mu_0\, I_1 \,I_2}{2\pi\, r}.
\end{equation}
This follows from Eq.~(\ref{e8.1}), which is valid for continuous wires as well as short
test wires. The force acting on the second wire is directed radially inwards towards
the first wire. The magnetic field-strength at the first wire due to the
current flowing in the second wire is $B= \mu_0 \,I_2/2\pi\, r$. This field
is orientated at right-angles to the first wire, so the force per unit length acting
on the first wire is equal and opposite to that acting on the second wire, 
according to Eq.~(\ref{e8.1}).  Equation~(\ref{e8.6}) is called {\em Amp\`{e}re's law}.

Incidentally, Eq.~(\ref{e8.6}) is the basis of the official SI definition of the
{\em ampere}, which is:
\begin{quote}
{\sf One ampere is the magnitude of the current which, when flowing in
each of two long parallel wires one meter apart, results in a force 
between the wires of exactly $2\times 10^{-7}$\,N per meter of length.}
\end{quote}
We can see that it is no accident that the constant $\mu_0$ has the
numerical value of {\em exactly} $4\pi\times 10^{-7}$. 
The SI system of units is based on four standard units: the {\em meter},
the {\em kilogram}, the {\em second}, and the {\em ampere}. Hence, the SI system is
sometime referred to as the MKSA system. All other units can be derived
from these four standard units. For instance, a coulomb is equivalent to
an ampere-second. You may be wondering why the ampere is the standard
electrical unit, rather than the coulomb, since the latter unit is
clearly more fundamental than the former. The answer is simple. It is very difficult
to measure charge accurately, whereas it is easy to accurately measure electric
current. Clearly, it makes  sense to define a standard unit in terms
of something which is easily measurable, rather than something which is
difficult to measure.

\subsection{The Lorentz Force}\label{s8.4}
The flow of an electric current down
a conducting  wire is ultimately due to  the movement of
electrically charged particles
(in most cases, electrons) along the wire.
 It seems reasonable, therefore, that
the force exerted on the wire when it is placed in a magnetic field is simply
the resultant of the  forces exerted on these moving charges. Let us
suppose that this is the case. 

Let $A$ be the 
(uniform) cross-sectional area of the wire, and let $n$ be the number density
of mobile charges in the wire. Suppose that the
mobile charges each have charge $q$ and drift velocity ${\bf v}$.
 We must assume that
the wire also contains stationary charges, of charge $-q$ and number density
$n$, say, so that the net charge density in the wire is zero. In most conductors, the
mobile charges are electrons, and the stationary charges are atoms.
The magnitude of the electric current flowing through the wire is simply the
number of coulombs per second which flow past a given point. In one second,
a mobile charge moves a distance $v$, so all of the charges contained in a
cylinder of cross-sectional area $A$ and length $v$ flow past a given point.
Thus, the magnitude of the current is $q\,n\, A\,v$. The direction of the 
current is the same as the direction of motion of the charges ({\em i.e.},
${\bf I}'\propto {\bf v}$), so the
vector current is ${\bf I}' = q\,n\,A\,{\bf v}$.
According to Eq.~(\ref{e8.1}), the force per unit length acting on the wire is
\begin{equation}
{\bf F} = {\bf I}'\times {\bf B} = q\,n \,A \,{\bf v}\times{\bf B}.
\end{equation}
However, a unit length of the wire contains $n\,A$ moving charges. So, assuming
that each charge is subject to an equal force from the magnetic field (we have
no reason to suppose otherwise), the magnetic
force acting on an individual charge is
\begin{equation}\label{e8.7}
{\bf f} = q\,{\bf v} \times{\bf B}.
\end{equation}
This formula implies  that the magnitude of the magnetic force exerted on a {\em moving}
charged particle is the product of the particle's charge, its
 velocity, the
magnetic field-strength, and the sine of the angle subtended between the
particle's direction of motion and the direction of the magnetic field. 
The force is directed at right-angles to both the magnetic field and the
instantaneous direction of motion.

We can combine the above equation with Eq.~(\ref{e3.12}) to give the force acting on a charge $q$ moving
with velocity ${\bf v}$ in an electric field ${\bf E}$ and a magnetic field
${\bf B}$:
\begin{equation}
{\bf f} = q\,{\bf E} + q\,{\bf v} \times{\bf B}.
\end{equation}
This is called the {\em Lorentz force law}, after the Dutch physicist
Hendrick Antoon Lorentz, who first formulated it. The electric
force on a charged particle is parallel to the local electric field.
The magnetic force, however, is perpendicular to both the local magnetic
field and the particle's direction of motion. No magnetic force is exerted on a
stationary charged particle.

The
equation of motion of a free particle of charge $q$ and
mass $m$ moving in electric and
magnetic fields is
\begin{equation} 
m\,{\bf a} = q\,{\bf E} + q\,{\bf v} \times{\bf B},
\end{equation}
according to the Lorentz force law. Here, ${\bf a}$ is the
particle's acceleration.
This equation of motion was verified in a famous experiment carried out
by the Cambridge physicist J.J.~Thompson in 1897. Thompson was investigating
{\em cathode rays}, a then mysterious form of radiation emitted by a heated
metal element held at a large negative voltage ({\em i.e.},  a cathode) with respect
to another metal element ({\em i.e.}, an anode)  in an evacuated tube. 
German physicists maintained that cathode rays were
a form of electromagnetic radiation, whereas British and French physicists suspected
that they were, in reality, a stream of charged particles. Thompson was able to
demonstrate that the latter view was correct. In Thompson's experiment, the
cathode rays pass though a region of {\em crossed}\/ electric and magnetic
fields (still in vacuum). The fields are perpendicular to the original
trajectory of the rays, and are also mutually perpendicular.

\begin{figure}
\epsfysize=2.5in
\centerline{\epsffile{Chapter08/fig8.04.eps}}
\caption{\em Thompson's experiment.}\label{f8.4}
\end{figure}

Let us analyze  Thompson's experiment.  Suppose that
the rays are originally traveling in the $x$-direction, and are subject to
a uniform electric field $E$ in the   $z$-direction, and a uniform magnetic
field $B$ in the $-y$-direction---see Fig.~\ref{f8.4}. Let us assume, as Thompson did, that cathode
rays are a stream of particles of mass $m$ and charge $q$. The
equation of motion of the particles in the $z$-direction is
\begin{equation}\label{e8.11}
m \,a_z = q\left(E - v \,B\right),
\end{equation}
where $v$ is the velocity of the particles in the $x$-direction, and $a_z$
 the acceleration of the particles in the $z$-direction.
Thompson started off his experiment by
only turning on the  electric field in his apparatus, and
measuring the
deflection $d$ of the rays in the $z$-direction after they  had traveled a
distance  $l$ through the  field. Now, a particle subject to a
constant acceleration $a_z$ in the $z$-direction is deflected a
distance $d=(1/2)\,a_z\,t^2$ in a time $t$. 
Thus, 
\begin{equation}\label{e8.12}
d = \frac{1}{2}\frac{q\,E}{m}\,t^2 = \frac{q}{m} \frac{E\,l^2}{2\,v^2},
\end{equation}
where the {\em time of flight}\/ $t$ is replaced by $l/v$. This replacement is only
valid if $d\ll l$ ({\em i.e.}, if the deflection of
the rays is small compared to the distance they travel
through the electric field), which is assumed to be the case. 
Next, Thompson  turned on
the magnetic field in his apparatus, and adjusted it so that the cathode rays were
no longer deflected. The lack of deflection implies that the net force on the
particles in the $z$-direction is zero. In other words, the electric and
magnetic forces balance exactly. It follows from Eq.~(\ref{e8.11})
that, with a properly adjusted magnetic field-strength,
\begin{equation}\label{e8.13}
v = \frac{E}{B}.
\end{equation}
Thus, Eqs.~(\ref{e8.12}) and (\ref{e8.13})
 can be combined and rearranged to give the charge to mass ratio of
the particles in terms of measured quantities:
\begin{equation}
\frac{q}{m} = \frac{2\,d \,E}{l^2\, B^2}.
\end{equation}
Using this method, Thompson inferred that cathode rays are made up of
negatively charged particles (the sign of the charge is obvious from the
direction of the deflection in the electric field) with a charge to mass
ratio of $-1.7\times 10^{11}~{\rm C\,kg}^{-1}$. A decade later, in 1908, the American Robert
Millikan performed his famous {\em oil drop} experiment in which he
 discovered that
mobile electric charges are quantized in units of $-1.6\times 10^{-19}$~C. 
Assuming that mobile electric charges and the particles which
make up cathode rays are one and the same thing,
 Thompson's and Millikan's experiments imply that the mass
of  these particles is  $9.4\times 10^{-31}$~kg. Of course, this is the mass of
an electron (the modern value is $9.1\times 10^{-31}$~kg), and  
$-1.6\times 10^{-19}$~C is the charge of an electron. Thus, cathode rays are, in fact,
streams of electrons which are  emitted from a heated cathode, and then
accelerated because of  the large voltage difference between the cathode and anode.

If a particle is subject to a force ${\bf f}$ which causes it
to displace by
$d{\bf r}$ then the work done on the particle by the
force is
\begin{equation}
W = {\bf f}\cdot d{\bf r} =
 f\,dr\,\cos\theta,
\end{equation}
where $\theta$ is the angle subtended between the force and the displacement. However, this angle is always $90^\circ$ for the force exerted by a magnetic field on
a charged particle, since the magnetic force is
always perpendicular to the particle's instantaneous direction of motion.
 It follows that
a magnetic field is unable to do work on a charged particle. 
 In other words, a
charged particle can never gain or lose energy due to interaction with
a magnetic field. On the other hand, a charged particle can certainly gain
or lose energy due to interaction with an electric field. 
Thus, magnetic
fields are often used in particle accelerators to guide charged particle motion ({\em e.g.}, in a circle), but the
actual acceleration is always performed by electric fields. 

\subsection{Charged Particle in a Magnetic Field}
Suppose that a particle of mass $m$ moves in a circular orbit of
radius $\rho$ with a constant speed $v$. As is well-known, the acceleration of the
particle is of magnitude $v^2/\rho$, and is always
directed towards the centre of the orbit. It follows that the
acceleration is always perpendicular to the particle's instantaneous
direction of motion. 

We have seen that the force exerted on a charged particle by a magnetic
field is always perpendicular to its instantaneous direction of motion.
Does this mean that the field causes the particle to execute a circular
orbit? Consider the case shown in Fig.~\ref{f8.5}. Suppose that a
 particle of positive charge $q$ and mass $m$ moves in a plane perpendicular
to a uniform magnetic field $B$. In the figure, the field points into
the plane of the paper. Suppose that the particle moves, in an
anti-clockwise manner,  with constant
speed $v$ (remember that the magnetic field cannot do work on the
particle, so it cannot affect its speed), in a  circular orbit of radius $\rho$.
The magnetic force acting on the particle is
of magnitude $f=q\,v\,B$ and, according to Eq.~(\ref{e8.7}), this force is always
directed towards the centre of the orbit. Thus, if
\begin{equation}
f = q\,v\,B = \frac{m\,v^2}{\rho},
\end{equation}
then we have a self-consistent picture. It follows that
\begin{equation}\label{e8.17}
\rho = \frac{m\,v}{q\,B}.
\end{equation}
The angular frequency of rotation of the particle ({\em i.e.}, the number of
radians the particle rotates through in one second) is 
\begin{equation}\label{e8.18}
\omega = \frac{v}{\rho} = \frac{q\,B}{m}.
\end{equation}
Note that this frequency, which is known as the {\em Larmor frequency}, does
not depend on the velocity of the particle. For a negatively charged particle,
the picture is exactly the same as described above, except that the particle moves in a
clockwise orbit. 

\begin{figure}
\epsfysize=2.5in
\centerline{\epsffile{Chapter08/fig8.05.eps}}
\caption{\em Circular motion of a charged particle in a magnetic field.}\label{f8.5}
\end{figure}

It is clear, from Eq.~(\ref{e8.18}), that the angular frequency of gyration of a charged
particle in a known magnetic field can be used to determine its charge to
mass ratio. Furthermore, if the speed of the particle is known, then
the radius of the orbit can also be used to determine $q/m$, via Eq.~(\ref{e8.17}).
This method is employed in High Energy Physics to identify particles from
photographs of the tracks which they leave in magnetized cloud chambers or bubble
chambers. It is, of course,  easy to differentiate positively charged particles
from negatively charged ones using the direction of deflection of the
particles in the magnetic field. 

We have seen that a charged particle placed in a magnetic field  executes a
circular orbit in the plane perpendicular to the direction of the field. 
Is this the most general motion of a charged particle in a magnetic field?
Not quite. We can also add an arbitrary drift along the direction
of the magnetic field. This follows because the force $q\,{\bf v}\times{\bf B}$
acting on the particle only depends on the component of the particle's velocity
which is {\em perpendicular} to the direction of magnetic field (the cross
product of two parallel vectors is always zero because the angle $\theta$
they subtend is zero). The combination of circular motion in the
plane perpendicular to the magnetic field, and uniform motion along the
direction of the 
field, gives rise to a {\em spiral}\/ trajectory of a charged particle in
a magnetic field, where the field forms the axis of the spiral---see Fig.~\ref{f8.6}.

\begin{figure}
\epsfysize=1.in
\centerline{\epsffile{Chapter08/fig8.06.eps}}
\caption{\em Spiral trajectory of a charged particle in a uniform magnetic field.}\label{f8.6}
\end{figure}

\subsection{The Hall Effect}
We have repeatedly stated that the mobile charges in
conventional
 conducting materials are negatively charged (they are, in fact, electrons).
Is there any direct experimental evidence that this is true? Actually, there is. We can
use a phenomenon called the {\em Hall effect} to determine whether the
mobile charges in a given conductor are positively or negatively charged. 
Let us investigate this effect. 

Consider a thin, flat,  uniform, ribbon of some conducting material which
is orientated such that its flat side is perpendicular to a uniform
magnetic field $B$---see Fig.~\ref{f8.7}. Suppose that we pass a current $I$ along the length
of the ribbon. There are two alternatives. Either the current
is carried by positive charges 
 moving from left to right (in the figure),
or it is carried by negative charges moving in the opposite direction. 

Suppose that the
 current is carried by positive charges moving from left to right.
These charges are deflected
{\em upward}\/ (in the figure) by the magnetic field. Thus, the upper edge of the ribbon becomes
positively            charged, whilst the lower edge becomes negatively charged.
Consequently, there is a {\em positive}\/ potential difference $V_H$ between the upper
and lower edges of the ribbon. This potential difference is called the {\em
Hall voltage}. 

Suppose, now, that the current is carried by negative charges
moving from right to left. These
charges are also deflected {\em upward}\/ by the magnetic field. Thus, the upper edge
of the ribbon becomes negatively charged, whilst the lower edge becomes
positively charged. It follows that the Hall voltage ({\em i.e.}, the
potential difference between the upper and lower edges of the ribbon)
is {\em negative}\/ in this case.

\begin{figure}
\epsfysize=2in
\centerline{\epsffile{Chapter08/fig8.07.eps}}
\caption{\em Hall effect for positive charge carriers (left) and negative
charge carriers (right).}\label{f8.7}
\end{figure}

Clearly, it is possible to determine the sign of the mobile charges in a
current carrying conductor by measuring the Hall voltage. If the voltage is
positive then the mobile charges are positive (assuming that the
magnetic field and the current are orientated as shown in the
figure), whereas if the voltage is
negative then the mobile charges are negative. If we were to perform
this experiment we would discover that the the mobile charges in metals
are always negative (because they
are electrons). However, in some types of semiconductor the mobile charges
turn out to be positive. These positive charge carriers are called {\em holes}.
Holes are actually  missing electrons in the atomic lattice  of the
semiconductor, but they act essentially like positive charges. 

Let us investigate the magnitude of the Hall voltage. Suppose that the mobile
charges each possess a charge $q$ and move along the ribbon with the
drift velocity $v_d$. The magnetic force on a given mobile charge
is of magnitude $q\,v_d\,B$, since the charge moves essentially
at right-angles to the magnetic field. In a steady-state, this force
is balanced by the electric force due to the build up of charges
on the upper and lower edges of the ribbon. If the Hall voltage is
$V_H$, and the width of the ribbon is $w$, then the electric
field pointing from the upper to the lower edge of the ribbon is
of magnitude $E= V_H/w$. Now, the electric force on a mobile charge
is $q\,E$. This force acts in opposition to the magnetic force. 
In a steady-state,
\begin{equation}
q\,E = \frac{q\,V_H}{w} = q\,v_d\,B,
\end{equation}
giving
\begin{equation}\label{e8.20}
V_H = v_d\,w\,B.
\end{equation}
Note that the Hall voltage is directly proportional to the magnitude of the
magnetic field. In fact, this property of the
Hall voltage  is exploited  in instruments, called {\em Hall probes},
which are used 
to measure magnetic field-strength. 

Suppose that the thickness of the conducting ribbon is $d$, and that it contains
$n$ mobile charge carriers per unit volume. It follows that the total current
flowing through the ribbon can be written
\begin{equation}\label{e8.21}
I = q\,n\,w\,d\,v_d,
\end{equation}
since all mobile charges contained in a rectangular volume of length $v_d$, width
$w$, and thickness $d$, flow past a given point on the ribbon in one second. 
Combining Eqs.~(\ref{e8.20}) and (\ref{e8.21}), we  obtain
\begin{equation}
V_H = \frac{I\,B}{q\,n\,d}.
\end{equation}
It is clear that the Hall voltage is proportional to the current flowing through
the ribbon, and the magnetic field-strength, and is inversely proportional
to the number density of mobile charges in the ribbon, and the thickness of
the ribbon. Thus, in order to construct a sensitive Hall probe
({\em i.e.}, one which produces a large Hall voltage in the
presence of a small magnetic field), we need to take  a thin ribbon of
some material which possesses relatively few mobile charges per unit
volume ({\em e.g.}, a semiconductor), and then run a large current through it. 

\subsection{Amp\`{e}re's Circuital Law}\label{s8.7}
Consider a long thin wire carrying a steady current $I$. Suppose that the
wire is orientated such that the current flows along the $z$-axis. 
Consider some closed loop $C$ in the $x$-$y$ plane which circles the wire in
an anti-clockwise direction, looking down the $z$-axis. Suppose that
$d{\bf r}$ is a short straight-line element of this loop. 
Let us form the dot product of this element with the local magnetic field
${\bf B}$. Thus,
\begin{equation}
dw = {\bf B}\cdot d{\bf r}
= B\,d r\,\cos\theta,
\end{equation}
where $\theta$ is the angle subtended between the direction of the line element and
the direction of the local magnetic field. We can calculate a $dw$
for
every line element which makes up the loop $C$. If we sum all of the
$dw$ values thus obtained, and take the limit as the
number of  elements goes to infinity, 
we obtain the {\em line integral}
 \begin{equation}
w =\oint_C{\bf B}\cdot d{\bf r}.
\end{equation}

What is the value of this
 integral? In general, this is a difficult question
to answer. However, let us consider a special case. Suppose that $C$ is a circle
of radius $r$ centred on the wire. In this case, the magnetic field-strength
is the same at all points on the loop. In fact,
\begin{equation}
B  = \frac{\mu_0\,I}{2\pi\,r}.
\end{equation}
Moreover, the field is everywhere parallel to the line elements which
make up the loop. Thus,
\begin{equation}
w = 2\pi\,r\,B = \mu_0\,I,
\end{equation}
or
\begin{equation}\label{e8.28}
\oint_C {\bf B} \cdot d{\bf r} = \mu_0\,I.
\end{equation}
In other words, the line integral of the magnetic field around some
circular loop $C$, centred on a current carrying wire, and
in the plane perpendicular to the wire, is equal to $\mu_0$
times the current flowing in the wire. Note that this answer is independent of
the radius $r$ of the loop: {\em i.e.}, the same result is
obtained by taking the line integral around {\em any}\/ circular loop centred
on the wire.

In 1826, Amp\`{e}re demonstrated that Eq.~(\ref{e8.28}) holds for {\em any}\/ closed loop which
circles around {\em any}\/ distribution of currents. Thus, Amp\`{e}re's circuital law
can be written:
\begin{quote}
{\sf The line integral of the magnetic field around some closed loop
is equal to the $\mu_0$ times the algebraic sum of the currents which pass through the
loop.}
\end{quote}
In forming the algebraic sum of the currents passing through the loop, those currents
 which the loop circles in an anti-clockwise direction (looking
against the direction of the current) count as positive currents, whereas those
 which the loop circles in a clockwise direction (looking
against the direction of the current) count as negative currents. 

Amp\`{e}re's circuital law is to magnetostatics (the study of the magnetic fields
generated by steady currents) what Gauss' law is to electrostatics
(the study of the electric fields generated by stationary charges). Like
Gauss' law, Amp\`{e}re's circuital law is particularly  useful in situations
which possess a high degree of symmetry. 

\subsection{Magnetic Field of a Solenoid}\label{s8.8}
A {\em solenoid}\/ is a tightly wound  helical coil of wire whose diameter is small compared
to its length. The magnetic field generated in the centre, or {\em core}, 
of  a current carrying solenoid
is essentially {\em uniform}, and is directed along the axis of the solenoid.
Outside the solenoid, the magnetic field is far weaker. Figure~\ref{f8.8} shows
(rather schematically) the magnetic field generated by a typical solenoid. 
The solenoid is wound from a single helical wire which carries a current $I$. 
The winding is sufficiently tight that each turn of the solenoid is well
approximated as 
a circular wire loop, lying in the plane perpendicular to the axis of
the solenoid, which carries a current $I$. Suppose that there
are $n$ such turns per unit axial length of the solenoid. What is the
magnitude of the magnetic field in the core of the solenoid?

\begin{figure}[h]
\epsfysize=3in
\centerline{\epsffile{Chapter08/fig8.08.eps}}
\caption{\em A solenoid.}\label{f8.8}
\end{figure}

In order to answer this question, let us apply Amp\`{e}re's circuital
law to the rectangular loop $abcd$. We must first find the line integral
of the magnetic field around $abcd$. Along $bc$ and $da$ the magnetic field
is essentially perpendicular to the loop, so there is no contribution to
the line integral from these sections of the loop. 
Along $cd$ the magnetic field is approximately uniform,
of magnitude $B$, say, and is directed parallel to the loop. Thus, the
contribution to the line integral from this section of the loop
is $B\,L$, where $L$ is the length of $cd$. 
Along $ab$ the magnetic field-strength is essentially negligible, so
this section of the loop makes no contribution to the line
integral. It follows that the line integral of the
magnetic field around $abcd$ is simply
\begin{equation}
w = B\,L.
\end{equation}
By  Amp\`{e}re's circuital law, this line integral is equal to $\mu_0$
times the algebraic sum of
the currents which flow through the loop $abcd$. Since the length of the
loop along the axis of the solenoid is $L$, the loop intersects $n\,L$
turns of the solenoid, each carrying a current $I$. Thus, the total
current which flows through the loop is $n\,L\,I$. This current counts as
a positive current since if we look against the direction of the
currents flowing in each turn 
 ({\em i.e.}, into the page in the figure), the loop $abcd$ circulates
these currents in an anti-clockwise direction. Amp\`{e}re's circuital law yields
\begin{equation}
B\,L  =  \mu_0\, n\,L\,I,
\end{equation}
which reduces to
\begin{equation}
B =\mu_0\,n\,I.
\end{equation}
Thus, the magnetic field in the core of a solenoid is directly
proportional to the product of the
current flowing around the solenoid and the number
of turns per unit length of the solenoid. This, result is {\em exact}\/ in the
limit in which the length of the solenoid is very much greater than its diameter.

\subsection{Origin of Permanent Magnetism}
We now know of two distinct methods of generating a magnetic field. We can either use a
permanent magnet, such as a piece of loadstone, or we can run a current
around an electric circuit. Are these two methods fundamentally different, or
are they somehow related to one another? Let us investigate further.

\begin{figure}[h]
\epsfysize=2.5in
\centerline{\epsffile{Chapter08/fig8.09.eps}}
\caption{\em Magnetic fields of a solenoid (left) and a bar magnet (right).}\label{f8.9}
\end{figure}

As illustrated in Fig.~\ref{f8.9}, the {\em external}\/ magnetic fields generated by
a solenoid and a conventional bar magnet are remarkably similar in
appearance. Incidentally,
these fields can easily be mapped out using iron filings. The above observation
allows us to formulate {\em two}\/ alternative hypotheses for the origin of the
magnetic field of a bar magnet. The first hypotheses is that the
field of a bar magnet is produced by solenoid-like currents which flow around the
outside of the magnet,  in an anti-clockwise direction as we look
along the magnet from its north to its south pole. There is no doubt, by analogy
with a solenoid, that such currents would generate the correct sort of field outside
the magnet. The second hypothesis is that the field is produced by a
positive {\em magnetic monopole}\/ located close to the north pole of the magnet, in combination with 
a  negative monopole of equal magnitude located close 
to the south pole of the magnet. What is a magnetic monopole? Well, it is basically
the magnetic equivalent of an electric charge. A positive magnetic monopole
is an isolated magnetic north pole. We would expect magnetic field-lines
to radiate away from such an object, just as electric field-lines radiate
away from a positive electric charge. Likewise, a negative magnetic monopole
is an isolated magnetic south pole. We would expect magnetic field-lines
to radiate towards such an object, just as electric field-lines radiate
towards a negative  electric charge. The magnetic field patterns generated by
both types of monopole are sketched in Fig.~\ref{f8.10}. If we place a positive monopole
close to the north pole of a bar magnet, and a negative monopole of the
same magnitude close to the
south pole, then the resultant magnetic field pattern 
is obtained by {\em superposing}\/
the fields generated by the two monopoles individually. 
As is easily  demonstrated, the field generated outside the magnet
is indistinguishable from that of a solenoid.

\begin{figure}[h]
\epsfysize=3in
\centerline{\epsffile{Chapter08/fig8.10.eps}}
\caption{\em Magnetic field-lines generated by magnetic monopoles.}\label{f8.10}
\end{figure}

We now have two alternative hypotheses to explain the origin of the magnetic field of
a bar magnet. What experiment could we perform in order to determine which of these
two hypotheses is  correct? Well, suppose that we snap
our bar magnet in two. What
happens according to each hypothesis? If we cut a solenoid in two then we just end up
with two shorter solenoids. So, according to our first hypothesis,
if we snap a bar magnet in two then we just end up with two smaller bar magnets. 
However, our second hypothesis predicts that if we snap a bar magnet in two
then we end up with two equal and opposite magnetic monopoles. Needless to say,
the former prediction is in accordance with experiment, whereas the latter
most certainly is not. Indeed, we can break a bar magnetic into as many separate
pieces as we like. Each piece will still act like a little bar magnet. 
No matter how
small we make the pieces, we cannot produce a magnetic monopole.
In fact, nobody has ever observed a magnetic monopole
experimentally, which  leads most physicists to conclude that {\em magnetic
monopoles do not exist}. Thus, we can conclude that the magnetic field of
a bar magnet is produced by solenoid-like currents flowing over the
surface of the magnet. But, what is the origin of these currents?

	In order to answer the last question, let us adopt a somewhat simplistic
model of the atomic structure of a bar magnet. Suppose that the north-south axis
of the magnet is aligned along the $z$-axis, such that the $z$-coordinate
of the magnet's north pole is larger
 than that of its south pole. Suppose, further, that the atoms which make up the
magnet are identical {\em cubes}\/ which are packed very closely together. Each atom carries
a {\em surface current}\/  which circulates in the $x$-$y$ plane in an
anti-clockwise direction (looking down the $z$-axis). When the atoms are
arranged in a uniform lattice, so as to form the
magnet, the interior surface currents cancel out, leaving a  current
which flows only on the outer surface of the magnet. This is illustrated in Fig.~\ref{f8.11}.  Thus, the solenoid-like currents which must flow over the
surface of a magnet in order to account for its associated magnetic field are, 
in fact, just the {\em resultant}\/ of currents which circulate in every
constituent  atom of  the magnet. But, what is the origin of these atomic currents?

\begin{figure}[h]
\epsfysize=3in
\centerline{\epsffile{Chapter08/fig8.11.eps}}
\caption{\em A schematic diagram of the current pattern in a permanent magnet.}\label{f8.11}
\end{figure}

Well, atoms consist of negatively charged electrons in orbit around positively
charged nuclei. A moving electric charge constitutes an electric current, so
there must be a current associated with every electron in an atom. In most
atoms, these currents cancel one another out, so that the atom carries
zero net current. However, in the atoms of {\em ferromagnetic}\/ materials
({\em i.e.}, iron, cobalt, and nickel) this cancellation is not complete,
so these atoms do carry a net current. 
Usually, the atomic currents are all jumbled up 
({\em i.e.}, they are not aligned in any particular plane)
so that they average to zero on
a
 macroscopic scale.
However, if a ferromagnetic material is
placed in a {\em strong}\/ magnetic field then the currents circulating in
each atom become {\em aligned}\/ such that they flow predominately
in the plane perpendicular to the
field. In this situation, the currents can combine together to form a
macroscopic magnetic field which reinforces the alignment field. In some
ferromagnetic materials, the atomic currents remain aligned
after the alignment field is switched off, so the macroscopic field generated by
these currents also remains. We call such materials {\em permanent magnets}. 

In conclusion, {\em all}\/ magnetic fields encountered in nature are generated by
{\em circulating currents}. There is no fundamental difference between the fields
generated by permanent magnets and those generated by currents flowing around
conventional electric circuits. In the former, case the currents which generate the
fields circulate on the atomic scale, whereas, in the latter case, the currents
circulate on a macroscopic scale ({\em i.e.}, the scale of the circuit). 

\subsection{Gauss' Law for Magnetic Fields}\label{s8.10}
Recall (from Sect.~\ref{s4.2}) that the electric flux through a closed
surface $S$ is 
written
\begin{equation}
\Phi_E = \oint_S {\bf E}\cdot d{\bf S}.
\end{equation}
Similarly, we can also define the {\em magnetic flux}\/  through a
closed surface as
\begin{equation}
\Phi_B = \oint_S {\bf B}\cdot d{\bf S}.
\end{equation}


According to Gauss' law (see Sect.~\ref{s4.2}), the electric flux 
through any  closed surface is directly
proportional to the net electric charge enclosed by that surface.
Given the very direct analogy which exists between an electric charge and a magnetic
monopole, we would expect to be able to formulate a second law which
states that the magnetic flux through any closed surface is directly
proportional to the number of magnetic monopoles enclosed by that surface.
However, as we have already discussed,  magnetic monopoles do not exist. It follows that the equivalent of Gauss' law
for magnetic fields reduces to:
\begin{quote}
{\sf The magnetic flux though any closed surface is zero.}
\end{quote}
This is just another way of saying that magnetic monopoles do not exist, and
that all magnetic fields are actually  generated by circulating currents. 

An immediate corollary of the above law is that the number of magnetic
field-lines which enter a closed surface is always equal to the number of
field-lines which leave the surface. In other words:
\begin{quote}
{\sf Magnetic field-lines form closed loops which never begin or end.}
\end{quote}
Thus, magnetic field-lines behave in a quite different manner to electric
field-lines, which begin on positive charges, end on negative charges,
and never form closed loops. Incidentally,
the statement that electric field-lines never form closed
loops follows from  the result that the work done
in taking an electric charge around a closed loop is always zero (see Sect.~\ref{spotn}). This
clearly cannot be  true if it is possible to take a charge around the path of
a closed electric field-line. Note, however, that this conclusion regarding
electric field-lines only holds for the electric fields generated by stationary
charges. 

\subsection{Galvanometers}\label{s8.11}
We have talked a lot about potential differences, currents, and 
resistances, but we
have not talked much about how these quantities can be measured. Let us now
investigate this topic. 

Broadly speaking, only electric currents can be measured directly. Potential
differences  and
resistances are usually {\em inferred}\/ from measurements of electric currents.
The most accurate method of measuring an electric current is by using a
device called a {\em galvanometer}. 

A galvanometer consists of a rectangular conducting coil which is free to pivot vertically
in an approximately uniform horizontal
magnetic field $B$---see Fig.~\ref{f8.12}. The magnetic field is usually generated
by a permanent magnet. Suppose that a current $I$ runs through the coil.
What are the forces exerted on the coil by the magnetic field? 
According to Eq.~(\ref{e8.1}), the
 forces exerted on those sections of the coil in which the
current runs in the horizontal plane are directed vertically up or down.
These forces are irrelevant, since they are absorbed by the support structure of
the coil, which does not allow the coil to move vertically.
Equation~(\ref{e8.1}) also implies that  the force exerted on
the section of the coil in which the current flows downward  is of magnitude
$F= I\,B\,L$, where $L$ the length of this section, and is directed out of
the page (in the figure). Likewise, the force exerted on
the section of the coil in which the current flows upward is also of
magnitude $F= I\,B\,L$, and is directed into the page. These two forces
exert a {\em torque} on the coil which tries to twist it about its vertical
axis in an anti-clockwise direction (looking from above). Using the
usual definition of  torque ({\em i.e.}, torque is the product of
the force and the perpendicular distance from the line of action of
the force to the axis of rotation), the net torque $\tau$ acting on the coil is
\begin{equation}
\tau = 2\,F\,\frac{D}{2} = I\,B\,L\,D = I\,B\,A.
\end{equation}
where $D$ is the horizontal width of the coil, and $A$ is its area. Note
that the two vertical sections of the coil give rise to equal contributions to the
torque. 
Strictly speaking, the above expression is only valid when the coil lies in the plane of the
magnetic field. However, galvanometers are  usually constructed with curved 
magnetic pole
pieces in order to ensure that, as the coil turns, it always remains in the plane of the magnetic field. 
It follows that, for  fixed magnetic field-strength, and  fixed coil area,
the torque exerted on the coil is directly proportional to the current
$I$.

\begin{figure}[h]
\epsfysize=3in
\centerline{\epsffile{Chapter08/fig8.12.eps}}
\caption{\em A galvanometer.}\label{f8.12}
\end{figure}

The coil in a galvanometer is usually suspended from a torsion wire. The
wire exerts a restoring torque on the coil which tries to twist it back to
its original position. The strength of this restoring torque is directly
proportional to the angle of twist ${\mit\Delta}\theta$.  It follows that,
in equilibrium, where the magnetic torque balances the restoring torque,
the angle of twist ${\mit\Delta}\theta$ is directly proportional
to the current $I$ flowing around  the coil. The angle of twist can be measured by 
attaching a pointer to the coil, or, even better, by mounting a mirror on the coil,
and reflecting a light beam off the mirror. Since ${\mit\Delta}\theta \propto I$, the
device can easily  be calibrated by running a known current through it.

There is, of course, a practical limit to how large the angle of twist  ${\mit\Delta}\theta$ can become in a galvanometer. If the torsion wire is twisted through
too great an angle
then it will deform permanently, and will eventually snap. 
In order to prevent this from happening, most galvanometers are equipped
with a ``stop'' which physically prevents the coil from twisting through more than
(say) $90^\circ$. Thus, there is a maximum current $I_{\rm fsd}$ which
a galvanometer can measure. This is usually referred to as the
{\em full-scale-deflection current}. The full-scale-deflection current in
 conventional galvanometers is usually pretty small: {\em e.g.}, $10\,\mu\,{\rm A}$. 
So, what do we do if we want to measure a large current?

What we do is to connect a {\em shunt resistor}\/ in parallel with the
galvanometer, so that most of the current flows through the resistor, and
only a small fraction of the current flows through the galvanometer itself. This is
illustrated in Fig.~\ref{f8.13}. Let the resistance of the galvanometer
be $R_G$, and the resistance of the shunt resistor be $R_S$. Suppose that
we want to be able to measure the total current $I$ flowing through the galvanometer
and the shunt resistor up to a maximum value of $I_{\rm max}$. 
We can achieve this if the current $I_G$ flowing through the galvanometer equals the
full-scale-deflection current $I_{\rm fsd}$ when $I=I_{\rm max}$. In this
case, the current $I_S= I-I_G$ flowing through the shunt resistor takes the
value $I_{\rm max}-I_{\rm fsd}$. The potential drop across the shunt resistor is
therefore $(I_{\rm max}-I_{\rm fsd})\,R_S$. This potential drop must match the
potential drop $I_{\rm fsd}\,R_G$ across the galvanometer, since the galvanometer is
connected in parallel with the shunt resistor. It follows that
\begin{equation}
(I_{\rm max}-I_{\rm fsd})\,R_S = I_{\rm fsd}\,R_G,
\end{equation}
which reduces to
\begin{equation}
R_S = \frac{I_{\rm fsd}}{I_{\rm max}-I_{\rm fsd}}\,R_G.
\end{equation}
Using this formula, we can always choose an appropriate shunt resistor to allow
a galvanometer to measure any current, no matter how large. For instance, if
the full-scale-deflection current is $I_{\rm fsd}=10\,\mu{\rm A}$, the maximum current
we wish to measure is $I_{\rm max} = 1\,{\rm A}$, and the resistance of
the galvanometer is $R_G = 40\,\Omega$, then the appropriate shunt resistance is
\begin{equation}
R_S = \frac{ 1\times 10^{-5}}{1 - 1\times 10^{-5}} \,40 \simeq 4.0\times 10^{-4}
\,\Omega.
\end{equation}
Most galvanometers are equipped with a dial which allows us to choose between
various alternative  ranges of currents which the device can measure: {\em e.g.},
0--$100$\,mA, 0--1\,A, or 0--10\,A. All the dial does is to switch
between  different
shunt resistors connected in parallel with the galvanometer itself. 
Note, finally, that the equivalent resistance of the galvanometer and its
shunt resistor is
\begin{equation}
R_{\rm eq} = \frac{1}{(1/R_G)+(1/R_S)} = \frac{I_{\rm fsd}}{I_{\rm max}}\,R_G.
\end{equation}
Clearly, if the full-scale-deflection current $I_{\rm fsd}$ is much less than the maximum current $I_{\rm max}$
which we wish to measure then the equivalent resistance is very small indeed. 
Thus, there is an advantage to making the full-scale-deflection current
of a galvanometer small. A small full-scale-deflection current implies a
small equivalent resistance of the galvanometer, which means that the
galvanometer can be connected into a circuit without seriously disturbing the
currents flowing around that
circuit. 

\begin{figure}[h]
\epsfysize=2in
\centerline{\epsffile{Chapter08/fig8.13.eps}}
\caption{\em Circuit diagram for a galvanometer measuring current.}\label{f8.13}
\end{figure}

A galvanometer can be used to measure potential difference as well as
current (although, in the former case, it is really measuring current). In
order to measure the potential difference $V$ between two points $a$ and $b$ in
some circuit, we connect a galvanometer, in series with a shunt resistor, across
these two points---see Fig.~\ref{f8.14}. The galvanometer draws a current $I$ from the circuit.
This current is, of course, proportional to the potential difference between
$a$ and $b$, which enables us to relate the reading on  the galvanometer
to the voltage we are trying to measure. Suppose that we wish to measure voltages in
the range 0 to $V_{\rm max}$. What is an appropriate choice of the
shunt resistance $R_S$? Well, the equivalent resistance of the
shunt resistor and the galvanometer is $R_S+R_G$, where $R_G$ is
the resistance of the galvanometer. Thus, the current flowing through the
galvanometer is $I= V/(R_S+R_G)$. We want this current to equal the full-scale-deflection current $I_{\rm fsd}$ of the galvanometer  when the potential
difference between points $a$ and $b$ attains its maximum allowed value
$V_{\rm max}$. It follows that
\begin{equation}
I_{\rm fsd} = \frac{V_{\rm max}}{R_S+R_G},
\end{equation}
which yields
\begin{equation}
R_S = \frac{V_{\rm max}}{I_{\rm fsd}} - R_G.
\end{equation}
Using this formula, we can always choose an appropriate shunt resistor to allow
a galvanometer to measure any voltage, no matter how large. For instance,
if the full-scale-deflection current is $I_{\rm fsd} = 10\,\mu{\rm A}$, 
the maximum voltage we wish to measure is $1000$\,V, and the resistance
of the galvanometer is $R_G = 40\,\Omega$, then the appropriate
shunt resistance is
\begin{equation}
R_S = \frac{1000}{1\times 10^{-5}} - 40 \simeq 10^8\,\Omega.
\end{equation}
Again, there is an advantage in making the full-scale-deflection
current of a galvanometer used as a voltmeter small, because, 
when it is properly set up, the galvanometer never
draws more current from the circuit than its full-scale-deflection current. If
this current is small then the galvanometer can measure potential differences
in a circuit without significantly perturbing the currents flowing
around that circuit. 

\begin{figure}[h]
\epsfysize=2in
\centerline{\epsffile{Chapter08/fig8.14.eps}}
\caption{\em Circuit diagram for a galvanometer measuring potential difference.}\label{f8.14}
\end{figure}

\subsection{Worked Examples}
\subsection*{\em Example 8.1: Earth's magnetic field}
\begin{figure*}[h]
\epsfysize=2.5in
\centerline{\epsffile{Chapter08/fig1.eps}}
\end{figure*}
{\em Question:}  In Texas, the
Earth's magnetic field is approximately uniform, and of magnitude
$B=10^{-4}$\,T. The horizontal component of the field is directed northward.
The field also has a vertical component which is directed into the ground. The
angle the field lines dip below the horizontal is $40^\circ$. A metal
bar of length $l=1.2$\,m carries a current of $I'=1.7$\,A. Suppose that the
bar is held horizontally such that the current flows from
East to West. What is the magnitude and direction of the magnetic
force on the bar? Suppose that the direction of the current is reversed.
What, now, 
is the magnitude and direction of the magnetic
force on the bar? Suppose that the bar is held vertically such the current
flows upward. What 
is the magnitude and direction of the magnetic
force on the bar? Suppose, finally, that the direction of the current is
reversed. What, now, is the magnitude and direction of the magnetic
force on the bar?\\
~\\
\noindent{\em Answer:} If the current in the bar flows horizontally
from East to West then the direction
of the current makes an angle of $90^\circ$ with the direction of the
magnetic field. So, from Eq.~(\ref{e8.1}), the magnetic  force {\em per unit length}
acting on the bar is
$$
F = I'\,B\,\sin 90^\circ = I'\,B = (1.7)\, (10^{-4})= 1.7\times 10^{-4}\,{\rm N\,m}^{-1}.
$$
Thus, the total force acting on the bar
is
$$
f = F\,l = (1.7\times 10^{-4})\,(1.2) = 2.04\times 10^{-4}\,{\rm N}.
$$
Using the right-hand rule, if the  index finger of a right-hand points horizontally from
East to West, and the middle finger points northward, but dips $40^\circ$ below
the horizontal, then the thumb points southward, but dips $50^\circ$ below
the horizontal. Thus, the force on the bar is directed southward, and
dips $50^\circ$ below
the horizontal.
If the current in the bar is reversed, so that it now flows horizontally from West to
East, then the angle subtended between the direction of current flow and
the direction of the magnetic field is still $90^\circ$, so the magnitude
of the force on the bar remains unchanged. According to the right-hand rule,
if the index finger of a right-hand   points horizontally from
West to East,  and the middle finger points northward, but dips $40^\circ$ below
the horizontal, then the thumb points northward, but is directed $50^\circ$ above
the horizontal.  Thus, the force on the bar is directed northward at an angle of
$50^\circ$ above
the horizontal. In other words, the new force points in exactly the opposite
direction to the old one. 

If the current in the bar flows vertically upward then the direction of
current flow subtends an angle of $50^\circ$ degrees with the direction
of the magnetic field. So, from Eq.~(\ref{e8.1}), the magnetic  force {\em per unit length}\/
acting on the bar is
$$
F = I'\,B\,\sin 50^\circ= (1.7)\, (10^{-4})\,(0.7660)
= 1.30\times 10^{-4}\,{\rm N\,m}^{-1}.
$$
Thus, the total force acting on the bar
is
$$
f = F\,l = (1.30\times 10^{-4})\,(1.2) = 1.56\times 10^{-4}\,{\rm N}.
$$
Using the right-hand rule, if the index finger of a right-hand   points vertically
upward,  and the middle finger points northward, but dips $40^\circ$ below
the horizontal, then the thumb points horizontally westward. Thus, the force on the bar is directed horizontally westward. If the current in the bar is reversed, so that
it flows vertically downward, then the force on the bar is of the same magnitude, but points in the opposite direction, which means that the new force
points horizontally eastward.


\subsection*{\em Example 8.2: Charged particle in magnetic field}
{\em Question:} Suppose that an electron is accelerated from
rest through a voltage
difference of $V=10^3$\,volts and then passes into a region containing a uniform
magnetic field of magnitude $B=1.2$\,T. The electron subsequently executes a
closed  circular
orbit in the plane perpendicular to the field. What is the radius of this
orbit? What is the angular frequency of gyration of the electron?\\
~\\
\noindent{\em Answer:} If an electron of mass $m_e
=9.11\times 10^{-31}$\,kg and charge $e=1.60\times 10^{-19}$\,C is accelerated
from rest through a potential difference $V$ then its final
kinetic energy is 
$$
\frac{1}{2}\,m_e\,v^2 = e\,V.
$$
Thus, the final velocity $v$ of the electron is given by
$$
v = \sqrt{\frac{2\,e\,V}{m_e}} = \sqrt{\frac{(2)\,(1.6\times 10^{-19})\,
(10^3)}{(9.11\times 10^{-31})}} = 1.87\times 10^{7}\,{\rm m\,s}^{-1}.
$$
The initial direction of motion of the electron is at right-angles
to the direction of the magnetic field, otherwise the orbit of the electron 
would be a spiral instead of a closed circle. Thus, we can use Eq.~(\ref{e8.17})
to calculate the radius $\rho$ of the orbit. We obtain
$$
\rho = \frac{m_e\,v}{e\,B} = \frac{ (9.11\times 10^{-31})\,( 1.87\times 10^{7})}
{(1.6\times 10^{-19})\,(1.2)} = 8.87\times 10^{-5}\,{\rm m}.
$$
The angular frequency of gyration $\omega$ of the electron comes from Eq.~(\ref{e8.18}):
$$
\omega = \frac{e\,B}{m_e} = \frac{ (1.6\times 10^{-19})\,(1.2)}{(9.11\times 10^{-31})}
= 2.11\times 10^{11}\,{\rm rad.\,s}^{-1}.
$$


\subsection*{\em Example 8.3: Amp\`{e}re's circuital law}
{\em Question:} A $z$-directed wire of radius $a$ carries a total
$z$-directed current $I$.
What is the magnetic field distribution, both inside and outside the wire, if the current
is evenly distributed throughout the wire? What is the magnetic field 
distribution if the current is
concentrated in a thin layer at the surface of the wire?\\
~\\
\noindent{\em Answer:} Since the current distribution possesses 
cylindrical symmetry, it is reasonable to suppose that the magnetic field
it generates also possesses cylindrical symmetry. By analogy with the magnetic
field generated by an infinitely thin $z$-directed wire, we expect the
magnetic field to circulate in the $x$-$y$ plane in an anti-clockwise direction
(looking against the direction of the current). Let us apply Amp\`{e}re's circuital
law to a circular loop in the $x$-$y$ plane which is centred on the centre of
the wire, and is of radius $r>a$. The magnetic field is everywhere tangential
to the loop, so the line integral of the magnetic field (taken
in an anti-clockwise sense, looking against the direction of the current) is
$$
w(r) = 2\pi \,r\,B(r),
$$
where $B(r)$ is the magnetic field-strength at radius $r$. According to Amp\`{e}re's
circuital law, this line integral is equal to $\mu_0$ times the total current enclosed by
the loop. The total current is clearly $I$, since the loop lies outside the wire.
Thus,
$$
w(r) = 2\pi \,r\,B(r) = \mu_0\,I,
$$
giving
$$
B(r) = \frac{\mu_0\,I}{2\pi\,r}
$$
for $r>a$. This is exactly the same field distribution as that generated by an
infinitely thin wire carrying the current $I$. Thus, we conclude that the magnetic field generated outside
a cylindrically symmetric $z$-directed current distribution is the same as if all of the
current were concentrated at the centre of the distribution. 
Let us now apply  Amp\`{e}re's circuital
law to a circular loop which  is of radius $r<a$. The line integral
of the magnetic field around this loop is simply $w(r) = 2\pi \,r\,B(r)$.
However, the current enclosed by the loop is equal to $I$ times the ratio of
the area of the loop to the cross-sectional area of the wire (since the
current is evenly distributed throughout the wire). Thus,  Amp\`{e}re's 
law yields
$$
2\pi \,r\,B(r) = \mu_0\,I\,\frac{r^2}{a^2},
$$
which gives
$$
B(r) = \frac{\mu_0\,I\,r}{2\pi\,a^2}.
$$
Clearly, the field inside the wire increases linearly with increasing distance
from the centre of the wire.

If the current flows along the outside of the wire then the magnetic
field distribution exterior to the wire is exactly the same as that
described above. However, there is no field inside the wire. This
follows immediately from Amp\`{e}re's circuital
law because the current enclosed by a circular loop whose radius is less than
the radius of the wire is clearly zero.
