\section{Electricity}\label{scharge}
\subsection{Historical Introduction}
We usually associate electricity with the 20th Century, during which it
revolutionized the lives of countless millions of ordinary people, in much the
same manner as steam power revolutionized lives in the 19th Century. 
It is, therefore, somewhat surprising to learn that  people have known about 
electricity for many
thousands of years. In about 1000 BC, the ancient Greeks started to navigate the
Black Sea, and opened up trade routes, via the river Dnieper, to the
Baltic region. Amongst the many trade items that the Greeks obtained from the Baltic
was a substance which they called ``electron'' (\gr{>hl'ektron}), but which we nowadays call {\em amber}. Amber is fossilized pine resin, and was used
by the Greeks, much  as it is used today, as a gem stone. However, in about 600 BC, the ancient Greek philosopher Thales of Miletus
discovered that amber possesses a rather peculiar property: {\em i.e.}, when it is rubbed with
 fur, it develops the ability to attract light objects, such as
feathers. For many centuries, this strange phenomenon was thought to be a  unique
property of amber.

In Elizabethan times,   the English physician William Gilbert  coined the word  ``electric''  (from the Greek
word for amber) to describe the above mentioned effect. It was later found  that many materials become electric when rubbed with certain other materials.
In 1733, the French chemist du Fay discovered that there are, in fact, {\em two}\/ different types of electricity.
When amber is rubbed with fur, it acquires so-called  ``resinous electricity.'' 
On the other hand, when glass is rubbed with silk, it
acquires so-called  ``vitreous electricity.'' Electricity
repels electricity of the same kind, but  attracts electricity of the opposite kind. At the time, it was thought that  electricity was created by friction.

Scientists in the 18th Century eventually 
 developed the concept of {\em electric charge}\/ in order to account for a large body
of 
observations made in countless electrical experiments. 
There are two types of charge: {\em positive}\/ (which is the same as vitreous), and {\em negative}\/ (which is the same as resinous). Like
charges repel one another, whilst opposite charges attract. When two bodies are
rubbed together, charge can be transfered from one to the other, but the total
charge remains constant. Thus, when amber is rubbed with fur, there is transfer of
charge such that the amber acquires a negative charge, and the fur an equal
positive charge. Likewise, when glass is rubbed with silk, the glass acquires
a positive charge, and the silk an equal negative charge. The idea that
electrical charge is a conserved quantity is attributed to the American scientist
Benjamin Franklin (who is also to blame for the unfortunate sign convention in electricity). The {\em law of  charge conservation}\/ can be written:
\begin{quote}
{\sf In any closed system, the total electric charge remains constant.}
\end{quote}
Of course, when  summing  charge, positive charges are represented as
positive numbers, and negative charges as negative numbers. 

In the 20th Century, scientists discovered that the atoms out of which ordinary matter is
composed consist of two components: a relatively massive, positively charged
nucleus, surrounded by a cloud of relatively light, negatively 
charged particles called {\em electrons}. Electrons and atomic
nuclii carry fixed electrical charges, and are essentially indestructible (provided that we neglect  nuclear reactions). 
Under normal circumstances, only 
the electrons are mobile. Thus, when amber is rubbed with fur, electrons
are transferred from the fur to the amber, giving the amber an excess of
electrons, and, hence, a negative charge, and  the fur a deficit of
electrons, and, hence, a positive charge. Substances normally
contain neither an excess nor a deficit of electrons, and are, therefore,
electrically neutral. 

The SI unit of electric charge is the {\em coulomb} (C). The charge on an
electron is $-1.602\times 10^{-19}$~C.

\subsection{Conductors and Insulators}
Suppose that we were to electrically charge two isolated metal spheres: one with a positive
charge, and the other with an equal negative charge. We could then perform
a number of simple experiments. For instance, we could connect the spheres 
together using
a length of string. In this case, we would find that the charges residing on
the two spheres were unaffected. Next, we could connect the spheres using a
copper wire. In this case, we would find that there was no charge
remaining on either sphere. Further investigation would reveal that charge
must have {\em flowed}\/ through the wire, from one sphere to the other, such that the
positive charge on the first sphere completely canceled the negative charge on the
second, leaving zero charge on either sphere. Substances can be classified
into two main groups, depending on whether they allow 
the free flow of electric charge. {\em Conductors}\/ allow charge to pass freely
through
them, whereas {\em insulators}\/ do not. Obviously, string is an insulator,
and copper is a conductor. As a general rule, substances which are good
conductors of heat are also good conductors of electricity. Thus,
all metals are conductors, whereas air, (pure) water, plastics, glasses,
and ceramics
are insulators. Incidentally, the distinction between conductors and
insulators was first made by the English scientist Stephen Gray in 1729.

Metals are good conductors (both of heat and electricity) because at least
one electron per atom is {\em free}: {\em i.e.}, it is not tied to
any particular atom, but is, instead, able to move freely throughout the
metal. In good insulators, such as glass, all of the electrons are tightly
bound to  atoms (which are fixed), and so there are no free electrons.

\subsection{Electrometers and Electroscopes}
Electric
charge is measured using a device called an {\em electrometer}, which consists
of a metal  knob connected via a conducting shaft to a flat, vertical metal
plate. A very light gold leaf, hinged at the top, is attached to the plate.
Both the plate and the gold leaf are enclosed in a  glass vessel to protect the delicate
leaf from air currents. When charge is deposited on the knob, some fraction
is conducted to the plate and the gold leaf, which consequently
repel one another, causing the  leaf to pull away from the plate. 
The angular deflection of the gold leaf with respect to the  plate is
proportional to the charge deposited on the knob. An electrometer can be
calibrated in such a manner that the angular deflection of the gold leaf 
can be used to calculate the absolute magnitude of the charge deposited on the knob.

An {\em electroscope}\/ is a somewhat cruder charge measuring device than an
electrometer, and  consists of two gold leaves, hinged at the
top,  in place of the metal plate and
the single leaf. When the knob is charged, the two leaves also become charged
and repel one another, which causes them to move apart. The mutual deflection
of the leaves can
be used as a rough measure of the amount of electric charge deposited on the knob. 

\subsection{Induced Electric Charge}\label{s3.4}
We have seen how an electroscope can be used to measure the absolute magnitude
of an electric charge. But, how can we determine the sign of the charge? In fact, this
is fairly straightforward. Suppose that an electroscope carries a charge 
of unknown sign. Consider what happens when we bring a negatively
charged amber rod, produced by rubbing the rod with fur, close to the knob of the
electroscope. The excess electrons in the rod repel the free electrons in the
knob and shaft of the electroscope. The repelled electrons move as far away
from the rod as possible, ending up in the gold leaves. Thus, the charge on the
leaves becomes more negative. If the original charge on the electroscope is
negative then the magnitude of the charge on the leaves increases in the
presence of the rod, and the leaves consequently move further apart. On the
other hand, if the original charge on the electroscope is
positive then the magnitude of the charge on the leaves decreases in the presence of the rod,
 and the leaves consequently move closer together. The general rule is that
the deflection of the leaves increases when a charge of the same sign is
brought close to the knob of the electroscope, and {\em vice versa}. The sign of the charge on 
an electroscope can easily be
determined in this manner.

Suppose that we bring a negatively charged rod close to the knob of an
uncharged electroscope. The excess electrons in the rod repel the free electrons
in the knob and shaft of the electroscope so that they collect in the gold
leaves, which, therefore, move apart.  It follows that  whenever a charged object is brought
close to the knob of an uncharged electroscope, the electroscope registers a
charge. Thus, an uncharged electroscope can be used to {\em detect}\/ electric charge
residing on nearby objects, without disturbing that charge. 

Suppose that we bring a negatively charged rod close to the knob of an uncharged
electroscope which is attached, via a conducting wire, to a large uncharged
conductor.  The excess electrons in the rod repel the free electrons
in the knob and shaft of the electroscope. The repelled electrons move as far away
from the rod as possible, which means that they flow down the wire into
the external conductor. Suppose that we disconnect the wire and then remove the
charged rod. By disconnecting the wire we have stranded the electrons which were
repelled down the wire on the external conductor. Thus, the electroscope, which
was initially uncharged, acquires a deficit of electrons. In other words,
the electroscope becomes positively charged. Clearly, by bringing a charged object
close to an uncharged electroscope, transiently  connecting the electroscope to
a large uncharged conductor, and then removing the object, we can {\em induce}
a charge of the opposite sign on the electroscope without affecting  the
charge on the object. This process is called charging by {\em induction}. 

But where are we going to find a large uncharged conductor?
Well, it turns out that we standing on one. 
The ground ({\em i.e.}, the Earth)
 is certainly large, and it turns out that it is also a reasonably
good electrical conductor. Thus, we can inductively charge an electroscope by transiently
connecting it to the ground ({\em i.e.},  ``grounding'' or
``earthing'' it) whilst it is in the
presence of a charged object. The most effective way of earthing  an
object is to connect it to a conducting wire which is attached, at the other end,
to a metal stake driven into the ground. A somewhat less effective way of
grounding an object is simply to touch it. It turns out that we are sufficiently
good electrical conductors that charge can flow though us to the ground. 

Charges can also be induced on good insulators, although to nothing like the
same extent that they can be induced on good conductors. Suppose that a negatively
charged amber rod is brought close to a small piece of paper (which is an insulator).
The excess electrons on the rod repel the electrons in the atoms which
make up the paper, but attract the positively charged nuclei. Since paper is an
insulator, the repelled electrons are not free to move through the paper. 
Instead, the atoms in 
the paper {\em polarize}: {\em i.e.}, they distort in such a manner
 that their nuclei move slightly towards, and their electrons slightly
away from, the rod. The electrostatic force of attraction between the excess electrons in the rod
and the atomic nuclei in the paper is slightly greater than the repulsion
between the electrons in the rod and those in the paper, since the electrons in the
paper are, on average, slightly further away from the rod than the nuclei (and
the force of electrostatic attraction falls off with increasing distance). Thus,
there is a net {\em attractive}\/ force between the rod and the paper. In fact, if the
piece of paper is sufficiently light then it can actually be picked up using  the
 rod. In summary, whenever a charged object is brought close to an
insulator, the atoms in the insulator polarize, resulting in
a net attractive force between the object and the insulator. This effect is
used commercially to  remove soot particles from the exhaust plumes of 
coal-burning power stations. 

\subsection{Coulomb's Law}
The first precise  measurement of the force between two electric charges was
performed by the French scientist Charles-Augustin de Coulomb in 1788. Coulomb concluded that: 
\begin{quote}{\sf The electrical force between two
charges at rest is directly proportional to the product of the charges, and
inversely proportional to the square of the distance between the charges}
\end{quote}
This law of force is nowadays known as {\em Coulomb's law}. 
Incidentally, an electrical force exerted between two stationary charges
is known as an {\em electrostatic}\/ force. 
In algebraic
form, Coulomb's law is written
\begin{equation}
f = \frac{q\,q'}{4\pi\epsilon_0\,r^2},
\end{equation}
where $f$ is the magnitude of the
force, $q$ and $q'$ are the magnitudes of the two charges
(with the appropriate signs), and $r$ is the distance between the two charges.
The force is repulsive if $f>0$, and attractive if $f<0$.
The universal constant
\begin{equation}
\epsilon_0 = 8.854\times 10^{-12}\,\,{\rm N^{-1}\, m^{-2}\,C^2}
\end{equation}
is called the {\em permittivity of free space} or the {\em permittivity
of the vacuum}. We can also write Coulomb's law in the form
\begin{equation}
f = k_e\, \frac{q \,q'}{r^2},
\end{equation}
where the constant of proportionality $k_e=1/4\pi\epsilon_0$ takes the value
\begin{equation}
k_e = 8.988\times 10^9\,\,{\rm N\,m^2\,C^{-2}}.
\end{equation}

Coulomb's law has an analogous form to Newton's law of gravitation, 
\begin{equation}
f = -G\,\frac{m\,m'}{r^2},
\end{equation}
with
electric charge playing the role of mass. One major difference
between the two laws is the sign of the force. 
The electrostatic force between two
like charges is repulsive ({\em i.e.}, $f>0$), whereas that between two
unlike charges is attractive ({\em i.e.}, $f<0$). On the other hand, the
gravitational force between two masses is always attractive (since there is
no such thing as a negative mass). Another major difference is the relative magnitude
of the two forces. For  instance, the electrostatic repulsion between two electrons is
approximately $10^{42}$ times larger than the corresponding
gravitational attraction. 

The electrostatic force ${\bf f}_{ab}$
exerted by a charge $q_a$ on a second charge $q_b$, located
a distance $r$ from the first charge, has the  magnitude 
\begin{equation}
f = \frac{q_a\,q_b}{4\pi\epsilon_0\,r^2}, 
\end{equation}
and is directed {\em radially away}\/ from the first charge if $f>0$, and {\em radially
towards}\/ it if $f<0$. The force 
${\bf f}_{ba}$ exerted by the second charge
on the first is equal and opposite to ${\bf f}_{ab}$, 
so that
\begin{equation}
{\bf f}_{ba} = - {\bf f}_{ab},
\end{equation}
in accordance with Newton's
third law of motion. 

Suppose that we have three point charges, $q_a$, $q_b$, and $q_c$. 
It turns out that electrostatic
forces are {\em superposable}. That is, the force ${\bf f}_{ba}$ exerted by 
$q_b$ on $q_a$ is completely unaffected by the presence of $q_c$. Likewise, the
force ${\bf f}_{ca}$ exerted by $q_c$ on $q_a$ is unaffected by the presence of $q_b$. 
Thus, the net force ${\bf f}_a$ acting on $q_a$ is the {\em resultant} of these two
forces: {\em i.e.},
\begin{equation}
{\bf f}_a = {\bf f}_{ba} + {\bf f}_{ca}.
\end{equation}
This rule can be generalized in a straightforward manner
to the case where there are more than three point
charges. 

\subsection{Electric Fields}
According to Coulomb's law, a charge $q$ exerts a force on a second charge $q'$,
and {\em vice versa}, even in a vacuum. But, how is this force
transmitted through empty space? In order to answer this question, physicists in the
19th Century 
developed the concept of an {\em electric field}. The idea is as follows. The
charge $q$ generates an electric field ${\bf E}({\bf r})$ which fills space.
The electrostatic force exerted on the second charge $q'$ is actually produced locally by the 
electric field ${\bf E}$ at the position of this charge, in accordance with Coulomb's law. Likewise, the charge $q'$ generates its
own electric field ${\bf E}'({\bf r})$ which also fills space. The equal and opposite reaction
force exerted on $q$ is  produced locally by the electric field ${\bf E}'$ at the
position of this charge, again, in accordance with Coulomb's law. Of course, an electric field
cannot exert a force on the charge which   generates it,
 in just the same way as we cannot pick ourselves  up with our own shoelaces. Incidentally, electric fields have a real physical existence, and are not just theoretical constructs invented by physicists to get around the
problem of the transmission of electrostatic
forces through vacuums. 
We can say this with certainty  because, as we shall see later, there is an {\em energy}\/
associated with
an electric field filling space. Indeed, it is actually possible to convert this energy into
heat or work, and {\em vice versa}. 

The electric field ${\bf E}({\bf r})$ generated by a set of fixed electric charges is a vector field which is defined as follows.
If ${\bf f}({\bf r})$ is the electrostatic force experienced by some small positive
test charge $q'$ located at a certain point ${\bf r}$ in space, then the electric field at
this point is simply the force divided by the magnitude of the test
charge. In other words,
\begin{equation}
{\bf E} = \frac{\bf f}{q'}.
\end{equation}
Electric field  has dimensions of force per unit charge, and
units of newtons per coulomb (${\rm N \,C}^{-1}$). Incidentally, the reason
that we specify a small, rather than a large,
 test charge is to avoid disturbing any of 
the  fixed charges
which generate the electric  field.

Let us use the above rule to reconstruct the electric field generated by
a point charge $q$. According to Coulomb's law, the electrostatic force
exerted by a point charge $q$ on a positive test charge $q'$, located a distance
$r$ from it, has the magnitude
\begin{equation}
f = \frac{q\,q'}{4\pi\epsilon_0\,r^2},
\end{equation}
and is directed radially away from the former charge if $q>0$, and radially
towards it if $q<0$. Thus, the electric field a distance $r$
away from a charge $q$ has the magnitude
\begin{equation}
E= \frac{q}{4\pi\epsilon_0\,r^2},
\end{equation}
and is directed radially away from the charge if $q>0$, and radially towards
the charge if $q<0$. Note that the field is independent of the magnitude
of the test charge. 

A corollary of the above definition of an electric field is that  a stationary charge
$q$  located in an electric field ${\bf E}$ experiences an electrostatic force
\begin{equation}\label{e3.12}
{\bf f} = q \,{\bf E},
\end{equation}
where ${\bf E}$ is the electric field at the location of the charge
(excluding the field produced by the charge itself). 

Since electrostatic forces are superposable, it follows that electric fields are also superposable.
For example, if we have three stationary
point charges, $q_a$, $q_b$, and $q_c$, located at three different points in space,
then the net electric field which fills space is simply the vector sum of the fields produced
by each point charge taken in isolation. 

\subsection{Worked Examples}
\subsection*{\em Example~3.1: Electrostatic force between three colinear point charges}
\noindent{\em Question:}~ A particle of charge $q_1=+6.0\,\mu{\rm C}$ is located
on the $x$-axis at coordinate $x_1=5.1\,{\rm cm}$. A second particle of
charge $q_2=-5.0\,\mu{\rm C}$ is placed on the $x$-axis at $x_2=-3.4\,{\rm cm}$. What is
the magnitude and direction of the total electrostatic force acting on a third particle
of charge $q_3=+2.0\,\mu{\rm C}$ placed at the origin ($x=0$)?\\
~\\
\noindent{\em Solution:}~The force $f$ acting  between charges 1 and 3 is
given by
$$
f = k_e\, \frac{q_1 \,q_3}{x_1^{~2}} = (8.988\times 10^9)\,\frac{(6\times 10^{-6})\,(2\times
10^{-6})}{(5.1\times 10^{-2})^2} = 
+41.68\,{\rm N}.
$$
Since $f>0$, the force is repulsive. This means that the force $f_{13}$ exerted by charge
1 on charge 3 is directed along the $-x$-axis 
({\em i.e.}, from charge 1 towards charge 3), and is of magnitude $|f|$. Thus,
$f_{13} = - 41.69\,{\rm N}$. Here, we adopt the convention that forces directed
along the $+x$-axis are positive, and {\em vice versa}. The force $f'$ acting between charges 2 and 3
is given by
$$
f' = k_e\, \frac{q_2 \,q_3}{|x_2|^{2}} = (8.988\times 10^9)\,\frac{(-5\times 10^{-6})\,(2\times
10^{-6})}{(3.4\times 10^{-2})^2} = 
-77.75\,{\rm N}.
$$
Since $f'<0$, the force is attractive. This means that the force $f_{23}$ exerted
by charge 2 on charge 3 is directed along the $-x$-axis ({\em i.e.}, from
charge 3 towards charge 2), and is of magnitude $|f'|$. Thus,
$f_{23} = - 77.75\,{\rm N}$. 

The resultant force $f_3$ acting on charge 3 is the
algebraic sum of the forces exerted by charges 1 and 2 separately (the sum is algebraic
because all the 
forces act along the $x$-axis). It follows that
$$
f_3 = f_{13} + f_{23} = - 41.69 - 77.75 =-119.22\,{\rm N}.
$$
Thus, the magnitude of the total force acting on charge 3 is
$119.22\,{\rm N}$, and the force is directed along the $-x$-axis (since $f_3<0$).

\subsection*{\em Example~3.2: Electrostatic force between three non-colinear point charges}
\begin{figure*}[b]
\epsfysize=2in
\centerline{\epsffile{Chapter03/fig1.eps}}
\end{figure*}
\noindent{\em Question:}~Suppose that three point charges, $q_a$, $q_b$, and $q_c$, are arranged at the
vertices of a right-angled triangle, as shown in the diagram. What is the magnitude and
direction of the electrostatic force acting on the third charge if
$q_a=-6.0\,\mu{\rm C}$, $q_b=+4.0\,\mu{\rm C}$, $q_c = +2.0\,\mu{\rm C}$,
$a=4.0$\,m, and $b=3.0$\,m?\\
~\\
\noindent{\em Solution:}~The magnitude $f_{ac}$ of the force ${\bf f}_{ac}$ exerted
by charge $q_a$ on charge $q_c$ is given by
$$
f_{ac} = k_e\,\frac{|q_a|\, q_c}{c^2}= (8.988\times 10^9)\,\frac{(6\times 10^{-6})\,(2\times 10^{-6})}
{(4^2+3^2)} = 4.31 \times 10^{-3} \,{\rm N},
$$
where use has been made of the Pythagorean theorem.
The force is attractive (since charges $q_a$ and $q_c$ are of opposite sign). Hence,
the  force is directed from charge $q_c$ towards charge $q_a$, as shown in the diagram.
The magnitude $f_{bc}$ of the force ${\bf f}_{bc}$ exerted
by charge $q_b$ on charge $q_c$ is given by
$$
f_{bc} = k_e\,\frac{q_b q_c}{b^2}= (8.988\times 10^9)\,\frac{(4\times 10^{-6})\,(2\times 10^{-6})}
{(3^2)} = 7.99 \times 10^{-3} \,{\rm N}.
$$
The force is repulsive (since charges $q_b$ and $q_c$ are of the same sign). Hence,
the force is directed from charge $q_b$ towards charge $q_c$, as shown in the diagram.
Now, the net force acting on charge $q_c$ is the sum of ${\bf f}_{ac}$ and ${\bf f}_{bc}$.
Unfortunately, since ${\bf f}_{ab}$ and ${\bf f}_{bc}$ are vectors pointing in {\em different}\/
directions, they {\em cannot}\/ be added together algebraically.  Fortunately, however,
their components along the $x$- and $y$-axes {\em can}\/ be added algebraically.
Now, it is clear, from the diagram, that ${\bf f}_{bc}$ is directed along the
$+x$-axis. If follows that
\begin{eqnarray}
f_{bc\,x} &=& f_{bc} =  7.99 \times 10^{-3} \,{\rm N},\nonumber\\[0.5ex]
f_{bc\,y} &=& 0.\nonumber
\end{eqnarray}
It is also clear, from the diagram, that ${\bf f}_{ac}$ subtends an angle
$$
\theta =\tan^{-1} (a/b) = \tan^{-1}(4/3) = 53.1^\circ
$$
with the $-x$-axis, and an angle $90^\circ-\theta$ with the $+y$-axis.
It follows from the conventional laws of vector projection that
\begin{eqnarray}
f_{ac\,x} &=& -f_{ac}\,\cos\theta =  -(4.31 \times 10^{-3})\,(0.6)=-2.59\times 10^{-3}\,{\rm N},
\nonumber\\[0.5ex]
f_{ac\,y} &=& f_{ac}\,\cos(90^\circ -\theta) = f_{ac} \,\sin\theta = (4.31 \times 10^{-3})\,(0.8) = 3.45 \times 10^{-3}\,{\rm N}.
\nonumber
\end{eqnarray}

The $x$- and $y$-components of the resultant force ${\bf f}_c$ acting on charge $q_c$ are given by
\begin{eqnarray}
f_{c\,x} &=&f_{ac\,x} + f_{bc\,x} =  -2.59\times 10^{-3} +  7.99 \times 10^{-3} =5.40\times 
10^{-3}\,{\rm N},\nonumber\\[0.5ex]
f_{c\,y} &=& f_{ac\,y} + f_{bc\,y} = 3.45 \times 10^{-3}\,{\rm N}.\nonumber
\end{eqnarray}
Thus, from the Pythagorean theorem, the magnitude of the resultant force is
$$
f_c = \sqrt{(f_{c\,x})^2 + (f_{c\,y})^2} = 6.4\times 10^{-3}\,{\rm N}.
$$
Furthermore, the resultant force subtends an
angle
$$
\phi = \tan^{-1} (f_{c\,y}/f_{c\,x}) = 32.6^\circ
$$
with the $+x$-axis, and an angle $90^\circ-\phi = 57.4^\circ$ with the $+y$-axis.

\subsection*{\em Example 3.3: Electric field generated by two point charges}
\begin{figure*}[h]
\epsfysize=1in
\centerline{\epsffile{Chapter03/fig2.eps}}
\end{figure*}
\noindent{\em Question:} Two point charges, $q_a$ and $q_b$, are separated by a distance
$c$. What is the electric field at a point halfway between the charges? 
What force would be exerted on a third charge $q_c$ placed at this point?
Take $q_a=50\,\mu{\rm C}$, $q_b = 100\,\mu{\rm C}$, $q_c=20\,\mu{\rm C}$,
and $c=1.00$ m.\\
~\\
\noindent{\em Solution:} Suppose that the line from $q_a$ to $q_b$ runs along
the $x$-axis. It is clear, from Coulomb's law, that the electrostatic
force exerted on any charge placed on this line is parallel to the $x$-axis.
Thus, the electric field at any point along  this line must also be aligned
along the $x$-axis. Let the $x$-coordinates of charges $q_a$ and $q_b$ be $-c/2$ and $+c/2$,
respectively. It follows that the origin ($x=0$) lies
halfway between the two charges. The electric field $E_a$ generated by charge $q_a$ at the
origin is given by
$$
E_a = k_e\,\frac{q_a}{(c/2)^2} = (8.988\times 10^9)\, \frac{(50\times 10^{-6})}{(0.5)^2}
= 1.80\times 10^6\,{\rm N\,C}^{-1}.
$$
The field is positive because it is directed along the $+x$-axis ({\em i.e.}, from
charge $q_a$ towards the origin). The electric field $E_b$ generated by charge $q_b$ at the
origin is given by
$$
E_b = -k_e\,\frac{q_b}{(c/2)^2} = -(8.988\times 10^9) \,\frac{(100\times 10^{-6})}{(0.5)^2}
= -3.60\times 10^6\,{\rm N\,C}^{-1}.
$$
The field is negative  because it is directed along the $-x$-axis ({\em i.e.}, from
charge $q_b$ towards the origin). The resultant field $E$ at the origin is the
algebraic sum  of $E_a$ and $E_b$ (since all fields are directed along the $x$-axis).
Thus,
$$
E = E_a + E_b =  -1.8 \times 10^6\,{\rm N\,C}^{-1}.
$$
Since $E$ is negative, the resultant field is directed along the $-x$-axis.

The force $f$ acting on a charge $q_c$ placed at the origin is simply
$$
f = q_c\,E = (20\times 10^{-6})\,(-1.8\times 10^6)=-36 \,{\rm N}.
$$
Since $f<0$, the force is directed along the $-x$-axis.


