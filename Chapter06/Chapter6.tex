\section{Capacitance}
\subsection{Charge Storage}
Consider a hollow metal sphere mounted on an insulating stand. The sphere is
 initially grounded so that no excess charge remains on it. Suppose that 
we introduce a metal ball,  suspended on an insulating thread, 
through a small hole in the sphere,
and then fill in the hole with a metal plug. Let the ball carry a charge
$+Q$. What distribution of charge is induced on the hollow sphere as a result
of introducing the positive  charge into the cavity?

To answer this question we make use of Gauss' law (see Sect.~\ref{s4.2})
\begin{equation}
{\mit\Phi}_E= \oint {\bf E}\cdot d{\bf S} = \frac{Q}{\epsilon_0}.
\end{equation}
Assuming that the metal ball is placed at the centre of the hollow sphere,
we can use symmetry arguments to deduce that the electric field 
depends only on the radial distance $r$ from the centre,
and is everywhere directed radially away from the ball. Let us choose a
spherical gaussian surface, centred on the ball, which runs through the
interior of the hollow metal sphere.    We know that the electric field inside a
conductor is everywhere  zero (see Sect.~\ref{szero}), so the electric flux
${\mit\Phi}_E$ through the surface is also zero. It follows from Gauss' law that
zero net charge is enclosed by the surface. Now,  there is a
charge $+Q$ on the ball at the centre of the hollow sphere, so there must  be an equal and
opposite 
 charge $-Q$ distributed over the interior surface of the sphere
(recall that any charge carried on a conductor must reside on its surface). 
Furthermore, since the sphere is insulated, and was initially uncharged, a
charge $+Q$ must be distributed over its exterior surface. Thus, when
the charge $+Q$ is introduced into the centre of the sphere, there is
a redistribution of charge in the sphere such that a positive charge $+Q$
is repelled to its exterior surface, leaving a negative charge $-Q$
on the interior surface. (In actuality, free electrons are attracted to
the interior surface, exposing positive charges on the exterior surface).
Further use of Gauss' law shows that the electric field between the charged ball and
the interior surface of the sphere is the same as that generated by a point
charge 
$+Q$ located at the centre of the sphere. Likewise, for the electric
field exterior to the sphere. The electric field inside the conducting
sphere is, of course,  zero.

Suppose, finally,  that the ball is moved so that it touches the inside of the hollow
sphere. The charge $-Q$ on the interior surface of the sphere cancels the
charge $+Q$ on the ball, leaving the charge $+Q$ distributed over 
its exterior surface. Thus, the effect of touching the
ball to the inside of the sphere is to transfer the charge $+Q$ from
the ball to the exterior surface of the sphere. In principle, we
can repeat this process, again and again, until a very large amount of charge
is accumulated on the outside of the
sphere. The idea of transferring charge from one conductor
to another by means of internal contact is the theoretical
 basis of the {\em Van de Graaff
generator}. In this type of device,  charge is continuously transmitted to a conducting sphere by
means of a moving belt charged by friction. 

\subsection{Breakdown}
Is there any practical limit to the charge $Q$ which can be accumulated on the conducting
sphere of a Van de Graaff generator? Well, we know that the field outside the sphere
is just the same as if the charge $Q$ were placed at the centre of the sphere. 
In fact,
the electric field is at its most intense just above the surface of the sphere, where
it has the magnitude $E = Q/(4\pi\epsilon_0\,a^2)$.  Here, $a$ is the radius of
the sphere.  Air (assuming that the sphere is surrounded by air)
is generally a very good insulator. However, air ceases to be an insulator
when the electric field-strength through it exceeds some critical value
which is about $E_{\rm crit} \sim 10^6\,\,{\rm V \,m}^{-1}$. This
phenomenon is known as {\em breakdown}, and is associated with the
formation of sparks. The explanation of breakdown is quite
straightforward. Air naturally contains a very small fraction of
ionized molecules (not enough to prevent air from being an insulator).
In an electric field, these ionized molecules are constantly 
being accelerated, and then crashing into
neutral molecules. As the strength of the field is increased, the ionized molecules
are accelerated to ever higher energies before crashing into the neutral molecules.
Eventually, a critical field-strength $E_{\rm crit}$ is reached at which  the
ionized molecules are accelerated to a sufficiently high  energy that they ionize the neutral
molecules when they hit them. At this point, a chain reaction takes place which
rapidly leads to the almost complete ionization of the air. Thus, the air makes
an almost instantaneous transition from a good insulator to a good conductor. 
It follows that the charge $Q$ on the conducting
sphere of a Van de Graaff generator can never exceed
the critical value $Q_{\rm crit} = 4\pi\epsilon_0\,a^2\, E_{\rm crit}$, because
for $Q\geq Q_{\rm crit}$ the  electric field around the sphere 
is sufficiently intense to
cause breakdown. Of course, when breakdown occurs the charge on the
sphere  is conducted to
earth. 

The phenomenon of breakdown sets an upper limit on the charge which
can be stored on a conductor. There is, however, another important factor which 
affects the onset of breakdown. This is best illustrated in the following
simple example. Suppose that we have two charged conducting spheres of radii $a$ and
$b$, respectively,  which are connected by a long conducting wire. The wire allows charge to
move back and forth between the spheres until they reach the same potential (recall
that the electric potential is uniform in a conductor). Let $Q_a$ be the
charge on the  first sphere, and $Q_b$ the charge on the second sphere. Of course,
the total charge $Q = Q_a  + Q_b$ carried by the two spheres is a conserved quantity.
The electric field generated by each sphere is the same as if the charge on that
sphere were concentrated at its centre. Assuming that the wire is sufficiently
long that the  two spheres do not
affect one another very much, the absolute potential of the first sphere
is $V_a = Q_a/(4\pi\epsilon_0\,a)$, whereas that of the second
sphere is $V_b = Q_b/(4\pi\epsilon_0\,b)$ [see Eq.~(\ref{e5.22})]. Since $V_a = V_b$, we find that
\begin{eqnarray}
\frac{Q_a}{Q} &=& \frac{a}{a+b},\\[0.5ex]
\frac{Q_b}{Q} &=& \frac{b}{a+b}.
\end{eqnarray}
Note that if the second sphere is much smaller than the first
({\em i.e.}, if $b\ll a$) then the larger sphere grabs the lion's
share of the charge:
\begin{equation}
\frac{Q_a}{Q_b} = \frac{a}{b} \gg 1.
\end{equation}
The electric field-strengths just above the surfaces of the two spheres are
$E_a = Q_a/(4\pi\epsilon_0\,a^2)$ and $E_b= Q_b/(4\pi\epsilon_0\,b^2)$,
respectively. Thus, the ratio of the field-strengths generated in the immediate
vicinities of the two spheres is 
\begin{equation}\label{e6.5}
\frac{E_b}{E_a} = \frac{Q_b}{Q_a}\frac{a^2}{b^2}=\frac{a}{b}.
\end{equation}
Clearly, if $b\ll a$ then the field just above the smaller sphere is
far stronger than that above the larger one. Suppose that  the total
charge $Q_0$ on the two spheres is gradually increased until breakdown
occurs. Since $E_b\gg E_a$, it follows that breakdown always occurs above the
smaller sphere. 

Equation~(\ref{e6.5}) is a special case of a far more general rule: 
{\em i.e.}, the electric
field-strength above some point on the surface of a conductor is inversely proportional
to the local radius of curvature of the surface. It is clear that
if we wish to store significant amounts of charge on a conductor
then the surface of the conductor must be made as smooth as
possible. Any sharp spikes on the surface  possess
relatively small radii of curvature. Intense local electric 
fields are generated above these spikes whenever the conductor is charged.
These fields can easily exceed the critical field for the breakdown
of air, leading to sparking, and the eventual loss of the charge on
the conductor. Sparking tends to be very destructive because of
its highly localized nature, which leads inevitably  to very large  electric currents,
and, hence, to  intense heating.

Clouds can acquire very large negative charges during thunderstorms. An equal
and opposite positive charge is induced on the surface of the Earth. 
The electric field generated between the clouds and the Earth can become
sufficiently large to cause breakdown in the atmosphere, giving rise to
the phenomenon which we call {\em lightning}.  Let us consider the various factors which determine where lightning strikes. Breakdown starts
at cloud level, as a so-called ``dark leader'' of ionized air traces out a path
towards the ground. When it comes within about 10 meters of  ground level, a second
dark leader comes up from the ground to meet it. Once the two leaders meet, and
a conducting path is established, the lightning strike proper occurs.
Note that, contrary to popular opinion, the lightning strike
travels {\em upwards}\/ from
the Earth to the clouds. It is clear that lightning ``strikes'' a particular
object on the ground because the object emits a dark leader: {\em i.e.},
because breakdown takes place just above the object.
In a thunderstorm, the ground, and the
objects upon it,  acts essentially like a charged conductor with a
convoluted surface.
Thus, any ``spikes'' on the ground ({\em e.g.}, a person standing in a field,
a radio mast, a lightning rod)
are comparatively more likely to be hit by lightning, because the electric field-strength
above these points is relatively large, which facilitates breakdown.

\subsection{Capacitance}
As we have seen, the amount of charge which can be stored on a conductor is limited
by the electric field-strength  just above its surface,
which is not allowed to exceed a certain critical value, $E_{\rm crit}$. 
Unfortunately, the field-strength varies from point to point across the surface (unless the
surface possesses a constant radius of curvature). It is, therefore,
 generally convenient
to parameterize the maximum field-strength above the surface of a conductor
in terms of the voltage difference $V$ between the conductor and either
infinity or another conductor.  The point  is that $V$, unlike the electric
field-strength, is a constant over the surface, and can, therefore,  be specified unambiguously. 

How do we tell the difference between a good  and a
bad  charge storage device? Well, a good charge storage device must be capable of
storing a large amount of charge without causing breakdown. Likewise,
a bad charge storage device is only capable of storing a small amount of
charge before breakdown occurs. Thus, if we place a charge $Q$
in a good storage device then the electric fields generated just above the surface of
the device should be comparatively weak. In other words, the voltage $V$
should be relatively small. A convenient measure of the ability of a device
to store electric charge is its {\em capacitance}, $C$, which is defined as the
ratio of $Q$ over $V$:
\begin{equation}\label{e6.6}
C = \frac{Q}{V}.
\end{equation}
Obviously, a good charge storage device possesses a high capacitance. Note
that the capacitance of a given charge storage device is a constant
which depends on the dimensions of the device, but is independent of either
$Q$ or $V$. This follows from the {\em linear}\/  nature of the laws of
electrostatics: {\em i.e.}, if we double the charge on the device, then we double the
electric fields generated around the device, and so we double the
voltage difference between the device and (say) infinity. In other words,
$V\propto Q$. The units of capacitance are called {\em farads}\/ (F), and 
are equivalent to coulombs per volt:
\begin{equation}
1\,{\rm F} \equiv 1\,{\rm C\,V}^{-1}.
\end{equation}
A farad is actually a pretty unwieldy
unit. In fact, most of the capacitors found in electronic circuits have capacitances  in the micro-farad
range. 

Probably the simplest type of capacitor is the so-called 
{\em parallel plate capacitor},
which consists of two parallel conducting plates, one carrying a charge $+Q$
and the other a charge $-Q$, separated by a distance $d$. Let $A$ be the area
of the two plates. It follows that the charge densities on the plates are
$\sigma$ and $-\sigma$, respectively, where $\sigma=Q/A$. Now, we have
already seen (in Sect.~\ref{s4.5}) that the electric field generated between
two oppositely charged parallel plates is uniform, and of magnitude
$E= \sigma/\epsilon_0$. The field is directed perpendicular to the
plates, and runs from the positively  to the negatively charged plate. 
Note that this result is only valid if the spacing between the plates is
much less than their typical dimensions.
According to Eq.~(\ref{e4.8}), the potential difference $V$ between the plates is
given by 
\begin{equation}
V = E \,d = \frac{\sigma\,d}{\epsilon_0} = \frac{Q\,d}{\epsilon_0\,A},
\end{equation}
where the positively charged plate is at the higher potential. It
follows from Eq.~(\ref{e6.6}) that the capacitance of a parallel plate capacitor
takes the form
\begin{equation}\label{e6.9}
C = \frac{\epsilon_0\,A}{d}.
\end{equation}
Note that the capacitance is proportional to the area of the plates, and
inversely proportional to their perpendicular spacing. It follows
that a good parallel plate capacitor possesses closely spaced plates of
large surface area. 

\subsection{Dielectrics}
Strictly speaking, the expression (\ref{e6.9}) for the capacitance of a parallel
plate capacitor is only valid if the region between
plates is a vacuum.
However, this expression turns out to be a pretty good approximation if the
region is filled with air. But, what happens if the 
region between the plates is filled  by an insulating
material such as glass or plastic? 

We could  investigate this question experimentally. 
Suppose that we started with a charged parallel plate capacitor, whose plates
were separated by a vacuum gap,  and which was disconnected from any battery
or other source of charge. We could measure the voltage difference $V_0$ between  the plates
using a voltmeter. Suppose that we 
inserted a slab of some insulating material ({\em e.g.}, glass)  into the gap between the plates, and then re-measured the
voltage difference between the plates. 
 We would find that the new voltage difference $V$ 
was {\em less}\/ than $V_0$, despite that fact that the charge $Q$ on the plates
was unchanged. Let  us denote the voltage ratio $V_0/V$ as $K$. 
Since, $C=Q/V$, it follows that
 the capacitance of the capacitor must have increased by a
factor $K$ when the insulating slab was inserted between the plates. 

An insulating
material which has the effect of increasing the capacitance of a vacuum-filled parallel
plate capacitor, when it is inserted between its plates, is called a
{\em dielectric}\/ material, and the factor $K$ by which the capacitance is
increased is called the {\em dielectric constant}\/ of that material. Of course, $K$ varies from material to material. A few
sample values are given in Table~\ref{t6.1}. Note, however, that $K$ is always greater
than unity, so filling the gap between the plates of a parallel plate
capacitor with a dielectric material always increases the capacitance of the
device to some extent. On the other hand, $K$ for air is only $0.06$ percent greater
than $K$ for a vacuum ({\em i.e.}, $K=1$), so an air-filled capacitor is virtually
indistinguishable from a vacuum-filled capacitor.
\begin{table}
\centering
\begin{tabular}{ll}\hline
Material & $K$ \\ \hline
Vacuum &1  \\
Air & 1.00059 \\
Water & 80\\
Paper & 3.5 \\
Pyrex & 4.5\\
Teflon & 2.1\\
\hline
\end{tabular}
\caption{\em Dielectric constants of various common materials.}\label{t6.1}
\end{table}

The formula for the capacitance of a dielectric-filled parallel plate capacitor
is
\begin{equation}\label{ecap}
C = \frac{\epsilon\,A}{d},
\end{equation}
where
\begin{equation}
\epsilon = K\,\epsilon_0
\end{equation}
is called the {\em permittivity} of the dielectric material between the plates.
Note that the permittivity $\epsilon$ of a dielectric 
material is always greater than
the permittivity of a vacuum $\epsilon_0$

How do we explain the reduction in voltage which occurs when we insert a
dielectric between the plates of a vacuum-filled parallel plate capacitor? 
Well, if the voltage difference between the plates is reduced then the
electric field between the plates must be reduced by the same factor. 
In other words, the electric field $E_0$ generated by the charge
stored on the capacitor plates  must be partially canceled out
by an opposing electric field $E_1$ generated  by the dielectric itself
when it is placed in an external electric field. What is the
cause of this opposing field? It turns out that the opposing field is
produced by the {\em polarization}\/ of the constituent molecules of the
dielectric when they are placed in an electric field (see Sect.~\ref{s3.4}). 
 If $E_0$ is sufficiently small then the degree of
polarization of each molecule is {\em proportional to}\/ the 
strength of the polarizing field
$E_0$. It follows that the strength of the  opposing field $E_1$ is also proportional
to $E_0$. In fact, the constant of proportionality is $1-1/K$, so
$E_1 = (1-1/K)\,E_0$. The net electric field between the plates is
$E_0 - E_1 = E_0/K$. Hence, both the field and
voltage between the plates are reduced by a factor
$K$ with respect to the vacuum case.
 In principle, the dielectric constant $K$ of a dielectric
material can be calculated from  the
properties of the molecules which make up the
material. In practice, this calculation is too difficult to perform, except
for very simple molecules. Note that the result that the degree of polarization of
a polarizable molecule is proportional to the external electric field-strength $E_0$ breaks down if $E_0$ becomes too large (just as Hooke's
law breaks down if we pull too hard on a spring). 
Fortunately, however, the field-strengths encountered in conventional
laboratory experiments are not generally large enough to invalidate this
result. 

We have seen that when a dielectric material of dielectric
constant $K$ is placed in the uniform
electric field generated between the plates of a parallel plate capacitor then
the material polarizes, giving rise to a reduction of the field-strength
between the plates by some factor $K$. Since there is nothing particularly
special about the electric field between the plates of a capacitor,
we surmise that this result is quite general. Thus, if space is filled
with a dielectric medium then Coulomb's law is rewritten as
\begin{equation}
f = \frac{q\,q'}{4\pi\epsilon\,r^2},
\end{equation}
and the formula for the electric field generated by a point charge becomes
\begin{equation}
E = \frac{q}{4\pi\epsilon\, r^2},
\end{equation}
{\em etc.} Clearly, in a dielectric medium, the laws of electrostatics
take exactly the same form as in a vacuum, except that the permittivity of
free space $\epsilon_0$ is replaced by the permittivity $\epsilon= K\,\epsilon_0$
of the medium. Dielectric materials have the general effect of
reducing the electric fields and potential differences generated by electric
charges. Such materials are extremely
useful because they inhibit breakdown. 
For instance, if we fill a parallel plate capacitor with a dielectric
material then we effectively increase the amount of charge we can store
on the device before breakdown occurs. 


\subsection{Capacitors in Series and in Parallel}
Capacitors are one of the standard components in electronic circuits. 
Moreover, complicated combinations of capacitors often occur
in practical circuits. It is,
therefore, useful to have a set of rules for finding the equivalent capacitance
of some general arrangement  of capacitors. It turns out that we can always find the
equivalent capacitance by repeated
application of {\em two} simple rules. These rules related to capacitors connected
in series and in parallel.

\begin{figure}[h]
\epsfysize=2.5in
\centerline{\epsffile{Chapter06/fig6.1.eps}}
\caption{\em Two capacitors connected in parallel.}\label{f6.1}
\end{figure}

Consider two capacitors connected in {\em parallel}: {\em i.e.}, with the
positively charged plates connected to a common ``input'' wire, and the negatively
charged plates attached to a common ``output'' wire---see Fig.~\ref{f6.1}. What is the equivalent capacitance
between the input and output wires? In this case, the potential
difference $V$ across the two capacitors is the same, and is equal to
the potential difference between the input and output wires. The total charge
$Q$, however, stored in the two capacitors is divided between the 
capacitors, since it must distribute itself such that the voltage across the
two is the same. Since the capacitors may have different capacitances, $C_1$ and $C_2$,
the charges $Q_1$ and $Q_2$ may also be different. The equivalent capacitance
$C_{\rm eq}$ of the pair of capacitors is simply the ratio $Q/V$, where
$Q=Q_1+Q_2$ is the total stored  charge. It follows that
\begin{equation}
C_{\rm eq} = \frac{Q}{V} = \frac{Q_1+Q_2}{V} = \frac{Q_1}{V} + \frac{Q_2}{V},
\end{equation}
giving
\begin{equation}\label{e6.14}
C_{\rm eq} = C_1 + C_2.
\end{equation}
Here, we have made use of the fact that the voltage $V$ is common to all three
capacitors. Thus, the rule is:
\begin{quote}
{\sf The equivalent capacitance of two capacitors connected in parallel
is the sum of the individual capacitances.}
\end{quote}
For $N$ capacitors connected in parallel, Eq.~(\ref{e6.14}) generalizes  to $C_{\rm eq} = \sum_{i=1}^N C_i$. 

\begin{figure}[h]
\epsfysize=1in
\centerline{\epsffile{Chapter06/fig6.2.eps}}
\caption{\em Two capacitors connected in series.}\label{f6.2}
\end{figure}

Consider two capacitors connected in {\em series}: {\em i.e.}, in a line such that
the positive plate of one is attached to the negative plate of the other---see
Fig.~\ref{f6.2}.
In fact, let us suppose that the positive plate of capacitor 1 is connected
to the ``input'' wire, the negative plate of capacitor 1 is connected to
the positive plate of capacitor 2, and the negative plate of capacitor
2 is connected to the ``output'' wire. 
What is the equivalent capacitance  between the input and output wires? 
In this case, it is important to realize that the charge $Q$ stored in
the two capacitors is the same. This is most easily seen by considering
the ``internal'' plates: {\em i.e.}, the negative plate of capacitor 1, and
the positive plate of capacitor 2. These plates are physically disconnected
from the rest of the circuit, so the total charge on them must
remain constant. Assuming, as seems reasonable, that these plates carry zero charge
when zero potential difference is applied across the two capacitors, it follows
that in the presence of a non-zero potential difference the charge $+Q$ on the positive
plate of capacitor 2 must be balanced by an equal and opposite charge
$-Q$ on the negative plate of capacitor 1. Since the negative plate of
capacitor 1 carries a charge $-Q$, the positive plate must carry a charge $+Q$.
Likewise, since the positive plate of capacitor 2 carries a charge $+Q$, the
negative plate must carry a charge $-Q$. The net result is that both capacitors
possess the same stored charge $Q$. The potential drops, $V_1$ and $V_2$, across
the two capacitors are, in general, different. However, the sum of these
drops equals the total potential drop $V$ applied across the input and output
wires: {\em i.e.}, $V=V_1+V_2$. The equivalent capacitance of the pair of
capacitors is again $C_{\rm eq} = Q/V$. 
Thus,
\begin{equation}
\frac{1}{C_{\rm eq}}=\frac{V}{Q}=\frac{V_1+ V_2}{Q} = \frac{V_1}{Q} + \frac{V_2}{Q},
\end{equation}
giving
\begin{equation}\label{e6.17}
\frac{1}{C_{\rm eq}} = \frac{1}{C_1} + \frac{1}{C_2}.
\end{equation}
Here, we have made use of the fact that the charge $Q$ is common to all three
capacitors.
Hence, the rule is:
\begin{quote}
{\sf The reciprocal of the equivalent capacitance of two capacitors connected in
series is the sum of the reciprocals of the individual capacitances.}
\end{quote}
For $N$ capacitors connected in series, Eq.~(\ref{e6.17}) generalizes to $1/C_{\rm eq}
=\sum_{i=1}^N (1/C_i).$

\subsection{Energy Stored by Capacitors}
Let us consider charging an initially uncharged  parallel plate 
capacitor by transferring a charge $Q$ from one
plate to the other, leaving the former plate with charge $-Q$ and the later
 with  charge $+Q$. Of course, 
once we have transferred some charge, an electric field is set up between
the plates which opposes any further  charge transfer.  In order to fully
charge the capacitor, we must do
work against this field, and this work becomes
 energy stored in the capacitor. Let us calculate this
energy.

Suppose that the capacitor plates carry a charge $q$ and that the
potential difference between the plates is $V$. The work we do in transferring
an {\em infinitesimal}\/ amount of charge $dq$ from the negative to the
positive plate is simply
\begin{equation}
dW = V\,dq.
\end{equation}
In order to evaluate the total work $W(Q)$ done in transferring
the total  charge $Q$ from one plate to the other, we can divide this charge into many small
increments $dq$, find the incremental work 
$dW$ done in transferring this incremental charge,
using the above formula, and
then sum all of these works. The only complication is that the potential
difference $V$ between the plates is a function of the total transferred
charge. In fact, $V(q) = q/C$, so
\begin{equation}
dW =\frac{q\,dq}{C}.
\end{equation}
Integration  yields
\begin{equation}
W(Q) = \int_0^Q\frac{q\,dq}{C}= \frac{Q^2}{2\,C}.
\end{equation}
Note, again, that the work $W$ done in charging the capacitor is
the same as the energy stored in the capacitor. Since $C=Q/V$, we can
write this stored energy  in one of three equivalent forms:
\begin{equation}
W = \frac{Q^2}{2\,C} = \frac{C\,V^2}{2} = \frac{Q \,V}{2}.
\end{equation}
These formulae are valid for any type of capacitor, since the arguments that we used
to derive them do not depend on any special property of parallel plate
capacitors. 

Where is the energy  in a parallel plate 
capacitor actually stored? Well, if we think about
it, the only place it could be stored is in the electric field generated 
between the plates. This insight allows us to calculate the energy (or, rather,
the energy density) of an electric field.

Consider a vacuum-filled parallel plate capacitor whose plates are of cross sectional area $A$,
and are spaced a distance $d$ apart. The electric field $E$ between the plates
is
approximately uniform, and of magnitude $\sigma/\epsilon_0$, where $\sigma
=Q/A$, and $Q$ is the charge stored on the plates. The electric field elsewhere is approximately zero. The potential difference
between the plates is $V=E\,d$. Thus, the energy stored in the capacitor
can be written
\begin{equation}
W = \frac{C\,V^2}{2} = \frac{\epsilon_0\,A\,E^2\,d^2}{2\,d} =
\frac{\epsilon_0\,E^2\,Ad}{2},
\end{equation}
where use has been made of Eq.~(\ref{e6.9}).
Now, $A\,d$ is the volume of the field-filled region between the plates, so if the
energy is stored in the electric field then the energy per unit volume,
 or  {\em energy density}, of the field must be
\begin{equation}\label{e6.23}
w = \frac{\epsilon_0\,E^2}{2}.
\end{equation}
It turns out that this result is quite general. Thus, we can calculate the energy
content of any electric field by dividing  space into little cubes, applying the
above formula to find the energy content of each cube, and then summing the
energies thus obtained to obtain the total energy. 

It is easily demonstrated that the energy density in a dielectric
medium is
\begin{equation}
w = \frac{\epsilon\, E^2}{2},
\end{equation}
where $\epsilon=K\,\epsilon_0$ is the permittivity of the medium. 
This energy density consists of two elements: the energy density $\epsilon_0\,E^2/2$
held in the electric field, and the energy density $(K-1)\,\epsilon_0\,E^2/2$
held in the dielectric medium (this represents the work done on the
constituent  molecules of the dielectric 
in order to polarize them). 

\subsection{Worked Examples}
\subsection*{\em Example 6.1: Parallel plate capacitor}
\noindent{\em Question:} A parallel plate capacitor consists of two metal
plates, each of area $A=150\,{\rm cm}^2$, separated by a vacuum gap
$d=0.60$ cm thick. What is the capacitance of this device? What potential
difference must be applied between the plates if the capacitor
is to hold a charge of magnitude $Q=1.00\times 10^{-3}\,\mu{\rm C}$ on each
plate?\\
~\\
\noindent{\em Solution:} Making use of formula (\ref{e6.9}), the capacitance
$C$ is given by
$$
C= \frac{(8.85\times 10^{-12})\,(150\times 10^{-4})}{(0.6\times 10^{-2})}
=2.21\times 10^{-11} = 22.1\,{\rm pF}.
$$
The voltage difference $V$ between the plates and
the magnitude of the charge $Q$ stored on each plate are related via
$C = Q/V$, or $V = Q/C$. Hence, if $Q=1.00\times 10^{-3}\,\mu{\rm C}$ then
$$
V = \frac{ (1.00\times 10^{-9})}{(2.21\times 10^{-11})} = 45.2\, {\rm V}.
$$


\subsection*{\em Example 6.2: Dielectric filled capacitor}
\noindent{\em Question:} 
A parallel plate capacitor has a plate area of $50\,{\rm cm}^2$ and a plate
separation of $1.0$ cm. A potential difference of $V_0= 200$ V is applied
across the plates with no dielectric present. The battery is then disconnected,
and a piece of Bakelite ($K=4.8$) is inserted which fills the region between
the plates. What is the capacitance, the charge on the plates, and the
potential difference between the plates, before and after the dielectric
is inserted?\\
~\\
\noindent{\em Answer:} Before the dielectric is inserted, 
the space between the plates is presumably filled
with air. Since the dielectric constant of air is virtually
indistinguishable from that of a vacuum, let us use the vacuum formula (\ref{e6.9}) to
calculate the initial capacitance $C_0$. Thus,
$$
C_0=\frac{\epsilon_0\,A}{d} = \frac{(8.85\times 10^{-12})\,(50\times 10^{-4})}
{(1\times 10^{-2})} = 4.4\,{\rm pF}.
$$
After the dielectric is inserted, the capacitance increases by a factor $K$,
which in this case is $4.8$, so the new capacitance $C$ is given by
$$
C= K\,C_0 = (4.8)\, (4.4 \times 10^{-12}) = 21\,{\rm pF}.
$$

Before the dielectric is inserted, the charge $Q_0$ on the plates is
simply
$$
Q_0 = C_0 V_0 = (4.4\times 10^{-12})\,(200) = 8.8\times 10^{-10} \,{\rm C}.
$$
After the dielectric is inserted, the charge  $Q$ is exactly the same, since the
capacitor is disconnected, and so the charge cannot leave the plates.
Hence,
$$
Q=Q_0 = 8.8\times 10^{-10} \,{\rm C}.
$$

The potential difference before the dielectric is inserted is given as
$V_0=200$ V. The potential difference $V$ after the dielectric is
inserted is simply
$$
V = \frac{Q}{C} =\frac{(8.8\times 10^{-10})}{(21\times 10^{-12})} = 42\, {\rm V}.
$$
Note, of course, that $V=V_0/K$. 

\subsection*{\em Example 6.3: Equivalent capacitance}
\begin{figure*}[h]
\epsfysize=2in
\centerline{\epsffile{Chapter06/fig1.eps}}
\end{figure*}
\noindent{\em Question:} 
A $1\,\mu{\rm F}$ and a $2\,\mu{\rm F}$ capacitor are connected in parallel, and this pair of
capacitors is then connected in series with a $4\,\mu{\rm F}$ capacitor, as
shown in the diagram.
What is the equivalent capacitance of the whole combination?
What is the charge on the $4\,\mu{\rm F}$ capacitor if 
the whole combination is connected across the terminals of
a $6$ V battery? Likewise, what are the charges on the 
$1\,\mu{\rm F}$ and $2\,\mu{\rm F}$ capacitors?\\
~\\
\noindent{\em Answer:} The equivalent capacitance of the   $1\,\mu{\rm F}$ and $2\,\mu{\rm F}$
capacitors connected in parallel is $1+2= 3\,\mu{\rm F}$.
When a $3\,\mu{\rm F}$ capacitor  is combined in series with a $4\,{\mu}{\rm F}$ capacitor,
the equivalent capacitance of the whole combination is given by
$$
\frac{1}{C_{\rm eq}} = \frac{1}{(3\times 10^{-6})} + \frac{1}{(4\times 10^{-6})} = \frac{(7)}{(12\times 10^{-6})}\,\,{\rm F}^{-1},
$$
and so
$$
C_{\rm eq} = \frac{(12\times 10^{-6})}{(7)} = 1.71\,\mu{\rm F}.
$$

The charge delivered by the $6$ V battery is 
$$
Q = C_{\rm eq} V = (1.71\times 10^{-6})\,(6) = 10.3\,\mu{\rm C}.
$$
This is the charge on the $4\,\mu{\rm F}$ capacitor, since one of the terminals
of the battery is connected directly to one of the plates of this capacitor.

The voltage drop across the $4\,\mu{\rm F}$ capacitor is 
$$
V_4 = \frac{Q}{C_4} = \frac{(10.3\times 10^{-6})}{(4\times 10^{-6})}=2.57\,{\rm V}.
$$
Thus, the voltage drop across the $1\,\mu{\rm F}$ and  $2\,\mu{\rm F}$ combination
must be $V_{12}=6-2.57 = 3.43\,{\rm V}$. The charge stored on the $1\,\mu{\rm F}$ 
is given by
$$
Q_1 = C_1\,V_{12} = (1\times 10^{-6})\,(3.43) = 3.42\,\mu{\rm C}.
$$
Likewise, the charge stored on the $2\,\mu{\rm F}$ capacitor is
$$
Q_2 = C_2\,V_{12} = (2\times 10^{-6})\,(3.43) = 6.84\,\mu{\rm C}.
$$
Note that the total charge stored on the $1\,\mu{\rm F}$ and  $2\,\mu{\rm F}$ combination
is $Q_{12} = Q_1+Q_2 = 10.3\,\mu{\rm C}$, which is the same as the charge
stored on the $4\,\mu{\rm F}$ capacitor. This makes sense because the $1\,\mu{\rm F}$ and  $2\,\mu{\rm F}$ combination
and the  $4\,\mu{\rm F}$ capacitor are connected in series.


\subsection*{\em Example 6.4: Energy stored in a capacitor}
\noindent{\em Question:} 
An air-filled parallel plate capacitor has a capacitance of $5.0$\,pF. A potential
of 100\,V is applied across the plates, which are $1.0$\,cm apart, using a
storage battery. What is the energy stored in the capacitor? Suppose that
the battery is disconnected, and the plates are moved until they are $2.0$\,cm
apart. What now is the energy stored in the capacitor? Suppose, instead, that
the battery is left connected, and the plates are again moved until they are
$2.0$\,cm apart. What is the energy stored in the capacitor in this case?\\
~\\
\noindent{\em Answer:} The initial energy stored in the
capacitor is
$$
W = \frac{C\,V^2}{2} = \frac{(5\times 10^{-12})\,(100)^2}{2}= 2.5\times 10^{-8}
\,{\rm J}.
$$

When the spacing between the plates is doubled, the capacitance of the capacitor
is halved to $2.5$\,pF. If the battery is disconnected then this process
takes place at constant charge $Q$. Thus, it follows from the
formula
$$
W = \frac{Q^2}{2\,C}
$$
that  the energy stored in the capacitor doubles. So, the
new energy is $5.0\times 10^{-8}$\,J. Incidentally, the increased energy
of the capacitor is accounted for by  the work done in pulling 
the capacitor plates apart (since these plates  are oppositely charged,
they attract one another). 

If the battery is left connected, then the capacitance is still halved, but now
the process takes place at constant voltage $V$. It follows from the
formula
$$
W = \frac{C\,V^2}{2}
$$
that  the energy stored in the capacitor is halved. 
So, the new energy is $1.25\times 10^{-8}$\,J. Incidentally, the energy lost
by the capacitor is given to the battery (in effect, it goes to re-charging the
battery). Likewise, the work done in pulling the plates apart is
also given to the battery. 
