\section{Geometric Optics}
\subsection{Introduction}
Optics deals with the propagation
of light through transparent media,
and its interaction with mirrors, lenses, slits, {\em etc}.
Optical effects  can be divided into two broad  classes. Firstly, those
which can be explained without reference to the fact that light is
fundamentally a wave phenomenon, and, secondly,  those which can only be explained
on the basis that light is a wave phenomenon. Let us, for the moment, 
consider the former
class of effects. It might seem somewhat surprising that any
 optical effects at all can be
accounted for without reference to  waves. After all,
 as we saw in Sect.~\ref{s11}, light really is a
wave phenomenon. It turns out, however, that wave effects are only 
crucially
important when the wavelength of the wave is either comparable to, or 
much
larger  than, the size of the objects with which it interacts
(see Sect.~\ref{s13}). 
When the wavelength of
the wave  becomes much smaller than the size of the objects with which 
it
interacts then the interactions can be accounted for in a very
simple  geometric manner, as explained in this section. 
Since the
wavelength of visible light is only of order a micron, it is very easy
 to
find situations in which its wavelength  is very much smaller
 than
the size of the objects with which it interacts. Thus,
``wave-less'' optics, which is usually called {\em geometric optics},
 has a
very wide range of applications. 

In geometric optics, light is treated as a set of {\em rays}, emanating
 from
a source, which propagate through transparent media according to
a set of {\em  three}\/ simple laws. The first law is the {\em law of
 rectilinear propagation},
which states that light rays propagating through a 
homogeneous transparent 
 medium  do so in straight-lines.
 The second law is the {\em law
of reflection}, which governs the interaction of light rays with
 conducting
surfaces ({\em e.g.}, metallic mirrors). The third law is the
{\em law of refraction}, which governs the behaviour of light rays as
they traverse a sharp boundary between two different transparent media
({\em e.g.}, air and glass). 

\subsection{History of Geometric Optics}
Let us first consider the law of {\em rectilinear propagation}. 
The earliest surviving optical treatise, Euclid's 
{\em Catoptrics}\footnote{Catoptrics is the ancient
Greek word for reflection.}
 (280\,BC),
recognized that light travels in straight-lines  in homogeneous media. 
However, following the teachings of Plato, Euclid (and all other ancient
Greeks) thought that  light rays  emanate from the eye, and 
intercept external objects, which are thereby  ``seen'' by the observer. 
The ancient Greeks also thought that the speed with which light
rays emerge from the eye is very high, if not infinite. After all,
they argued,
an observer with his eyes closed can open them and immediately
see the distant stars. 

Hero of Alexandria, in his {\em Catoptrics}\/ (first century BC), 
also maintained
that light travels with infinite speed. His argument was by analogy with
the free fall of objects. If we throw an object horizontally with a
relatively small velocity then it manifestly does not move in a
straight-line. However, if we throw an object horizontally with a
relatively large velocity then it appears to move in
a straight-line to begin with, but eventually deviates from this
path. The larger the velocity with which the object is thrown, the longer
the initial period of apparent rectilinear motion. Hero reasoned that
 if
an object were thrown with an infinite velocity then it would move in
a straight-line forever. Thus, light, which travels in a straight-line,
must move with an infinite velocity. The erroneous idea that light 
travels with
an {\em infinite}\/
 velocity persisted until 1676, when
 the
Danish astronomer Olaf R\"{o}mer demonstrated that light must have a
{\em  finite}\/
velocity, using his   timings
of the successive  eclipses of the satellites of Jupiter, as they passed
 into the
shadow of the planet. 

The first person to realize that light actually travels
from the object seen to the eye was the Arab philosopher
``Alhazan'' (whose real name was Abu'ali al-hasan ibn al-haytham),
who published a book on optics in about 1000~AD.

The law of {\em reflection}\/ was correctly formulated in Euclid's book. 
Hero of Alexandria demonstrated that, by adopting the rule that light 
rays
always travel between two points by the {\em shortest path} (or, more 
rigorously,
the extremal path), it is possible to derive the law of reflection 
using geometry.

The law of {\em refraction}\/ was studied experimentally by Claudius Ptolemy
(100-170\,AD), and is reported in Book~V of his {\em Catoptrics}.
 Ptolemy formulated a very inaccurate 
version of
the law of refraction, which only works when the light rays are almost
 normally
incident on the interface in question. Despite its obvious inaccuracy,
 Ptolemy's
theory of refraction persisted for nearly 1500 years. 
The true law of refraction was discovered empirically by the Dutch
mathematician  Willebrord Snell in 1621. However, the French philosopher
Ren\'{e} Descartes was the first to publish, in his
{\em La Dioptrique} (1637), the now familiar
formulation of the law of refraction in terms of sines. Although  there was much controversy at 
the time  regarding plagiarism, Descartes was apparently unaware of
Snell's work.
Thus, in English speaking countries the law
of refraction is called ``Snell's law'', but in French speaking
 countries
it is called ``Descartes' law''.

In 1658, the French mathematician Pierre de Fermat demonstrated that all
three of the laws of geometric optics can be accounted for
 on the 
assumption  
that
light always travels between two points on the path which takes the 
{\em least time}\/ (or, more
rigorously, the extremal time). Fermat's ideas were an extension of those of Hero of
Alexandria. Fermat's (correct) derivation of the law of refraction 
depended crucially
on his (correct) assumption that light travels {\em more slowly}\/ in dense 
media than it
does in air. Unfortunately, many famous scientists, 
including Newton, maintained
 that
light travels {\em faster}\/ in dense media than it does in air. This
erroneous idea held
up progress in optics for over one hundred years, and
 was not conclusively disproved until the mid-nineteenth 
century. 
Incidentally, 
Fermat's principle of least time can only  be justified 
 using
wave theory.

\begin{figure}
\epsfysize=3in
\centerline{\epsffile{Chapter12/fig12.01.eps}}
\caption{\em An opaque object illuminated by a point light source.}\label{f12.1}
\end{figure}

\subsection{Law of Geometric Propagation}
According to geometric optics, an opaque object
illuminated by a point source of light casts a sharp shadow
whose dimensions can be calculated using geometry. The method
of calculation is very straightforward.
The source emits {\em light-rays} uniformly  in all directions.
These rays can be represented as straight lines radiating 
from the source. The light-rays propagate away from
the source until they encounter an opaque object, at which point
they stop. This is illustrated in Fig.~\ref{f12.1}.

\begin{figure}
\epsfysize=3in
\centerline{\epsffile{Chapter12/fig12.02.eps}}
\caption{\em An opaque object illuminated by an extended light source.}\label{f12.2}
\end{figure}

For an extended light 
source, each element of the source emits light-rays, just
like a point source.  Rays emanating from different
elements of the source are assumed  not to interfere
with one another. Figure~\ref{f12.2} shows how the shadow
cast by an opaque sphere  illuminated by a spherical
light source is calculated using a small
number of critical light-rays. The 
shadow consists of
a perfectly black disk called the {\em umbra}, surrounded by
a ring of gradually diminishing darkness called the
{\em penumbra}. In the umbra, {\em all} of the light-rays emitted
by the source are blocked by the opaque sphere, whereas in the penumbra
only {\em some}\/ of the rays emitted by the source are blocked
by the sphere. As was well-known to the ancient Greeks, if
the light-source represents the Sun, and the opaque sphere  the
 Moon,
then at a  point on the Earth's surface which is situated inside
the umbra the Sun is totally eclipsed, whereas at a point
on the Earth's surface which is situated in the penumbra
the Sun is only partially eclipsed. 

\begin{figure}
\epsfysize=3in
\centerline{\epsffile{Chapter12/fig12.03.eps}}
\caption{\em Relationship between wave-fronts and light-rays.}\label{f12.3}
\end{figure}

In the wave picture of light, a {\em wave-front}\/ is defined as a
surface joining all adjacent points on a wave that have the same
phase ({\em e.g.}, all maxima, or minima, of the electric field).  A
light-ray is simply a line which runs perpendicular to the wave-fronts
at all points along the path of the wave. This is illustrated
in Fig.~\ref{f12.3}. Thus, the law of rectilinear propagation
of light-rays
also specifies how wave-fronts propagate through homogeneous
media. Of course, this law is only valid in the limit where the wavelength of
the wave is much smaller than the dimensions of any obstacles which it encounters. 

\subsection{Law of Reflection}
The law of reflection governs the reflection of light-rays off
 smooth conducting surfaces, such as polished metal or metal-coated
glass mirrors.

Consider a light-ray incident on a plane mirror, as shown in Fig.~\ref{f12.4}.
The law of reflection states that the incident ray, the reflected
ray, and the normal to the surface of the mirror all lie in the
{\em same plane}. Furthermore, the angle of reflection $r$ is {\em
equal}\/
to the angle of incidence $i$. Both angles are measured with
respect to the normal to the mirror.

\begin{figure}
\epsfysize=2.5in
\centerline{\epsffile{Chapter12/fig12.04.eps}}
\caption{\em The law of reflection}\label{f12.4}
\end{figure}

The law of reflection also holds for non-plane mirrors, provided
that the normal at any
point on the mirror is understood to be the outward pointing
normal to the local tangent plane of the mirror at that point. 
For rough surfaces, the law of reflection remains valid. It
predicts 
that  rays incident at slightly different points on the
surface are reflected in completely different directions, because
the normal to a rough surface varies in direction very strongly from
point to point on the surface. This type of reflection is called
{\em diffuse reflection}, and is what enables us to see non-shiny 
 objects. 

\subsection{Law of Refraction}
The law of refraction, which is generally
known as {\em Snell's law}, governs the behaviour of light-rays as
they propagate across a sharp interface between two 
transparent dielectric media. 

Consider a light-ray incident on a plane interface between two
transparent dielectric media, labelled 1 and 2, as shown in Fig.~\ref{f12.5}.
The law of refraction states that the incident ray, the refracted ray,
and the normal to the interface, all lie in the {\em same plane}. 
Furthermore, 
\begin{equation}
n_1\,\sin\theta_1 = n_2\,\sin\theta_2,
\end{equation}
where $\theta_1$ is  the angle subtended between the incident ray and
the normal to the interface, and $\theta_2$ is the angle subtended between the
refracted ray and the normal to the
interface. The quantities $n_1$ and $n_2$ are
termed the {\em refractive indices}\/ of media 1 and 2, respectively.
Thus, the law of refraction predicts that a light-ray always
deviates more towards
 the normal in the optically denser medium: {\em i.e.},
the medium with the higher refractive index. Note that $n_2>n_1$ in the figure. The law of refraction also holds for non-planar
interfaces, provided that the normal to the interface at any given point
is understood to be the normal to the local tangent plane of the
interface at that
point.

\begin{figure}
\epsfysize=3in
\centerline{\epsffile{Chapter12/fig12.05.eps}}
\caption{\em The law of refraction.}\label{f12.5}
\end{figure}

By definition, the refractive index $n$ of a dielectric medium
of dielectric constant $K$ is given by
\begin{equation}
n = \sqrt{K}.
\end{equation}
Table~\ref{t12.1} shows the refractive indices of some common 
materials (for yellow light of wavelength $\lambda= 589$\,nm). 
\begin{table}
\begin{tabular}{ll}\hline
Material & $n$ \\[0.5ex] \hline
Air (STP) & 1.00029\\
Water & 1.33 \\
Ice & 1.31\\
Glass: &\\
~~Light flint & 1.58\\
~~Heavy flint & 1.65\\
~~Heaviest flint & 1.89\\
Diamond & 2.42\\ 
\end{tabular}
\centering
\caption{\em Refractive indices of some common materials at $\lambda=589$ nm.}\label{t12.1}
\end{table}

The law of refraction follows directly from  the fact that
the speed $v$ with which  light propagates through a dielectric medium 
is {\em inversely proportional} to the refractive index of the
medium (see Sect.~\ref{s11.3}). In fact, 
\begin{equation}
v  =\frac{c}{n},
\end{equation}
where $c$ is the speed of light in a vacuum. Consider two
parallel light-rays, $a$ and $b$, incident at an angle $\theta_1$ with
respect to the normal to the interface between two dielectric media,
1 and 2. Let the refractive indices of the two media be $n_1$ and
$n_2$ respectively, with $n_2>n_1$. It is clear from Fig.~\ref{f12.6}
that ray $b$ must move from  point
$B$ to point $Q$, in medium 1, in the same time interval,
 ${\mit\Delta} t$, in which
ray $a$ moves between points $A$ and $P$, in medium 2. Now, the
speed of light in medium 1 is $v_1=c/n_1$, whereas the speed
of light in medium 2 is $v_2=c/n_2$. It follows that the
length $BQ$ is given by $v_1\,{\mit\Delta} t$, whereas the
length $AP$ is given by $v_2\,{\mit\Delta} t$. By trigonometry,
\begin{equation}
\sin\theta_1 = \frac{BQ}{AQ} = \frac{v_1\,{\mit\Delta} t}{AQ},
\end{equation}
and
\begin{equation}
\sin\theta_2 = \frac{AP}{AQ} = \frac{ v_2\,{\mit\Delta} t}{AQ}.
\end{equation}
Hence,
\begin{equation}
\frac{\sin\theta_1}{\sin\theta_2} = \frac{v_1}{v_2} = \frac{n_2}{n_1},
\end{equation}
which can be rearranged to give  Snell's law. Note that the
lines $AB$ and $PQ$ represent wave-fronts in media 1 and 2, respectively, and,
therefore, cross rays $a$ and $b$ at right-angles. 

\begin{figure}
\epsfysize=3.5in
\centerline{\epsffile{Chapter12/fig12.06.eps}}
\caption{\em Derivation of Snell's law.}\label{f12.6}
\end{figure}

When  light passes from one dielectric medium to another
its velocity $v$ changes, but its frequency $f$ remains {\em unchanged}.
Since, $v=f\,\lambda$ for all waves, where $\lambda$ is the wavelength,
it follows that the wavelength of light must also change as it crosses
an interface between two different media. 
Suppose that light propagates
 from medium 1 to medium 2. Let $n_1$ and $n_2$ be the refractive
indices of the two media, respectively. The ratio of the
wave-lengths in the two media is given by
\begin{equation}
\frac{\lambda_2}{\lambda_1} = \frac{v_2/f}{v_1/f} = 
\frac{v_2}{v_1}=\frac{n_1}{n_2}.
\end{equation}
Thus, as light moves from  air to glass its
wavelength {\em decreases}. 

\subsection{Total Internal Reflection}
An interesting effect known as {\em total internal reflection}\/
can occur when light attempts to move from a medium having a
given refractive index  to a medium having a {\em lower} refractive index.
Suppose that light crosses an interface from medium 1 to medium 2,
where $n_2< n_1$. According to Snell's law,
\begin{equation}\label{e12.8}
\sin\theta_2 = \frac{n_1}{n_2} \,\sin\theta_1.
\end{equation}
Since $n_1/n_2>1$, it follows that $\theta_2 > \theta_1$. 
For relatively small angles of incidence, part of the light
is refracted into the less optically dense medium, and part
is reflected (there is always some reflection at an interface). 
When the angle of incidence $\theta_1$ is such that the
angle of refraction $\theta_2= 90^\circ$, the refracted ray
runs along the interface between the two media. This particular
angle of incidence is called the {\em critical angle}, $\theta_c$. 
For $\theta_1>\theta_c$, there is {\em no}\/ refracted ray. Instead, all of
the light incident on the interface is reflected---see Fig.~\ref{f12.7}. This effect
is called {\em total internal reflection}, and occurs whenever the
angle of incidence exceeds the critical angle. 
Now when $\theta_1=\theta_c$, we have $\theta_2=90^\circ$, and so
$\sin\theta_2 = 1$. It follows from Eq.~(\ref{e12.8}) that
\begin{equation}\label{e12.9}
\sin\theta_c = \frac{n_2}{n_1}.
\end{equation}

\begin{figure}
\epsfysize=3.5in
\centerline{\epsffile{Chapter12/fig12.07.eps}}
\caption{\em Total internal reflection.}\label{f12.7}
\end{figure}

Consider a fish (or a diver) swimming in a  clear 
pond. As Fig.~\ref{f12.8}
makes clear, if the fish looks upwards it sees the sky, but if
it looks at too large an angle to the vertical it sees the bottom
of the pond reflected on the surface of the water. The critical angle
to the vertical at which the fish first sees the reflection of
the bottom of the pond is, of course, equal to the critical angle
$\theta_c$ for total internal reflection at an air-water interface.
From Eq.~(\ref{e12.9}), this critical angle is given by
\begin{equation}
\theta_c = \sin^{-1}(1.00/1.33) = 48.8^\circ,
\end{equation}
since the refractive index of air is approximately unity, and
the refractive index of water is $1.33$. 

\begin{figure}
\epsfysize=3in
\centerline{\epsffile{Chapter12/fig12.08.eps}}
\caption{\em A fish's eye view.}\label{f12.8}
\end{figure}

When total internal reflection occurs at an interface the interface in question acts
as a {\em perfect reflector}. This allows $45^\circ$ crown glass
prisms to be used,
in place of mirrors, to reflect light in binoculars. This is
illustrated in Fig.~\ref{f12.9}. The angles of incidence on the
sides of the prism are all $45^\circ$, which is greater than
the critical angle $41^\circ$ for crown glass (at an air-glass
interface). 

\begin{figure}[h]
\epsfysize=2.5in
\centerline{\epsffile{Chapter12/fig12.09.eps}}
\caption{\em Arrangement of prisms used in binoculars.}\label{f12.9}
\end{figure}

Diamonds, for which $n=2.42$, have a critical angle $\theta_c$ which
is only $24^\circ$. The facets on a diamond are cut in such a manner
that much of the incident light on the diamond is reflected many
times by successive total internal reflections before it
escapes. This effect gives rise to the 
characteristic sparkling of cut diamonds.

Total internal reflection enables light to be transmitted inside
thin glass fibers. The light is internally reflected off the
sides of the fiber, and, therefore, follows the path of the fiber.
Light can actually be transmitted around corners using a glass fiber,
provided that the bends in the fiber are not too sharp, so that the
light always strikes the sides of the fiber at angles greater than the
critical angle. The whole field  of {\em fiber optics},
with its many useful applications,  is based on this
effect. 

\subsection{Dispersion}
When a wave is refracted into a dielectric medium whose refractive
index {\em  varies}\/ with wavelength then the angle of refraction
also varies with wavelength. If the incident wave is not
monochromatic, but is, instead, composed of a mixture of waves of
different
wavelengths, then each component wave is refracted through
a {\em  different}\/ angle. This phenomenon is called {\em dispersion}. 

\begin{figure}
\epsfysize=4in
\centerline{\epsffile{Chapter12/fig12.10.eps}}
\caption{\em Refractive indices of some common materials as functions of
wavelength.}\label{f12.10}
\end{figure}

Figure~\ref{f12.10} shows the refractive indices of
some common materials  as functions of wavelength in the
visible range. It can be seen that the refractive index
always {\em decreases}\/ with increasing wavelength in the visible range.
In other words, violet light is always refracted 
{\em more strongly}\/ than
red light.

\begin{figure}
\epsfysize=3in
\centerline{\epsffile{Chapter12/fig12.11.eps}}
\caption{\em Dispersion of light by a parallel-sided glass slab.}\label{f12.11}
\end{figure}

Suppose that a  parallel-sided glass slab is placed in a beam of white
light. Dispersion takes place inside the slab, but, since
the rays which emerge from the slab
all run {\em parallel} to one another, the
dispersed colours  recombine to form white light again, and no dispersion
is observed except at the very edges of the beam.
This is illustrated in Fig.~\ref{f12.11}. 
 It follows that the
dispersion of white light through a parallel-sided glass slab is
not generally a noticeable effect. 

\begin{figure}
\epsfysize=2.5in
\centerline{\epsffile{Chapter12/fig12.12.eps}}
\caption{\em Dispersion of light by a glass prism.}\label{f12.12}
\end{figure}

Suppose that a glass prism is placed in a beam of white light.
Dispersion takes place inside the prism, and, since the
emerging rays are {\em not parallel}\/ for different colours, the
dispersion is clearly noticeable, especially if the emerging
rays are projected onto a screen which is placed a
 long way from the prism. This
is illustrated in Fig.~\ref{f12.12}. It is clear that a glass prism
is far more effective at separating white
light into its component
colours than a parallel-sided glass slab (which explains why 
 prisms are generally employed to perform this task). 

\subsection{Rainbows}
The most well-known, naturally occurring phenomenon which involves the
dispersion of light is a {\em rainbow}. A rainbow is an
{\em arc}
 of light, with an angular radius of $42^\circ$, centred
on a direction which is {\em  opposite}\/
 to that of the Sun in the
sky ({\em i.e.}, it is centred on the
direction of propagation of
the Sun's rays)---see Fig.~\ref{f12.13}. Thus, if the Sun is low in the sky ({\em i.e.}, close to the horizon) we see almost
a full semi-circle. If the Sun is higher in the sky we see a smaller arc, and
if the Sun is more than $42^\circ$ above the horizon then there is
no rainbow (for viewers on the Earth's surface). Observers on a
hill may see parts of the rainbow below the horizontal:
{\em i.e.},  an arc
greater than a semi-circle. Passengers on an airplane can sometimes
see a full circle. 

\begin{figure}[h]
\epsfysize=3in
\centerline{\epsffile{Chapter12/fig12.13.eps}}
\caption{\em A rainbow.}\label{f12.13}
\end{figure}

The colours of a rainbow vary smoothly from red on the outside
of the arc to violet on the inside. A rainbow has a
diffuse inner edge, and a sharp outer edge.
Sometimes a {\em secondary arc}\/
is observed. This is fainter and larger
(with an angular radius of $50^\circ$) than the primary
arc, and the order of the colours is reversed ({\em i.e.}, red is
on the inside, and violet on the outside). The secondary
arc has a diffuse outer edge, and a sharp inner edge. The sky between
the two arcs sometimes appears to be less bright than the sky elsewhere.
This region is called {\em Alexander's dark band}, in honour of
Alexander of Aphrodisias who described it some 1800 years ago.

Rainbows have been studied  since ancient times.
Aristotle wrote extensively on rainbows in his 
{\em De Meteorologica},\footnote{``On Weather''.} and even speculated
that a rainbow is caused by the reflection of sunlight from the
drops of water in a cloud. 

The first scientific study of rainbows was performed by Theodoric,
professor of theology at Freiburg, in the fourteenth century.
He studied the path of a light-ray through a spherical globe of water
in his laboratory, and suggested that the globe be thought of as a
model of a single falling raindrop. A ray, from the Sun, entering the drop,
is refracted at the air-water interface, undergoes
 internal reflection from the inside surface of the drop, and
then leaves the drop in a backward direction, after being again
refracted at the surface. Thus, looking away from the
Sun, towards a cloud of raindrops, one sees an enhancement of light due
to these rays. Theodoric did not explain why this enhancement
is concentrated at a particular angle from the direction
of the Sun's rays, or
why the light is split  into different colours.

The first person to give a full  explanation of how a
rainbow is formed was Ren\'{e} Descartes. He showed mathematically
that if one traces the path through a spherical raindrop of
parallel light-rays entering the drop at different points on
its surface, each  emerges in a different direction, but there
is a concentration of emerging rays at an angle of
$42^\circ$ from the reverse  direction to  the incident rays, in exact
agreement with the observed
angular size of rainbows. Furthermore, since some colours are
refracted more than others in a
raindrop, the ``rainbow angle'' is slightly different for each
colour, so a raindrop disperses the Sun's light into a
set of nearly overlapping coloured arcs. 

Figure~\ref{f12.13} illustrated Descartes' theory in more detail.
It shows parallel light-rays
 entering a spherical raindrop. Only
rays entering the upper half contribute to the rainbow effect. Let us
follow the rays, one by one, from the top down to the middle
of the drop. We observe the following pattern. Rays which enter near the
top of the drop emerge going in almost the reverse direction,
but  a few
degrees below the horizontal. Rays entering a little
further below the top emerge at a greater angle below the horizontal. 
Eventually, we reach a critical ray, called the {\em rainbow ray}, which
emerges in an angle $42^\circ$ below the horizontal. Rays
entering the drop lower than the rainbow ray emerge at an angle
 less than $42^\circ$. Thus, the rainbow ray is the
one  which 
 deviates
{\em most}\/ from the reverse direction to the incident rays.
This variation, with $42^\circ$ being the maximum angle of deviation
from the  reverse direction,
leads to a bunching of rays at that angle, and, hence, to  an 
unusually bright 
arc  of reflected light centred around 
 $42^\circ$ from the reverse direction. The arc has a sharp outer edge,
since reflected light {\em cannot}\/ deviate by more than $42^\circ$ from the reverse direction,
 and a diffuse inner edge, since light {\em can}\/
deviate by less than $42^\circ$ from the reverse direction:
 $42^\circ$ is
just the {\em most likely} angle of deviation. Finally, since the
rainbow angle varies slightly with wavelength  (because
the refractive index of water varies slightly with wavelength), the
arcs corresponding to
each colour appear at slightly different angles relative to
the reverse direction to  the incident  rays. We
expect violet light to be refracted more strongly than red light
in a raindrop. It is, therefore, clear, from Fig.~\ref{f12.14},
that the red arc deviates slightly more from the  reverse direction
to  the incident rays than
the violet arc. In other words, violet is concentrated on the inside
of the rainbow, and red is concentrated on the outside. 

\begin{figure}
\epsfysize=3.5in
\centerline{\epsffile{Chapter12/fig12.14.eps}}
\caption{\em Descarte's theory of the rainbow.}\label{f12.14}
\end{figure}

Descartes was also able to show that light-rays which are
internally  reflected
{\em twice} inside a raindrop emerge concentrated at an
angle of $50^\circ$ from  the reverse direction to the incident
 rays. Of course,
this angle corresponds exactly to the angular size of the secondary
rainbow sometimes seen outside the first. This rainbow
is naturally less intense than the primary rainbow, since a
light-ray  loses some of its intensity at each reflection
or refraction event. Note that $50^\circ$ represents the
angle of {\em maximum}\/ deviation of doubly reflected light
from the reverse direction  ({\em i.e.}, doubly reflected
light can deviate by more than this angle, but not by less). Thus,
we expect the secondary rainbow to have a diffuse outer edge, and
a sharp inner edge. We  also expect doubly reflected violet light to be
refracted more strongly in a raindrop than 
doubly reflected red light. It follows, from Fig.~\ref{f12.15},
 that the red secondary arc deviates slightly less
from the reverse direction to the incident
 rays than the violet secondary arc.
In other words, red is concentrated on the inside of the
secondary rainbow, and violet  on the outside.
Since no reflected light emerges between the primary and secondary
rainbows ({\em i.e.}, in the angular range $42^\circ$ to $50^\circ$,
relative to the reverse direction),
 we naturally expect this region of the sky to look
slightly less bright than the other surrounding regions of the sky,
which explains Alexander's dark band.

\begin{figure}
\epsfysize=3.5in
\centerline{\epsffile{Chapter12/fig12.15.eps}}
\caption{\em Rainbow rays for the primary and secondary arcs of a rainbow.}\label{f12.15}
\end{figure}

\subsection{Worked Examples}
\subsection*{\em Example 12.1: The corner-cube reflector}
\begin{figure*}[h]
\epsfysize=3in
\centerline{\epsffile{Chapter12/fig1.eps}}
\end{figure*}
{\em Question:} Two mirrors are placed at right-angles to one another. Show
that a light-ray incident from {\em any}\/ direction
in the plane perpendicular to both mirrors is reflected
through $180^\circ$.\\
~\\
{\em Answer:} Consider the diagram. We are effectively being asked to prove that
$\alpha=i_1$, for any value of $i_1$. Now, from trigonometry,
$$
i_2 = 90^\circ - r_1.
$$
But, from the law of reflection, $r_1=i_1$ and $i_2=r_2$, so
$$
r_2=90^\circ - i_1.
$$
Trigonometry also yields 
$$
\alpha = 90^\circ - r_2.
$$
It follows from the previous two equations that
$$
\alpha = 90^\circ-(90^\circ - i_1) = i_1.
$$
Hence, $\alpha = i_1$, for all values of $i_1$. 

It can easily be appreciated that a combination of {\em three}\/
mutually perpendicular mirrors would reflect a light-ray incident from
{\em any}\/ direction through $180^\circ$. Such a combination of
mirrors is
called a {\em corner-cube reflector}. Astronauts on the Apollo 11
mission (1969) left  a panel  of
corner-cube reflectors on the surface of the Moon. These reflectors
have been used ever since to measure the Earth-Moon distance
via laser range finding (basically, a laser beam is fired from
the Earth, reflects off the corner-cube reflectors on the
Moon, and then returns to the Earth. The time of travel of the beam
can easily be converted into the Earth-Moon distance). The Earth-Moon
distance can be measured to within an accuracy of $3\,{\rm cm}$ using
this method. 

\subsection*{\em Example 12.2: Refraction}
{\em Question:} A light-ray of wavelength $\lambda_1 = 589$\,nm 
traveling through air is incident on a smooth, flat slab of
crown glass (refractive index 1.52) at an angle of $\theta_1=
30.0^\circ$ to the normal. What is the angle of refraction?
What is the wavelength $\lambda_2$ of the light inside the glass?
What is the frequency $f$ of the light inside the glass?\\
~\\
{\em Answer:} Snell's law can be written
$$
\sin\theta_2 = \frac{n_1}{n_2}\,\sin\theta_1.
$$
In this case, $\theta_1=30^\circ$, $n_1\simeq 1.00$ (here, we neglect the
slight deviation of the refractive index of air from that of a
vacuum), and $n_2= 1.52$. Thus,
$$
\sin\theta_2 =\frac{(1.00)}{(1.52)}\,(0.5)=0.329,
$$
giving 
$$
\theta_2 = 19.2^\circ
$$
as the angle of refraction (measured with respect to the normal). 

The wavelength $\lambda_2$ of the light inside the glass is given by
$$
\lambda_2 = \frac{n_1}{n_2}\,\lambda_1 = \frac{(1.00)}{(1.52)}\,(589)
= 387.5\,{\rm nm}.
$$

The frequency $f$ of the light inside the glass is exactly the same
as the frequency outside the glass, and is given by
$$
f = \frac{c}{n_1\,\lambda_1} = \frac{(3\times 10^8)}{(1.00)\,(589\times 10^{-9})}
=5.09\times 10^{14} \,{\rm Hz}.
$$

