\section{Electric Potential}\label{spotn}
\subsection{Electric Potential Energy}\label{s3.1}
Consider a charge $q$ placed in a uniform electric field
${\bf E}$ ({\em e.g.}, the field between two oppositely charged, parallel
conducting plates). Suppose that we {\em very slowly}\/ displace the charge by a vector displacement ${\bf r}$ in a straight-line. How much work must we perform
in order to achieve this? Well, the force ${\bf F}$ we must exert on the charge is
equal and opposite to the electrostatic force $q\,{\bf E}$ experienced by the charge
({\em i.e.}, we must overcome the electrostatic force on the charge before we are
free to move it around). 
The amount of work $W$ we would perform in displacing the charge is simply the product of the force  ${\bf F}=-
q\,{\bf E}$ we
exert, and the displacement of the charge {\em in the direction of this force}.
Suppose  that the displacement vector subtends an angle $\theta$ with
the electric field ${\bf E}$. It follows that
\begin{equation}\label{e5.1}
W = {\bf F} \!\cdot\! {\bf  r} = -q\,{\bf E}\!\cdot\! {\bf r} = -q\, E \,r\,
\cos\theta.
\end{equation}
Thus, if we move a positive charge in the direction of the electric field then we
do negative work ({\em i.e.}, we gain energy). Likewise,
if we move a positive charge in the opposite direction to the electric field
then we do positive work ({\em i.e.}, we lose energy). 

Consider  a set of point  charges,
distributed in space, which are rigidly clamped in position so that they cannot
move. We already know how to calculate the electric field ${\bf E}$ generated by such a
charge distribution (see Sect.~\ref{scharge}). In general, this electric
field is going to be
non-uniform. Suppose that we place a charge $q$ in the field, at point $A$, say,
and 
then slowly
move it along some curved path to a different point $B$. How much work must we perform in order to
achieve this? Let us split up the charge's path from point $A$ to point
$B$ into a series of $N$ straight-line segments, where the $i$th segment
is of length ${\mit\Delta}r_i$ and subtends an angle $\theta_i$ with the
local electric field $E_i$. If we make $N$ sufficiently large then we can
adequately represent any curved path between $A$ and $B$, and we can also ensure
that $E_i$ is approximately uniform along the $i$th path segment. By a simple
generalization of  Eq.~(\ref{e5.1}), the work $W$ we must perform in moving
the charge from point $A$ to point $B$ is 
\begin{equation}\label{e5.2}
W =- q\sum_{i=1}^N E_i\,{\mit\Delta}r_i\,\cos\theta_i.
\end{equation}
Finally, taking the limit in which $N$ goes to infinity, the right-hand side of
the above expression becomes a line integral:
\begin{equation}\label{e5.3}
W = - q\,\int_A^B{\bf E}\cdot d{\bf r}.
\end{equation}

Let us now consider the special case where point $B$ is identical with point
$A$. In other words, the case in which we move  the charge around a {\em closed loop}\/ in the electric
field. How much work must we perform in order to achieve this? 
It is, in fact, possible to prove, using rather high-powered mathematics, that
the net work performed when a charge is moved around a closed loop in an
electric field generated by fixed charges is {\em zero}. However,
we do not need to be  mathematical geniuses to appreciate that this is
a sensible result.
Suppose, for the sake of argument, that  the net work performed when we  take a charge around some
closed loop in an electric field is non-zero. In other words, we  lose energy
every time we take the charge around the loop in one direction, but gain energy
every time we take the charge around the loop in the opposite direction. This
follows from Eq.~(\ref{e5.2}), because when we switch the direction of circulation
around the loop the electric field $E_i$ on the $i$th path segment is unaffected, but,
since the charge is moving along the segment in the opposite direction,
$\theta_i \rightarrow 180^\circ +\theta_i$, and, hence, $\cos\theta_i\rightarrow
-\cos\theta_i$. Let us choose to move the charge around the loop in the direction
in which we gain energy. So, we move the charge once around the loop, and
we gain a certain amount of energy in the process. Where does this energy come from? Let us 
consider the possibilities. Maybe the electric field of the movable charge 
does negative work on the fixed charges, so
that the latter charges lose energy in order to compensate for the energy which
we gain? But, the  fixed charges  cannot move, and so
it is impossible to do work on them. Maybe the electric field
loses energy in order to compensate for the energy which
we gain? (Recall, from the previous section, that there is an energy associated
with an electric field which fills space).  But, all of the charges ({\em i.e.}, the
fixed charges and the movable charge)
 are in the same position before and after we take the
movable charge around the loop, and so the electric field is the same before and
after (since, by Coulomb's law, the electric field only depends on the positions
and magnitudes of the charges), and, hence, the energy of the field must be
the same before and after. Thus, we have a situation in which we take a
charge around a closed loop in an electric field, and gain energy in the process,
but nothing loses energy. In other words, the energy appears out of
``thin air,'' which clearly violates the first law of thermodynamics. 
The only way in which we can avoid this absurd conclusion is
if we adopt the following rule:
\begin{quote}
{\sf The work done in taking a charge around a closed loop in an electric
field generated by fixed charges is   zero.}
\end{quote}
 

One  corollary of the above rule is that the work done in moving  a
charge between two points $A$ and $B$ in such an electric field is {\em independent}
of the path taken between these points. This is easily proved. Consider
 two different paths, 1 and 2, between points $A$ and $B$.
 Let the work done in taking the charge from $A$ to $B$ along
path 1 be $W_1$, and the work done in taking the charge from $A$ to $B$ along
path 2 be $W_2$. Let us  take the charge
from $A$ to $B$ along path 1, and then from $B$ to $A$ along path 2. The net
work done in taking the charge around this closed loop is $W_1-W_2$. 
Since we know this work must be zero, it immediately follows that $W_1=W_2$. Thus,
we have a new rule:
\begin{quote}
{\sf The work done in taking a charge between two points in an electric field
generated by fixed charges is independent of the path taken between the points.}
\end{quote}

A force which has the special property that the work done in overcoming it
in order to move a body between two points in space is independent of the
path taken between these points is called a {\em conservative force}. 
The electrostatic force between stationary charges is clearly a
conservative force. Another example of a conservative force is the force
of gravity (the work done in lifting a mass only depends on the difference
in height between the beginning and end points, and not on the path
taken between these points). Friction is an obvious example of a non-conservative
force. 

Suppose that we move a charge $q$ very slowly from point $A$ to point $B$
in an electric field generated by fixed charges. The work $W$ which we must perform in order to
achieve this  can be calculated
using Eq.~(\ref{e5.3}). Since we lose the energy $W$ as the charge moves from $A$ to
$B$, something must gain this energy. Let us, for the moment, suppose that this
something is the charge. Thus, the charge {\em gains}\/ the energy $W$ when
we move it from point $A$ to point $B$. What is the nature of this energy gain?
It certainly is not a gain in kinetic energy, since we are moving the particle
 {\em slowly}:
{\em i.e.}, such that it always possesses negligible kinetic energy. 
In fact, if we think carefully, we can see that the gain in energy of the
charge depends only on its {\em position}. For a fixed starting point $A$, the work $W$ 
done in taking the charge from point $A$ to point $B$ depends only on the
position of point $B$, and not, for instance, on the route taken between
$A$ and $B$. We usually call energy a body possess by virtue of its position 
{\em potential energy}: {\em e.g.}, a mass has a certain {\em gravitational potential energy}
which depends on its height above the ground. Thus, we can say that when
a charge $q$ is taken from point $A$ to point $B$ in an electric field generated by fixed charges its
{\em electric potential energy}\/ $P$ increases by an amount $W$:
\begin{equation}\label{e5.4}
P_B - P_A = W.
\end{equation}
Here, $P_A$ denotes the electric  potential energy of the charge at point $A$,
{\em etc.} This definition uniquely defines the {\em difference} in the potential energy
between points $A$ and  $B$ (since $W$ is independent of the path taken
between these points), but the absolute value of the potential energy
at point $A$ remains arbitrary.

We have seen that when a charged particle is taken from point $A$ to point $B$ in an electric field its electric potential energy increases by the amount specified
in Eq.~(\ref{e5.4}). But, how does the particle store this energy? In fact, the particle
does not store the energy at all. Instead, the energy is stored in the electric field
surrounding the particle. It is possible to calculate this increase in the
energy of the field directly (once we know the formula which links the energy density
of an electric field to the magnitude of the field), but it is a very tedious 
calculation. It is far easier to calculate the work $W$ done in taking the
charge from point $A$ to point $B$, via Eq.~(\ref{e5.3}), and then use the
conservation of energy to conclude that the energy of the electric field must
have increased by an amount $W$. The fact that we conventionally ascribe this energy
increase to the particle, rather than the field, via the concept of electric
potential energy, does not matter for all practical purposes. For instance, we call the
money which we have in the bank ``ours,'' despite the fact that the bank has possession of it,
because we know that the bank will return the money  to us any time we ask them. 
Likewise, when we move a charged particle in an
electric field from point $A$ to point $B$ then the energy of the field increases by an amount $W$
(the work which we perform in moving the particle from $A$ to $B$), but we can
safely  associate
this energy increase with the particle because we know that if the particle is
moved back to point $A$ then the field will give all of the
energy back to the particle {\em without loss}. Incidentally, we can be sure that
the field returns the energy  to the particle without loss because if there
were any loss then this would imply that non-zero work is done in taking a charged
particle around a closed loop in an electric field generated by fixed charges. We call a force-field which
stores energy without loss a {\em conservative field}. Thus, an electric field, or rather
an {\em electrostatic field}\/ ({\em i.e.}, an electric field generated by
stationary charges), is  conservative. It should be clear, from the above
discussion, that the concept of potential energy is only meaningful if the field
which generates the force in question is conservative. 

A gravitational field is another example of a conservative field. It turns out
that when we lift a body through a certain height the increase in gravitational
potential energy of the body is actually stored in the surrounding
gravitational field ({\em i.e.}, in the distortions of space-time around the
body). It is possible to determine the increase in energy of the gravitational
field directly, but it is a very difficult 
calculation involving General Relativity. 
On the other hand, it is very easy to calculate the work done in lifting the body. 
Thus, it is convenient to calculate the increase in the energy
of the field  from the work done, and
then to ascribe this energy increase to the body, via the concept of
gravitational potential energy. 

In conclusion, we can evaluate the increase in electric potential energy of a charge when
it is taken between two different points in an electrostatic field from the
work done in moving the charge between these two points. The energy is actually
stored in the electric field surrounding the charge, but we can safely ascribe
this energy to the charge, because we know that the field stores the energy without loss,
and will return the energy to the charge whenever it is required to do so by the
laws of Physics. 

\subsection{Electric Potential}
Consider a charge $q$ placed in an electric field generated by fixed charges.
Let us chose some arbitrary reference point $A$ in the field.
At this point, the  electric
potential energy of the charge is defined to be zero. This 
{\em uniquely}\/ specifies the
electric potential energy of the charge at every other point in the field. 
For instance, the electric potential energy $P_B$ at some point $B$ is
simply the work $W$ done in moving the charge from $A$ to $B$ along any path. Now,
$W$ can be calculated using Eq.~(\ref{e5.3}). It is clear, from this equation, that
$P_B$ depends  both on the particular charge $q$ which we place in the
field, and the magnitude and direction of the electric field along the chosen route between points $A$ and $B$. However, it is also clear that
$P_B$ is {\em directly proportional}\/ to the magnitude of the
charge $q$. Thus, if the electric
potential energy of a charge $q$ at point $B$ is $P_B$ then the electric potential
energy of a charge $2\,q$ at the same point is $2 \,P_B$. We can exploit
this fact to define a quantity known as the {\em electric potential}. The
difference in electric potential between two points $B$ and $A$ in an electric
field is simply the work done in moving some charge between the two points
divided by the magnitude of the charge. Thus,
\begin{equation}\label{e5.5}
V_B - V_A = \frac{W}{q},
\end{equation}
where $V_A$ denotes the electric potential at point $A$, {\em etc.} This definition
uniquely defines the difference in electric potential between points $A$ and
$B$, but the absolute value of the potential at point $A$ remains arbitrary. 
We can therefore, without loss of generality, set the potential at point $A$ equal to
zero. It follows that the potential energy of a charge $q$ at some point $B$
is simply the product of the magnitude of the charge and the electric
potential $V_B$ at that point:
\begin{equation}
P_B = q\,V_B.
\end{equation}
It is clear, from a comparison of Eqs.~(\ref{e5.3}) and (\ref{e5.4}), that the electric
potential at point $B$ (relative to point $A$) is solely a property of the
electric field, and is, therefore, the same for any charge placed at that point. 
We shall see exactly  how the electric potential is related to the electric
field later on. 

The dimensions of electric potential are work (or energy) per unit charge.
The units of electric potential are, therefore, joules per coulomb (${\rm J\,C}^{-1}$).
A joule per coulomb is usually referred to as a volt (V): {\em i.e.},
\begin{equation}
1\,{\rm J\, C^{-1}} \equiv 1\,{\rm V}.
\end{equation}
 Thus, the
alternative (and more conventional) units of electric potential are volts. 
The difference in electric potential between two points in an electric field
is usually referred to as the {\em potential difference}, or even the difference
in ``voltage,''
between the two points. 

A battery is a convenient tool for generating a difference in electric potential
between two points in space. For instance, a twelve volt (12\,V) battery 
generates an electric field, usually via some chemical process, which is such
that the potential difference $V_+-V_-$ between its positive and negative
terminals  is twelve volts. This means that in order to move a
positive charge of 1 coulomb from the negative to the positive terminal of
the battery we must do 12 joules of work against the electric field. (This is
true irrespective of the route taken between the two terminals). 
This implies that the electric field must be directed predominately from the
positive to the negative terminal. 

More generally, in order to move a charge $q$ through a potential difference ${\mit\Delta}V$ we must do work $W=q\,{\mit\Delta} V$, and the electric potential energy
of the charge increases by an amount ${\mit\Delta}P = q\,{\mit\Delta} V$ in the
process. Thus, if we move an electron, for which $q=-1.6\times 10^{-19}$\,C,
through a potential difference of minus 1 volt then we must do
$1.6\times 10^{-19}$ joules of work. This amount of work (or energy) is
called an {\em electronvolt} (eV): {\em i.e.},
\begin{equation}
1\,{\rm eV} \equiv 1.6\times 10^{-19}\,{\rm J}.
\end{equation}
The electronvolt is a convenient measure of energy in atomic physics.
For instance, the energy required to
break up a hydrogen atom into a free electron and a free proton is $13.6$\,eV.

\subsection{Electric Potential and  Electric Field}\label{s5.3}
We have seen that the difference in electric potential between two
arbitrary points in space is a function of the electric field which permeates space,
but is independent of the test charge used to measure this difference. 
Let us investigate the relationship between electric potential and the electric
field. 

Consider a charge $q$ which is slowly moved an infinitesimal  distance $dx$
along the $x$-axis. Suppose that the difference in electric potential
between the final and initial positions of the charge is $dV$.
By definition, the change $dP$ 
in the charge's electric potential energy 
is given by
\begin{equation}
dP = q \,dV
\end{equation}
From Eq.~(\ref{e5.1}), the work $W$ which we perform in moving the charge is
\begin{equation}
W = -q\,E\,dx\,\cos\theta,
\end{equation}
where $E$ is the local electric field-strength, and $\theta$ is the angle subtended
between the direction of the field and the $x$-axis. By definition,
$E\,\cos\theta=E_x$, where $E_x$ is the $x$-component of the local electric field.
Energy conservation demands that ${\mit\Delta P} = W$
({\em i.e.}, the increase in the charge's energy matches the
work done on the charge), or
\begin{equation}
q \,dV = -q\,E_x\,dx,
\end{equation}
which reduces to
\begin{equation}\label{e5.11}
E_x = - \frac{dV}{dx}.
\end{equation}
We call the quantity $dV/dx$ the {\em gradient}\/ of the
electric potential in the $x$-direction. It basically measures how fast
the potential $V$ varies as the coordinate $x$ is changed (but the
coordinates $y$ and $z$ are held constant). Thus, the above formula is saying
that the $x$-component of the electric field at a given point in space is equal
to {\em minus}\/ the local gradient of the electric potential in the
$x$-direction.  

According to Eq.~(\ref{e5.11}), electric field strength has dimensions
 of potential difference
over length. It follows that the units of electric field are volts
per meter (${\rm V\, m^{-1}})$. 
Of course, these new units are entirely equivalent to
newtons per coulomb: {\em i.e.},
\begin{equation}
1\,{\rm V \,m^{-1}} \equiv 1\,{\rm N\, C^{-1}}.
\end{equation}

Consider the special case of a uniform $x$-directed electric field $E_x$ 
generated by two uniformly charged parallel planes normal to the $x$-axis. It is
clear, from Eq.~(\ref{e5.11}), that if $E_x$ is to be constant between the plates
then $V$ must vary {\em linearly}\/ with $x$ in this region. In fact, it is
easily shown that
\begin{equation}\label{e5.13}
V(x) = V_0 - E_x\,x,
\end{equation}
where $V_0$ is an arbitrary constant. According to Eq.~(\ref{e5.13}), the electric potential $V$ {\em decreases}\/
continuously as we move
along the direction of the electric field. Since a positive charge is
accelerated in this direction, we  conclude that  positive charges are
accelerated {\em down} gradients in the electric potential, in much the same manner
as masses fall down gradients of gravitational potential (which is, of course,
proportional to height). Likewise, negative charges are accelerated {\em up}
gradients in the electric potential. 

According to Eq.~(\ref{e5.11}), the $x$-component of the electric field is equal
to minus the gradient of the electric potential in the $x$-direction.
Since there is nothing special about the $x$-direction, analogous rules
must exist for the $y$- and $z$-components of the field. 
These three rules can be combined to give
\begin{equation}\label{e5.14}
{\bf E} = - \left(\frac{dV}{dx}, 
\frac{dV}{dy}, \frac{dV}{dz}
\right).
\end{equation}
Here, the $x$ derivative is taken at constant $y$ and $z$, {\em etc.}
The above expression shows how the electric field ${\bf E}({\bf r})$, which is a vector field, is related to the electric
potential $V({\bf r})$, which is a scalar field. 

We have seen that electric fields are superposable. That is, the electric
field generated by a set of charges distributed in space is
simply the {\em vector sum} of the electric fields generated by each charge
taken separately. Well, if electric fields are superposable, it follows
from Eq.~(\ref{e5.14}) that electric potentials must also be superposable. Thus,
the electric potential generated by a set of charges distributed in space
is just the {\em scalar sum} of the potentials generated by each charge taken in isolation. Clearly, it is far easier to determine the potential generated by a set
of charges than it is to determine the electric field, since we can 
sum the potentials
generated by the individual charges algebraically, and do not have to worry about
their directions (since they have no directions). 
 

Equation~(\ref{e5.14}) looks rather forbidding. Fortunately, however, it is possible
to rewrite this equation in a more appealing form. Consider two neighboring
points $A$ and $B$. Suppose that $d{\bf r}  = (dx, dy,
dz)$ is the vector displacement of point $B$ relative to point $A$. 
 Let $dV$ be the difference in electric potential
between these two points. 
Suppose that we travel from $A$ to $B$ by first moving a distance $dx$
along the $x$-axis, then moving $dy$ along the $y$-axis,
and finally moving $dz$ along the $z$-axis. The net  increase 
in the electric potential $dV$ as we move from $A$ to $B$ 
is simply the sum of the increases $d_x V$ as we move along the $x$-axis,
$d_y V$ as we move along the $y$-axis, and $d_z V$ as we move along the $z$-axis:
\begin{equation}
dV = d_x V + d_y V + d_z V.
\end{equation}
But, according to Eq.~(\ref{e5.14}), $d_x V =- E_x\,dx$, {\em etc.}
So, we obtain
\begin{equation}
dV =- E_x\,dx - E_y \,dy -E_z\,dy,
\end{equation}
which is equivalent to 
\begin{equation}\label{e5.18}
dV = -{\bf E}\!\cdot\! d{\bf r}= -E\,dr \,\cos\theta,
\end{equation}
where $\theta$ is the angle subtended between the vector $d{\bf r}$ and
the local electric field ${\bf E}$. Note that $dV$ attains
its most negative value when $\theta=0$. In other words, the direction of the
electric field at point $A$ corresponds to the direction in which the electric
potential $V$ decreases most rapidly. A positive charge placed at point $A$
is accelerated in this direction. Likewise, a negative charge placed at $A$ is
accelerated in the direction in which the potential increases most rapidly
({\em i.e.}, $\theta = 180^\circ$). Suppose that we move from point $A$ to a neighboring point
$B$ in a direction perpendicular to that of the local electric
field ({\em i.e.}, $\theta = 90^\circ$). In this case, it follows from Eq.~(\ref{e5.18}) that the  points $A$ and $B$ lie at the same electric potential ({\em i.e.}, $dV=0$). The locus of all the points in the vicinity of point $A$ which lie at the
same potential as $A$ is  a plane perpendicular to the direction of the
local  electric
field. More generally, the surfaces of constant electric potential, the so-called
{\em equipotential surfaces}, exist as a set of non-interlocking surfaces which are
everywhere perpendicular to the direction of the electric field. Figure~\ref{f5.4} shows the
equipotential surfaces (dashed lines) and electric field-lines (solid lines)
generated by a positive point charge. In this case, the equipotential surfaces are
spheres centred on the charge. 

\begin{figure}[h]
\epsfysize=2.5in
\centerline{\epsffile{Chapter05/fig5.1.eps}}
\caption{\em The equipotential surfaces (dashed lines) and the electric
field-lines (solid lines) of a positive point charge.}\label{f5.4}
\end{figure}

In Sect.~\ref{snormal}, we found that the electric field immediately above the surface of
a conductor is directed perpendicular to that surface. Thus, it is clear that the
surface of a conductor must correspond to an equipotential surface. In fact, since there
is no electric field inside a conductor (and, hence, no gradient in the electric
potential), it follows that the whole conductor ({\em i.e.}, both the surface and the
interior)  lies at the same electric potential. 

\subsection{Electric Potential of a Point Charge}
Let us calculate the electric potential $V({\bf r})$ generated by a point charge $q$ located at
the origin. It is fairly obvious, by symmetry, and also by looking at Fig.~\ref{f5.4}, that
 $V$ is a
function of  $r$ only, where $r$ is the radial distance 
from the origin. Thus, without loss of generality, we can restrict our
investigation to the
potential $V(x)$ generated  along the positive $x$-axis. The $x$-component of the electric
field generated along this axis takes the form 
\begin{equation}
E_x(x)= \frac{q}{4\pi\epsilon_0\,x^2}.
\end{equation}
 Both the
$y$- and $z$-components of the field are zero. According to Eq.~(\ref{e5.11}), $E_x(x)$ and
$V(x)$ are related via 
\begin{equation}
E_x(x) = - \frac{dV(x)}{dx}.
\end{equation}
Thus, by integration,
\begin{equation}
V(x)= \frac{q}{4\pi\epsilon_0\,x} + V_0,
\end{equation}
where $V_0$ is an arbitrary constant. Finally, making use of the
fact that $V = V(r)$, we obtain
\begin{equation}\label{e5.22}
V({\bf r}) = \frac{q}{4\pi\epsilon_0\,r}.
\end{equation}
Here, we have adopted the common convention that the potential at infinity
is zero. A potential defined according to this convention is  called
an {\em absolute potential}.

Suppose that we have $N$ point charges distributed in space. Let the
$i$th charge $q_i$ be located at position vector ${\bf r}_i$. Since
electric potential is superposable, and is also a scalar quantity, the
absolute potential at position vector ${\bf r}$ is simply the
algebraic sum of the potentials generated by each charge taken in
isolation:
\begin{equation}
V({\bf r}) = \sum_{i=1}^N \frac{q_i}{4\pi\epsilon_0\,|{\bf r} - {\bf r}_i|}.
\end{equation}
The work $W$ we would perform in taking a charge $q$ from infinity and slowly moving
it to point ${\bf r}$ is the same as  the increase in electric potential
energy of the charge during its journey [see Eq.~(\ref{e5.4})]. This,
by definition, is equal to the product of the charge $q$ and the increase in
the electric potential. This, finally, is the same as $q$ times the
absolute potential at point ${\bf r}$: {\em i.e.}, 
\begin{equation}
W = q\,V({\bf r}).
\end{equation}

\subsection{Worked Examples}
\subsection*{\em Example~5.1: Charge in a uniform electric field}
\noindent{\em Question:} A charge of $q=+1.20\,\mu{\rm C}$ is placed in a
uniform $x$-directed electric field of magnitude $E_x=1.40\times 10^3\,{\rm N\,C^{-1}}$.
How much work must be performed in order to move the charge a distance $c=-3.50$ cm in the
$x$-direction? What is the potential difference between the initial and final positions
of the charge? If the electric field is produced by two oppositely 
charged parallel plates
separated by a distance $d=5.00$ cm, what is the potential difference between the
plates?\\
~\\
{\em Solution:} Let us denote the initial and final positions of the charge
$A$ and $B$, respectively. The work which we must perform in order to move the
charge from $A$ to $B$ is minus the product of the electrostatic force on the
charge due to the electric field 
(since the force we exert on the charge is minus this force)
and the distance that the charge moves in the
direction of this force [see Eq.~(\ref{e5.1})]. Thus,
$$
W = -q \,E_x\, c= -(1.2\times 10^{-6})\,(1.40\times 10^3)\,(-3.50\times 10^{-2}) =
+5.88 \times 10^{-5}\,{\rm J}.
$$
Note that the work is positive. This makes sense, because we would have to do real
work ({\em i.e.}, we would lose energy) in order to move a positive charge in the opposite direction
to an electric field ({\em i.e.}, against the direction of the electrostatic
force acting on the charge).

The work done on the charge goes to increase its electric
potential energy, so  $P_B - P_A = W$. By definition, this increase in
potential energy is equal to the product of the potential difference
$V_B- V_A$ between points $B$ and $A$, and the magnitude of the charge $q$.
Thus,
$$
q\,(V_B- V_A)= P_B - P_A = W  = - q\,E_x\,c,
$$
giving
$$
V_B - V_A = -E_x\,c= -(1.40\times 10^3)\,(-3.50\times 10^{-2})=49.0\,{\rm V}.
$$
Note that the electric field is directed from point $B$ to point $A$, and
that the former point is  at a higher potential than the latter.

It is clear, from the above formulae, that the magnitude of the potential difference
between two points in a uniform electric field is simply the product of the
electric field-strength and the distance between the two points (in the direction
of the field). Thus, the potential difference between the two metal plates is
$$
{\mit\Delta}V = E_x\,d = (1.40\times 10^3)\,(5.00\times 10^{-2})=70.0\,{\rm V}.
$$
If the electric field is directed from plate 1 (the positively charged plate)
to plate 2 (the negatively charged plate) then the former plate is at the
higher potential. 

\subsection*{\em Example~5.2: Motion of  an electron in an electric field}
\noindent{\em Question:} 
An electron in a television set is accelerated from the cathode to the
screen through a potential difference of +1000 V. The screen is 35 mm from
the cathode. What is the net change in the potential energy of the electron
during the acceleration process? 
How much work is done by the electric field in accelerating the
electron?
What is the speed of the electron when it strikes
the screen?\\
~\\
{\em Solution:} Let call the cathode point $A$ and the screen point $B$. We
are told that the potential difference between points $B$ and $A$ is +1000 V,
so 
$$
V_B - V_A = 1000\,{\rm V}.
$$
By definition, the difference in electric potential energy of some charge $q$ at
points $B$ and $A$ is the product of the charge  and
the difference in  electric potential
between these points. Thus,
$$
P_B- P_A = q\,(V_B - V_A)= (-1.6\times 10^{-19})\,(1000) = -1.6\times 10^{-16}\,{\rm J},
$$
since $q=-1.6\times 10^{-19}\,{\rm C}$ for an electron.
Note that the potential energy of the electron {\em decreases}
as it is accelerated towards the screen. As we have seen, the electric
potential energy of a charge is actually held in the surrounding electric field.
Thus, a decrease in the potential energy of the charge corresponds to a reduction in the
energy of the field. In this case, the energy of
the field decreases because it does work $W'$ on the charge. 
Clearly, the work done
 ({\em i.e.}, energy lost) by the field equals the decrease in potential energy
of the charge,
$$
W' = - {\mit\Delta}P.
$$
Thus, 
$$
W' = 1.6\times 10^{-16}\,{\rm J}.
$$

The total energy $E$ of the electron is made up of two components---the electric
potential energy $P$, and the kinetic energy $K$. Thus,
$$
E = P + K.
$$
Of course,
$$
K = \frac{1}{2} \,m\, v^2,
$$
where $m=9.11\times 10^{-31}$ kg is the mass of the electron, and $v$ 
its speed. By  conservation of energy, $E$ is a constant of the motion, so
$$
K_B - K_A = {\mit\Delta} K = - {\mit\Delta}P.
$$
In other words, the decrease in electric potential energy of the electron, as 
it is accelerated towards the screen, is offset by a corresponding increase in its kinetic
energy.
Assuming that the electron starts from rest ({\em i.e.} $v_A=0$), it follows that
$$
\frac{1}{2}\,m\,v_B^{~2} =  - {\mit\Delta}P,
$$
or
$$
v_B = \sqrt{\frac{-2\,{\mit\Delta}P}{m}} = \sqrt{
\frac{-2\,(-1.6\times 10^{-16})}{9.11\times
10^{-31}}} = 1.87\times 10^7\,{\rm m\,s^{-1}}.
$$
Note that the distance between the cathode and the screen is immaterial in this
problem. The final speed of the electron is entirely determined by
its charge, its initial velocity,
 and the potential difference through which it is accelerated. 


\subsection*{\em Example~5.3: Electric potential due to point charges}
\noindent{\em Question:}~ A particle of charge $q_1=+6.0\,\mu{\rm C}$ is located
on the $x$-axis at the point $x_1=5.1\,{\rm cm}$. A second particle of
charge $q_2=-5.0\,\mu{\rm C}$ is placed on the $x$-axis at $x_2=-3.4\,{\rm cm}$. What is
the absolute electric potential at the origin ($x=0$)? How much work must we perform in
order to slowly move a charge of $q_3=-7.0\,\mu{\rm C}$ from infinity to the origin, whilst keeping
the other two charges fixed?\\
~\\
\noindent{\em Solution:}~The absolute electric potential at the origin due to the first
charge is 
$$
V_1 = k_e\,\frac{q_1}{x_1} = (8.988\times 10^9)\,\frac{(6\times 10^{-6})}{(5.1\times 10^{-2})}= 1.06 \times 10^6\,{\rm V}.
$$
Likewise, the absolute electric potential at the origin due to the second charge is
$$
V_2 = k_e\,\frac{q_2}{|x_2|} = (8.988\times 10^9)\,\frac{(-5\times 10^{-6})}{(3.4\times 10^{-2})}=-1.32\times 10^6\,{\rm V}.
$$
The net potential $V$ at the origin is simply the algebraic sum of the potentials due to each charge
taken in isolation. Thus,
$$
V = V_1 + V_2 = -2.64\times 10^5\,{\rm V}.
$$

The work $W$ which we must perform in order to slowly moving a charge $q_3$ from infinity to the origin is
simply the product of the charge and the potential difference $V$ between the end and beginning
points. Thus,
$$
W = q_3\, V = (-7\times 10^{-6})\,(-2.64\times  10^5) = 1.85\,{\rm J}.
$$

\subsection*{\em Example~5.4: Electric potential due to point charges}
\begin{figure*}[h]
\epsfysize=2in
\centerline{\epsffile{Chapter05/fig1.eps}}
\end{figure*}
\noindent{\em Question:}~Suppose that three point charges, $q_a$, $q_b$, and $q_c$, are arranged at the
vertices of a right-angled triangle, as shown in the diagram. What is the absolute electric potential of the third charge if
$q_a=-6.0\,\mu{\rm C}$, $q_b=+4.0\,\mu{\rm C}$, $q_c = +2.0\,\mu{\rm C}$,
$a=4.0$\,m, and $b=3.0$\,m? Suppose that the third charge, which is initially
at rest,  is repelled to infinity by the
combined electric field of the other two charges, which are held fixed. What is the final  kinetic energy
of the third charge?\\
~\\
\noindent{\em Solution:}~The absolute electric potential of the third charge due to the presence of the
first charge is 
$$
V_a = k_e\,\frac{q_a}{c} = (8.988\times 10^9)\,\frac{(-6\times 10^{-6})}{(\sqrt{4^2+3^2})}= - 1.08 \times 10^4\,{\rm V},
$$
where use has been made of the Pythagorean theorem.
Likewise, the absolute electric potential of the third charge due to the presence of the second
charge is
$$
V_b = k_e\,\frac{q_b}{b} = (8.988\times 10^9)\,\frac{(4\times 10^{-6})}{(3)}=1.20\times 10^4\,{\rm V}.
$$
The net absolute potential of the third charge $V_c$ is simply the algebraic sum of the
potentials due to the other two charges taken in isolation. Thus,
$$
V_c = V_a + V_b =1.20\,\times 10^3\,{\rm V}.
$$

The change in electric potential energy of the third charge as it moves from its initial
position to infinity is the product of the third charge, $q_c$, and the difference in electric
potential ($-V_c$) between infinity and the initial position. It follows that
$$
{\mit\Delta}P = -q_c\,V_c = -(2\times 10^{-6})\,(1.2\times 10^3)=-2.40\times 10^{-3}\,{\rm J}.
$$
This decrease in the potential energy of the
charge is offset by a corresponding increase ${\mit\Delta}K= -{\mit\Delta}P$
in its kinetic energy. Since the initial kinetic energy of the third charge is zero (because it
is initially at rest), the final kinetic energy is simply
$$
K = {\mit\Delta}K = -{\mit\Delta}P = 2.40\times 10^{-3}\,{\rm J}.
$$
