\section{Electric Current}
\subsection{Electric Circuits}
A battery is a device possessing  a {\em positive}\/ and a {\em
negative}\/
terminal. Some process, usually a chemical reaction, takes place inside the
battery which causes positive charge to migrate towards the positive terminal,
and {\em vice versa}. This process
continues until the electric field set up between the two terminals is
sufficiently strong to inhibit any further charge migration.

An {\em electric circuit}\/ is a conducting path, external
to the battery,  which allows charge to
flow from one terminal 
to the other. A simple circuit might consist of a
single strand of metal wire linking the positive and negative terminals.
A more realistic circuit possesses multiple branch points, so that charge can take
many different paths between the two terminals. 

Suppose that a (positive) charge $q$ is driven around the external
circuit, from the positive to
the negative terminal, by the electric field set up between the
terminals. The work done on the charge by this field during its
journey  is $q\,V$, where $V$ is the difference
in electric potential between the positive and negative terminals. We usually refer to $V$ as the {\em voltage} of the battery:
{\em e.g.}, when we talk of a 6 volt battery, what we actually  mean is
that the potential difference between its two terminals is 6 V. Note, from
Sect.~\ref{spotn}, that the electrical work $q\,V$ done on the charge 
is completely independent of the route it takes between the
 terminals. In other words, although there are, in general,
 many
different paths through the external circuit which the charge could take in order
to get from the positive to the negative terminal of the 
battery, the electrical energy which
the charge acquires in making this journey is always the same. Since, when analyzing
electrical circuits, we are primarily interested in {\em energy}\/ ({\em i.e.}, 
in the transformation of the chemical energy of the battery into
heat energy in some electric heating element, or mechanical
energy in some electric motor, {\em etc.}), 
it follows that the  property of a battery which primarily concerns us is its
{\em voltage}. Hence, we do not have to map out the  electric field
generated by a battery in order to calculate how much energy this field gives to
a charge $q$  which goes around some external circuit connected to it.
All we need to know is the potential difference $V$ between the two terminals
of the battery. This is obviously an enormous simplification. 

This section is only concerned with {\em steady-state}\/ electric circuits
powered by batteries of constant voltage. Thus, the rate at which electric
charge flows out of the positive terminal of the battery into the external
circuit must match the rate at which charge flows from the circuit into
the negative terminal of the battery, otherwise charge would build up in
either the battery or the circuit, which would not correspond to a steady-state
situation. The rate at which charge flows out of the positive terminal
is termed the {\em electric current}\/ flowing out of the battery. Likewise,
the rate at which charge flows into the negative terminal is termed the current
flowing into the battery. Of course, these two currents must be the same in
a steady-state. 
Electric current is measured in units of amperes (A), which are
equivalent to coulombs per second:
\begin{equation}
1\,{\rm A} \equiv 1\,{\rm C \,s}^{-1}.
\end{equation}
We can define the electric current $I$ flowing at any particular point in the
external circuit as follows. If an amount of charge $dQ$ flows past this point
in an infinitesimal  time interval $dt$ then
\begin{equation}
I= \frac{d Q}{dt}.
\end{equation}
By convention, the direction of the current is taken to be the
direction positive charges would have to move in order to account for the
flow of charge.
In a steady-state, the current at all points in the external circuit must
remain constant in time. We call this type of circuit a {\em direct current}\/ (DC)
circuit because the current always flows in the same direction. There is
a second type of circuit, called an {\em alternating current}\/ (AC) circuit, in
which the current periodically switches direction. 

Consider a simple circuit in which a steady current $I$ flows 
around a single conducting wire connecting the positive and negative
terminals of a battery of voltage $V$. Let us suppose that the current is
carried by positive charges flowing around the external circuit from the positive to
the negative terminal. In reality, the current is
carried by negative charges ({\em i.e.}, by electrons) flowing in the opposite
direction, but for most purposes we can safely ignore this rather inconvenient fact.
Every charge $q$ which flows around the external circuit experiences a
potential drop $V$. In order to flow around the circuit again, the charge must
be raised  to the potential of the positive terminal
of the battery. This process occurs inside the battery, as the charge migrates from the
negative to the positive terminal. The energy $q \,V$
required to move the charge between the two terminals is derived  from the
energy released by the chemical reactions taking place inside the
battery. 

The simple circuit described above is somewhat analogous to a small ski resort. 
The charges flowing around the external circuit are like people skiing
down the ski-slope. The charges flow down a gradient of electric
potential just as the people ski down a gradient of gravitational potential.
Note that the good skiers who ski directly down the slope acquire exactly the
same gravitational energy as the poor skiers who ski from side to side. 
In both cases, the total acquired energy depends only on the
difference in height between the top and bottom of the slope. Likewise, charges
flowing around  an external  circuit acquire the same electrical energy no matter what route
they take, because the acquired energy only depends on the potential difference
between the two terminals of the battery. Once the people in our ski resort
reach the bottom of the
slope, they must be lifted to the top  in a ski-lift 
before they can ski down it again. Thus, the ski-lift in our  resort plays an analogous
role to the battery in our circuit. Of course, the ski-lift must
expend non-gravitational energy in order to lift skiers to the top of the slope, in just the
same manner as the battery must expend non-electrical energy to move charges up a potential
gradient. If the ski-lift runs out of energy then the circulation of skiers
in the resort rapidly stops. Likewise, if the battery runs out of energy ({\em i.e.},
if the battery ``runs down'') then the current in the external circuit stops flowing. 

\subsection{Ohm's Law}
Consider, again, a simple circuit in which a steady current $I$ flows through
a single conducting wire connecting the positive and negative terminals
of a battery of voltage $V$. What is the relationship between the current
$I$ flowing in the wire and the potential difference $V$ applied across
the two ends of the wire by  the battery? If we were to investigate this relationship experimentally we would quickly conclude that the current
$I$ is {\em directly proportional}\/ to the potential difference  $V$. In other words,
\begin{equation}
V = I\,R,
\end{equation}
where the constant of proportionality $R$ is termed the (electrical) {\em resistance}
of the wire. The above formula is called {\em Ohm's law}\/ after its
discoverer, the early nineteenth century German physicist Georg Simon Ohm. 
The unit of electrical resistance is the ohm ($\Omega$), which is
equivalent to a volt per ampere:
\begin{equation}
1\,\Omega \equiv 1\,{\rm V\,A}^{-1}.
\end{equation}

There is a slight discrepancy between what we are saying
now, and what we said earlier. In Sect.~\ref{spotn}, we maintained that the
 electric field inside a conductor  is zero. However,  if there is a potential difference
$V$ between the beginning and the end of a conducting wire, as described above, 
 then there must be an
electric field running along the length of the wire. In fact, if the
wire is straight, and the electric potential decreases uniformly with
distance traveled along the wire, then the longitudinal electric
field-strength is given by $E = V/L$ (see Sect.~\ref{s5.3}), where $L$ is the length of the wire. 
The earlier result that there is zero electric field inside a
conductor is equivalent to saying that conductors possess zero electrical
resistance. This follows because if $R$ is zero then the electric field, and, hence, the potential
difference $V$, must be zero, otherwise an infinite current would flow
according to Ohm's law. It turns out that good conductors ({\em i.e.}, 
copper, silver, aluminium, and most other metals)  possess 
  non-zero electrical
resistances. However, these resistances are generally so small that
if we were to connect the terminals of a battery together using a
wire fashioned out of a good conductor then the current which would flow
in the wire, according to Ohm's law, would be so large that it would
damage both the wire and the battery. We usually call such a circuit
a {\em short-circuit}. In order to prevent excessively large currents from
flowing, conventional electric circuits contain components, called {\em resistors},
whose electrical resistance is many orders of magnitude greater than that
of the conducting wires in the circuit. When we apply Ohm's
law, $V=I\,R$,  to a circuit, we usually only count the net resistance $R$ of 
all the
resistors in the circuit, and  neglect the resistances of the interconnecting
wires. This means that all of the major drops in electric potential, as we
travel around the circuit from one terminal of the battery to the other,
take place inside the resistors. The drop in potential in the conducting
wires themselves is usually negligible. 
Thus, to all intents and purposes, good conductors, and wires
made out of good conductors,  act as if they
have zero resistance, and contain zero electric field. 

\subsection{Resistance and Resistivity}\label{s7.3}
Let us attempt to find a microscopic explanation for electrical
resistance and Ohm's law. Now, electric current in metals, and most other conductors found 
in conventional electric circuits (good or bad), is carried by free electrons.
Consider a uniform  wire of cross-sectional area $A$ and length $L$ made of some
conducting material. Suppose that the potential difference between
the two ends of the wire is $V$. The longitudinal electric field inside
the wire is therefore $E=V/L$. Consider a free  electron
 of charge $q$ and mass $m$ inside the
wire. The electric field in the wire exerts a force $f=q\,E$ on the electron, causing
it to accelerate with an acceleration $a= q\,E/m$ along the direction of
the wire. However, the electron does not accelerate for ever. Eventually,
it crashes into one of the atoms in the wire. Since atoms are far more massive
than electrons, the electron loses all forward momentum every time it hits
an atom (just as we would lose all forward momentum if we ran into a brick wall). 
Suppose that the average  time interval  between
collisions is $\tau$. Of course,   this 
characteristic time interval depends on the size and number density of the atoms
in the wire. 
Immediately after the electron hits  an atom (at $t=0$, say)  its forward velocity $v$ is
zero. The electron is then accelerated by the electric
field, so $v = (q\,E/m)\,t$. The final velocity of the electron is
$v = (q\,E/m)\,\tau$, and its average velocity is
\begin{equation}
v_d = \frac{q\,E\,\tau}{2\,m}.
\end{equation}
In fact,  on average, the electron acts as though it drifts along the
wire with the constant velocity $v_d$. This velocity
 is therefore called the {\em drift velocity}.

Suppose that there are $N$ free electrons per unit volume in the wire. All
of these electrons effectively drift along the wire with the drift velocity $v_d$. 
Thus, the total charge which passes any particular point on the wire in
a time interval $dt$ is $dQ = q\,N\,(A\,v_d\,dt)$.
This follows because all free electrons contained in a tube of
length $v_d\,dt$ and cross-sectional area $A$ pass the point in
question in the time interval $dt$. 
The electric current $I$ flowing in the wire is given by
\begin{equation}
I= \frac{dQ}{d t} = \frac{q^2\,N\,\tau}{2\,m} \,\frac{A}{L} \,V.
\end{equation}
This equation can be rearranged to give Ohm's law,
\begin{equation}
V= I\,R,
\end{equation}
where
\begin{equation}\label{e7.8}
R = \rho\,\frac{L}{A},
\end{equation}
and
\begin{equation}
\rho = \frac{2\,m}{q^2\,N\,\tau}.
\end{equation}
Thus, we can indeed account for Ohm's law on a microscopic level. 
According to Eq.~(\ref{e7.8}), the resistance of a wire is proportional
to its length, and inversely proportional to its cross-sectional area.
The constant of proportionality  $\rho$ is called the {\em resistivity}\/ of the material making
up the wire. The units of resistivity are ohm-meters ($\Omega\,{\rm m}$). Table~\ref{t7.1}
below shows the resistivities of some common metals at $0^\circ$\,C.

\begin{table}\centering
\begin{tabular}{l l}\hline
Material & $\rho \,\,\,(\Omega\,{\rm m})$\\ \hline
Silver & $1.5\times 10^{-8}$\\
Copper & $1.7\times 10^{-8} $\\
Aluminium & $2.6\times 10^{-8} $\\
Iron & $8.85\times 10^{-8}$\\\hline
\end{tabular}
\caption{\em Resistivities of some common metals at $0^\circ$\,C.}\label{t7.1}
\end{table}

\subsection{Emf and Internal Resistance}
Now, real batteries are constructed from  materials which possess  non-zero resistivities.
It follows that  real batteries are not just pure voltage sources. They also possess
{\em internal resistances}. 
Incidentally, a pure voltage
source is usually referred to as an {\em emf}\/ (which stands for {\em electromotive force}). Of course,
emf is measured in units of volts. 
A battery can be modeled as an emf ${\cal E}$ connected in series with a resistor
$r$, which represents its internal resistance. Suppose that such
a battery is used to drive a current $I$ through an external load resistor $R$, as
shown in Fig.~\ref{f7.1}.
Note that in circuit diagrams an emf ${\cal E}$ is represented as two closely spaced parallel
lines of unequal length. The electric potential of the longer line is greater than
that of the shorter one by $+{\cal E}$ volts. A resistor is represented as
a zig-zag line. 

\begin{figure}
\epsfysize=2.5in
\centerline{\epsffile{Chapter07/fig7.1.eps}}
\caption{\em A battery of emf ${\cal E}$ and internal resistance $r$ connected
to a load resistor of resistance $R$.}\label{f7.1}
\end{figure}

Consider the battery in the figure. The voltage $V$ of the battery is
defined as the difference in electric potential between its positive and
negative terminals: {\em i.e.}, the points $A$ and $B$, respectively. As we move from $B$ to
$A$, the electric potential increases by $+{\cal E}$ volts as we cross the
emf, but then decreases by $I\,r$ volts as we cross the internal resistor. 
The voltage drop across the resistor follows from Ohm's law, which implies that
the drop in voltage  across a resistor $R$, carrying a current
$I$, is $I\,R$ in the direction in which the
current flows. Thus, the voltage $V$ of the battery is related to its emf 
${\cal E}$ and internal resistance $r$ via
\begin{equation}
V = {\cal E} - I\,r.
\end{equation}
Now, we usually think of the emf of a battery as being essentially constant (since it
 only depends on the chemical reaction going on inside the battery, which converts
chemical energy into electrical energy), so  we  must conclude that the voltage of a
battery actually {\em decreases}\/ as the  current drawn from it increases. 
In fact, the voltage only equals the
emf when the current is negligibly small. The current draw
from the battery cannot normally exceed the critical value
\begin{equation}
I_0=\frac{\cal E}{r},
\end{equation}
since
for $I>I_0$ the voltage $V$  becomes negative (which can only happen
if the load resistor $R$ is also negative: this is essentially impossible).
It follows that if we short-circuit a battery, by connecting its
positive and negative terminals together using  a conducting wire of negligible resistance,
the current drawn from the battery is limited by its internal resistance.
In fact, in this case, the current is equal to the maximum possible
current
$I_0$. 

A real battery is usually characterized in terms of
its  emf ${\cal E}$ ({\em i.e.}, its
voltage at zero current), and the maximum current $I_0$ which it can supply.
 For instance, a standard {\em dry cell}\/ ({\em i.e.}, the sort of
battery used to power calculators and torches) is usually rated at $1.5\,{\rm V}$
and (say) $0.1\,{\rm A}$. Thus, nothing really catastrophic is going to
happen if we short-circuit a dry cell. We will run the battery down in a
comparatively short space of time, but no dangerously large current is going to 
flow. On the other hand, a car battery is usually rated at $12\,{\rm V}$
and something like $200\,{\rm A}$ (this is the sort of current needed to
operate a starter motor). It is clear that a car battery must have a much
lower internal resistance than a dry cell. It follows  that if
we were foolish enough to short-circuit a car battery the result would be
fairly catastrophic (imagine all of the energy needed to turn over the engine of
a car  going into a thin wire connecting the battery terminals together).

\subsection{Resistors in Series and in Parallel}
Resistors are probably  the most commonly occurring 
components in  electronic circuits. 
Practical circuits often contain very complicated combinations of resistors.
It is, therefore, useful to have a set of rules for finding the equivalent
resistance of some general arrangement of resistors. It turns out that we can
always find the equivalent resistance by repeated application of
{\em two}\/ simple rules. These rules relate to resistors connected in series and
in parallel.

\begin{figure}
\epsfysize=1.25in
\centerline{\epsffile{Chapter07/fig7.2.eps}}
\caption{\em Two resistors connected in series.}\label{f7.2}
\end{figure}

Consider two resistors connected in {\em series}, as shown in Fig.~\ref{f7.2}.
It is clear that the same current $I$ flows through both  resistors.
For, if this were not the case, charge would build up in one or other
of the resistors, which would not correspond to
 a steady-state situation (thus violating
the fundamental assumption of this section). Suppose that the potential drop
from point $B$ to point $A$ is $V$. This drop is the sum of the potential
drops $V_1$ and $V_2$ across the two resistors $R_1$ and $R_2$, respectively.
Thus,
\begin{equation}
V = V_1 + V_2.
\end{equation}
According to Ohm's law, the equivalent   resistance $R_{\rm eq}$ between
 $B$ and $A$ is the ratio of the potential drop $V$ across these points
and the current $I$ which flows between them. Thus,
\begin{equation}
R_{\rm eq} = \frac{V}{I} = \frac{V_1+V_2}{I} = \frac{V_1}{I} + \frac{V_2}{I},
\end{equation}
giving
\begin{equation}\label{e7.14}
R_{\rm eq} = R_1 + R_2.
\end{equation}
Here, we have made use of the fact that the current $I$ is common to
all three resistors. Hence, the rule is
\begin{quote}
{\sf The equivalent resistance of two resistors connected in series is the
sum of the individual resistances.}
\end{quote}
For $N$ resistors connected in series, Eq.~(\ref{e7.14}) generalizes
to $R_{\rm eq} = \sum_{i=1}^N R_i$. 

\begin{figure}
\epsfysize=2in
\centerline{\epsffile{Chapter07/fig7.3.eps}}
\caption{\em Two resistors connected in parallel.}\label{f7.3}
\end{figure}

Consider two resistors connected in {\em parallel}, as shown in Fig.~\ref{f7.3}. It
is clear, from the figure, that the potential drop $V$ across the two resistors is the
same. In general, however, the currents $I_1$ and $I_2$ which flow
through resistors $R_1$ and $R_2$, respectively, are different. 
According to Ohm's law, the equivalent resistance $R_{\rm eq}$
between $B$ and $A$ is the ratio of the potential drop
$V$ across these points and the current $I$
which flows between them. This current must equal the sum of the
currents $I_1$ and $I_2$ flowing through the two resistors, otherwise
charge would build up at one or both of the junctions in the circuit.
Thus,
\begin{equation}
I = I_1 + I_2.
\end{equation}
It follows that 
\begin{equation}
\frac{1}{R_{\rm eq}} = \frac{I}{V} = \frac{I_1+I_2}{V} = \frac{I_1}{V}
+\frac{I_2}{V},
\end{equation}
giving
\begin{equation}\label{e7.17}
\frac{1}{R_{\rm eq}} = \frac{1}{R_1} + \frac{1}{R_2}.
\end{equation}
Here, we have made use of the fact that the potential drop
$V$ is common to all three resistors. Clearly, the rule is
\begin{quote}
{\sf The reciprocal of the equivalent resistance of two resistances 
connected in parallel is the sum of the reciprocals of the
individual resistances.}
\end{quote}
For $N$ resistors connected in parallel, Eq.~(\ref{e7.17}) generalizes to
$1/R_{\rm eq} = \sum_{i=1}^N (1/R_i)$. 

\subsection{Kirchhoff's Rules} 
We now know just about all that we need to know about emfs and resistors. However, 
it would be convenient  if we could distill our knowledge into
a number of handy  rules which could then be used to analyze any DC circuit. 
This is essentially what the German physicist Gustav Kirchhoff did in 1845
when he proposed {\em two}\/ simple rules for dealing with DC circuits. 

Kirchhoff's first rule applies to {\em junction points}\/ in DC circuits ({\em i.e.},
points at which three or more wires come together). The junction rule is:
\begin{quote}
{\sf The sum of all the currents entering any junction point is equal to
the sum of all the currents leaving that junction point.}
\end{quote}
This rule is easy to understand. As we have already remarked, if this rule
were not satisfied then charge would build up at the junction points, violating our
fundamental steady-state assumption. 

Kirchhoff's second rule applies to {\em loops}\/ in DC circuits. The loop rule is:
\begin{quote}
{\sf The algebraic sum of the changes in electric potential encountered in a
complete traversal of any closed circuit is equal to zero.}
\end{quote}
This rule is also easy to understand. We have already seen (in Sect.~\ref{spotn})
that zero net work is done
in slowly moving a charge $q$ around some closed loop in an electrostatic field.
Since the work done is equal to the product of the charge $q$ and the difference
${\mit\Delta}V$ in electric potential between the beginning and end points of
the loop, it follows that this difference must be zero. Thus, if we apply this
result to the special case of a
 loop in a DC circuit, we immediately arrive at Kirchhoff's second rule.
When using this rule, we first pick a closed loop in the DC circuit that
we are analyzing. Next, we decide whether we are going to traverse this
loop in a clockwise or an anti-clockwise direction (the choice is arbitrary).
If a source of emf ${\cal E}$ is traversed in the direction of increasing potential
then 
the change in potential is $+{\cal E}$. 
However, if the emf is traversed in the opposite direction then the change in potential is $-{\cal E}$. If a resistor $R$, carrying a
current $I$, is traversed in the direction of current flow then the change in
potential is $-I\,R$. Finally, if the resistor is traversed in the
opposite direction then the change in potential is $+I\,R$. 

The currents flowing around a general DC circuit can always be found by applying 
Kirchhoff's first rule to all junction points, 
Kirchhoff's second rule to all loops, and then solving the 
simultaneous algebraic equations thus obtained. This procedure works
no matter how complicated the circuit in question is ({\em e.g.}, Kirchhoff's
rules are used in the semiconductor industry to analyze the incredibly
complicated circuits, etched onto the surface of silicon wafers, which are used to
construct the central processing units of computers).  

\subsection{Capacitors in DC Circuits}
Capacitors do not play an important role in DC circuits because it
is impossible for a steady current to flow across  a capacitor. If an
uncharged 
capacitor $C$ is connected across the terminals of a battery of voltage $V$
then a {\em transient}\/ current flows as the capacitor plates charge up.
However, the current stops flowing as soon as the charge $Q$ on the positive plate
reaches the value $Q=C\,V$. At this point, the electric field between the
plates cancels  the effect of the electric field generated by the battery,
and there is no further movement of charge. Thus, if a capacitor is
placed in a DC circuit then, as soon as its plates have charged up, the capacitor
effectively behaves like a {\em break}\/ in the circuit. 

\subsection{Energy in DC Circuits}
Consider a simple circuit in which a battery of voltage $V$ drives a
current $I$ through a resistor of resistance $R$. 
As we have seen, the battery is continuously doing work 
by raising the potentials
of charges which flow into its negative terminal and then flow out of its
positive terminal. How much work does the battery do per unit time?
In other words, what is the power output of the battery? 

Consider a (positive) charge $q$ which flows through the battery from the  negative
terminal to the positive terminal. The battery raises the potential of the
charge by $V$, so the work the battery does on the charge is $q\,V$. 
The total amount of charge which flows through the battery per unit time is,
by definition, equal to the current $I$ flowing through the battery. Thus, the
amount of work the battery does per unit time is simply the product of
the work done  per unit 
charge, $V$, and the charge passing through the battery per unit
time, $I$. In other words,
\begin{equation}
P = V\,I,
\end{equation}
where $P$, of course, stands for the power output of the battery. Thus,
the rule is
\begin{quote}
{\sf The power in a DC circuit is the product of the voltage and the current}.
\end{quote}
This rule does not just apply to batteries. If a current $I$ flows through
some component of a DC circuit which has a potential {\em drop}\/ $V$ in the
direction of current flow then that component {\em gains}\/ the energy per unit time
$V\,I$ at the expense of the rest of the circuit, and {\em vice versa}. 
Incidentally, since the SI unit of power is the watt (W), it follows that 
\begin{equation}
1\,{\rm W} \equiv 1\,{\rm V}\cdot 1\,{\rm A}.
\end{equation}

Consider a resistor $R$ which carries a current $I$. According to Ohm's
law, the potential drop across the resistor is $V= I\, R$. Thus, the
energy gained by the resistor per unit time is
\begin{equation}
P = V \,I = I^2\,R = \frac{V^2}{R}.
\end{equation}
In what form does the resistor acquire this energy? In turns out that
the energy is dissipated as {\em heat} inside the resistor. This effect
is known as {\em Joule heating}. Thus, the above formula 
gives the electrical heating power of a resistor. Electrical energy is converted
into heat ({\em i.e.}, random motion of the atoms which make up the resistor)
as the electrically accelerated free electrons inside the resistor collide with the atoms and,
thereby, transfer all of their kinetic energy to the atoms. It is this
energy which appears  as heat on a macroscopic scale (see Sect.~\ref{s7.3}). 

Household electricity bills depend on  the amount of electrical
 {\em energy} the household in question uses during  a given accounting period,
since the energy usage determines how much coal or gas was burnt on 
the household's behalf
in the local power station during this period. The conventional unit of 
electrical energy usage employed
by utility companies is the {\em kilowatthour}. If electrical energy
is consumed for 1 hour at the rate of 1 kW (the typical rate of consumption of
a single-bar  electric fire) then the total energy usage is
one  kilowatthour (kWh). It follows that 
\begin{equation}
1\,{\rm kWh} =  (1000)\,(60)\,(60) = 3.6\times 10^6\,{\rm J}.
\end{equation}

\subsection{Power and Internal Resistance}\label{s7.9}
Consider a simple circuit in which a battery of emf ${\cal E}$ and internal 
resistance $r$ drives a current $I$ through an external resistor of resistance $R$
(see Fig.~\ref{f7.1}). The external resistor is usually referred
to as the {\em load resistor}.  It could stand for either an electric light,
an electric heating element, or, maybe, an electric motor. The 
basic purpose of
the circuit is to transfer energy from the battery to the load, where it actually
does something useful for us ({\em e.g.}, lighting
a light bulb, or lifting a weight). Let us see to what extent the internal resistance
of the battery interferes with this process.

The equivalent resistance of the circuit is $r+R$ (since the load resistance is
in series with the internal resistance), so the current flowing in the
circuit is given by
\begin{equation}
I = \frac{{\cal E}}{r+R}.
\end{equation}
The power output of the emf is simply
\begin{equation}
P_{\cal E} = {\cal E}\,I = \frac{{\cal E}^2}{r+R}.
\end{equation}
The power dissipated as heat by the internal resistance of the battery is
\begin{equation}
P_r = I^2\,r = \frac{ {\cal E}^2\,r}{(r+R)^2}.
\end{equation}
Likewise, the power transferred to the load is
\begin{equation}\label{e7.25}
P_R = I^2\,R = \frac{ {\cal E}^2\,R}{(r+R)^2}.
\end{equation}
Note that 
\begin{equation}
P_{\cal E} =P_r + P_R.
\end{equation}
Thus, some of the power output of the battery is immediately lost as heat dissipated by the
internal resistance of the battery. The remainder is transmitted to the load. 

Let $y = P_R/({\cal E}^2/r)$ and $x=R/r$. It follows from
Eq.~(\ref{e7.25}) that
\begin{equation}
y = \frac{x}{(1+x)^2}.
\end{equation}
The function $y(x)$ increases monotonically from zero for 
increasing $x$ in the range $0<x<1$, attains
a maximum value of $1/4$ at $x=1$, and then decreases monotonically with increasing
$x$ in the range $x>1$. In other words, if the load resistance $R$ is varied at
constant ${\cal E}$ and $r$ then the transferred power attains a maximum
value of 
\begin{equation}
(P_R)_{\rm max} = \frac{{\cal E}^2}{4 \,r}
\end{equation}
when $R=r$. This is a very important result in electrical engineering.
Power transfer between a voltage source and an external load is at its most efficient when the
resistance of the load matches the internal resistance of the voltage source.
If the load resistance is too low then most of the power output of the voltage
source is dissipated as heat inside the source itself. If the load resistance
is too high then the current which flows in the circuit is too low to
transfer energy to the load at an appreciable rate. Note that in the optimum case,
$R=r$, only {\em half}\/ of the power output of the voltage source
is transmitted to the load. The other half is dissipated as heat inside
the source. 
 Incidentally, electrical engineers call the process by which the resistance of
a load is matched to that of the power supply {\em impedance matching}\/
(impedance is just a fancy name for resistance).

\subsection{Worked Examples}
\subsection*{\em Example 7.1: Ohm's law}
\noindent{\em Question:} What is the resistance at $0^\circ$\,C of a $1.0$\,m
long piece of no.~5 gauge copper wire (cross-sectional area $16.8\,{\rm mm}^2$)?
What voltage must be applied across the two ends of the wire to produce a
current of 10\,A through it?\\
~\\
\noindent{\em Answer:} Using the basic equation $R=\rho\,L/A$, and the
value of $\rho$ for copper given in Tab.~\ref{t7.1}, we have
$$
R = \frac{(1.7\times 10^{-8})\,(1.0)}{(16.8\times 10^{-6})} = 1.0\times 10^{-3}\,
\Omega.
$$

Using Ohm's law $V=I\,R$, we obtain
$$
V = (10)\, (1.0\times 10^{-3}) = 1.0\times 10^{-2} \,{\rm V}.
$$

\subsection*{\em Example 7.2: Equivalent resistance}
\begin{figure*}[h]
\epsfysize=2.5in
\centerline{\epsffile{Chapter07/fig1.eps}}
\end{figure*}
\noindent{\em Question:} A $1\,\Omega$ and a $2\,{\rm \Omega}$ resistor are connected in parallel,
and this pair of resistors is connected in series with a $4\,\Omega$ resistor.
What is the equivalent resistance of the whole combination? What is the
current flowing through the $4\,\Omega$ resistor if the whole combination is
connected across the terminals of a $6\,{\rm V}$ battery (of negligible
internal resistance)? Likewise, what are the currents flowing through the
$1\,\Omega$  and $2\,\Omega$ resistors?\\
~\\
\noindent{\em Answer:} The equivalent resistance of the
 $1\,\Omega$  and $2\,\Omega$ resistors is
$$
\frac{1}{R_{eq}'} = \frac{1}{1} + \frac{1}{2} = \frac{3}{2}\,\Omega^{-1},
$$
giving $R_{\rm eq}' = 0.667\,{\rm \Omega}$. When a $0.667\,{\rm \Omega}$
resistor is combined in series with a $4\,\Omega$ resistor, the
equivalent resistance is $R_{\rm eq} = 0.667 + 4 = 4.667\,\Omega$. 

The current driven by the $6\,{\rm V}$ battery is
$$
I = \frac{V}{R_{\rm eq}} = \frac{(6)}{(4.667)} = 1.29\,{\rm A}.
$$
This is the current flowing through the $4\,\Omega$ resistor, since one end of
this resistor is connected directly to the battery, with no intermediate
junction points. 

The voltage drop across the $4\,\Omega$ resistor is
$$
V_4 = I\,R_4 = (1.29)\,(4) =5.14\,{\rm V}.
$$
Thus, the voltage  drop across the $1\,\Omega$  and $2\,\Omega$ combination is
$V_{12} = 6-5.14 = 0.857\,{\rm V}$. The current flowing through the $1\,\Omega$ 
resistor is given by
$$
I_1 = \frac{V_{12}}{R_1} = \frac{(0.857)}{(1)}= 0.857\,{\rm A}.
$$
Likewise, the current flowing through the $2\,\Omega$ 
resistor is 
$$
I_2 = \frac{V_{12}}{R_2} = \frac{(0.857)}{(2)}= 0.429\,{\rm A}.
$$
Note that the total current flowing through the $1\,\Omega$  and $2\,\Omega$ combination is
$I_{12} = I_1 + I_2 = 1.29\,{\rm A}$, which is the same as the current flowing through
the $4\,\Omega$ resistor. This makes sense because the  $1\,\Omega$  and $2\,\Omega$ combination
is connected in series with the $4\,\Omega$ resistor.

\subsection*{\em Example 7.3: Kirchhoff's rules}
\noindent{\em Question:}
Find the three currents $I_1$, $I_2$, and $I_3$ in the circuit shown in the diagram,
where $R_1= 100\,\Omega$, $R_2=10\,\Omega$, $R_3=5\,\Omega$, ${\cal E}_1=12$\,V
and ${\cal E}_2=6$\,V.\\
~\\
\noindent{\em Answer:} Applying the junction rule to point $a$, and assuming that
the currents flow in the direction shown (the initial choice of 
directions of the currents is arbitrary), we have
$$
I_1 = I_2 + I_3.
$$
There is no need to apply the junction rule again at point $b$, since if the
above equation is satisfied then this rule is automatically satisfied at $b$. 

\begin{figure*}[h]
\epsfysize=3in
\centerline{\epsffile{Chapter07/fig2.eps}}
\end{figure*}


Let us apply the loop rule by going around the various loops in the circuit
in a clockwise direction. For loop $abcd$, we have
$$
-\,I_2\,R_2+{\cal E}_1 - I_1\,R_1=0.
$$
Note that both the terms involving resistors are negative, since we cross
the resistors in question in the direction of nominal current flow.
Likewise, the term involving the emf is  positive since we traverse      the
emf in question from the negative to the positive plate. For loop
$aefb$, we find
$$
-I_3\,R_3 -{\cal E}_2 -{\cal E}_1 + I_2\,R_2  = 0.
$$
There is no need to apply the loop rule to the full loop $defc$, since
this loop is made up of loops $abcd$ and $aefb$, and the loop rules for these two
loops therefore already contain all of the information 
which would be obtained by applying the
loop rule to $defc$. 

Combining the junction rule with the first loop rule, we obtain
$$
(R_1+R_2)\,I_2 + R_1\,I_3 = {\cal E}_1.
$$
The second loop rule can be rearranged to give
$$
-R_2\,I_2 +R_3\,I_3 = -({\cal E}_1 +{\cal E}_2).
$$
The above two equations are a pair of simultaneous algebraic equations
for the currents $I_2$ and $I_3$, and can be solved using the standard method
for solving such equations.
Multiplying the first equation by $R_2$, the second by $(R_1+R_2)$, and adding
the resulting equations, we obtain
$$
(R_1 R_2 + R_2 R_3+ R_1 R_3)\,I_3 = - R_1\,{\cal E}_1 -
(R_1+R_2)\,{\cal E}_2,
$$
which can be rearranged to give
$$
I_3 = - \frac{ R_1\,{\cal E}_1 +
(R_1+R_2)\,{\cal E}_2}{R_1 R_2 + R_2 R_3+ R_1 R_3},
$$
or
$$
I_3 = - \frac{ (100)\,(12)+(110)\,(6)}{(1000+50 + 500)} =- \frac{(1860)}{(1550)}=
-1.2\,{\rm A}.
$$
Likewise, multiplying the first equation by $R_3$, the second by $R_1$, and
taking the difference of the resulting equations, we obtain
$$
(R_1 R_2 + R_2 R_3+ R_1 R_3)\,I_2 = (R_1+R_3)\,{\cal E}_1 
+ R_1\,{\cal E}_2,
$$
which can be rearranged to give
$$
I_2 = \frac{ (R_1+R_3)\,{\cal E}_1 
+ R_1\,{\cal E}_2}{R_1 R_2 + R_2 R_3+ R_1 R_3},
$$
or
$$
I_2 = \frac{ (105)\,(12) + (100)\,(6) }{ (1000+50 + 500)} = \frac{(1860)}{(1550)}
= 1.2\,{\rm A}.
$$
Finally, from the junction rule,
$$
I_1 = I_2 + I_3 = -1.2+1.2 = 0\,{\rm A}.
$$
The fact that $I_3=-1.2\,{\rm A}$ indicates that this current is of magnitude
$1.2$\,A, but flows in the opposite direction to that which we initially
guessed. In fact, we can see that a current of $1.2$\,A circulates in an
anti-clockwise direction in the lower loop of the circuit, whereas zero current
circulates in the upper loop. 

\subsection*{\em Example 7.4: Energy in DC circuits}
\noindent{\em Question:}  A 150\,W light bulb is connected to a 120\,V line.
What is the current drawn from the line? What is the resistance of the light
bulb whilst it is burning? How much energy is consumed if the light is
kept on for 6 hours? What is the cost of this energy at 8 cents/kWh?\\
~\\
\noindent{\em Answer:} Since power is equal to $I\,V$, it follows that
$$
I = \frac{P}{V} = \frac{(150)}{(120)} = 1.25\,{\rm A}.
$$
From Ohm's law, the resistance of the light bulb is
$$
R = \frac{V}{I} = \frac{(120)}{(1.25)} = 96 \,\Omega.
$$
The energy $W$ consumed is the product of the power $P$ (the energy consumed per
unit time) and the time period $t$ for which the light is on, so
$$
W = P\,t = (150)\,(6)\,(60)\,(60) = 3.24\times 10^6\,{\rm J}.
$$
Since, $1\,{\rm kWh} \equiv 3.6\times 10^6\,{\rm J}$, it
follows that
$$
W = \frac{(3.24\times 10^6)}{(3.6\times 10^6)} = 0.9\,{\rm kWh}.
$$
The cost $c$ of the electricity is product of the number of kilowatthours used
and the cost per kilowatthour, so
$$
c = (0.9)\,(0.08) = 0.072\,\,{\rm dollars} = 7.2\,\,{\rm cents}.
$$
